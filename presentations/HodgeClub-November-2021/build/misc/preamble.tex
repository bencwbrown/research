%
% XeTeX specific
%
\RequireXeTeX
\usepackage{xltxtra}
\usepackage{fontspec}
\usepackage{xunicode}

% -----------------------------------
% font config
% -----------------------------------

\setsansfont{Linux Biolinum}
\setromanfont{Linux Libertine}
\setmonofont[Scale=0.9]{Consolas}

% -----------------------------------
% math
% -----------------------------------

\usepackage{amsmath}
\usepackage{amsfonts}
\usepackage{amssymb}
\usepackage{amsthm}
\usepackage{mathrsfs}   % \mathscr
\usepackage{stmaryrd}   % \lightning

% -----------------------------------
% grammar and textstyle
% -----------------------------------

\usepackage{polyglossia}
\setdefaultlanguage[variant=british]{english}
\usepackage{csquotes}
% underlining
\usepackage{soul} % ulem redifines \emph, this sucks

% -----------------------------------
% colors
% -----------------------------------

\usepackage{xcolor}
\usepackage{colortbl}

% listing colors
% https://kuler.adobe.com/Color-palette-color-theme-4004722
\definecolor{keyword}{RGB}{239,33,74}
\definecolor{number}{RGB}{243,202,22}
\definecolor{comment}{RGB}{126,158,19}
\definecolor{string}{RGB}{6,129,128}


% -----------------------------------
% links and references
% -----------------------------------

\usepackage{hyperref}
\hypersetup{
    colorlinks=false,
    % hide bookmarks
    pdfpagemode=UseNone,
    % meta
    pdfauthor={\presentationAuthor},
    pdftitle={\presentationTitle}
}
\usepackage[english]{cleveref}

% -----------------------------------
% bibliography and glossaries
% -----------------------------------

\usepackage[
    backend=biber,
    style=alphabetic-verb
    ]{biblatex}
\bibliography{presentation}

\usepackage{glossaries}
\input{glossary}
\makeglossaries

% -----------------------------------
% graphics
% -----------------------------------

\usepackage{graphicx}
\usepackage{tikz}
\usetikzlibrary{shapes.geometric, arrows, shadows, positioning}
\usepackage{adforn} % ornaments, used in titlepage

% -----------------------------------
% beamer theme
% -----------------------------------

% you can't locate the theme in a subfolder without shooting yourself in the knee
\usetheme{alinz}

% -----------------------------------
% listings and pseudocode
% -----------------------------------

\Crefname{lstlisting}{Listing}{Listings}
\crefname{lstlisting}{listing}{listings}

\usepackage{listings}
\lstset{
    basicstyle=\footnotesize\ttfamily\color{lightdark},
    backgroundcolor=\color{blockbg},
    numbers=left,
    %numbersep=6pt,
    numberstyle=\scriptsize\color{granite},
    commentstyle=\sffamily\itshape\color{sea},
    keywordstyle=\bfseries\color{raspberry},
    stringstyle=\itshape\color{lake},
    showstringspaces=false,
    breaklines=true,
    breakatwhitespace=true,
    frame=lr,
    framerule=0pt,
    framesep=6pt,
    captionpos=b
}
% for pseudocode
\usepackage[slide,linesnumbered,algoruled]{algorithm2e}

\input{misc/javascript}
%\input{misc/html5}

% -----------------------------------
% custom commands
% -----------------------------------

\usepackage[utf8]{inputenc} % allow utf-8 input
\usepackage[T1]{fontenc}    % use 8-bit T1 fonts
\usepackage{hyperref}       % hyperlinks
\usepackage{url}            % simple URL typesetting
\usepackage{booktabs}       % professional-quality tables
\usepackage{nicefrac}       % compact symbols for 1/2, etc.
\usepackage{microtype}      % microtypography
% \usepackage{lipsum}		% Can be removed after putting your text content
\usepackage{graphicx}
\usepackage{epstopdf}
\usepackage{url}
\usepackage{setspace}
\usepackage{enumitem}
\usepackage{parskip}
\usepackage{IEEEtrantools}
\usepackage{mathtools}
\usepackage{tensor}
\usepackage{yfonts}
\usepackage{dsfont}

%%%%%%%%%%%%%%%%%%% Custom packages
\usepackage{braket}
\usepackage{todo}
\usepackage{xargs}                      % Use more than one optional parameter in a new commands
\usepackage{tikz}
\usepackage{tikz-cd}

\newcommand{\ie}{\emph{i.e.} }
\newcommand{\eg}{\emph{e.g.} }
\newcommand{\cf}{\emph{cf.} }
\newcommand{\ra}{\rightarrow}
\newcommand{\la}{\leftarrow}
\newcommand{\lra}{\longrightarrow}
\newcommand{\lla}{\longleftarrow}
\newcommand{\lbracket}{\left(}
\newcommand{\rbracket}{\right)}

\newcommand{\al}{\alpha}
\newcommand{\w}{\omega}
\newcommand{\W}{\Omega}
\newcommand{\e}{\epsilon}
\newcommand{\K}{K\"ahler }
\newcommand{\into}{\hookrightarrow}
\newcommand{\PP}{\mathbb{P}}
\newcommand{\RR}{\mathbb{R}}
\newcommand{\CC}{\mathbb{C}}
\newcommand{\QQ}{\mathbb{Q}}
\newcommand{\FF}{\mathbb{F}}
\newcommand{\ZZ}{\mathbb{Z}}
\newcommand{\NN}{\mathbb{N}}
\newcommand{\HH}{\mathbb{H}}
\newcommand{\vp}{\varphi}
\newcommand{\mcA}{\mathcal{A}}
\newcommand{\mcC}{\mathcal{C}}
\newcommand{\mcE}{\mathcal{E}}
\newcommand{\mcF}{\mathcal{F}}
\newcommand{\mcG}{\mathcal{G}}
\newcommand{\mcH}{\mathcal{H}}
\newcommand{\mcL}{\mathcal{L}}
\newcommand{\mcO}{\mathcal{O}}
\newcommand{\mcR}{\mathcal{R}}
\newcommand{\mfg}{\mathfrak{g}}
\newcommand{\mfh}{\mathfrak{h}}
\newcommand{\mfk}{\mathfrak{k}}
\newcommand{\mft}{\mathfrak{t}}

\newcommand{\sssslash}{\mathbin{/\mkern-6mu/\mkern-6mu/\mkern-6mu/}}
\newcommand{\half}{\frac{1}{2}}
\newcommand{\thalf}{\tfrac{1}{2}}