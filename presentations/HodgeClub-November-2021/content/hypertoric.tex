\section{Hypertoric Manifolds}

\begin{frame}
    \begin{itemize}
        \item $M_{\lambda} \cong T^{\ast}\CC\PP^{2}$ is an example of a \textcolor{red}{hypertoric manifold}.\
        \item Had an exact sequence:
        \[
            1 \ra S^{1} \overset{\imath}{\hookrightarrow} T^{3} \overset{\pi}{\rightarrow} T^{3}/S^{1} \cong T^{2} \ra 1.    
        \]
        \item $\ker(\pi) = \Image(\imath)$, where $\imath(t) = (t,t,t)$.
        \item Differentiating, $\imath_{\ast}(1) = (1,1,1)$.
        \[
            0 \ra \RR \overset{\imath_{\ast}}{\hookrightarrow} \RR^{3} \overset{\pi_{\ast}}{\rightarrow} \RR^{2} \ra 0.    
        \]
        \item $\ker(\pi_{\ast}) = \Span_{\RR}\{(1,0,-1),\, (0,1,-1)\}$,
        \[
        \implies \pi_{\ast} = \begin{bmatrix} 1 & 0 & -1 \\ 0 & 1 & -1 \end{bmatrix} = \begin{bmatrix} u_{1} & u_{2} & u_{3} \end{bmatrix}.
        \]
    \end{itemize}
\end{frame}

\begin{frame}{Generally}
    \begin{itemize}
        \item $\Image(\bar{\mu}_{\RR})$ is hyperplane arrangement $\{H_{1}, \ldots, H_{n} \}$ in $\RR^{d}$.
        \item The $u_{i}$ are the normals to the $H_{i}$.
        \item $\pi_{\ast} : \RR^{n} \ra \RR^{d}$, via $\pi : e_{i} \mapsto u_{i}$.
        \item Set $\mfk = \ker(\pi_{\ast}) \subset \RR^{n} \rightsquigarrow$ inclusion $\imath_{\ast} : \mfk \into \RR^{n}$. 
    \end{itemize}
    \begin{equation*}
        \xymatrix @R=-0.2pc {
        0 \ar[r] & \mfk \ar[r]^{\imath_{\ast}} & \RR^{n} \ar[r]^{\pi_{\ast}} & \RR^{d} \ar[r] & 0, \\
                 &                                & e_{i} \ar@{|->}[r] & u_{i} & \\
        }
        \end{equation*}
    \begin{itemize}
        \item $T^{n} \acts T^{\ast}\CC^{n}$ diagonally, moment map $\mu_{\HK} : T^{\ast}\CC^{n} \ra \RR^{3n}$.
        \item $K  = \exp(\mfk) \acts T^{\ast}\CC^{n}$ too via inclusion $\imath : K \into T^{n}$.
        \item Induces (real) moment map $\imath^{\ast} \circ \mu_{\RR} : T^{\ast}\CC^{n} \ra \mfk^{\ast}$, that we reduce with respect to.
    \end{itemize}
\end{frame}

\begin{frame}
    \begin{itemize}
        \item $T^{n}/K \cong T^{d} \acts M_{\lambda} := T^{\ast}\CC^{n} \sssslash_{\lambda} K$, \& $\lambda = (\alpha,0) \in \mfk^{\ast}\otimes \RR^{3}$.
        \item Now $T^{d}$ moment map can be written down
        \begin{align*}
            \phi_{\RR}[z,w] &= \tfrac{1}{2}\left(\sum_{i=1}^{n} |z_{i}|^{2} - |w_{i}|^{2} - \alpha \right)e_{i} \in \ker(\imath^{\ast}) \cong \RR^{d} \\
            \phi_{\CC}[z,w] &=\sum_{i=1}^{n}(z_{i}w_{i})e_{i} \in \ker(\imath_{\CC}^{\ast}) \cong \CC^{d}.
        \end{align*}
    \end{itemize}
\end{frame}

\begin{frame}
    \begin{itemize}
    \item Let's use combinatorics! Choose a lift $\beta = (\beta_{1}, \ldots, \beta_{n}) \in \RR^{n}$ such that $\imath^{\ast}(\beta) = \alpha$.
    \item $\{H_{1}, \ldots, H_{n}\}$ hyperplane arrangement in $\RR^{d}$:
    \[
        H_{i} = \{x \in \RR^{d}\, |\, \langle x, u_{i} \rangle = \beta_{i} \}.
    \]
    \item Half-spaces
    \begin{align*}
        F_{i} &= \{x \in \RR^{d}\, |\, \langle x, u_{i} \rangle \geq \beta_{i} \}, \\
        G_{i} &= \{x \in \RR^{d}\, |\, \langle x, u_{i} \rangle \leq \beta_{i} \}.
    \end{align*}
    \item So $H_{i} = F_{i} \cap G_{i}$.
    \end{itemize}
\end{frame}

\begin{frame}
    \begin{block}{Lemma}
        If $w_{i} = 0$, then $\phi_{\RR}[z,w] \in F_{i}$, and if $z_{i} = 0$, then $\phi_{\RR}[z,w] \in G_{i}$.
    \end{block}
    Recall:
    \begin{align*}
        \phi_{\RR}[z,w] &= \tfrac{1}{2}\left(\sum_{i=1}^{n} |z_{i}|^{2} - |w_{i}|^{2} - \beta_{i} \right)e_{i} \in \ker(\imath^{\ast}), \\
        \mu_{\RR}(z,w) &= \tfrac{1}{2}\left(\sum_{i=1}^{n} |z_{i}|^{2} - |w_{i}|^{2}\right)e_{i} \in \RR^{n}.
    \end{align*}
    \begin{block}{Proof, \textcolor{blue}{N. Proudfoot (2004)}}
        If $w_{i} = 0$, 
        \[
            \langle \phi_{\RR}[z,w], u_{i} \rangle = \langle \mu_{\RR}(z,w), e_{i} \rangle - \beta_{i} = \tfrac{1}{2}(|z_{i}|^{2} - |w_{i}|^{2}) - \beta_{i} \geq 0
        \]
        so $\phi_{\RR}[z,w] \in F_{i}$.
    \end{block}
\end{frame}

\begin{frame}
    \begin{itemize} 
    \item As $\phi_{\CC}[z,w] = \sum_{i=1}^{n}(z_{i}w_{i})e_{i} = 0$, for
    \[
        \mathcal{E}_{A} = \{[z,w] \in M_{\alpha}\ |\ z_{i} = 0 \text{ for all } i \in A,\, w_{i} = 0 \text{ for all } i \not\in A \} 
    \]
    we have that $\mathcal{E}_{A} \subseteq \phi_{\CC}^{-1}(0)$, for each subset $A$.
    \item For $A = \emptyset$, all $w$'s are zero $\rightsquigarrow$ get $X := \{[z,0] \in M\}$, which is a \textcolor{red}{symplectic toric subvariety} \`a la the \textcolor{red}{Delzant construction}.
    \item Its moment map image is the convex polytope
    \[
        \Delta_{\emptyset} = \bigcap_{i=1}^{n} F_{i}.    
    \]
    \item Can be proven that $T^{\ast}X \subseteq M$, open and densely.
    \end{itemize}
\end{frame}

\begin{frame}
    \begin{itemize}
        \item More generally,
        \[
            \Delta_{A} = \left(\bigcap_{i \not\in A} F_{i}\right) \cap \left( \bigcap_{i \in A} G_{i} \right),
        \]
        is a convex (not necessarily bounded) polytope in $\RR^{d}$.
        \item Each $\Delta_{A}$ corresponds to a $2d$-dimensional K\"ahler submanifold $X_{A} \subset M$, and is compact if $\Delta_{A}$ is bounded.
    \end{itemize}
\end{frame}

\begin{frame}{Adjectives for Hyperplane Arrangements}
    Hyperplane arrangement $\{H_{1}, \ldots, H_{n}\}$ with normals $\{u_{1}, \ldots, u_{n}\}$:
    \begin{itemize}
        \item A hyperplane arrangement is \textcolor{red}{rational} if $u_{i} \in \ZZ^{d}$, for each $1 \leq i \leq n$,
        \item A hyperplane arrangement is \textcolor{red}{simple} if any subset of $m$ hyperplanes intersect in codimension $m$,
        \item A hyperplane arrangement is \textcolor{red}{smooth} if every subset of $d$ vectors from the $\{u_{i}\}$ forms a basis of $\ZZ^{d}$.
    \end{itemize}
\end{frame}

\begin{frame}
    \begin{itemize}
        \item If an arrangement is simple \textcolor{red}{but not} smooth, then we get a hypertoric \textcolor{red}{orbifold},
        \item Example:
        \[
            u_{1} = (1,0),\quad u_{2} = (0,1), \quad u_{3} = (-1,-2), \text{ in } \RR^{2}.
        \]
        \item $u_{1}$ and $u_{3}$ do not form a basis of $\ZZ^{2}$.
        \item In this case
        \[
            1 \ra K \into T^{3} \ra T^{2} \ra 1,    
        \]
        has $K \cong S^{1} \times \ZZ/2\ZZ$ (i.e. if arrangement is non-smooth, $K$ is not connected).
    \end{itemize}
\end{frame}

\begin{frame}
    \begin{itemize}
        \item For $K \cong S^{1} \times \ZZ/2\ZZ$,
    \end{itemize}
\end{frame}