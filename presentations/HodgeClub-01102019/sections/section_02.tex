\section{Geometric Invariant Theory}

In this section, we shall show that the phenomena that occurred in the previous examples happens in much more generality. To start with, let us assume that a \emph{complex} Lie group $G$ is the complexification of a maximal compact subgroup $K$, so $G = K_{\CC}$. This is one of the many equivalent definitions for a Lie group to be \emph{reductive}, and is a necessary and sufficient condition for the coordinate subring of invariant functions under the action of $G$, denoted $R(X)^{G}$, for some algebraic variety, to be finitely generated (this is Nagata's theorem).

Mumford's geometric invariant theory associates to an action of $G$ on $X$ a projective `quotient' variety, denoted $X \sslash G$, which is the projective variety associated to the subring $R(X)^{G}$ of $G$-invariants in the coordinate ring $R(X)$ of $X$.

\subsection{The $\Spec$ and $\Proj$ Constructions}

Let $R$ be a finitely generated integral domain over $\CC$, which means that there exists a surjective $\CC[x_{1},\ldots, x_{n}] \doublerightarrow R$ whose kernel $I \lhd \CC[x_{1},\ldots, x_{n}]$ is a prime ideal. For the purpose of this talk, we shall define $\Spec(R)$ to be the subset of $\CC^{n}$ on which all the polynomials in $I$ vanish. Of course, $\Spec(R)$ can be equipped with either the Zariski topology or the analytic topology, along with their respective sheaves of rings, though I will ignore these technicalities. If we do not want to list a generating set for $R$, then the definition for $\Spec(R)$ is that it is the set of maximal ideals in $R$.

\begin{ex}
	Let $R = \CC[x_{1},\ldots, x_{n}]$, then by the Nullstellensatz, any maximal ideal $I \lhd R$ is of the form $I = (x_{1} - p_{1},\ldots, x_{n} - p_{n})$, with each $p_{i} \in \CC$. Thus $\Spec(R) \cong \CC^{n}$.
\end{ex}

Suppose that a ring $R$ is equipped with a grading
$$
R = \bigoplus_{m=0}^{\infty}R_{m},
$$
which allows us to define an action of the group $\CC^{\ast}$ on the ring $R$ by setting $t \cdot f := t^{m}f$, for any $f \in R_{m}$ and $t \in \CC^{\ast}$. This in turn induces an action of $\CC^{\ast}$ on $\Spec(R)$ by setting $(t^{-1} \cdot f)(x) := f(t\cdot x)$ (the inverse exponent is required to make the action associative, or `compatible'). A fixed point $x \in \Spec(R)$ is fixed if and only if $t\cdot x = x$ for all $t \in \CC^{\ast}$, which is the same as
$$
f(x) = f(t\cdot x) = (t^{-1} \cdot f)(x),\qquad \text{for every } f \in R.
$$
If $f \in R_{0}$, then this is true trivially, whereas if $f \in R_{m}$ with $m \neq 0$, then this condition is equivalent to $f(x) = 0$. Hence the fixed-point set of $\CC^{\ast}$ is equal to the vanishing set of the ideal generated by all the functions of non-zero weight.

We call the following ideal
$$
R_{+} = \bigoplus_{m > 0} R_{m}
$$
the \emph{irrelevant ideal} of $R$, which is exactly the vanishing ideal of the fixed-point set for the action of $\CC^{\ast}$ on $\Spec(R)$. Thus the fixed-point set is equal to $\Spec(R/R_{+}) \cong \Spec(R_{0})$.

\begin{defn}
	For a graded ring $R = \oplus_{m = 0}^{\infty}R_{m}$, define the \emph{projective spectrum} of $R$ to be
	$$
	\Proj(R) := (\Spec(R) \setminus \Spec(R_{0}))/\CC^{\ast}.
	$$
\end{defn}

\begin{ex}
	For $R = \CC[x_{0}, \ldots, x_{n}]$, where $\deg(x_{i}) = 1$ for each $0 \leq i \leq n$ (that is, $\CC^{\ast}$ acts on each $x_{i}$ with weight 1), we have that $R_{0} = \CC$ and the irrelevant ideal is $R_{+} = (x_{0}, \ldots, x_{n})$. Thus, we have a $\CC^{\ast}$-equivariant embedding $\Spec(R) = \CC^{n+1}$ with $\Spec(R/R_{+}) = \Spec(R_{0}) \cong \{0\}$, so
	$$
	\Proj(R) = (\Spec(R) \setminus \Spec(R_{0}) )/\CC^{\ast} = (\CC^{n+1} \setminus \{0\} )/\CC^{\ast} = \PP^{n}.
	$$
\end{ex}

\subsection{Affine GIT Quotients}

Suppose that $R$ is a finitely generated integral domain over $\CC$, so that $X = \Spec(R)$ defines a complex affine algebraic variety $X$. Further suppose that $G$ is a reductive algebraic group acting on $R$, so that its $\CC$-subalgebra of invariants $R^{G}$ is finitely generated by Nagata's theorem.

\begin{defn}
	The \emph{affine GIT quotient} of $X$ with respect to an action of $G$ is
	$$
		X \sslash G := \Spec(R^{G}).
	$$
\end{defn}

\begin{prop}
	There is a surjective map from $X$ to $X \sslash G$, satisfying: any two points in $X$ lie in the same fibre of this map if and only if the closures of their $G$-orbits intersect non-trivially.
\end{prop}

Let us investigate what happens if we naively take this usual quotient; let $X = \CC^{2} = \Spec(\CC[x,y])$ and $G = \CC^{\ast}$ act on $X$ as
$$
	\lambda \cdot (x,y) = (\lambda x, \lambda^{-1}y).
$$
Then $\CC[x,y]^{G} = \CC[xy] \cong \CC[z]$ and the GIT quotient map should be
$$
	\varphi:X \longrightarrow X \sslash G = \Spec(\CC[z]) \cong \CC,\qquad (x,y) \longmapsto xy = z.
$$
However the closures of the non-closed three orbits
$$
	\CC^{\ast}\cdot (x,y),\quad \CC^{\ast} \cdot (0,y),\quad \CC^{\ast}\cdot (0,0),
$$
are then identified, so $\vp^{-1}(0)$ is equal to the union of the two coordinate axes. Thus away from the coordinate axes, the $\CC^{\ast}$-orbits are closed and they can be separated, yet the $\CC^{\ast}$-orbits of each coordinate axis are identified, and cannot be \emph{separated} (the italics are to emphasise that separatedness in algebraic geometry is analogous to the Hausdorff property in topology).

This discussion motivates the following definitions:
\begin{defn}
	Let $X = \Spec(R)$ and let $G$ be a reductive algebraic group that acts on $R$, and thus on $X$. Then:
	\begin{enumerate}
		\item A point $x \in X$ is \emph{stable} if its orbit is closed in $X$ and $\dim G_{x} = 0$, \ie its stabiliser $G_{x}$ is finite. Denote by $X^{s}$ the stable locus of points in $X$.
		\item A point $x \in X$ is \emph{semi-stable} if zero does not belong in the closure of its orbit under the $G$-action, \ie $0 \not\in \overline{G \cdot x}$. Denote by $X^{ss}$ the semi-stable locus of points in $X$.
		\item A point $x \in X$ is \emph{unstable} if it is not semi-stable.
	\end{enumerate}
\end{defn}

So in the previous example, all the points disjoint from the $x$ and $y$-coordinate axes were stable, yet the points in them were unstable. So after removing the unstable points, we would get a well-behaved quotient.

\subsection{Projective GIT Quotients}

To construct something similar for projective algebraic varieties, we need to remove some bad points from the orbit where the rational morphism induced by $R^{G} \hookrightarrow R$ is undefined. This is similar to the removal of the origin in $\CC^{n+1}$ when we constructed $\PP^{n}$.

\begin{defn}
	For a linear action of a reductive group $G$ on a projective variety $X \subseteq \PP^{n}$, we denote by $X \sslash G$ the projective variety $\Proj(R^{G})$ associated to the finitely generated $\CC$-algebra $R^{G}$ of $G$-invariant functions, where $R = R(X)$ is the homogeneous coordinate ring of $X$. The inclusion $R^{G} \hookrightarrow R$ defines a rational map
	$$
		\varphi : X \rightarrow X \sslash G
	$$
	which is undefined on the \emph{null-cone}
	$$
		N_{R^{G}}(X) := \{ x \in X \st f(x) = 0, \text{ for all } f \in R_{+}^{G} \}.
	$$
	Define the \emph{semi-stable locus} to be the complement in $X$ of the null-cone, $X^{ss} := X \setminus N_{R^{G}}(X)$, then the \emph{projective GIT quotient} for the action of $G$ on $X \subseteq \PP^{n}$ is the morphism $\vp: X^{ss} \rightarrow X\sslash G$.
\end{defn}

\begin{ex}
	Let $G = \CC^{\ast}$ act on $X = \PP^{n}$ by
	$$
		t \cdot [x_{0}: x_{1}: \ldots : x_{n}] = [t^{-1}x_{0}: tx_{1}: \ldots : tx_{n}].
	$$
	Here, the homogeneous coordinate ring is $R = \CC[x_{0},\ldots, x_{n}]$ which is graded by degree. It is not hard to see that the $G$-invariant subring is 
	$$
		R^{G} = \CC[x_{0}x_{1}:\ldots, x_{0}x_{n}] \cong \CC[y_{1},\ldots, y_{n}],
	$$ which corresponds to the projective variety $X \sslash G  = \Proj(R^{G}) \cong \PP^{n-1}$. Since we have chosen explicit generators for both $R$ and $R^{G}$, we can write down the rational morphism
	$$
		\vp: \PP^{n} = X \longrightarrow X \sslash G = \PP^{n-1},\qquad [x_{0}: x_{1}: \ldots : x_{n}] \longmapsto [x_{0}x_{1} : \ldots : x_{0}x_{n} ],
	$$
	from which it is clear that the null-cone is
	$$
		N_{R^{G}}(X) = \{ [x_{0}:\ldots : x_{n}] \in \PP^{n} \st x_{0} = 0, \text{ or } x_{1}, \ldots, x_{n} = 0   \}
	$$
	is the projective variety defined by the homogeneous ideal $I = (x_{0}x_{1}, \ldots, x_{0}x_{n})$. Thus
	$$
		X^{ss} = \{ [x_{0}: \ldots : x_{n}] \in \PP^{n} \st x_{0} \neq 0 \text{ and } (x_{1},\ldots, x_{n}) \neq (0,\ldots,0) \} \cong \CC^{n} \setminus \{0\}.
	$$
	Therefore
	$$
		\vp: X^{ss} = \CC^{n} \setminus \{0\} \longrightarrow X \sslash G = \PP^{n-1}.
	$$
\end{ex}













