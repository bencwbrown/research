\section{The Heisenberg Group and Abstract Configurations}

\subsection{The Heisenberg Group and Schr{\"o}dinger Representation}

We largely follow \cite{Hulek_1983} for most of this section and we set $V = \CC^{n}$ for brevity. We saw in the previous section that the action of the $n$-torsion subgroup $E_{\tau}[n]$ of the elliptic curve $E_{\tau} = \CC/\Lambda_{\tau}$ induced an action on $\PP^{n-1} = \PP(V)$.  Consider the transformations\footnote{For the purposes of this report and for the content of this section to be closer to the literature, we consider $\sigma(x_{i}) = x_{i-1}$ rather than $\sigma(x_{i}) = x_{i+1}$ as it was in the previous sections.}
\begin{equation*}
	\begin{split}
	\sigma: x_{i} &\longmapsto x_{i-1},\\
	\tau: x_{i} &\longmapsto \e^{i}x_{i},
	\end{split}
\end{equation*}
where $\e = e^{2\pi i/n}$ is a primitive $n$-th root of unity and $i \in \ZZ/n\ZZ$. As an abstract group, $E_{\tau}[n] \cong \ZZ/n\ZZ \times \ZZ/n\ZZ$ and in particular is commutative. However the transformations $\sigma$ and $\tau$ fail to commute, but only very little: $[\sigma,\tau] = \e \cdot\Id_{n}$.\\

\begin{defn}[\cite{Hulek_1983}]
	The subgroup $H_{n} \subset \GL(V)$ generated by $\sigma$ and $\tau$ is called the \emph{Heisenberg group of level $n$}. The representation of $H_{n}$ defined by the inclusion is called the \emph{Schr{\"o}dinger representation of the Heisenberg group.}\\
\end{defn}

\begin{remark}
	\begin{enumerate}
		\item The centre of the Heisenberg group $H_{n}$ equals
		\begin{equation*}
		\mu_{n} = \{ \e^{m} \cdot\Id_{V} \st m \in \ZZ/n\ZZ  \}
		\end{equation*}
		and the group $H_{n}$ is a central extension
		\begin{equation*}
		1 \longrightarrow \mu \longrightarrow H_{n} \longrightarrow \ZZ/n\ZZ \times \ZZ/n\ZZ \longrightarrow 0,
		\end{equation*}
		where $\sigma$ and $\tau$ are mapped to $(1,0)$ and $(0,1)$ respectively. The order of $H_{n}$ is $n^{3}$ and in fact, if $n = p \geq 3$ is a prime number then $H_{p}$ is the unique group of order $p^{3}$ with exponent $p$.
		
		\item When $n$ is odd, the Heisenberg group $H_{n}$ actually lies within $\SL(V)$.
		
		\item The commutator map induces a non-degenerate skew-symmetric bilinear form
		\begin{equation*}
			(\ ,\ ):(\ZZ/n\ZZ\times \ZZ/n\ZZ) \times (\ZZ/n\ZZ \times \ZZ/n\ZZ) \rightarrow \ZZ/n\ZZ
		\end{equation*}
		given by
		\begin{equation}
			\label{commutator}
			\e^{(\alpha,\beta)} = \frac{1}{n}\Tr(\alpha\beta \alpha^{-1}\beta^{-1}).
		\end{equation}
				
		\item The Schr{\"o}dinger representation of $H_{n}$ is an irreducible representation of $H_{n}$, and if $p$ is a prime number then we can describe all irreducible representations of $H_{p}$. To do this let
		\begin{equation*}
			\rho: H_{p} \rightarrow \GL(V)
		\end{equation*}
		be the Schr{\"o}dinger representation, and denote the corresponding $H_{p}$-module by $V^{1}$. The Schr{\"o}dinger representation gives rise to $p-1$ irreducible $H_{p}$-modules $V^{i}$, $i = 1,\ldots, p-1$ of dimension $p$ in the following way:
		\begin{equation*}
			\begin{split}
			\rho^{i}: H_{p} &\longrightarrow \GL(V),\\
			\rho^{i}(\sigma) &:= \sigma,\\
			\rho^{i}(\tau) &:= \tau^{i}.
			\end{split}
		\end{equation*}
		In addition, $\ZZ/p\ZZ \times \ZZ/p\ZZ$ and hence also $H_{p}$ has $p^{2}$ characters which we will denote by $V^{kl}$ with $k,l \in \ZZ/p\ZZ$. Since the sum over the squares of the dimensions of the irreducible representations described is equal to
		\begin{equation*}
			(p-1)p^{2} + p^{2} = p^{3} = |H_{p}|,
		\end{equation*}
		this is a complete list of irreducible $H_{p}$-modules.
		
		\item As the Schr{\"o}dinger representation is irreducible, it follows that our elliptic curves are in fact normal, i.e. they span their respective $\PP^{n-1}$ under the action of the Heisenberg group.
	\end{enumerate}
\end{remark}

\subsection{Abstract Configurations}
Again here, we largely follow \cite{Hulek_1983}.
In this section we restrict to the case where $n = p \geq 3$ is a prime number.
Recall that there are exactly $p+1$ subgroups $\ZZ_{p} \subset \ZZ_{p} \times \ZZ_{p}$, that are generated by $(0,1)$ and $(1,l)$, $l \in \ZZ_{p}$ respectively. To describe the configurations acted upon the the Heisenberg group, we need to determine all hyperplane $H \subset \PP^{p-1}$ which are invariant under one of these subgroups.

We start with the subgroup generated by $(0,1)$. Clearly
\begin{equation*}
	\tau(H) = H\quad\text{if and only if}\quad H_{k} = \{ x_{-k} = 0\}.
\end{equation*}
Further we note that
\begin{equation*}
	H_{k} = \sigma^{k}(H_{0}).
\end{equation*}

Now we will determine all hyperplane $H$ such that
\begin{equation*}
	\tau^{l}\sigma(H) = H.
\end{equation*}
First of all, due to $\sigma$, the equation of any such $H$ must be of the form
\begin{equation*}
	x_{0} + \sum_{m=1}^{p-1}\lambda_{m}x_{m} = 0.
\end{equation*}
Then invariance under $\tau^{l}\sigma$ is equivalent to
\begin{equation*}
	\begin{split}
	\lambda_{1}^{p} &= 1,\\
	\lambda_{m} &= \lambda_{1}^{m}\cdot \e^{\tfrac{1}{2}m(m-l)},\quad\text{for } m = 2,\ldots, p-1.
	\end{split}
\end{equation*}
Hence in setting
\begin{equation*}
	\lambda_{1} = \e^{-\tfrac{1}{2}(p-1)l - k}
\end{equation*}
for some $k \in \ZZ_{p}$, the other $\lambda_{m}$'s become
\begin{equation*}
	\lambda_{m} = \e^{\tfrac{m}{2}(m-p)l - mk}.
\end{equation*}
It follows that the $p$ hyperplanes
\begin{equation*}
	H_{kl} = \bigg\{ \sum_{m=0}^{p-1} \e^{\tfrac{m}{2}(m-p)l - mk}x_{m} = 0 \st k = 0,\ldots, p-1\bigg\}
\end{equation*}
are exactly the hyperplanes invariant under $\tau^{l}\sigma$. Also we observe that
\begin{equation*}
	H_{kl} = \tau^{k}(H_{0l}).
\end{equation*}

To summarise:\\

\begin{prop}[\cite{Hulek_1983}]
	\label{invariant_hyperplanes}
	For each of the $p+1$ subgroups $\ZZ_{p} \subset \ZZ_{p} \times \ZZ_{p}$ there are exactly $p$ hyperplanes which are invariant under this subgroup.
\end{prop}

Now we look at the involution $\imath(x_{m}) = x_{-m}$ on $V = \CC^{p}$. It defines a decomposition of $V$ into eigenspaces, namely
\begin{equation*}
	\CC^{p} = E^{+} \oplus E^{-},
\end{equation*}
where
\begin{equation*}
	\begin{split}
	E^{+} &= \big\langle e_{0},\ e_{1} + e_{p-1},\ \ldots\ ,\ e_{\tfrac{p-1}{2}} + e_{\tfrac{p+1}{2}}\big\rangle,\\
	E^{-} &= \big\langle e_{1} - e_{p-1},\ \ldots\ ,\ e_{\tfrac{p-1}{2}} - e_{\tfrac{p+1}{2}}\big\rangle.
	\end{split}
\end{equation*}
Clearly $\dim E^{+} = \frac{1}{2}(p+1)$ and $\dim E^{-} = \frac{1}{2}(p-1)$.\\

\begin{lemma}\cite{Hulek_1983}
$E^{-} = H_{0} \cap H_{0,0} \cap \ldots \cap H_{0,p-1}$.
\end{lemma}
\begin{proof}
	\begin{enumerate}
		\item We shall first show that $E^{-}$ is contained in this intersection. Clearly $E^{-} \subseteq H_{0}$. Furthermore since $H_{0l}$ is given by
		\begin{equation*}
		\sum_{m=0}^{p-1}\lambda_{m}^{l}x_{m} = 0,\qquad \text{where}\qquad \lambda_{m}^{l} = \e^{\tfrac{1}{2}m(m-p)l}.
		\end{equation*}
		The assertion now follows from
		\begin{equation*}
		\lambda_{p-m}^{l} = \e^{\tfrac{1}{2}(p-m)(-m)} = \e^{\tfrac{1}{2}m(m-p)l} = \lambda_{m}^{l}.
		\end{equation*}
		\item To finish the proof, we need to show that the $\tfrac{1}{2}(p+1)$ of the hypersurfaces $H_{0}$ and $H_{0l}$ are independent. This boils down to examining the matrix
		\begin{equation*}
		V = 
		\begin{bmatrix}
		1 & 1 & \lambda_{0} & \lambda_{0}^{2} & \ldots & \lambda_{0}^{p-1}\\
		0 & 1 & \lambda_{1} & \lambda_{1}^{2} & \ldots & \lambda_{1}^{p-1}\\
		\vdots & \vdots & \vdots & \vdots & \ddots & \ldots\\
		0 & 1 & \lambda_{p-1} & \lambda_{p-1}^{2} & \ldots & \lambda_{p-1}^{p-1}
		\end{bmatrix}
		\end{equation*}
		Using the formula for the Vandermonde determinant, namely
		\begin{equation*}
		\det(V) = \prod_{0\leq i < j \leq p-1}(\lambda_{j} - \lambda_{i}),
		\end{equation*}
		it suffices to see that $\tfrac{1}{2}(p+1)$ of the $\lambda_{m}$'s are different. Hence we look at
		\begin{equation*}
		\lambda_{2k} = \e^{k(2k-p)} = \e^{2k^{2}}.
		\end{equation*}
		It suffices to see that $2k^{2} \not\equiv 2l^{2}$ mod $p$ for $k\neq l \in \{0,\ldots, \tfrac{p-1}{2}\}$. But this is clearly so since othersince $p$ would divide $2(k-l)(l+k)$, which is impossible.
	\end{enumerate}
\end{proof}

We now define for all $k,l \in \ZZ_{p}$ the subspaces
\begin{equation*}
	E_{kl} := \tau^{k}\sigma^{l}(E^{-}).
\end{equation*}

\begin{lemma}
	$E_{kl} \cap E_{k^{\prime}l^{\prime}} = 0$ if $(k,l) \neq (k^{\prime}, l^{\prime})$.
\end{lemma}

\begin{proof}
	We will prove that
	\begin{equation*}
		E_{00} \cap E_{-k,-l} = 0
	\end{equation*}
	for $(k,l) \neq (0,0)$. To see this, assume that
	\begin{equation}
	\label{subspace_1}
		x = \sum_{m=0}^{p-1}x_{m}e_{m} \in E_{00}\cap E_{-k,-l}.
	\end{equation}
	Then, since $x \in E_{00}$ it follows that $x_{m} = -x_{-m}$. On the other hand, since $x \in E_{-k,-l}$, it follows that $\tau^{k}\sigma^{l}(x) \in E_{00}$. This is equivalent to
	\begin{equation}
	\label{subspace_2}
		x_{m+l}\e^{2mk} = -x_{-m+l}.
	\end{equation}
	If $l = 0$ and $k\neq 0$, it follows from (\ref{subspace_1}) and \ref{subspace_2} that $x = 0$. Hence assume $l\neq 0$. by (\ref{subspace_1}) it follows that $x_{0} = 0$. Setting $m = -l$ in (\ref{subspace_2}) implies $x_{2l} = 0$, which because of (\ref{subspace_1}) leads to $x_{-2l} = 0$. Using again (\ref{subspace_2}), this time with $m = -3l$ we find $x_{4l} = 0$. Continuing in this fashion one finds that $x = 0$.
\end{proof}

We want to sum up the situation as follows: we have found $p(p+1)$ hyperplanes which we denoted by $H_{k}$ and $H_{kl}$ respectively. Moreover we have constructed $p^{2}$ subspaces $E_{kl}$ of dimension $\tfrac{1}{2}(p-1)$. Each of the spaces $E_{kl}$ is contained in exactly $p+1$ of the hyperplanes and is in fact their common intersection. On the other hand, each of the hyperplanes $H_{k}$ and $H_{kl}$ contains exactly $p$ of the subspaces $E_{kl}$ and is indeed spanned by any two of them. In particular, we can say:\\

\begin{prop}[\cite{Hulek_1983}]
	The $p(p+1)$ hyperplanes $H_{k}$ and $H_{kl}$ together with the $p^{2}$ subspaces $E_{kl}$ form a configuration of type $(p^{2}_{p+1}, p(p+1)_{p})$.\\
\end{prop}

We have not said anything in this section about the relation of this configuration to the embedded elliptic curve $E_{\tau}$ in $\PP^{p-1}$. Because of
\begin{equation*}
	x_{m}(0) = -x_{-m}(0)
\end{equation*}
it follows that the (projective) space $E_{00} = E^{-}$ contains the origin $O$. Hence each of the subspaces $E_{kl}$ goes through exactly one of the $p$-torsion points $E_{\tau}$, namely $P_{kl} = \tfrac{k + l \tau}{p}$. In fact, this is the only point of intersection of $E_{kl}$ with $E_{\tau}$.

Since the hyperplanes $H_{k}$ and $H_{kl}$ are invariant under some subgroup $\ZZ_{p} \subset E_{\tau}[p]$, it follows that they all contain exactly $p$ of the $p$-torsion points. On the other hand the hyperplanes are determined by these points. The exact relation is given by
\begin{equation*}
	\begin{split}
	H_{k} \ni \{ mP_{01} + kP_{10}\st m \in \ZZ_{p}\} &= \bigg\{ \frac{k + m\tau}{p} \st m \in \ZZ_{p} \bigg\},\\
	H_{kl} \ni \{mP_{1l} + kP_{01}\st m \in \ZZ_{p} \}&= \bigg\{ \frac{m(1 + l\tau)}{p} \st m \in \ZZ_{p}\bigg\}.
	\end{split}
\end{equation*}

We summarise this as follows:\\

\begin{prop}\cite{Hulek_1983}
	Each of the hyperplanes $H_{k}$ and $H_{kl}$ intersect $E_{\tau}$ in exactly $p$ of the $p$-torsion points. The union of all $p$ hyperplanes belonging to a fixed subgroup $\ZZ_{p} \subset E_{\tau}[p]$ contains all $p^{2}$ hyperosculating points of $E_{\tau}$.\\
\end{prop}

In Section \ref{Hesse_example}, we shall use the results of this section to scrutinise the configuration that the singular members of the Hesse pencil give rise to.

\subsection{The Normaliser of the Heisenberg Group}

We largely follow \cite{Nieto_1992} in this section, and begin by citing a theorem:\\

\begin{theorem}[Discrete Theorem of Mumford-Stone-von Neumann, \cite{Mumford_1966}]
	\label{schrodinger}
	The Schr{\"o}dinger representation of $H_{n}$ is the unique irreducible representation of $H_{n}$ such that the centre operates by natural scalar multiplication.
\end{theorem}

The normaliser $N_{n}$ of the Heisenberg group $H_{n}$ is defined to be
\begin{equation*}
	N_{n} = \{ n \in \SL(n,\CC) \st n\cdot H_{n} \cdot n^{-1} = H_{n}\}.
\end{equation*}
Conjugation defines a group homomorphism
\begin{equation*}
	N_{n} \rightarrow \Aut(H_{n}) \rightarrow \SL(2,\ZZ/n\ZZ),
\end{equation*}
where
\begin{equation*}
\begin{split}
\Aut(H_{n}) &\longrightarrow \SL(2,\ZZ/n\ZZ)\\
f &\longmapsto \bar{f}			
\end{split}
\end{equation*}
and $\bar{f}$ is the induced group automorphism to the quotient $H_{n}/\mu_{n}$, which is well-defined since $f(\mu_{n}) = \mu_{n}$ by Theorem \ref{schrodinger}, and must belong to $\SL(2,\ZZ/n\ZZ)$ since the automorphism preserves the bilinear form (\ref{commutator}).

If $n \in H_{n}$, then the inner automorphism $t \mapsto n\cdot t \cdot n^{-1}$ of $H_{n}$ induces the identity of $\ZZ/n\ZZ \times \ZZ/n\ZZ = H_{n}/\mu_{n}$. Conversely we claim that:\\
\begin{claim}[\cite{Nieto_1992}]
	For any element $n \in N_{n}$ such that conjugation with this element induces the identity on $\ZZ/n\ZZ\times \ZZ/n\ZZ$, then $n$ belong to $H_{n}$.
\end{claim}

\begin{proof}
	 By assumption, $n \cdot z \cdot n^{-1} = c\cdot z$ with $c \in \mu_{n}$. The scalar $c$ depends only on the class $\bar{z}$ of $z$ in $\ZZ/n\ZZ \times \ZZ/n\ZZ$, and $\bar{z} \mapsto c$ defines a homomorphism $\ZZ/n\ZZ \times \ZZ/n\ZZ \rightarrow \mu_{n}$. The form $(\ ,\ )$ from (\ref{commutator}) on $\ZZ/n\ZZ \times \ZZ/n\ZZ$ is non-degenerate, so there is some $z_{n} \in H_{n}$ such that
	 \begin{equation*}
	 	c(\bar{z}) = \e^{(\bar{z}_{n}, \bar{z})}
	 \end{equation*}
	 for all $z \in H_{n}$. This shows that
	 \begin{equation*}
	 	z_{n} \cdot z \cdot z_{n}^{-1} = c(\bar{z}) \cdot z = n\cdot z \cdot n^{-1}
	 \end{equation*}
	 for all $z$. Schur's Lemma then implies that $n$ differs from $z_{n}$ by a scalar factor in $\mu_{n}$, hence $n \in H_{n}$.
\end{proof}

This assertion proves that the following sequence is exact:
\begin{equation*}
	1 \longrightarrow H_{n} \longrightarrow N_{n} \longrightarrow \SL(2,\ZZ/n\ZZ).
\end{equation*}
We will come back to proving that the last map is surjective in our examples for $n = 3$ and 5, as it has a very interesting geometric interpretation that we would like to emphasise in its own right.

\subsection{Examples}

We now want to apply the results of the previous subsection to those of our elliptic curves $E_{\tau}$ found in Section \ref{embeddings}. We begin with when $n = p = 3$ with our Hesse pencil.

\subsubsection{The Hesse Pencil}
\label{Hesse_example}
An in-depth summary of the nature of the Hesse pencil is \cite{Dolgachev_2006}, which we borrow heavily from in this section.
We recall that the Hesse pencil is given by the family of elliptic normal curves of degree 3, embedded in $\PP^{2}$ by the homogeneous cubic polynomial
\begin{equation*}
E_{\lambda}: x_{0}^{3} + x_{1}^{3} + x_{2}^{3} - 3\lambda x_{0}x_{1}x_{2} = 0,\quad \lambda \in \PP^{1},
\end{equation*}
which has four singular members precisely when $\lambda \in \{1,\e,\e^{2},\infty\}$, see Table \ref{tab:Hesseconfig}.
\begin{table}[h!]
	\renewcommand{\arraystretch}{1.2}
	\begin{center}
		\caption{The inflection lines and the inflection points in the Hesse configuration.}
		\begin{tabular}{clcc}\label{tab:Hesseconfig}
			\\\HLINE
			Singular curves&Subgroup&Inflection lines&Inflection Points\\\hline
			$E_{\infty}$&$\left.\begin{array}{l} \\ \langle\tau\rangle \\ \end{array}\right.$&
			$\begin{array}{l}H_{0}\\H_{2}\\H_{1} \end{array}\left\{\begin{array}{l}x_0=0\\x_1=0\\x_2=0\end{array}\right.$&
			$\begin{array}{l}E_{00},E_{10},E_{20}\\E_{02},E_{12},E_{22}\\E_{01},E_{11},E_{21}\end{array}$ \\\hline 
			$E_{1}$&
			$\left.\begin{array}{l} \\ \langle\sigma\rangle \\ \end{array}\right.$&
			$\begin{array}{l}H_{00}\\H_{10}\\H_{20} \end{array}\left\{\begin{array}{l}x_0+x_1+x_2=0\\x_0+\e x_1+\e^{2}x_2=0\\x_0+\e^{2}x_1+\e x_2=0\end{array}\right.$&$\begin{array}{l}E_{00},E_{01},E_{02}\\E_{20},E_{21},E_{22}\\E_{10},E_{11},E_{12} \end{array}$  \\\hline
			$E_{\e^{2}}$&
			$\left.\begin{array}{l} \\ \langle\tau\sigma\rangle \\ \end{array}\right.$&
			$\begin{array}{l}H_{01}\\H_{11}\\H_{21} \end{array}\left\{\begin{array}{l}x_0+\e x_1+\e x_2=0\\x_0+\e^{2}x_1+x_2=0\\x_0+x_1+\e^{2}x_2=0\end{array}\right.$&$\begin{array}{l}E_{00},E_{12},E_{21}\\E_{02},E_{20},E_{22}\\E_{01},E_{10},E_{22} \end{array}$  \\\hline
			$E_{\e}$&
			$\left.\begin{array}{l} \\ \langle\tau^{2}\sigma\rangle \\ \end{array}\right.$&
			$\begin{array}{l}H_{02}\\H_{12}\\H_{22} \end{array}\left\{\begin{array}{l}x_0+\e^{2}x_1+\e^{2}x_2=0\\x_0+x_1+\e x_2=0\\x_0+\e x_1+x_2=0\end{array}\right.$&$\begin{array}{l}E_{00},E_{11},E_{22}\\E_{01},E_{12},E_{20}\\E_{02},E_{10},E_{21} \end{array}$  \\
			\HLINE
		\end{tabular}
	\end{center}
\end{table}
%//////////////////// TABLE ////////////////////%
There are also 9 base-points to the pencil which coincide with the 9 inflection point that make up the $3$-torsion subgroup $E_{\lambda}[3]$\footnote{It is well known that three points on an elliptic curve add up to zero, if and only if they lie on an inflection line. Since the Hesse pencil is comprised of the Fermat cubic and a term proportional to its Hessian, this is no surprise.}. They are:
\begin{eqnarray}
\begin{array}{ccc}
E_{00} = (0:1:-1),    & E_{01} = (1:-1:0),	  & E_{02} = (1:0:-1), \\
E_{10} = (0:1:-\e),   & E_{11} = (1:-\e:0),   & E_{12} = (1:0:-\e^2),\\
E_{20} = (0:1:-\e^2), & E_{21} = (1:-\e^2:0), & E_{22} = (1:0:-\e).
\end{array}
\end{eqnarray}


There are twelve singular points of each triangles, which we call the \emph{vertices of the triangles}, which are:
\begin{eqnarray}
\label{vertices}
\begin{array}{ccc}
v^{(10)}_1=(1:0:0),        & v^{(10)}_2=(0:1:0),        & v^{(10)}_3=(0:0:1),\\
v^{(01)}_1=(1:1:1),   & v^{(01)}_2=(1:\e:\e^2), & v^{(01)}_3=(1:\e^{2}:\e),\\
v^{(11)}_1=(\e:1:1), & v^{(11)}_2=(1:\e:1),   & v^{(11)}_3=(1:1:\e),\\
v^{(21)}_1 = (\e^{2}:1:1), & v^{(21)}_{2} = (1:\e^{2}: 1), & v^{(21)}_{3} = (1:1:\e^{2}).
\end{array}
\end{eqnarray}

Each subgroup of the Heisenberg group $H_{3}$ has three vertices in (\ref{vertices}) as fixed points, namely
\begin{equation*}
	\begin{split}
	\fix(\langle \sigma \rangle) = \{ v^{(01)}_{1}, v^{(01)}_{2}, v^{(01)}_{2}\},\qquad
	&\fix(\langle \tau \rangle) = \{ v^{(10)}_{1}, v^{(10)}_{2}, v^{(10)}_{3}\},\\
	\fix(\langle \tau\sigma \rangle) = \{ v^{(11)}_{1}, v^{(11)}_{2}, v^{(11)}_{2}\},\qquad
	&\fix(\langle \tau^{2}\sigma \rangle) = \{ v^{(21)}_{1}, v^{(21)}_{2}, v^{(21)}_{3}\},\\
	\end{split}
\end{equation*}
with each set constituting a degenerate orbit of a given cyclic subgroup of order three under the action of $\ZZ/3\ZZ \times \ZZ/3\ZZ$ in $\PP^{2}$. The Hesse pencil gives rise to a $(9_{4}, 12_{3})$ configuration, known as the ``Wendepunkts configuration'', \cite{Hulek_1983}.

We continue our discussion of the normaliser $N_{3}$ for the Heisenberg group $H_{3}$, by showing that the map
\begin{equation*}
	N_{3} \rightarrow \SL(2,\ZZ/3\ZZ)
\end{equation*}
is surjective. To this end, consider the two transformations, originally considered by Bianchi \cite{Bianchi_1880},
\begin{equation}
\label{hesse_transform}
	\delta =
	\frac{1}{\e^{2} - \e}\begin{pmatrix}
	1 & 1 & 1 \\
	1 & \e & \e^{2}\\
	1 & \e^{2} & \e
	\end{pmatrix},
	\quad
	\nu =
	\frac{1}{\e^{2}}
	\begin{pmatrix}
	1 & 0 & 0\\
	0 & \e & 0\\
	0 & 0 & \e
	\end{pmatrix}.
\end{equation}
They act via conjugation via
\begin{equation*}
	\begin{split}
	\delta\cdot\sigma\cdot\delta^{-1} = \tau^{2},\qquad &\delta\cdot\tau\cdot\delta^{-1} = \sigma,\\
	\nu\cdot\sigma\cdot\nu^{-1} = \sigma\tau^{2},\qquad
	&\nu\cdot\tau\cdot\nu^{-1} = \tau = \tau,\\
	\end{split}
\end{equation*}
so indeed $\delta,\nu \in N_{3}$ and their images in $\SL(2,\ZZ/3\ZZ)$ are
\begin{equation*}
	\bar{\delta} = \begin{pmatrix}
	0 & 2\\
	1 & 0
	\end{pmatrix},
	\qquad
	\bar{\nu} = \begin{pmatrix}
	1 & 2\\
	0 & 1
	\end{pmatrix},
\end{equation*}
and these two matrices generate $\SL(2,\ZZ/3\ZZ)$. Hence the sequence
\begin{equation*}
	1 \longrightarrow H_{3} \longrightarrow N_{3} \longrightarrow \SL(2,\ZZ/3\ZZ) \longrightarrow 1
\end{equation*}
is exact, and the normaliser $N_{3}$ is equal to the semi-direct product \cite{Horrocks_1973}
\begin{equation*}
	N_{3} = H_{3} \rtimes \SL(2,\ZZ/3\ZZ),
\end{equation*}
where $\SL(2,\ZZ/3\ZZ)$ acts by its natural linear action on $H_{3}$, and its order is $|N_{3}| = 27\cdot 24 = 648$. Let us investigate the action of both $\delta$ and $\nu$ on the singular members of the Hesse pencil. Substituting them in, we find that their action on the projective parameter $\lambda \in \PP^{1}$ is the following:
\begin{equation*}
	\begin{split}
	\bar{\delta}(\lambda) &= \e^{2} \lambda,\\
	\bar{\nu}(\lambda) &= \frac{\lambda + 2}{\lambda - 1},
	\end{split}
\end{equation*}
and in particular:
\begin{equation*}
	\begin{split}
	\bar{\delta}(1) = \e^{2},\quad \bar{\delta}(\e) = 1,\quad \bar{\delta}(\e^{2}) = \e,\quad \bar{\delta}(\infty) = \infty,\\
	\bar{\nu}(1) = \infty,\quad \bar{\nu}(\e) = \e^{2},\quad \bar{\nu}(\e^{2}) = \e,\quad \bar{\nu}(\infty) = 1.
	\end{split}
\end{equation*}
Now topologically, $\PP^{1}$ is equivalent to the 2-sphere, $S^{2}$. So, by stereographically projecting $\PP^{1}$ onto $S^{2}$, we can view the four singular values for $\lambda$ of the Hesse pencil as inscribing the vertices of a tetrahedron. Then the projective action of $\bar{\delta}$ and $\bar{\nu}$ permutes the vertices. Indeed, this follows from the well-known fact that $\SL(2,\ZZ/3\ZZ)/\mu_{3} \cong \PSL(2,\ZZ/3\ZZ) \cong A_{4}$. Moreover, when mapped to $\PGL(3,\CC)$ we see that
\begin{equation*}
	N_{3}/\CC^{\ast} \cong \ZZ/3\ZZ\times \ZZ/3\ZZ \rtimes \SL(2,\ZZ/3\ZZ),
\end{equation*}
which has order $216$. It is therefore isomorphic to the Hessian group $G_{216}$ of order 216, since they both have the same order and both are the groups that preserve the Hesse pencil \cite{Dolgachev_2006}.

We finish this section with an interesting connection between the Hesse pencil and Shioda's modular surface $S(3)$\footnote{For references on Shioda's modular surfaces $S(n)$, see \cite{Barth_1985}}, in which we follow \cite{Dolgachev_2006}:
consider the rational map
\begin{equation*}
\PP^{2} \dashrightarrow \PP^{1},\quad (x_{0}:x_{1}:x_{2}) \mapsto (x_{0}x_{1}x_{2}: x_{0}^{3} + x_{1}^{3} + x_{2}^{3}),
\end{equation*}
which is not defined at the nine base points of the pencil. Let
\begin{equation*}
\pi:S(3) \longrightarrow \PP^{2}
\end{equation*}
denote the blow up of the base points. This is a rational map such that the composition of rational maps $S(3) \rightarrow \PP^{2} \dashrightarrow \PP^{1}$ is a regular map
\begin{equation*}
\phi:S(3) \longrightarrow \PP^{1},
\end{equation*}
whose fibres are isomorphic to the members of the Hesse pencil. In fact, the map $\phi$ defines a structure of a minimal elliptic surface on $S(3)$, called \emph{Shioda's modular surface of level 3}, whose nine sections that define the $3$-torsion subgroup in each fibre come from the exceptional curves that arise from the blowing up process, and the action of $\Gamma = \langle \sigma, \tau \rangle$ on $\PP^{2}$ lifts to $S(3)$, and be identified with the Mordell-Weil group of the elliptic surface and its action with the translation action \cite{Dolgachev_2006}.

Let $\bar{\phi}:S(3)/\Gamma \rightarrow \PP^{1}$ be the induced morphism from $\phi$.\\

\begin{prop}[\cite{Dolgachev_2006}]
	\label{shioda}
	The quotient surface $S(3)/\Gamma$ has four singular points of type $A_{2}$, or equivalently of type $I_{3}$ in Kodaira's classification\footnote{In Kodaira's classification, a singular fibre of type $I_{n}$ is a connected cycle of $n$-cycles, each cycle being isomorphic to $\PP^{1}$. In this case each cycle is one of the hyperplanes $H_{k}$ and $H_{kl}$, $k,l\in\ZZ/3\ZZ$.} of singular fibres \cite{Kodaira_1963} given by the singular orbits of the vertices (\ref{vertices}). The minimal resolution of the singularities are isomorphic to $S(3)$, and up to these resolutions, $\bar{\phi}$ is isomorphic to $\phi$.
\end{prop}

\begin{proof}
	The group $\Gamma$ preserves each singular member of the Hesse pencil and any of its subgroups leaves invariant the vertices of one of the triangles. Without loss of generality, assume that the triangle is $x_{0}x_{1}x_{2} = 0$. The the subgroup of $\Gamma$ stabilising its vertices is $\langle \tau\rangle$, with acts locally at the point $x_{1} = x_{2} = 0$ by
	\begin{equation*}
	\begin{split}
	\PP^{2} &\longrightarrow \CC^{2}\\
	(x_{0}:x_{1}:x_{2}) &\longmapsto \bigg( u = \frac{x_{1}}{x_{0}}, v = \frac{x_{2}}{x_{0}}   \bigg),
	\end{split}
	\end{equation*}
	and hence the local action of $\tau$ is $\tau\cdot (u,v) = (\e u, \e^{2}v)$. Hence the vertices give four singular points of type $A_{2}$ in $S(3)/\Gamma$, locally given by the equation $xy + z^{3} = 0$.
	
	The multiplication-by-3 map, $[3]:E\rightarrow E:\ x\mapsto 3x$ is a surjective map of degree $3^{2}$, with kernel $E[3]$ by Proposition \ref{torsion_prop}. The quotient map by $\Gamma$ acts of each member of the Hesse pencil as the map $[3]$, which implies that the quotient of the surface $S(3)$ by the group $\Gamma$ is isomorphic to $S(3)$ over the open subset $U=\PP^{1}\setminus \{\infty,1,\e,\e^{2}\}$. The map $\bar{\phi}:S(3)/\Gamma$ induced by the map $\phi$ has four singular fibres. Each fibre is an irreducible rational curve with a double point of type $A_{2}$. Let $\pi:S(3)^{\prime}\rightarrow S(3)/\Gamma$ be a minimal resolution of the four singular points of $S(3)/\Gamma$. Then the composition $\bar{\phi}\circ\pi:S(3)^{\prime} \rightarrow \PP^{1}$ is an elliptic surface isomorphic to $\phi:S(3)\rightarrow \PP^{1}$ over the open subset of the base $\PP^{1}$. Moreover $\bar{\phi}\circ \pi$ and $\phi$ have singular fibres of the same types, thus $S(3)^{\prime}$ is a minimal elliptic surface. As it is known that a birational isomorphism of minimal elliptic surfaces is an isomorphism \cite{Dolgachev_2006}, this implies that $\bar{\phi}\circ\pi$ is isomorphic to $\phi$.
\end{proof}

\subsubsection{The Bianchi Quintic}

We now turn our attention to the $n = 5$ case, where $E_{\tau}$ is cut out the the set-theoretic intersection of the five quadrics
\begin{equation*}
Q_{i}(x_{0},\ldots,x_{4}) = x_{i}^{2} + ax_{i+2}x_{i+3} + \tfrac{1}{a}x_{i+1}x_{i+4},\qquad i \in \ZZ/5\ZZ,
\end{equation*}
where $a \in \PP^{1}$ is a parameter depending solely on $\tau$. Let
\begin{equation*}
	S_{a} = \bigcap_{i\in\ZZ/5\ZZ} Q_{i}(x_{0},\ldots,x_{4})
\end{equation*}
be the variety in $\PP^{4}$ by the $Q_{i}$.
We quote the following result from \cite{Barth_1985}:\\

\begin{prop}
	For each $a \in \PP^{1}$, $S_{a}$ is a curve in $\PP^{4}$. If $a \in \PP^{1}\setminus\Gamma$, where
	\begin{equation*}
		\Gamma = \{0,\infty, -(1/2)(1\pm 5)\e^{k} \st k = 0,\ldots 4\} \subset \PP^{1},
	\end{equation*}
	where $\e = \exp(2\pi i /5)$, the curve $S_{a}$ is a smooth elliptic curve. On the other hand, if $a \in \Gamma$, then $S_{a}$ is a connected cycle of five lines which is defined as type $I_{5}$ in Kodaira's classification of singular fibres \cite{Kodaira_1963}. The twelve points in $\Gamma$ can be identified with the twelve vertices of an isosahedron inscribed within $S^{2} \cong \PP^{1}$.
\end{prop}

Analogously to our Hesse pencil, we want to derive a similar result to the action of the Heisenberg group on the twelve singular points for $a \in \Gamma$. To this end we will show the homomorphism
\begin{equation*}
	N_{5} \longrightarrow \SL(2,\ZZ/5\ZZ)
\end{equation*}
is surjective, so that the normaliser of $N_{5}$ of $H_{5}$ in $\SL(5,\CC)$ is the semi-direct product
\begin{equation*}
	N_{5} = H_{5} \rtimes \SL(2,\ZZ/5\ZZ).
\end{equation*}
To show this, we again define two matrices $\delta,\nu\in \SL(5,\CC)$, given by\footnote{Used both in \cite{Horrocks_1973} and $\cite{Bianchi_1880}$.}
\begin{equation*}
\delta = -\frac{1}{\sqrt{5}}\begin{pmatrix}
1 & 1 & 1 & 1 & 1\\
1 & \e & \e^{2} & \e^{3} & \e^{4}\\
1 & \e^{2} & \e^{4} & \e & \e^{3}\\
1 & \e^{3} & \e & \e^{4} & \e^{2}\\
1 & \e^{4} & \e^{3} & \e^{3} & \e\\
\end{pmatrix},
\qquad
\nu = \begin{pmatrix}
1 & 0 & 0 & 0 & 0\\
0 & \e & 0 & 0 & 0\\
0 & 0 & \e^{4} & 0 & 0\\
0 & 0 & 0 & \e ^{4} & 0\\
0 & 0 & 0 & 0 & \e
\end{pmatrix},
\end{equation*}
so that
\begin{equation*}
\begin{split}
\delta\cdot\sigma\cdot\delta^{-1} = \tau^{4},\qquad &\delta\cdot\tau\cdot\delta^{-1} = \sigma,\\
\nu\cdot\sigma\cdot\nu^{-1} = \sigma\tau^{2},\qquad
&\nu\cdot\tau\cdot\nu^{-1} = \tau = \tau.\\
\end{split}
\end{equation*}
Therefore their images in $\SL(2,\ZZ/5\ZZ)$ are
\begin{equation*}
	\bar{\delta} = \begin{pmatrix}
	0 & 4\\
	1 & 0
	\end{pmatrix},
	\quad
	\bar{\nu} = \begin{pmatrix}
	1 & 2\\
	0 & 1
	\end{pmatrix},
\end{equation*}
and these generate $\SL(2,\ZZ/5\ZZ)$. So the map $N_{5} \rightarrow \SL(2,\ZZ/5\ZZ)$ is surjective and $N_{5}$ can be written as the semi-direct product
\begin{equation*}
	N_{5} = H_{5} \rtimes \SL(2,\ZZ/5\ZZ).
\end{equation*}
Its order is $|N_{5}| = 125\cdot 120 = 15000$, and this group is the famous automorphism group of the Horrocks-Mumford bundle \cite{Horrocks_1973}, whose order is listed in the title of their article. We however focus on the central quotient
\begin{equation*}
	\SL(2,\ZZ/5\ZZ)/\mu_{5} \cong \PSL(2,\ZZ/5\ZZ),
\end{equation*}
since $\PSL(2,\ZZ/5\ZZ) \cong A_{5}$, the \emph{icosahedral group} \cite{klein2007lectures}. Thus its action on $\PP^{1} \cong S^{2}$ can be viewed analogously to that of the tetrahedral group $A_{4}$ in the previous section, namely that its action on $\PP^{1}$ can be identified with the permutations between the twelve singular points in $\Gamma$.