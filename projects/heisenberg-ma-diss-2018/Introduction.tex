\section{Introduction}

In this report we want to investigate the symmetries of elliptic curves $E \subseteq \PP^{n-1}$ of degree $n\geq 3$, over the complex numbers $\CC$. Section 2 introduces this notion by viewing the an elliptic curve as the quotient of $\CC$ by a lattice, and also the complementary viewpoint. Whilst completely elementary, opens up the discussion for subsequent sections.

After this motivation, in Section 3 we introduce theta functions relative to a lattice, and study some of their properties. In particular, we define theta functions of weight $n$ for some positive integer $n$ with form a vector space of dimension $n$. The motivation behind this result is that for $n \geq 3$, the basis theta functions define a holomorphic embedding of the elliptic curve (as a complex torus) into $\PP^{n-1}$, which by Chow's Theorem defines an algebraic variety.

Section 4 is dedicated to the relations satisfied between the theta functions, as to describe the model of the elliptic curve that is given rise to. We start off with the case when $n = 3$, and find that the embedded elliptic curve takes on the form of a homogeneous cubic polynomial that belongs to the Hesse family. We then generalise this result to all positive $n \geq 4$, and prove that the embedded elliptic curve is described the the set-theoretic intersection of $n(n-3)/2$ linearly independent quadrics, and discuss some interesting cases for low values of $n$ which exhibit some interesting symmetries.

In Section 5 we introduce the Heisenberg group, which lifts the $n$-translation point action to $\PP^{n-1}$, and discuss study the invariant hyperplanes of this action. Afterwards, we then study the normaliser of the Heisenberg group, which is its group of automorphisms. To finish, we apply these results to our previously determined models of the elliptic curves.