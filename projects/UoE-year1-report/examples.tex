
\chapter{Examples}

In this section we shall apply the above theory to construct two examples of toric symplectic varieties, not just to illustrate the aforementioned theory, but also because these two examples will be necessary for further discussion on hypertoric varieties and quantisation later on in this report.

\section{Complex Projective Space, $\CC\PP^{n}$}

\subsection{Delzant Construction}

Let $e_{i}$, $i = 1,\ldots,3$, be the standard basis of $\RR^{3}$, and let $\pi: \RR^{3} \rightarrow \RR^{2}$ be given by
\begin{equation*}
	\pi(e_{i}) =
	\begin{cases}
	e_{i},\qquad &\text{for } i=1,2,\\
	-e_{1}-e_{2},\qquad &\text{for } i=3,
	\end{cases}
\end{equation*}
and label $\pi(e_{i}) = u_{i}$ Observe that $\pi$ is represented by the matrix
\begin{equation*}
	\pi = \begin{bmatrix}
	1 & 0 & -1 \\
	0 & 1 & -1
	\end{bmatrix}
\end{equation*}
whose kernel is the span of the diagonal hyperplane, $\ker \pi = \Span_{\RR}(1, 1, 1) \subset \RR^{3}$. Denoting $\mf{n} := \ker\pi$, then the inclusion map $i :\mf{n} \hookrightarrow \RR^{3}$ is just the diagonal embedding, and its transpose $i^{\ast}:(\RR^{3})^{\ast} \rightarrow \mf{n}^{\ast}$ is just summation. Exponentiating and letting $T^{n+1}$ act on $\CC^{n+1}$ diagonally (which is Hamiltonian), we get the moment map
\begin{equation*}
	\mu:\CC^{3} \longrightarrow (\RR^{3})^{\ast},\qquad \mu(z) = \frac{1}{2}\Big(|z_{1}|^{2}, |z_{2}|^{2}, |z_{3}|^{2}   \Big) - \lambda,
\end{equation*}
where $\lambda = (\lambda_{1}, \lambda_{2}, \lambda_{3}) \in (\RR^{3})^{\ast}$ is a constant, which has to have integral components if we are to anticipate a Delzant polytope. The moment map for the Hamiltonian action of $N$ on $\CC^{3}$ via inclusion is subsequently
\begin{equation*}
	i^{\ast} \circ \mu:\CC^{3} \rightarrow \mf{n}^{\ast},\qquad (i^{\ast} \circ \mu)(z) = \frac{1}{2}\sum_{i=1}^{3}(|z_{i}|^{2} - \lambda_{i}),
\end{equation*}
so that the zero level-set $(i^{\ast} \circ \mu)^{-1}(0) \subseteq \CC^{d}$ is
\begin{equation*}
	(i^{\ast} \circ \mu)^{-1}(0) = \{z \in \CC^{3} \st |z_{1}|^{2} + |z_{2}|^{2} + |z_{3}|^{2} = 2(\lambda_{1} + \lambda_{2} + \lambda_{2}) \}.
\end{equation*}
For this example, we now set $\lambda_{1} = \lambda_{2} = 2$ and $\lambda_{3} = k\in \ZZ_{\geq 0}$, and also letting $|z_{i}|^{2} = 2x_{i}$, where we use $x_{i} \in (\RR_{\geq 0})^{\ast}$ represent the image of $\mu$ in $(\RR^{3})^{\ast}$, image of the moment map $\phi:X \rightarrow (\RR^{2})^{\ast}$ for the symplectic quotient $X = (i^{\ast} \circ \mu)^{-1}(0)/N$ above is now
\begin{equation*}
	\{ x \in (\RR_{\geq 0}^{3})^{\ast} \st x_{1} + x_{2} + x_{3} = k \} \cong \{ (x,y) \in \RR_{\geq 0}^{2} \st x + y \leq k \} =: \Delta \subseteq \RR^{2}.
\end{equation*}
Here, $\Delta$ is an isosceles triangle in $\RR^{2}$ with two of the sides with length $k$. Also, since $(i^{\ast} \circ \mu)^{-1}(0) \cong S^{5}$, and $N\cong S^{1}$ acts on this level-set diagonally, we see that
\begin{equation*}
	(i^{\ast} \circ \mu)^{-1}(0)/N \cong S^{5}/S^{1} \cong \CC\PP^{2}
\end{equation*}
that is the complex projective plane.

Mutatis mutandi, it is not hard to see that letting the same $N \cong S^{1}$ act on $\CC^{n+1}$, we get 
\begin{equation*}
	(i^{\ast} \circ \mu)^{-1}(0)/N \cong S^{2n+1}/S^{1} \cong \CC\PP^{n}.
\end{equation*}

