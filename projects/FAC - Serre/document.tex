\title{
    FAISCEAUX ALGEBRIQUES COHERENTS
}

\author{
    PaR JEAN-PIERRE SERRE
}

(Reçu le 8 Octobre 1954)

INTRODUCTION

\begin{abstract}
    On sait que les méthodes cohomologiques, et particulièrement la théorie des faisceaux, jouent un rôle croissant, non seulement en théorie des fonctions de plusieurs variables complexes (cf. [5]), mais aussi en géométrie algébrique classique (qu'il me suffise de citer les travaux récents de Kodaira-Spencer sur le théorème de Riemann-Roch). Le caractère algébrique de ces méthodes laissait penser qu'il était possible de les appliquer également à la géométrie algébrique abstraite; le but du présent mémoire est de montrer que tel est bien le cas.
\end{abstract}

Le contenu des différents chapitres est le suivant:

Le Chapitre I est consacré à la théorie générale des faisceaux. Il contient les démonstrations des résultats de cette théorie qui sont utilisés dans les deux autres chapitres. Les diverses opérations algébriques que l'on peut effectuer sur les faisceaux sont décrites au $\S 1$; nous avons suivi d'assez près l'exposé de Car$\tan ([2],[5]) .$ Le $\S 2$ contient l'étude des faisceaux cohérents de modules; ces faisceaux génêralisent les faisceaux analytiques cohérents (cf. [3], [5]), et jouissent de propriétés tout analogues. Au $\S 3$ sont définis les groupes de cohomologie d'un espace $X$ à valeurs dans un faisceau $\mathfrak{F} .$ Dans les applications ultérieures, $X$ est une variété algébrique, munie de la topologie de Zariski, donc n'est pas un espace topologique séparé, et les méthodes utilisées par Leray [10], ou Cartan [3] (basées sur les "partitions de l'unité", ou les faisceaux "fins") ne lui sont pas applicables; aussi avons-nous dû revenir au procédé de Čech, et définir les groupes de cohomologie $H^{q}(X, \mathfrak{F})$ par passage à la limite sur des recouvrements ouverts de plus en plus fins. Une autre difficulté, liée à la non-séparation de $X$, se rencontre dans la "suite exacte de cohomologie" (cf. $\mathrm{n}^{\mathrm{nos}} 24$ et 25$):$ nous n'avons pu établir cette suite exacte que dans des cas particuliers, d'ailleurs suffisants pour les applications que nous avions en vue (cf. $n^{\text {os }} 24$ et 47$)$.

Le Chapitre II débute par une définition des variétés algébriques, analogue à celle de Weil ([17], Chap. VII), mais englobant le cas des variétés réductibles (signalons à ce propos que, contrairement à l'usage de Weil, nous ne réservons pas le terme de "variété" aux seules variétés irréductibles); nous définissons la structure de variété algébrique par la donnée d'une topologie (la topologie de Zariski) et d'un sous-faisceau du faisceau des germes de fonctions (le faisceau des anneaux locaux). Un faisceau algébrique cohérent sur une variété algébrique $V$ est simplement un faisceau cohérent de $\mathcal{O}_{v}$-modules, $\mathcal{O}_{v}$ désignant le faisceau des anneaux locaux de $V$; nous en donnons divers exemples au $\S 2 .$ Le $\S 3$ est consacré aux variétés affines. Les résultats obtenus sont tout à fait semblables à ceux relatifs aux variétés de Stein (cf. [3], [5]): si $\mathfrak{F}$ est un faisceau algébrique cohérent sur la variété affine $V$, on a $H^{Q}(V, F)=0$ pour tout $q>0$, et $\mathcal{F}_{x}$ est engendré par $H^{0}(V, \mathfrak{})$ quel que soit $x \in V$. De plus $(\S 4), \mathfrak{F}$ est bien déterminé par $H^{0}(V, \mathfrak{F})$ considéré comme module sur l'anneau de coordonnées de $V$.

Le Chapitre III, relatif aux variétés projectives, contient les résultats essentiels de ce travail. Nous commençons par établir une correspondance entre faisceaux algébriques cohérents $\mathcal{F}$ sur l'espace projectif $X=\mathbf{P}_{r}(K)$ et $S$-modules gradués vérifiant la condition (TF) du no 56 (S désignant l'algèbre de polynômes $\left.K\left[t_{0}, \cdots, t_{r}\right]\right)$; cette correspondance est biunivoque si l'on convient d'identifier deux $S$-modules qui ne diffèrent que par leurs composantes homogènes de degrés assez bas (pour les énoncés précis, voir $\mathrm{n}^{\text {os }} 57,59$ et 65 ). A partir de là, toute question portant sur $\mathfrak{F}$ peut être traduite en une question portant sur le $S$-module associé $M$. C'est ainsi que nous donnons au $\S 3$ un procédé permettant de déterminer algébriquement les $H^{q}(X, \mathfrak{F})$ à partir de $M$, ce qui nous permet notamment d'étudier les propriétés des $H^{q}(X, F(n))$ pour $n$ tendant vers $+\infty$ (pour la définition de $\mathfrak{F}(n)$, voir $\left.\mathrm{n}^{\circ} 54\right) ;$ les résultats obtenus sont énoncés aux $\mathrm{n}^{\mathrm{os}} 65$ et 66 . Au $\S 4$, nous mettons les groupes $H^{q}(X, \mathcal{F})$ en relation avec les foncteurs Exts introduits par Cartan-Eilenberg $[6] ;$ ceci nous permet, au $\S 5$, d'étudier le comportement des $H^{q}(X, \mathcal{F}(n))$ pour $n$ tendant vers $-\infty$, et de donner une caractérisation homologique des variétés " $k$-fois de première espèce". Le $\S 6$ expose quelques propriétés élémentaires de la caractéristique d'Euler-Poincaré d'une variété projective, à valeurs dans un faisceau algébrique cohérent.

Nous montrerons ailleurs comment on peut appliquer les résultats généraux du présent mémoire à divers problèmes particuliers, et notamment étendre au cas abstrait le "théorème de dualité" de [15], ainsi qu'une partie des résultats de Kodaira-Spencer sur le théorème de Riemann-Roch; dans ces applications, les théorèmes des nos 66,75 et 76 jouent un rôle essentiel. Nous montrerons également que, lorsque le corps de base est le corps des complexes, la théorie des faisceaux algébriques cohérents est essentiellement identique à celle des faisceaux analytiques cohérents (cf. [4]).

\section{TABLE DES MATIERES}

Chapiteg I. Faisceaux

§1. Opérations sur les faisceaux. ........................ 199

$\S 2$. Faisceaux cohérents de modules............................ 207

$\S$ 3. Cohomologie d'un espace à valeurs dans un faisceau $\ldots \ldots \ldots \ldots \ldots 212$

\$4. Comparaison des groupes de cohomologie de recouvrements différents. . . 219 CHAPITRE II. VARIÉTÉS ALGÉBRIQUES-FAISCEAUX ALGÉBRIQUES COHÉRENTS SUR LES VARIÉTÉS AFFINES

§1. Variétés algébriques.. ..................................................

$\S 2$. Faisceaux algébriques cohérents ........................ 230

$\S 3$. Faisceaux algébriques cohérents sur les variétés affines $\ldots \ldots \ldots \ldots \ldots 233$

$\S$ 4. Correspondance entre modules de type fini et faisceaux algébriques $\mathrm{~ c o h e ́ r e n t s ~ . . . . . . . . . . . . . . . . . . . . . . . . . . . . . . . . . . . . . . . . . . . . . . . . . . . . . . . . . . . . . . . . . . . . . . . . . . . . . . . . . . . . . . . . . . . . . . . . . . . . . . . . . . . . . . . . . . . . . . . . . . . . . . . . . . . . . . . . . . . . . . . . . . . . . . . . . . . . . . . . . . . . . . . . . . . . . . . . . . . . . . . . . . . . . . . . . . . . . . . . . . . . . . . . . . . . . . . . . . . . . . . . . . . . . . . . . . . . . . . . . . . . . . . . . . . . . . . . . . . . . . . . . . . . . . . . . . . . . . . . . . . . . . . . . . . . . . . . . . . . . . . . . . . . . . . . . . . . . . . . . . . . . . . . . . . . . . . . . . . . . . . . . . . . . . . . . . . . . . . . . . . . . . . . . . . . . . . . . . . . . . . . . . . . . . . . . . . . . . . . . . . . . . . . . . . . . . . . . . . . . . . . . . . . . . . . . . . . . . . . . . . . . . . . . . . . . . . . . . . . . . . . . . . . . . . . . . . . . . . . . . . . . . . . . . . . . . . . . . . . . . . . . . . . . . . . . . . . . . . . . . . . . . . . . . . . . . . . . . . . . . . . . . . . . . . . . . . . . . . . . . . . . . . . . . . . . . . . . . . . . . . . . . . . . . . . . . . . . . . . . . . . . . . . . . . . . . . . . . . . . . . . . . . . . . . . . . . . . . . . . . . . . . . . . . . . . . . . . . . . . . . . . . . . . . . . . . . . . . . . . . . . . . . . . . . . . . . . . . . . . . . . . . . . . . . . . . . . . . . . . . . . . . . . . . . . . . . . . . . . . . . . . . . . . . . . . . . . . . . . . . . . . . . . . . . . . . . . . . . . . . . . . . . . . . . . . . . . . . . . . . . . . . . . . . . . . . . . . . . . . . . . . . . . . . . . . . . . . . . . . . . . . . . . . . . . . . . . . . . . . . . . . . . . . . . . . . . . . . . . . . . . . . . . . . . . . . . . . . . . . . . . . . . . . . . . . . . . . . . . . . . . . . . . . . . . . . . . . . . . . . . . . . . . . . . . . . . . . . . . . . . . . . . . . . . . . . . . . . . . . . . . . . . . . . . . . . . . . . . . . . . . . . . . . . . . . . . . . . . . . . . . . . . . . . . . . . . . . . . . . . . . . . . . . . . . . . . . . . . . . . . . . . . . . . . . . . . . . . . . . . . . . . . . . . . . . . . . . . . . . . . . . . . . . . . . . . . . . . . . . . . . . . . . . . . . . . . . . . . . . . . . . . . . . . . . . . . . . . . . . . . . . . . . . . . . . . . . . . . . . . . . . . . . . . . . . . . . . . . . . . . . . . . . . . . . . . . . . . . . . . . . . . . . . . . . . . . . . . . . . . . . . . . . . . . . . . . . . . . . . . . . . . . . . . . . . . . . . . . . . . . . . . . . . . . . . . . . . . . . . . . . . . . . . . . . . . . . . . . . . . . . . . . . . . . . . . . . . . . . . . . . . . . . . . . . . . . . . . . . . . . . . . . . . . . . . . . . . . . . . . . . . . . . . . . . . . . . . . . . . . . . . . . . . . . . . . . . . . . . . . . . . . . . . . . . . . . . . . . . . . . . . . . . . . . . . . . . . . . . . . . . . . . . . . . . . . . . . . . . . . . .}$ ChapitRe III. Fatsceatu algébergues coHÉrents sUR LES vARIÉTÉs PROJECTIVES

§1. Variétés projectives . .............................................................................................................................................................................................................................................................................................................

§2. Modules gradués et faisceaux algébriques cohérents sur l'espace $\mathrm{~ p r o j e c t i f ~ . . . . . . . . . . . . . . . . . . . . . . . . . . . . . . . . . . . . . . . . . . . . . . . . . . . . . . . . . . . . . . . . . . . . . . . . . . . . . . . . . . . . . . . . . . . . . . . . . . . . . . . . . . . . . . . . . . . . . . . . . . . . . . . . . . . . . . . . . . . . . . . . . . . . . . . . . . . . . . . . . . . . . . . . . . . . . . . . . . . . . . . . . . . . . . . . . . . . . . . . . . . . . . . . . . . . . . . . . . . . . . . . . . . . . . . . . . . . . . . . . . . . . . . . . . . . . . . . . . . . . . . . . . . . . . . . . . . . . . . . . . . . . . . . . . . . . . . . . . . . . . . . . . . . . . . . . . . . . . . . . . . . . . . . . . . . . . . . . . . . . . . . . . . . . . . . . . . . . . . . . . . . . . . . . . . . . . . . . . . . . . . . . . . . . . . . . . . . . . . . . . . . . . . . . . . . . . . . . . . . . . . . . . . . . . . . . . . . . . . . . . . . . . . . . . . . . . . . . . . . . . . . . . . . . . . . . . . . . . . . . . . . . . . . . . . . . . . . . . . . . . . . . . . . . . . . . . . . . . . . . . . . . . . . . . . . . . . . . . . . . . . . . . . . . . . . . . . . . . . . . . . . . . . . . . . . . . . . . . . . . . . . . . . . . . . . . . . . . . . . . . . . . . . . . . . . . . . . . . . . . . . . . . . . . . . . . . . . . . . . . . . . . . . . . . . . . . . . . . . . . . . . . . . . . . . . . . . . . . . . . . . . . . . . . . . . . . . . . . . . . . . . . . . . . . . . . . . . . . . . . . . . . . . . . . . . . . . . . . . . . . . . . . . . . . . . . . . . . . . . . . . . . . . . . . . . . . . . . . . . . . . . . . . . . . . . . . . . . . . . . . . . . . . . . . . . . . . . . . . . . . . . . . . . . . . . . . . . . . . . . . . . . . . . . . . . . . . . . . . . . . . . . . . . . . . . . . . . . . . . . . . . . . . . . . . . . . . . . . . . . . . . . . . . . . . . . . . . . . . . . . . . . . . . . . . . . . . . . . . . . . . . . . . . . . . . . . . . . . . . . . . . . . . . . . . . . . . . . . . . . . . . . . . . . . . . . . . . . . . . . . . . . . . . . . . . . . . . . . . . . . . . . . . . . . . . . . . . . . . . . . . . . . . . . . . . . . . . . . . . . . . . . . . . . . . . . . . . . . . . . . . . . . . . . . . . . . . . . . . . . . . . . . . . . . . . . . . . . . . . . . . . . . . . . . . . . . . . . . . . . . . . . . . . . . . . . . . . . . . . . . . . . . . . . . . . . . . . . . . . . . . . . . . . . . . . . . . . . . . . . . . . . . . . . . . . . . . . . . . . . . . . . . . . . . . . . . . . . . . . . . . . . . . . . . . . . . . . . . . . . . . . . . . . . . . . . . . . . . . . . . . . . . . . . . . . . . . . . . . . . . . . . . . . . . . . . . . . . . . . . . . . . . . . . . . . . . . . . . . . . . . . . . . . . . . . . . . . . . . . . . . . . . . . . . . . . . . . . . . . . . . . . . . . . . . . . . . . . . . . . . . . . . . . . . . . . . . . . . . . . . . . . . . . . . . . . . . . . . . . . . . . . . . . . . . . . . . . . . . . . . . . . . . . . . . . . . . . . . . . . . . . .}$

$\S 3$. Cohomologie de l'espace projectif à valeurs dans un faisceau algé$\mathrm{~ b r i q u e ~ c o h e ́ r e n t . . . . . . . . . . . . . . . . . . . . . . . . . . . . . . . . . . . . . . . . . . . . . . . . . . . . . . . . . . . . . . . . . . . . . . . . . . . . . . . . . . . . . . . . . . . . . . . . . . . . . . . . . . . . . . . . . . . . . . . . . . . . . . . . . . . . . . . . . . . . . . . . . . . . . . . . . . . . . . . . . . . . . . . . . . . . . . . . . . . . . . . . . . . . . . . . . . . . . . . . . . . . . . . . . . . . . . . . . . . . . . . . . . . . . . . . . . . . . . . . . . . . . . . . . . . . . . . . . . . . . . . . . . . . . . . . . . . . . . . . . . . . . . . . . . . . . . . . . . . . . . . . . . . . . . . . . . . . . . . . . . . . . . . . . . . . . . . . . . . . . . . . . . . . . . . . . . . . . . . . . . . . . . . . . . . . . . . . . . . . . . . . . . . . . . . . . . . . . . . . . . . . . . . . . . . . . . . . . . . . . . . . . . . . . . . . . . . . . . . . . . . . . . . . . . . . . . . . . . . . . . . . . . . . . . . . . . . . . . . . . . . . . . . . . . . . . . . . . . . . . . . . . . . . . . . . . . . . . . . . . . . . . . . . . . . . . . . . . . . . . . . . . . . . . . . . . . . . . . . . . . . . . . . . . . . . . . . . . . . . . . . . . . . . . . . . . . . . . . . . . . . . . . . . . . . . . . . . . . . . . . . . . . . . . . . . . . . . . . . . . . . . . . . . . . . . . . . . . . . . . . . . . . . . . . . . . . . . . . . . . . . . . . . . . . . . . . . . . . . . . . . . . . . . . . . . . . . . . . . . . . . . . . . . . . . . . . . . . . . . . . . . . . . . . . . . . . . . . . . . . . . . . . . . . . . . . . . . . . . . . . . . . . . . . . . . . . . . . . . . . . . . . . . . . . . . . . . . . . . . . . . . . . . . . . . . . . . . . . . . . . . . . . . . . . . . . . . . . . . . . . . . . . . . . . . . . . . . . . . . . . . . . . . . . . . . . . . . . . . . . . . . . . . . . . . . . . . . . . . . . . . . . . . . . . . . . . . . . . . . . . . . . . . . . . . . . . . . . . . . . . . . . . . . . . . . . . . . . . . . . . . . . . . . . . . . . . . . . . . . . . . . . . . . . . . . . . . . . . . . . . . . . . . . . . . . . . . . . . . . . . . . . . . . . . . . . . . . . . . . . . . . . . . . . . . . . . . . . . . . . . . . . . . . . . . . . . . . . . . . . . . . . . . . . . . . . . . . . . . . . . . . . . . . . . . . . . . . . . . . . . . . . . . . . . . . . . . . . . . . . . . . . . . . . . . . . . . . . . . . . . . . . . . . . . . . . . . . . . . . . . . . . . . . . . . . . . . . . . . . . . . . . . . . . . . . . . . . . . . . . . . . . . . . . . . . . . . . . . . . . . . . . . . . . . . . . . . . . . . . . . . . . . . . . . . . . . . . . . . . . . . . . . . . . . . . . . . . . . . . . . . . . . . . . . . . . . . . . . . . . . . . . . . . . . . . . . . . . . . . . . . . . . . . . . . . . . . . . . . . . . . . . . . . . . . . . . . . . . . . . . . . . . . . . . . . . . . . . . . . . . . . . . . . . . . . . . . . . . . . . . . . . . . . . . . . . . . . . . . . . . . . . . . . . . . . . . . . . . .}$

§4. Relations avec les foncteurs $\mathrm{Ext}_{s}^{q} \ldots \ldots$

\$5. Applications aux faisceaux algébriques cohérents................... 267

$\$ 6$. Fonction caractéristique et genre arithmétique. ................ 274

\section{CHAPITRE I. FAISCEAUX}

\section{$\S 1 .$ Opêrations sur les faisceaux}

1. Définition d'un faisceau. Soit $X$ un espace topologique. Un faisceau de groupes abéliens sur $X$ (ou simplement un faisceau) est constitué par:

(a) Une fonction $x \rightarrow \mathfrak{F}_{x}$ qui fait correspondre à tout $x \in X$ un groupe abélien $\mathfrak{F}_{x}$,

(b) Une topologie sur l'ensemble $\mathfrak{F}$, somme des ensembles $\mathfrak{F}_{x}$.

Si $f$ est un élément de $\mathfrak{F}_{x}$, nous poserons $\pi(f)=x$; l'application $\pi$ est appelée la projection de $\mathfrak{F}$ sur $X$; la partie de $\mathfrak{F} \times \mathcal{F}$ formée des couples $(f, g)$ tels que $\pi(f)=\pi(g)$ sera notée $\mathscr{+} \mathfrak{F}$

Ces définitions étant posées, nous pouvons énoncer les deux axiomes auxquels sont soumises les données (a) et (b):

(I) Pour tout $f \in \mathcal{F}$, il existe un voisinage $V$ de $f$ et un voisinage $U$ de $\pi(f)$ tels que la restriction de $\pi$ à $V$ soit un homéomorphisme de $V$ sur $U$.

(Autrement dit, $\pi$ doit être un homéomorphisme local).

(II) L'application $f \rightarrow-f$ est une application continue de $\mathfrak{F}$ dans $\mathfrak{F}$, et l'application $(f, g) \rightarrow f+g$ est une application continue de $\mathfrak{F}+\mathfrak{F}$ dans $\mathfrak{F} .$

On observera que, même si $X$ est séparé (ce que nous n'avons pas supposé), $\mathfrak{F}$ n'est pas nécessairement séparé, comme le montre l'exemple du faisceau des germes de fonctions (cf. $\left.\mathrm{n}^{\circ} 3\right)$.

ExEmple de faisceau. $G$ étant un groupe abélien, posons $\mathcal{F}_{x}=G$ pour tout $x \in X ;$ l'ensemble $\mathfrak{F}$ peut être identifié au produit $X \times G$, et, si on le munit de la topologie produit de la topologie de $X$ par la topologie discrète de $G$, on obtient un faisceau, appelé le faisceau constant isomorphe à $G$, et souvent identifié à $G$.

2. Sections d'un faisceau. Soit $\mathcal{F}$ un faisceau sur l'espace $X$, et soit $U$ une partie de $X$. On appelle section de $\mathfrak{F}$ au-dessus de $U$ une application continue $s: U \rightarrow \mathcal{F}$ telle que $\boldsymbol{\pi} \circ s$ soit l'application identique de $U$. On a donc $s(x) \in \mathfrak{F}_{x}$ pour tout $x \in U$. L'ensemble des sections de $\mathcal{F}$ au-dessus de $U$ sera désigné par $\Gamma(U, \mathfrak{F})$; l'axiome (II) entraîne que $\Gamma(U, \mathcal{F})$ est un groupe abélien. Si $U \subset V$, et si $s$ est une section au-dessus de $V$, la restriction de $s$ à $U$ est une section audessus de $U$; d'où un homomorphisme $\rho_{v}^{V}: \Gamma(V, \mathfrak{}) \rightarrow \Gamma(U, \mathfrak{F})$.

Si $U$ est ouvert dans $X, s(U)$ est ouvert dans $\mathcal{F}$, et, lorsque $U$ parcourt une base d'ouverts de $X$, les $s(U)$ parcourent une base d'ouverts de $\mathfrak{F}$ : ce n'est qu'une autre façon d'exprimer l'axiome (I).

Notons encore une conséquence de l'axiome (I): Pour tout $f \in \mathcal{F}_{x}$, il existe une section $s$ au-dessus d'un voisinage de $x$ telle que $s(x)=f$, et deux sections jouissant de cette propriété coincident dans un voisinage de $x$. Autrement dit, $\mathfrak{F}_{x}$ est limite inductive des $\Gamma(U, \mathfrak{F})$ suivant l'ordonné filtrant des voisinages de $x$.

3. Construction de faisceaux. Supposons donnés, pour tout ouvert $U \subset X$, un groupe abélien $\mathfrak{F}_{U}$, et, pour tout couple d'ouverts $U \subset V$, un homomorphisme $\varphi_{U}^{V}: \mathscr{F}_{V} \rightarrow \mathfrak{F}_{U}$, de telle sorte que la condition de transitivité $\varphi_{U}^{V} \circ \varphi_{V}^{W}=\varphi_{U}^{W}$ soit vérifiée chaque fois que $U \subset V \subset W$.

La collection des $\left(\mathfrak{F}_{U}, \varphi_{U}^{V}\right)$ permet alors de définir un faisceau $\mathfrak{F}$ de la manière suivante:

(a) On pose $\mathfrak{F}_{x}=\lim \mathfrak{F}_{U}$ (limite inductive suivant l'ordonné filtrant des voisinages ouverts $U$ de $x$ ). Si $x$ appartient à l'ouvert $U$, on a donc un homomorphisme canonique $\varphi_{x}^{U}: \mathfrak{F}_{U} \rightarrow \mathfrak{F}_{x}$.

(b) Soit $t \in \mathcal{F}_{U}$, et désignons par $[t, U]$ l'ensemble des $\varphi_{x}^{U}(t)$ pour $x$ parcourant $U$; on a $[t, U] \subset \mathfrak{F}$, et on munit $\mathfrak{F}$ de la topologie engendrée par les $[t, U]$. Ainsi, un élément $f \in \mathcal{F}_{x}$ admet pour base de voisinages dans $\mathfrak{F}$ les ensembles $[t, U]$ où $x \in U$ et $\varphi_{x}^{U}(t)=f$

On vérifie aussitôt que les données (a) et (b) satisfont aux axiomes (I) et (II), autrement dit, que $\mathcal{F}$ est bien un faisceau. Nous dirons que c'est le faisceau défini par le système $\left(\mathfrak{F}_{U}, \varphi_{U}^{v}\right)$.

Si $t \in \mathfrak{F}_{U}$, l'application $x \rightarrow \varphi_{x}^{U}(t)$ est une section de $\mathfrak{F}$ au-dessus de $U$; d'où un homomorphisme canonique $\iota: \mathcal{F}_{U} \rightarrow \Gamma(U, \mathfrak{F})$

Proposition 1. Pour que $\iota: \mathfrak{F}_{U} \rightarrow \Gamma(U, \mathfrak{F})$ soit injectif, ${ }^{1}$ il faut et il suffit que la condition suivante soit vérifiée:

Si un élément $t \in \mathcal{F}_{U}$ est tel qu'il existe un recouvrement ouvert $\left\{U_{i}\right\}$ de $U$ avec $\boldsymbol{\varphi}_{U_{i}}^{U}(t)=0$ pour tout $i$, alors $t=0$

Si $t \in \mathcal{F}_{U}$ vérifie la condition précédente, on a

$$
\varphi_{x}^{v}(t)=\varphi_{x}^{V_{i}} \circ \varphi_{U_{i}}^{U}(t)=0
$$

si $x \in U_{i}$

ce qui signifie que $\iota(t)=0 .$ Inversement, supposons que l'on ait $\iota(t)=0$, avec $t \in \mathfrak{F}_{U} ;$ puisque $\varphi_{x}^{U}(t)=0$ pour $x \in U$, il existe un voisinage ouvert $U(x)$ de $x$ tel que $\varphi_{U(x)}^{V}(t)=0$, par définition de la limite inductive. Les $U(x)$ forment alors un recouvrement ouvert de $U$ vérifiant la condition de l'énoncé.

Proposition $2 .$ Soit $U$ un ouvert de $X$, et supposons que $\iota: \mathfrak{F}_{V} \rightarrow \Gamma(V, \mathfrak{F})$ soit injectif pour tout ouvert $V \subset U$. Pour que $\iota: \mathfrak{F}_{U} \rightarrow \Gamma(U, \mathfrak{F})$ soit surjectif $^{1}$ (donc bijectif), il faut et il suffit que la condition suivante soit vérifiée:

Pour tout recouvrement ouvert $\left\{U_{i}\right\}$ de $U$, et tout système $\left\{t_{i}\right\}, t_{i} \in \mathcal{F}_{U_{i}}$, tels que

${ }^{1}$ Rappelons (cf. [1]) qu'une application $f: E \rightarrow E^{\prime}$ est dite injective si $f\left(e_{1}\right)=f\left(e_{2}\right)$ entraîne $e_{1}=e_{2}$, surjective si $f(E)=E^{\prime}$, bijective si elle est à la fois injective et surjective. Une application injective (resp. surjective, bijective) est appelée une injection (resp. une surjection, une bijection). $\varphi_{U i}^{U_{i}} \cap v_{i}\left(t_{i}\right)=\varphi_{U_{i}^{j}}^{U j} \cap_{j}\left(t_{j}\right)$ pour tout couple $(i, j)$, il existe $t \in \mathfrak{F}_{U}$ avec $\varphi_{U_{i}}^{U}(t)=t_{i}$ pour tout $i$.

La condition est nécessaire: chaque $t_{i}$ définit une section $s_{i}=\iota\left(t_{i}\right)$ au-dessus de $U_{i}$, et l'on a $s_{i}=s_{j}$ au-dessus de $U_{i}$ n $U_{j} ;$ il existe donc une section $s$ au-dessus de $U$ qui coincide avec $s_{i}$ sur $U_{i}$ pour tout $i$;si $\iota: \mathfrak{F}_{U} \rightarrow \Gamma(U, \mathfrak{F})$ est surjectif, il existe $t \in \mathfrak{F}_{U}$ tel que $\iota(t)=s .$ Si l'on pose $t_{i}^{\prime}=\varphi_{U_{i}}^{U}(t)$, la section définie par $t_{i}^{\prime}$ sur $U_{i}$ n'est autre que $s_{i} ;$ d'où $\iota\left(t_{i}-t_{i}^{\prime}\right)=0$ ce qui entraîne $t_{i}=t_{i}^{\prime}$ vu que $\iota$ est supposé injectif.

La condition est suffisante: si $s$ est une section de $\mathcal{F}$ au-dessus de $U$, il existe un recouvrement ouvert $\left\{U_{i}\right\}$ de $U$, et des éléments $t_{i} \in \mathfrak{F}_{U_{i}}$ tels que $\iota\left(t_{i}\right)$ soit égal à la restriction de $s$ à $U_{i} ;$ il s'ensuit que les éléments $\varphi_{v_{i}^{V}}^{V_{v}}\left(t_{i}\right)$ et $\varphi_{U_{i}^{j}}^{V j}{ }^{V_{j}}\left(t_{j}\right)$ définissent la même section sur $U_{i}$ n $U_{j}$, donc sont égaux, d'après l'hypothèse faite sur $\iota$. Si $t \in \mathcal{F}_{U}$ est tel que $\varphi_{U_{i}}^{U}(t)=t_{i}, \iota(t)$ coïncide avec $s$ sur chaque $U_{i}$ donc sur $U$, cqfd.

Proposition $3 .$ Si $\mathfrak{F}$ est un faisceau de groupes abéliens sur $X$, le faisceau défini par le système $\left(\Gamma(U, \mathfrak{}), \rho_{U}\right)$ est canoniquement isomorphe à $\mathfrak{F}$.

Cela résulte immédiatement des propriétés des sections énoncées au n $2 .$

La Proposition 3 montre que tout faisceau peut être défini par un système $\left(\mathfrak{F}_{U}, \varphi_{U}^{v}\right)$ convenable. On notera que des systèmes différents peuvent définir le même faisceau $\mathfrak{F}$; toutefois, si l'on impose aux $\left(\mathfrak{F}_{U}, \varphi_{U}^{v}\right)$ de vérifier les conditions des Propositions 1 et 2, il n'y a (à un isomorphisme près) qu'un seul système possible: celui formé par les $\left(\Gamma(U, \mathcal{F}), \rho_{U}^{V}\right)$.

ExEMPLE. Soit $G$ un groupe abélien, et prenons pour $\mathfrak{F}_{v}$ l'ensemble des fonctions sur $U$ a valeurs dans $G ;$ définissons $\varphi_{U}^{V}: \mathfrak{F}_{v} \rightarrow \mathfrak{F}_{U}$ par l'opération de restriction d'une fonction. On obtient ainsi un système $\left(\mathfrak{F}_{U}, \rho_{U}^{V}\right)$, d'où un faisceau $\mathfrak{F}$, appelé faisceau des germes de fonctions à valeurs dans $G$. On vérifie tout de suite que le système $\left(\mathfrak{F}_{v}, \varphi_{U}^{V}\right)$ vérifie les conditions des Propositions 1 et $2 ;$ il en résulte que l'on peut identifier les sections de $\mathcal{F}$ sur un ouvert $U$ avec les eléments de $\mathcal{F}_{U}$.

4. Recollement de faisceaux. Soit $\mathfrak{F}$ un faisceau sur $X$, et soit $U$ une partie de $X ;$ l'ensemble $\pi^{-1}(U) \subset \mathfrak{F}$, muni de la topologie induite par celle de $\mathcal{F}$, forme un faisceau sur $U$, appelé le faisceau induit par $\mathfrak{F}$ sur $U$, et noté $\mathfrak{F}(U)$ (ou même F, lorsqu'aucune confusion n'est à craindre).

Nous allons voir que, inversement, on peut définir un faisceau sur $X$ au moyen de faisceaux sur des ouverts recouvrant $X$

Proposition 4. Soit $\mathfrak{U}=\left\{U_{i}\right\}_{\text {i\epsilon }}$ un recouvrement ouvert de $X$, et, pour chaque $i \in I$, soit $\mathfrak{F}_{i}$ un faisceau sur $U_{i}$; pour tout couple $(i, j)$, soit $\theta_{i j}$ un isomorphisme de $\mathfrak{F}_{j}\left(U_{i} \operatorname{n} U_{j}\right)$ sur $\mathcal{F}_{i}\left(U_{i}\right.$ n $\left.U_{j}\right)$; supposons que l'on ait $\theta_{i j} \circ \theta_{j k}=\theta_{i k}$ en tout point de $U_{i}$ ก $U_{j}$ n $U_{k}$, pour tout système $(i, j, k)$,

Il existe alors un faisceau $\mathfrak{F}$, et, pour chaque $i \in I$, un isomorphisme $\eta_{i}$ de $\mathfrak{F}\left(U_{i}\right)$ $\operatorname{sur} \mathfrak{F}_{i}$, tels que $\theta_{i j}=\eta_{i} \circ \eta_{j}^{-1}$ en tout point de $U_{i}$ n $U_{j} .$ De plus, $\mathfrak{F}$ et les $\eta_{i}$ sont déterminés à un isomorphisme près par la condition précédente.

L'unicité de $\left\{\mathcal{F}, \eta_{i}\right\}$ est évidente; pour en démontrer l'existence, on pourrait définir $\mathfrak{F}$ comme espace quotient de l'espace somme des $\mathfrak{F}_{i}$. Nous utiliserons plutôt le procédé du $\mathrm{n}^{\circ} 3:$ si $U$ est un ouvert de $X$, soit $\mathfrak{F}_{U}$ le groupe dont les éléments sont les systèmes $\left\{s_{k}\right\}_{k \in I}$, avec $s_{k} \in \Gamma\left(U \cap U_{k}, \mathcal{F}_{k}\right)$, et $s_{k}=\theta_{k j}\left(s_{j}\right)$ sur $U$ \cap $U_{j}$ \cap $U_{k} ;$ si $U \subset V$, on définit $\varphi_{U}^{V}$ de façon évidente. Le faisceau défini par le système $\left(\mathcal{F}_{U}, \varphi_{U}^{V}\right)$ est le faisceau $\mathfrak{F}$ cherché; de plus, si $U \subset U_{i}$, l'application qui fait correspondre au système $\left\{s_{k}\right\} \in \mathfrak{F}_{U}$ l'élément $s_{i} \in \Gamma\left(U_{i}, \mathfrak{F}_{i}\right)$ est un isomorphisme de $\mathcal{F}_{U}$ sur $\Gamma\left(U, \mathfrak{F}_{i}\right)$, d'après la condition de transitivité; on obtient ainsi un isomorphisme $\eta_{i}: \mathscr{F}\left(U_{i}\right) \rightarrow \mathfrak{F}_{i}$ qui répond évidemment à la condition posée.

On dit que le faisceau $\mathfrak{F}$ est obtenu par recollement des faisceaux $\mathfrak{F}_{i}$ au moyen des isomorphismes $\theta_{i j}$.

5. Extension et restriction d'un faisceau. Soient $X$ un espace topologique, $Y$ un sous-espace fermé de $X, \mathfrak{F}$ un faisceau sur $X .$ Nous dirons que $\mathfrak{F}$ est concentré sur $Y$, ou est nul en dehors de $Y$ si l'on a $\mathfrak{F}_{x}=0$ pour tout $x \in X-Y$.

Proposition 5. Si le faisceau $\mathfrak{F}$ est concentré sur $Y, l^{\prime} h o m o m o r p h i s m e$

$$
\rho_{Y}^{x}: \Gamma(X, \mathscr{}) \rightarrow \Gamma(Y, \mathscr{F}(Y))
$$

est bijectif.

Si une section de $\mathscr{F}$ au-dessus de $X$ est nulle au-dessus de $Y$, elle est nulle partout puisque $\mathfrak{F}_{x}=0$ si $x \notin Y$, ce qui montre que $\rho_{Y}^{X}$ est injectif. Inversement, soit $s$ une section $\operatorname{de} \mathfrak{F}(Y)$ au-dessus de $Y$, et prolongeons $s$ à $X$ en posant $s(x)=0$ si $x \notin Y$; l'application $x \rightarrow s(x)$ est évidemment continue sur $X-Y$; d'autre part, si $x \in Y$, il existe une section $s^{\prime}$ de $F$ au-dessus d'un voisinage $U$ de $x$ telle que $s^{\prime}(x)=s(x)$; comme $s$ est continue sur $Y$ par hypothèse, il existe un voisinage $V$ de $x$, contenu dans $U$, et tel que $s^{\prime}(y)=s(y)$ pour tout $y \in V \cap Y ; \mathrm{du}$ fait que $\mathcal{F}_{\nu}=0$ si $y \notin Y$, on a aussi $s^{\prime}(y)=s(y)$ pour $y \in V-V$ n $Y$; donc $s$ et $s^{\prime}$ coïncident sur $V$, ce qui prouve que $s$ est continue au voisinage de $Y$, donc continue partout. Il s'ensuit que $\rho_{Y}^{X}$ est surjectif, ce qui achève la démonstration.

Nous allons maintenant montrer que le faisceau $\mathcal{F}(Y)$ détermine sans ambiguïté le faisceau $\mathcal{F}$ :

Proposirion 6. Soit $Y$ un sous-espace fermé d'un espace $X$, et soit $\mathcal{S}$ un faisceau sur $Y$. Posons $\mathfrak{F}_{x}=\mathrm{S}_{x}$ si $x \in Y, \mathfrak{F}_{x}=0$ si $x \notin Y$, et soit $\mathfrak{F}$ l'ensemble somme des $\mathfrak{F}_{x} .$ On peut munir $\mathfrak{F}$ d'une structure de faisceau sur $X$, et d'une seule, telle que $\mathcal{F}(Y)=\mathcal{G}$

Soit $U$ un ouvert de $X ;$ si $s$ est une section de $\mathcal{S}$ sur $U$ n $Y$, prolongeons $s$ par 0 sur $U-U \cap Y ;$ lorsque $s$ parcourt $\Gamma(U \cap Y, S)$ on obtient ainsi un groupe $\mathcal{F}_{J}$ d'applications de $U$ dans $\mathfrak{F} .$ La Proposition 5 montre que, si $\mathfrak{F}$ est muni d'une structure de faisceau telle que $\mathfrak{F}(Y)=\mathrm{S}$, on a $\mathfrak{F}_{U}=\Gamma(U, \mathfrak{F})$, ce qui prouve l'unicité de la structure en question. Son existence se montre par le procédé du $\mathrm{n}^{\circ} 3$, appliqué aux $\mathfrak{F}_{U}$, et aux homomorphismes de restriction $\varphi_{U}: \mathfrak{F}_{U} \rightarrow \mathfrak{F}_{\boldsymbol{V}}$.

On dit que le faisceau $\mathfrak{F}$ est obtenu en prolongeant le faisceau $\mathrm{S}$ par 0 en dehors de $Y$; on le note $\mathcal{G}^{\boldsymbol{Y}}$, ou même $\mathcal{S}$, si aucune confusion ne peut en résulter.

6. Faisceaux d'anneaux et faisceaux de modules. La notion de faisceau définie au $\mathrm{n}^{\circ} 1$ est celle de faisceau de groupes abéliens. Il est clair qu'il existe des définitions analogues pour toute structure algébrique (on pourrait même définir les "faisceaux d'ensembles", où $\mathscr{F}_{x}$ ne serait muni d'aucune structure algébrique, et où l'on postulerait seulement l'axiome (I)). Dans la suite, nous rencontrerons principalement des faisceaux d'anneaux et des faisceaux de modules:

Un faisceau d'anneaux $\alpha$ est un faisceau de groupes abéliens $Q_{x}, x \in X$, où chaque $\alpha_{x}$ est muni d'une structure d'anneau telle que l'application $(f, g) \rightarrow f . g$ soit une application continue de $\alpha+a$ dans $a$ (les notations étant celles du $\left.\mathrm{n}^{\circ} 1\right)$. Nous supposerons toujours que chaque $\alpha_{x}$ possède un élément unité, variant continûment avec $x$.

Si $Q$ est un faisceau d'anneaux vérifiant la condition précédente, $\Gamma(U, \alpha)$ est un anneau à élément unité, et $\rho_{U}^{V}: \Gamma(V, \alpha) \rightarrow \Gamma(U, \alpha)$ est un homomorphisme unitaire si $U \subset V$. Inversement, si l'on se donne des anneaux $\alpha_{U}$ à élément unité, et des homomorphismes unitaires $\varphi_{U}^{V}: Q_{V} \rightarrow Q_{U}$, vérifiant $\varphi_{U}^{V} \circ \varphi_{V}^{W}=\varphi_{U}^{W}$, le faisceau a défini par le système $\left(\alpha_{U}, \varphi_{U}^{V}\right)$ est un faisceau d'anneaux. Par exemple, si $G$ est un anneau à élément unité, le faisceau des germes de fonctions a valeurs dans $G$ (défini au $\mathrm{n}^{\circ} 3$ ) est un faisceau d'anneaux.

Soit $Q$ un faisceau d'anneaux. Un faisceau $\mathfrak{F}$ est appelé un faisceau de $Q$-modules si chaque $\mathfrak{F}_{x}$ est muni d'une structure de $\alpha_{x}$-module unitaire (à gauche, pour fixer les idées), variant "continûment" avec $x$, au sens suivant: si $a+\mathfrak{F}$ est la partie de $\alpha \times \mathfrak{F}$ formée des couples $(a, f)$ tels que $\pi(a)=\pi(f)$, l'application $(a, f) \rightarrow a . f$ est une application continue de $\alpha+\mathscr{F}$ dans $\mathfrak{F}$.

Si $\mathfrak{F}$ est un faisceau de $Q$-modules, $\Gamma(U, \mathfrak{F})$ est un module unitaire sur $\Gamma(U, \alpha)$. Inversement, supposons a défini par le système $\left(Q_{U}, \varphi_{U}^{V}\right)$ comme ci-dessus, et soit $\mathfrak{\text { le faisceau défini par le système }}\left(\mathfrak{F}_{U}, \psi_{U}^{v}\right)$, où chaque $\mathfrak{F}_{U}$ est un $\alpha_{U}$-module unitaire, avec $\psi_{U}^{V}(a . f)=\varphi_{U}^{V}(a) \cdot \psi_{U}^{V}(f)$ si $a \in Q_{V}, f \in \mathcal{F}_{V} ;$ alors $\mathfrak{F}$ est un faisceau de $\alpha$-modules.

Tout faisceau de groupes abéliens peut être considéré comme un faisceau de Z-modules, $\mathbf{Z}$ désignant le faisceau constant, isomorphe à l'anneau des entiers. Ceci nous permettra, par la suite, de nous borner à étudier les faisceaux de modules.

7. Sous-faisceau et faisceau quotient. Soient $Q$ un faisceau d'anneaux, $\mathscr{F}$ un faisceau de $Q$-modules. Pour tout $x \in X$, soit $\mathcal{S}_{x}$ un sous-ensemble de $\mathcal{F}_{x} .$ On dit que $\mathrm{S}=\cup \mathrm{G}_{x}$ est un sous-faisceau $\operatorname{de} \mathcal{F}$ si :

(a) $\mathcal{S}_{x}$ est un sous- $\boldsymbol{\alpha}_{x}$-module de $\mathfrak{F}_{x}$ pour tout $x \in X$,

(b) S est un sous-ensemble ouvert de $\mathfrak{F}$.

La condition (b) peut encore s'exprimer ainsi:

(b') Si x est un point de $X$, et si s est une section de $\mathfrak{F}$ au-dessus d'un voisinage de $x$ telle que $s(x) \in \mathcal{S}_{x}$, on a $s(y) \in \mathrm{S}_{y}$ pour tout y assez voisin de $x$.

Il est clair que, si ces conditions sont vérifiées, $S$ est un faisceau de $Q$-modules.

Soit $\mathcal{G}$ un sous-faisceau de $\mathfrak{F}$, et posons $\mathfrak{H}_{x}=\mathfrak{F}_{x} / \mathrm{S}_{x}$ pour tout $x \in X .$ Munissons $\mathfrak{H}=\cup \mathfrak{C}_{x}$ de la topologie quotient de la topologie de $\mathfrak{F} ;$ on voit aisément que l'on obtient ainsi un faisceau de $Q$-modules, appelé faisceau quotient de $\mathcal{F}$ par $\mathrm{S}$, et noté F/S. On peut en donner une autre définition, utilisant le procédé du no 3 : si $U$ est un ouvert de $X$, posons $\mathfrak{F}_{U}=\Gamma(U, \mathfrak{F}) / \Gamma(U, \mathrm{~S}), \varphi_{U}^{V}$ étant l'homomorphisme défini par passage au quotient à partir de $\rho_{U}^{V}: \Gamma(V, \mathfrak{F}) \rightarrow \Gamma(U, \mathfrak{F}) ;$ le faisceau défini par le système $\left(\mathfrak{C}_{U}, \varphi_{U}^{V}\right)$ n'est autre que $\mathfrak{C}$.

L'une ou l'autre définition de $\mathfrak{H}$ montre que, si $s$ est une section de $\mathfrak{C}$ au voisinage de $x$, il existe une section $t$ de $\mathfrak{F}$ au voisinage de $x$, telle que la classe de $t(y) \bmod \mathcal{S}_{y}$ soit égale à $s(y)$ pour tout $y$ assez voisin de $x$. Bien entendu, ceci n'est plus vrai globalement, en général: si $U$ est un ouvert de $X$, on a seulement la suite exacte:

$$
0 \rightarrow \Gamma(U, \mathrm{~S}) \rightarrow \Gamma(U, \mathfrak{F}) \rightarrow \Gamma(U, \mathfrak{F})
$$

l'homomorphisme $\Gamma(U, \mathfrak{F}) \rightarrow \Gamma(U, \Im{C})$ n'étant pas surjectif en général (cf. $\left.\mathrm{n}^{\circ} 24\right)$.

8. Homomorphismes. Soient $\alpha$ un faisceau d'anneaux, $\mathfrak{F}$ et $\mathcal{S}$ deux faisceaux de $Q$-modules. Un Q-homomorphisme (ou encore un homomorphisme $Q$-linéaire, ou simplement un homomorphisme) de $\mathfrak{F}$ dans $\mathcal{G}$ est la donnée, pour tout $x \in X$, d'un $Q_{x}$-homomorphisme $\varphi_{x}: \mathfrak{F}_{x} \rightarrow \mathcal{S}_{x}$, de telle sorte que l'application $\varphi: \mathfrak{F} \rightarrow \mathbf{S}$ définie par les $\varphi_{x}$, soit continue. Cette condition peut aussi s'exprimer en disant que, si $s$ est une section de $\mathfrak{F}$ au-dessus de $U, x \rightarrow \varphi_{x}(s(x))$ est une section de $\mathcal{G}$ au-dessus de $U$ (section que l'on notera $\varphi(s)$, ou $\varphi \circ s) .$ Par exemple, si est un sous-faisceau de $\mathcal{F}$, l'injection $\mathcal{S} \rightarrow \mathfrak{F}$, et la projection $\mathfrak{F} \rightarrow \mathfrak{F} / \mathrm{S}$, sont des homomorphismes.

Proposition 7. Soit $\varphi$ un homomorphisme de $\mathfrak{F}$ dans $\mathrm{S}$. Pour tout $x \in X$, soit $\mathfrak{r}_{x}$ le noyaude $\varphi_{x}$, et soit $\mathrm{g}_{x}$ l'image de $\varphi_{x} .$ Alors $\mathfrak{I}=\mathrm{U} \mathfrak{N}_{x}$ est un sous-faisceau de $\mathfrak{F}, \boldsymbol{g}=\mathrm{U} g_{x}$ est un sous-faisceau de $\mathrm{S}$, et $\varphi$ définit un isomorphisme de $\mathfrak{F} / \mathfrak{N}$ sur $\mathrm{g}$.

Puisque $\varphi_{x}$ est un $Q_{x}$-homomorphisme, $\mathfrak{N}_{x}$ et $g_{x}$ sont des sous-modules de $\mathfrak{F}$ et de respectivement, et $\varphi_{x}$ définit un isomorphisme de $\mathfrak{F}_{x} / \mathfrak{r}_{x}$ sur $g_{x} .$ D'autre part, si $s$ est une section locale de $\mathcal{F}$, telle que $s(x) \in \mathfrak{M}_{x}$, on a $\varphi \circ s(x)=0$, d'où $\varphi \circ s(y)=0$ pour $y$ assez voisin de $x$, d'où $s(y) \in \mathfrak{l}_{y}$, ce qui montre que $\mathfrak{N}$ est un sous-faisceau de $\mathfrak{F} .$ Si $t$ est une section locale de $\mathcal{S}$, telle que $t(x) \in \mathrm{g}_{x}$, il existe une section locale $s$ de $F$, telle que $\varphi \circ s(x)=t(x)$, d'où $\varphi \circ s=t$ au voisinage de $x$, ce qui montre que $g$ est un sous-faisceau de $\mathcal{S}$, isomorphe à $\mathfrak{F} / \mathfrak{r}$.

Le faisceau $\mathfrak{N}$ est appelé le noyau de $\varphi$, et noté $\operatorname{Ker}(\varphi)$; le faisceau $g$ est appelé l'image de $\varphi$, et noté $\operatorname{Im}(\varphi) ;$ le faisceau $\mathcal{S} / \mathrm{g}$ est appelé le conoyau de $\varphi$, et noté $\operatorname{Coker}(\varphi)$. Un homomorphisme $\varphi$ est dit injectif, ou biunivoque, si chacun des $\varphi_{x}$ est injectif, ce qui équivaut à $\operatorname{Ker}(\varphi)=0 ;$ il est dit surjectif si chacun des $\varphi_{x}$ est surjectif, ce qui équivaut à $\operatorname{Coker}(\varphi)=0 ;$ il est dit bijectif s'il est à la fois injectif et surjectif, auquel cas la Proposition 7 montre que c'est un isomorphisme de $\mathfrak{F}$ sur $\mathcal{G}$, et $\varphi^{-1}$ est aussi un homomorphisme. Toutes les définitions relatives aux homomorphismes de modules peuvent se transposer de même aux faisceaux de modules; par exemple, une suite d'homomorphismes est dite exacte si l'image de chaque homomorphisme coincide avec le noyau de l'homomorphisme suivant. Si $\varphi: \mathfrak{F} \rightarrow S$ est un homomorphisme, les suites:

$$
\begin{array}{ll}
    0 \rightarrow \operatorname{Ker}(\varphi) \rightarrow \mathfrak{F} \rightarrow \operatorname{Im}(\varphi) & \rightarrow 0 \\
    0 \rightarrow \operatorname{Im}(\varphi) \rightarrow \mathcal{G} \rightarrow \operatorname{Coker}(\varphi) & \rightarrow 0
\end{array}
$$

sont des suites exactes. Si $\varphi$ est un homomorphisme de $\mathfrak{\text { dans }}$ S, l'application $s \rightarrow \varphi \circ s$ est un $\Gamma(U, \alpha)$ homomorphisme de $\Gamma(U, \mathcal{F})$ dans $\Gamma(U, \mathrm{~S}) .$ Inversement, supposons $\alpha, \mathfrak{F}, \mathrm{S}$ définis par des systèmes $\left(Q_{U}, \varphi_{U}^{V}\right),\left(\mathfrak{F}_{U}, \psi_{U}^{\bar{V}}\right),\left(\mathcal{S}_{v}, \chi_{U}^{V}\right)$, comme au $\mathrm{n}^{\circ} 6$, et donnons-nous, pour tout ouvert $U \subset X$, un $Q_{v}$-homomorphisme $\varphi_{U}: \mathfrak{F}_{U} \rightarrow \mathcal{S}_{v}$ tel que $\chi_{v} \circ \varphi_{V}=\varphi_{U} \circ \psi_{U}^{v} ;$ par passage à la limite inductive, les $\varphi_{U}$ définissent un homomorphisme $\varphi: \mathscr{F} \rightarrow \mathrm{S}$.

9. Somme directe de deux faisceaux. Soient $\alpha$ un faisceau d'anneaux, $\mathfrak{F}$ et $\mathcal{S}$ deux faisceaux de $Q$-modules; pour tout $x \in X$, formons le module $\mathcal{F}_{x}+\mathrm{S}_{x}$ somme directe de $\mathfrak{F}_{x}$ et $\mathcal{G}_{x} ;$ un élément de $\mathfrak{F}_{x}+\mathrm{S}_{x}$ est un couple $(f, g)$, avec $f \in \mathcal{F}_{x}$ et $g \in \mathcal{S}_{x}$. Soit $\mathfrak{F}$ l'ensemble somme des $\mathfrak{F}_{x}+\mathrm{S}_{x}$ lorsque $x$ parcourt $X$; on peut identifier $\mathfrak{H}$ à la partie de $\mathcal{F} \times \mathcal{S}$ formés des couples $(f, g)$ tels que $\pi(f)=\pi(g)$. Si l'on munit $\mathfrak{F C}$ de la topologie induite par celle de $\mathfrak{F} \times \mathrm{G}$, on vérifie immédiatement que $\mathfrak{C}$ est un faisceau de $\boldsymbol{Q}$-modules; on l'appelle la somme directe de $\mathfrak{F}$ et de $\mathcal{S}$, et on le note $\mathfrak{F}+\mathrm{S}$. Toute section de $\mathcal{F}+\mathrm{S}$ sur un ouvert $U \subset X$ est de la forme $x \rightarrow(s(x), t(x))$ où $s$ et $t$ sont des sections de $\mathfrak{F}$ et $\mathcal{S}$ sur $U$; en d'autres termes, $\Gamma(U, \mathfrak{F}+\mathcal{S})$ est isomorphe à la somme directe $\Gamma(U, \mathfrak{F})+\Gamma(U, \mathrm{~S})$.

La définition de la somme directe s'étend par récurrence à un nombre fini de Q-modules. En particulier, le faisceau somme directe de $p$ faisceaux isomorphes à un même faisceau $\mathfrak{F}$ sera noté $\mathfrak{F}^{p}$.

10. Produit tensoriel de deux faisceaux. Soient $a$ un faisceau d'anneaux, $\mathfrak{F}$ un faisceau de Q-modules à droite, $\mathcal{S}$ un faisceau de $Q$-modules à gauche. Pour tout $x \in X$, posons $\mathfrak{H}_{x}=\mathfrak{F}_{x} \otimes \mathcal{S}_{x}$, le produit tensoriel étant pris sur l'anneau $Q_{\alpha}$ (cf. par exemple [6], Chap. II, §2); soit 3 C l'ensemble somme des $\mathfrak{H}_{x}$.

Proposition 8. Il existe sur l'ensemble If une structure de faisceau et une seule telle que, si s et $t$ sont des sections de $\mathfrak{F}$ et de $\mathrm{S}$ sur un ouvert $U$, l'application $x \rightarrow s(x) \otimes t(x) \in \mathfrak{H}_{x}$ soit une section de $\mathfrak{H}$ au-dessus de $U$.

Le faisceau $\mathfrak{S C}$ ainsi défini est appelé le produit tensoriel (sur $\mathbb{Q}$ ) de $\mathfrak{F}$ et de $\mathrm{S}$, et on le note $\mathfrak{F} \otimes_{\mathrm{a}} \mathrm{S} ;$ si les $\mathrm{Q}_{x}$ sont commutatifs, c'est un faisceau de Q-modules.

Si $\mathfrak{H}$ est muni d'une structure de faisceau vérifiant la condition de l'énoncé, et si $s_{i}$ et $t_{i}$ sont des sections de $\mathfrak{F}$ et de $\mathcal{S}$ au-dessus d'un ouvert $U \subset X$, l'application $x \rightarrow \sum s_{i}(x) \otimes t_{i}(x)$ est une section de $\mathfrak{C}$ sur $U .$ Or tout $h \in \mathfrak{H}_{x}$ peut s'écrire sous la forme $h=\sum f_{i} \otimes g_{i}, f_{i} \in \mathfrak{F}_{x}, g_{i} \in \mathcal{S}_{x}$, donc aussi sous la forme $\sum s_{i}(x) \otimes t_{i}(x)$, où $s_{i}$ et $t_{i}$ sont des sections définies dans un voisinage $U$ de $x$; il en résulte que toute section de $\mathfrak{H}$ doit être localement égale à une section de la forme précédente, ce qui démontre l'unicité de la structure de faisceau de $\mathfrak{C}$.

Montrons maintenant son existence. Nous pouvons supposer que $Q, \mathfrak{F}, \mathrm{G}$, sont définis par des systèmes $\left(Q_{U}, \varphi_{U}^{V}\right),\left(\mathfrak{F}_{U}, \psi_{U}^{V}\right),\left(\mathcal{S}_{U}, \chi_{U}^{V}\right)$ comme au $\mathrm{n}^{\circ} 6$. Posons alors $\mathfrak{H}_{U}=\mathfrak{F}_{U} \otimes \mathrm{S}_{v}$, le produit tensoriel étant pris sur $Q_{U} ;$ les homomorphismes $\psi_{v}^{v}$ et $\chi_{v}^{v}$ définissent, par passage au produit tensoriel, un homomorphisme $\eta_{U}^{v}: \mathfrak{C}_{V} \rightarrow \mathfrak{C}_{U} ;$ en outre, on a $\lim _{x \in U} \mathfrak{C}_{U}=\lim _{x \in U} \mathfrak{F}_{U} \otimes \lim _{x \in U} S_{U}=\mathscr{C}_{x}$ le produit tensoriel étant pris sur $Q_{x}$ (commutation du produit tensoriel avec les limites inductives, cf. par exemple [6], Chap. VI, Exer. 18). Le faisceau défini par le système $\left(\mathfrak{C}_{U}, \eta_{U}^{V}\right)$ peut donc être identifié à $\mathfrak{H}$, et $\mathfrak{F C}$ se trouve ainsi muni d'une structure de faisceau répondant visiblement à la condition imposée. Enfin, si les $Q_{x}$ sont commutatifs, on peut supposer que les $Q_{U}$ le sont aussi (il suffit de prendre pour $\alpha_{U}$ l'anneau $\left.\Gamma(U, Q)\right)$, donc $\mathfrak{C}_{U}$ est un $\alpha_{U}$-module, et $\mathfrak{C}$ est un faisceau de $Q$-modules.

Soient maintenant $\varphi$ un $Q$-homomorphisme de $\mathcal{F}$ dans $\mathcal{F}^{\prime}$, et $\psi$ un $Q$-homomorphisme de $S$ dans $\mathcal{G}^{\prime} ;$ alors $\varphi_{x} \otimes \psi_{x}$ est un homomorphisme (de groupes abéliens, en général-de $\alpha_{x}$-modules, si $\alpha_{x}$ est commutatif), et la définition de $\mathfrak{F} \otimes_{\mathrm{a}} \mathrm{S}$ montre que la collection des $\varphi_{x} \otimes \psi_{x}$ est un homomorphisme de $\mathcal{F} \otimes_{a} \mathcal{S}$ dans $\mathcal{F}^{\prime} \otimes_{\mathfrak{a}} \mathrm{S}^{\prime} ;$ cet homomorphisme est noté $\varphi \otimes \psi ;$ si $\psi$ est l'application identique, on écrit $\varphi$ au lieu de $\varphi \otimes 1$.

Toutes les propriétés usuelles du produit tensoriel de deux modules se transposent d'elles-mềmes au produit tensoriel de deux faisceaux de modules. Par exemple, toute suite exacte:

$$
\mathfrak{F} \rightarrow \mathfrak{F}^{\prime} \rightarrow \mathfrak{F}^{\prime \prime} \rightarrow 0
$$

donne naissance à une suite exacte:

$$
\mathfrak{F} \otimes_{\mathrm{a}} \mathcal{S} \rightarrow \mathfrak{F}^{\prime} \otimes_{a} \mathcal{S} \rightarrow \mathfrak{F}^{\prime \prime} \otimes_{a} \mathcal{G} \rightarrow 0
$$

On a des isomorphismes canoniques:

$$
\mathscr{F} \otimes_{a}\left(S_{1}+\mathcal{S}_{2}\right) \approx \mathfrak{F} \otimes_{a} S_{1}+\mathfrak{F} \otimes_{a} \mathcal{S}_{2}, \quad \mathfrak{F} \otimes_{a} Q \approx \mathfrak{F}
$$

et (en supposant les $Q_{x}$ commutatifs, pour simplifier les notations):

$$
\mathfrak{F} \otimes_{a} \mathcal{S} \approx \mathcal{G} \otimes_{a} \mathfrak{F}, \quad \mathfrak{F} \otimes_{a}\left(\mathcal{G} \otimes_{\mathrm{a}} \mathfrak{K}\right) \approx\left(\mathfrak{F} \otimes_{a} \mathcal{S}\right) \otimes_{\mathrm{a}} \mathfrak{K}
$$

11. Faisceau des germes d'homomorphismes d'un faisceau dans un autre faisceau. Soient $\alpha$ un faisceau d'anneaux, $\mathfrak{F}$ et $\mathcal{S}$ deux faisceaux de $Q$-modules. Si $U$ est un ouvert de $X$, soit $\mathfrak{C}_{U}$ le groupe des homomorphismes de $\mathfrak{F}(U)$ dans $\mathcal{S}(U)$ (nous dirons également "homomorphisme de $\mathcal{F}$ dans $\mathcal{S}$ au-dessus de $U^{\prime \prime}$ a la place de "homomorphisme de $\mathfrak{F}(U)$ dans $\mathrm{S}(U)$ ''). L'opération de restriction d'un homomorphisme définit $\varphi_{U}^{V}: \mathfrak{C}_{V} \rightarrow \mathfrak{C}_{U} ;$ le faisceau défini par le système $\left(\mathfrak{C}_{U}, \varphi_{U}^{V}\right)$ est appelé le faisceau des germes d'homomorphismes de $\mathfrak{F}$ dans $\mathrm{S}$, et on le note $\operatorname{Hom}_{a}(F, \mathcal{S})$. Si les $Q_{x}$ sont commutatifs, $\operatorname{Hom}_{\mathfrak{a}}(F, \mathrm{~S})$ est un faisceau de Q-modules.

Un élément de $\operatorname{Hom}_{a}(\mathfrak{F}, \mathcal{S})_{x}$, étant un germe d'homomorphisme de $\mathfrak{F}$ dans $\mathrm{S}$ au voisinage de $x$, définit sans ambiguité un $Q_{x}$-homomorphisme de $\mathcal{F}_{x}$ dans $\mathcal{S}_{x}$; d'où un homomorphisme canonique

$$
\rho: \operatorname{Hom}_{\mathfrak{a}}(\mathfrak{F}, \mathcal{S})_{x} \rightarrow \operatorname{Hom}_{\mathfrak{a _ { z }}}\left(\mathfrak{F}_{x}, \mathrm{~S}_{x}\right)
$$

Mais, contrairement à ce qui se passait pour les opérations étudiées jusqu'a présent, l'homomorphisme $\rho$ n'est pas en général une bijection; nous donnerons au no 14 une condition suffisante pour qu'il le soit.

Si $\varphi: F^{\prime} \rightarrow \mathfrak{F}$ et $\psi: \mathcal{S} \rightarrow \mathcal{G}^{\prime}$ sont des homomorphismes, on définit de façon évidente un homomorphisme

$$
\operatorname{Hom}_{\alpha}(\varphi, \psi): \operatorname{Hom}_{a}(\mathcal{F}, \mathcal{G}) \rightarrow \operatorname{Hom}_{\mathfrak{a}}\left(\mathfrak{F}^{\prime}, \mathcal{G}^{\prime}\right)
$$

Toute suite exacte: $0 \rightarrow \mathcal{\rightarrow} G^{\prime} \rightarrow \mathcal{G}^{\prime \prime}$ donne naissance à une suite exacte:

$$
0 \rightarrow \operatorname{Hom}_{a}(\mathfrak{F}, \mathcal{G}) \rightarrow \operatorname{Hom}_{\mathfrak{a}}\left(\mathscr{F}, \mathcal{G}^{\prime}\right) \rightarrow \operatorname{Hom}_{\mathfrak{a}}\left(\mathcal{F}, \mathcal{G}^{\prime \prime}\right)
$$

On a également des isomorphismes canoniques: $\operatorname{Hom}_{a}(Q, \rho) \approx S$,

$$
\begin{aligned}
    &\operatorname{Hom}_{a}\left(\mathfrak{F}, \mathcal{S}_{1}+\mathcal{S}_{2}\right) \approx \operatorname{Hom}_{\mathfrak{a}}\left(\mathfrak{F}, \mathrm{S}_{1}\right)+\operatorname{Hom}_{\mathfrak{a}}\left(\mathfrak{F}, \mathcal{G}_{2}\right) \\
    &\operatorname{Hom}_{\mathfrak{a}}\left(\mathfrak{F}_{1}+\mathfrak{F}_{2}, \mathrm{G}\right) \approx \operatorname{Hom}_{\mathfrak{a}}\left(\mathfrak{F}_{1}, \mathrm{G}\right)+\operatorname{Hom}_{\mathfrak{a}}\left(\mathfrak{F}_{2}, \mathrm{G}\right)
\end{aligned}
$$

\section{§2. Faisceaux cohérents de modules}

Dans ce paragraphe, $X$ désigne un espace topologique, et $\alpha$ un faisceau d'anneaux sur $X$. On suppose que tous les $Q_{x}, x \in X$, sont commutatifs et possèdent un élément unité variant continûment avec $x .$ Tous les faisceaux considérés jusqu'au $\mathrm{n}^{\circ} 16$ sont des faisceaux de $Q$-modules, et tous les homomorphismes sont des $Q$-homomorphismes.

12. Définitions. Soit $\mathcal{F}$ un faisceau de $Q$-modules, et soient $s_{1}, \cdots, s_{p}$ des sections de $\mathfrak{F}$ au-dessus d'un ouvert $U \subset X .$ Si l'on fait correspondre a toute famille $f_{1}, \cdots, f_{p}$ d'éléments de $a_{x}$ l'élément $\sum_{i=1}^{i=p} f_{i} \cdot s_{i}(x)$ de $\mathfrak{F}_{x}$, on obtient un homomorphisme $\varphi: Q^{p} \rightarrow \mathfrak{F}$, défini au-dessus de l'ouvert $U$ (de facon plus correcte, $\varphi$ est un homomorphisme de $Q^{p}(U)$ dans $F(U)$, avec les notations du $\left.\mathrm{n}^{\circ} 4\right)$. Le noyau $R\left(s_{1}, \cdots, s_{p}\right)$ de l'homomorphisme $\varphi$ est un sous-faisceau de $a^{p}$, appelé faisceau des relations entre les $s_{i} ;$ l'image de $\varphi$ est le sous-faisceau de $\mathfrak{F}$ engendré par les $s_{i}$. Inversement, tout homomorphisme $\varphi: \mathbb{Q}^{p} \rightarrow \mathfrak{F}$ définit $\mathrm{de}_{\mathrm{s}}$ sections $s_{1}, \cdots, s_{p}$ de $\mathfrak{F}$ par les formules:

$$
s_{1}(x)=\varphi_{x}(1,0, \cdots, 0), \quad \cdots, \quad s_{p}(x)=\varphi_{x}(0, \cdots, 0,1)
$$

DÉFINITION 1. Un faisceau de Q-modules $\mathfrak{F}$ est dit de type fini s'il est localement engendré par un nombre fini de ses sections.

Autrement dit, pour tout point $x \in X$, il doit exister un ouvert $U$ contenant $x$, et un nombre fini de sections $s_{1}, \cdots, s_{p}$ de $\mathfrak{F}$ au-dessus de $U$, tels que tout élément de $\mathfrak{F}_{y}, y \in U$, soit combinaison linéaire, à coefficients dans $Q_{y}$, des $s_{i}(y) .$ D'après ce qui précède, il revient au même de dire que la restriction de $\mathfrak{F}$ à $U$ est isomorphe à un faisceau quotient d'un faisceau $\alpha^{p}$.

Proposition 1. Soit $\mathfrak{F}$ un faisceau de type fini. Si $s_{1}, \cdots, s_{p}$ sont des sections de $\mathfrak{F}$, définies au-dessus d'un voisinage d'un point $x \in X$, et qui engendrent $\mathfrak{F}_{x}$, elles engendrent $\mathcal{F}_{y}$ pour tout y assez voisin de $x$.

Puisque $\mathfrak{F}$ est de type fini, il y a un nombre fini de sections de $\mathcal{F}$ au voisinage de $x$, soient $t_{1}, \cdots, t_{q}$, qui engendrent $\mathcal{F}_{y}$ pour $y$ assez voisin de $x$. Puisque les $s_{i}(x)$ engendrent $\mathcal{F}_{x}$, il existe des sections $f_{i j}$ de $Q$ au voisinage de $x$ telles que $t_{i}(x)=\sum_{j=1}^{j=p} f_{i j}(x) \cdot s_{j}(x) ;$ il s'ensuit que, pour $y$ assez voisin de $x$, on a:

$$
t_{i}(y)=\sum_{j=1}^{j=p} f_{i j}(y) \cdot s_{j}(y)
$$

ce qui entraîne que les $s_{j}(y)$ engendrent $\mathfrak{F}_{y}$, cqfd. Dérinitron 2. Un faisceau de Q-modules $\mathfrak{F}$ est dit cohérent si:

(a) $\mathfrak{F}$ est de type fini,

(b) Si $s_{1}, \cdots, s_{p}$ sont des sections de $\mathfrak{F}$ au-dessus d'un ouvert $U \subset X$, le faisceau des relations entre les $s_{i}$ est un faisceau de type fini (sur l'ouvert $U$ ).

On notera le caractère local des définitions 1 et $2 .$

Proposition 2. Localement, tout faisceau cohérent est isomorphe au conoyau d'un homomorphisme $\varphi: Q^{q} \rightarrow \alpha^{p}$.

Cela résulte immédiatement des définitions, et des remarques qui précèdent la définition $1 .$

Proposition 3. Tout sous-faisceau de type fini d'un faisceau cohérent est un faisceau cohérent.

En effet, si un faisceau $\mathfrak{F}$ vérifie la condition (b) de la définition 2, il est évident que tout sous-faisceau de $\mathfrak{F}$ la vérifie aussi.

\section{Principales propriétés des faisceaux cohérents.}

THÉorème 1. Soit $0 \rightarrow \mathfrak{F} \stackrel{\alpha}{\rightarrow} \mathrm{S} \stackrel{\beta}{\rightarrow} \mathfrak{H} \rightarrow 0$ une suite exacte d'homomorphismes. Si deux des trois faisceaux $\mathfrak{F}, \mathrm{G}, \mathfrak{C}$ sont cohérents, le troisième l'est aussi.

Supposons $\mathcal{S}$ et $\mathfrak{F C}$ cohérents. Il existe donc localement un homomorphisme surjectif $\gamma: \mathrm{a}^{p} \rightarrow \mathcal{G}$; soit $\boldsymbol{g}$ le noyau de $\beta \circ \gamma$; puisque $\mathfrak{F}$ est cohérent, $s$ est un faisceau de type fini (condition (b)); donc $\gamma(g)$ est un faisceau de type fini, donc cohérent d'après la Proposition 3 ; comme $\alpha$ est un isomorphisme de $\mathfrak{F}$ sur $\gamma(g)$, il en résulte bien que $\mathcal{F}$ est cohérent.

Supposons $\mathfrak{F}$ et $\mathcal{S}$ cohérents. Puisque $\mathcal{S}$ est de type fini, $\mathfrak{C}$ est aussi de type fini, et il reste à montrer que $\mathfrak{H}$ vérifie la condition (b) de la définition $2 .$ Soient $s_{1}, \cdots, s_{p}$ un nombre fini de sections de $\mathfrak{H}$ au voisinage d'un point $x \in X .$ La question étant locale, on peut supposer qu'il existe des sections $s_{1}^{\prime}, \cdots, s_{p}^{\prime}$ de S telles que $s_{i}=\beta\left(s_{i}^{\prime}\right)$. Soient d'autre part $n_{1}, \cdots, n_{q}$ un nombre fini de sections de $\mathfrak{F}$ au voisinage de $x$, engendrant $\mathfrak{F}_{y}$ pour $y$ assez voisin de $x$. Pour qu'une famille $f_{1}, \cdots, f_{p}$ d'éléments de $\alpha_{y}$ appartiennent à $Q\left(s_{1}, \cdots, s_{p}\right)_{y}$, il faut et il suffit qu'il existe $g_{1}, \cdots, g_{q} \in Q_{y}$ tels que

$$
\sum_{i=1}^{i=p} f_{i} \cdot s_{i}^{\prime}=\sum_{j=1}^{j=q} g_{j} \cdot \alpha\left(n_{j}\right) \text { en } y
$$

Or le faisceau des relations entre les $s_{i}^{\prime}$ et les $\alpha\left(n_{j}\right)$ est de type fini, puisque $S$ est cohérent. Le faisceau $R\left(s_{1}, \cdots, s_{p}\right)$, image du précédent par la projection canonique de $\alpha^{p+q}$ sur $Q^{p}$ est donc de type fini, ce qui achève de montrer que $\mathfrak{H}$ est cohérent.

Supposons $\mathfrak{F}$ et $\mathfrak{H}$ cohérents. La question étant locale, on peut supposer $\mathfrak{F}$ (resp. $\mathfrak{C}$ ) engendré par un nombre fini de sections $n_{1}, \cdots, n_{q}\left(\right.$ resp. $\left.s_{1}, \cdots, s_{p}\right)$; en outre on peut supposer qu'il existe des sections $s_{i}^{\prime}$ de $\mathcal{G}$ telles que $s_{i}=\beta\left(s_{i}^{\prime}\right)$. Il est alors clair que les sections $s_{i}^{\prime}$ et $\alpha\left(n_{j}\right)$ engendrent $\mathcal{G}$, ce qui prouve que $\mathrm{S}$ est un faisceau de type fini. Soient maintenant $t_{1}, \cdots, t_{r}$ un nombre fini de sections de $\mathcal{S}$ au voisinage d'un point $x$; puisque $\mathfrak{H}$ est cohérent, il existe des sections $f_{j}^{i}$ de $Q^{r}(1 \leqq i \leqq r, 1 \leqq j \leqq s)$, définies au voisinage de $x$, et qui engendrent le faisceau des relations entre les $\beta\left(t_{i}\right)$. Posons $u_{j}=\sum_{i=1}^{\ell=r} f_{j}^{2} \cdot t_{i} ;$ puis- que $\sum_{i=1}^{i=r} f_{j}^{i} \cdot \beta\left(t_{i}\right)=0$, les $u_{j}$ sont contenus dans $\alpha(\mathfrak{})$, et, comme $\mathfrak{F}$ est cohérent, le faisceau des relations entre les $u_{j}$ est engendré, au voisinage de $x$, par un nombre fini de sections, soient $g_{k}^{j}(1 \leqq j \leqq s, 1 \leqq k \leqq t) .$ Je dis que les $\sum_{j=1}^{j=s} g_{k}^{j} \cdot f_{j}^{i}$ engendrent le faisceau $R\left(t_{1}, \cdots, t_{r}\right)$ au voisinage de $x$; en effet, si $\sum_{i=1}^{i=r} f_{i} \cdot t_{i}=0$ en $y$, avec $f_{i} \in Q_{y}$, on $a \sum_{i=1}^{i=r} f_{i} \cdot \beta\left(t_{i}\right)=0$, et il existe $g_{j} \in Q_{y}$ avec $f_{i}=\sum_{j=1}^{j=8} g_{j} \cdot f_{j}^{i}$ en écrivant que $\sum_{i=1}^{i=r} f_{i} \cdot t_{i}=0$, on obtient $\sum_{j=1}^{j=8} g_{j} \cdot u_{j}=0$, d'où le fait que le système des $g_{j}$ est combinaison linéaire des systèmes $g_{k}^{j}$, ce qui démontre notre assertion. Il s'ensuit que $\mathcal{S}$ vérifie la condition (b), ce qui achève la démonstration.

CoroulalRe. La somme directe d'une famille finie de faisceaux cohérents est un faisceau cohérent.

THÉonème 2. Soit $\varphi$ un homomorphisme d'un faisceau cohérent $\mathfrak{F}$ dans un faisceau cohérent G. Le noyau, le conoyau, et l'image de $\varphi$ sont alors des faisceaux cohérents.

Puisque $\mathfrak{F}$ est cohérent, $\operatorname{Im}(\varphi)$ est de type fini, donc cohérentd'après la Proposition $3 .$ En appliquant le Théorème 1 aux suites exactes:

$$
\begin{aligned}
    0 \rightarrow \operatorname{Ker}(\varphi) \rightarrow \mathfrak{F} \rightarrow \operatorname{Im}(\varphi) & \rightarrow 0 \\
    0 \rightarrow \operatorname{Im}(\varphi) & \rightarrow \mathcal{G} \rightarrow \operatorname{Coker}(\varphi) & \rightarrow 0
\end{aligned}
$$

on voit alors que $\operatorname{Ker}(\varphi)$ et Coker $(\varphi)$ sont cohérents.

CorolLAIRe. Soient $\mathfrak{F}$ et $\mathcal{S}$ deux sous-faisceaux cohérents d'un faisceau cohérent JC. Les faisceaux $\mathcal{F}+\mathcal{S}$ et $\mathfrak{F} \cap \mathcal{S}$ sont cohérents.

Pour $\mathfrak{F}+\mathrm{S}$, cela résulte de la Proposition $3 ;$ quant à $\mathfrak{F} \cap \mathrm{S}$, c'est le noyau de $\mathfrak{F} \rightarrow \mathfrak{F} / \mathrm{S} .$

14. Opérations sur les faisceaux cohérents. Nous venons de voir que la somme directe d'un nombre fini de faisceaux cohérents est un faisceau cohérent. Nous allons démontrer des résultats analogues pour les foncteurs $\otimes$ et Hom.

Proposition 4. Si $\mathfrak{F}$ et $\mathcal{S}$ sont deux faisceaux cohérents, $\mathfrak{F} \otimes_{\mathrm{a}} \mathrm{S}$ est un faisceau cohérent.

D'aprè la Proposition $2, \mathfrak{F}$ est localement isomorphe au conoyau d'un homomorphisme $\varphi: Q^{q} \rightarrow \mathbb{a}^{p} ;$ donc $\mathfrak{F} \otimes_{a} \mathcal{S}$ est localement isomorphe au conoyau de $\varphi: \mathbb{Q}^{q} \otimes_{a} \mathrm{G} \rightarrow \mathbb{C}^{p} \otimes_{a} \mathcal{G} .$ Mais $\alpha^{q} \otimes_{a} \mathcal{S}$ et $Q^{p} \otimes_{a} \mathcal{S}$ sont respectivement isomorphes à $\mathcal{G}^{a}$ et à $\mathcal{}^{p}$, qui sont cohérents (Corollaire au Théorème 1). Donc $\mathfrak{F} \otimes_{\mathfrak{a}} \mathcal{S}$ est cohérent (Théorème 2 ).

Proposition $5 .$ Soient $\mathcal{F}$ et $\mathcal{S}$ deux faisceaux, $\mathfrak{F}$ étant cohérent. Pour tout $x \in X$, le module ponctuel $\operatorname{Hom}_{a}(\mathfrak{F}, \mathrm{G})_{x}$ est isomorphe à $\operatorname{Hom}_{\alpha_{x}}\left(\mathfrak{F}_{x}, \mathrm{~S}_{x}\right)$.

De façon plus précise, montrons que l'homomorphisme

$$
\rho: \operatorname{Hom}_{\mathfrak{a}}(\mathfrak{F}, \mathrm{S})_{x} \rightarrow \operatorname{Hom}_{\boldsymbol{a}_{x}}\left(\mathfrak{F}_{x}, \mathrm{~S}_{x}\right)
$$

défini au $n^{\circ} 11$, est bijectif. Soit tout d'abord $\psi: \mathfrak{F} \rightarrow \mathrm{G}$ un homomorphisme défini au voisinage de $x$, et nul sur $\mathcal{F}_{x} ;$ du fait que $\mathfrak{F}$ est de type fini, on conclut aussitôt que $\psi$ est nul au voisinage de $x$, ce qui prouve que $\rho$ est injectif. Montrons que $\rho$ est surjectif, autrement dit que, si $\varphi$ est un $\alpha_{x}$-homomorphisme de $\mathfrak{F}_{x}$ dans $\mathcal{S}_{x}$, il existe un homomorphisme $\psi: \mathfrak{F} \rightarrow \mathrm{S}$, défini au voisinage de $x$, et tel que $\psi_{x}=\varphi$. Soient $m_{1}, \cdots, m_{v}$ un nombre fini de sections de $\mathfrak{F}$ au voisinage de $x$, engendrant $\mathfrak{F}_{y}$ pour tout $y$ assez voisin de $x$, et soient $f_{j}^{i}(1 \leqq i \leqq p$, $1 \leqq j \leqq q$ ) des sections de $\alpha^{p}$ engendrant $R\left(m_{1}, \cdots, m_{p}\right)$ au voisinage de $x$. II existe des sections locales de $\mathcal{G}$, soient $n_{1}, \cdots, n_{p}$, telles que $n_{i}(x)=\varphi\left(m_{i}(x)\right)$. Posons $p_{j}=\sum_{i=1}^{i=p} f_{j}^{i} \cdot n_{i}, 1 \leqq j \leqq q ;$ les $p_{j}$ sont des sections locales de $\mathcal{S}$ qui s'annulent en $x$, donc en tous les points d'un voisinage $U$ de $x$. Il s'ensuit que, pour $y \in U$, la formule $\sum f_{i} \cdot m_{i}(y)=0$ avec $f_{i} \in Q_{y}$, entraîne $\sum f_{i} \cdot n_{i}(y)=0$; pour tout élément $m=\sum f_{i} \cdot m_{i}(y) \in \mathfrak{F}_{y}$, on peut donc poser:

$$
\psi_{y}(m)=\sum_{i=1}^{i=p} f_{i} \cdot n_{i}(y) \in \mathcal{S}_{y}
$$

cette formule définissant $\psi_{y}(m)$ sans ambiguité. La collection des $\psi_{y}, y \in U$, constitue un homomorphisme $\psi: \mathfrak{F} \rightarrow \mathrm{S}$, défini au-dessus de $U$, et tel que $\psi_{x}=\varphi$, ce qui achève la démonstration.

Proposrion 6. Si $\mathfrak{F}$ et $\mathcal{S}$ sont deux faisceaux cohérents, $\operatorname{Hom}_{a}(\mathcal{F}, \mathcal{G})$ est un faisceau coherent.

La question étant locale, on peut supposer, d'après la Proposition 2, que l'on a une suite exacte: $Q^{q} \rightarrow Q^{p} \rightarrow \mathfrak{F} \rightarrow 0$. Il résulte alors de la Proposition précédente que la suite:

$$
0 \rightarrow \operatorname{Hom}_{a}(\Im, S) \rightarrow \operatorname{Hom}_{a}\left(Q^{p}, \mathcal{G}\right) \rightarrow \operatorname{Hom}_{a}\left(Q^{q}, \mathcal{S}\right)
$$

est exacte. Or le faisceau $\mathrm{Hom}_{\alpha}\left(Q^{p}, \mathrm{G}\right)$ est isomorphe à $\mathrm{G}^{p}$, donc est cohérent, et de même pour $\operatorname{Hom}_{a}\left(a^{q}, \mathcal{S}\right)$. Le Théorème 2 montre alors que $\operatorname{Hom}_{a}(F, \mathrm{G})$ est cohérent.

15. Faisceaux cohérents d'anneaux. Le faisceau d'anneaux $Q$ peut être regardé comme un faisceau de $Q$-modules; si ce faisceau de $Q$-modules est cohérent, nous dirons que $Q$ est un faisceau cohérent d'anneaux. Comme $Q$ est évidemment de type fini, cela signifie que $Q$ vérifie la condition (b) de la Proposition 2. Autrement dit:

DéFinition 3. Le faisceau a est un faisceau cohérent d'anneaux si le faisceau des relations entre un nombre fini de sections de Q au-dessus d'un ouvert $U$ est un faisceau de type fini sur $U$

ExEMPLES. (1) Si $X$ est une variété analytique complexe, le faisceau des germes de fonctions holomorphes sur $X$ est un faisceau cohérent d'anneaux, d'après un théorème de K. Oka (cf. [3], exposé XV, ou $[5], \S 5)$.

(2) Si $X$ est une variété algébrique, le faisceau des anneaux locaux de $X$ est un faisceau cohérent d'anneaux (cf. $n^{\circ} 37$, Proposition 1 ).

Lorsque $Q$ est un faisceau cohérent d'anneaux, on a les résultats suivants:

Proposition 7 . Pour qu'un faisceau de Q-modules soit cohérent, il faut et il suffit que, localement, il soit isomorphe au conoyau d'un homomorphisme $\varphi: \mathbb{Q}^{q} \rightarrow \mathrm{C}^{p}$.

La nécessité n'est autre que la Proposition $2 ;$ la suffisance résulte de ce que $\alpha^{p}$ et $\alpha^{q}$ sont cohérents, et du Théorème $2 .$ Proposition 8. Pour qu'un sous-faisceau de $Q^{p}$ soit cohérent, il faut et il suffit qu'il soit de type fini.

C'est un cas particulier de la Proposition $3 .$

CorollalRe. Le faisceau des relations entre un nombre fini de sections d'un faisceau cohérent est un faisceau cohérent.

En effet, ce faisceau est de type fini, par définition même des faisceaux cohérents.

Proposition 9. Soit $\mathcal{F}$ un faisceau cohérent de a-modules. Pour tout $x \in X$, soit $\mathrm{g}_{x} l$ l'idéal de $Q_{x}$ formé des $a \in \mathbb{Q}_{x}$ tels que a. $f=0$ pour tout $f \in \mathfrak{F}_{x} .$ Les $\boldsymbol{g}_{x}$ forment un faisceau cohérent d'idéaux (appelé l'annulateur de $\mathfrak{F}$ ).

En effet, $s_{x}$ est le noyau de l'homomorphisme: $\boldsymbol{a}_{x} \rightarrow \operatorname{Hom}_{a_{x}}\left(\mathfrak{F}_{x}, \mathfrak{F}_{x}\right) ;$ on applique alors les Propositions 5 et 6, et le Théorème $2 .$

Plus généralement, le transporteur $\mathfrak{F}: \mathrm{S}$ d'un faisceau cohérent $\mathcal{S}$ dans un sousfaisceau cohérent $\mathfrak{F}$ est un faisceau cohérent d'idéaux (c'est l'annulateur de $\mathrm{S} / \mathfrak{F})$

16. Changement d'anneaux. Les notions de faisceau de type fini, et de faisceau cohérent, sont relatives à un faisceau d'anneaux a déterminé. Lorsqu'on considèrera plusieurs faisceaux d'anneaux, on dira "de type fini sur $Q$ ", " $Q$-cohérent", pour préciser qu'il s'agit de faisceaux de $Q$-modules.

ThÉonème 3. Soient Q un faisceau cohérent d'anneaux, $\boldsymbol{g}$ un faisceau cohérent d'idéaux de a. Soit $\mathfrak{F}$ un faisceau de Q/g-modules. Pour que $\mathfrak{F}$ soit $\mathrm{Q} / \mathrm{g}$-cohérent, il faut et il suffit qu'il soit Q-coherent. En particulier, Q/g est un faisceau cohérent d'anneaux.

Il est clair que "de type fini sur $\alpha$ " équivaut à "de type fini sur $\mathbb{Q} / \mathrm{g}$ ". D'autre part, si $\mathscr{F}$ est $\boldsymbol{Q}$-cohérent, et si $s_{1}, \cdots, s_{p}$ sont des sections de $\mathfrak{F}$ sur un ouvert $U$, le faisceau des relations entre les $s_{i}$, à coefficients dans $Q$, est de type fini sur $Q$; il s'ensuit aussitôt que le faisceau des relations entre les $s_{i}$, à coefficients dans $Q / g$, est de type fini sur $Q / g$, puisque c'est l'image du précédent par l'application canonique $Q^{p} \rightarrow(\alpha / g)^{p} .$ Donc $\mathfrak{F}$ est $\mathrm{a} / \mathrm{g}$-cohérent. En particulier, puisque $Q / g$ est $Q$-cohérent, il est aussi $Q / g$-cohérent, autrement dit, $Q / g$ est un faisceau cohérent d'anneaux. Inversement, si $\mathfrak{F}$ est $\mathbb{Q} / \mathrm{s}$-cohérent, il est localement isomorphe au conoyau d'un homomorphisme $\varphi:(\mathbb{Q} / \mathrm{g})^{\mathrm{q}} \rightarrow(\mathbb{Q} / \mathrm{g})^{p}$, et comme $Q / g$ est $Q$-cohérent, $\mathfrak{F}$ est $Q$-cohérent, d'après le Théorème 2 .

17. Extension et restriction d'un faisceau cohérent. Soit $Y$ un sous-espace fermé de l'espace $X .$ Si $\mathcal{S}$ est un faisceau sur $Y$, nous noterons $\mathcal{S}^{\boldsymbol{X}}$ le faisceau obtenu en prolongeant $\mathcal{S}$ par 0 en dehors de $Y$; c'est un faisceau sur $X\left(\right.$ cf. $\left.\mathrm{n}^{\circ} 5\right)$. Si $Q$ est un faisceau d'anneaux sur $Y, \alpha^{X}$ est un faisceau d'anneaux sur $X$, et, si $\mathscr{F}$ est un faisceau de $Q$-modules, $\mathscr{^}{X}$ est un faisceau de $Q^{x}$-modules.

Proposition 10 . Pour que $\mathfrak{F}$ soit de type fini sur $\boldsymbol{Q}$, il faut et il suffit que $\mathfrak{F}^{\mathrm{X}}$ soit de type fini sur $\alpha^{X}$.

Soit $U$ un ouvert de $X$, et soit $V=U \cap Y$. Tout homomorphisme $\varphi: Q^{p} \rightarrow \mathfrak{F}$ au-dessus de $V$ définit un homomorphisme $\varphi^{x}:\left(Q^{X}\right)^{p} \rightarrow \mathfrak{F}^{X}$ au-dessus de $U$, et réciproquement; pour que $\varphi$ soit surjectif, il faut et il suffit que $\varphi^{x}$ le soit. La proposition résulte immédiatement de là.

On démontre de même:

Proposition 11. Pour que $\mathfrak{F}$ soit Q-cohérent, il faut et il suffit que $\mathfrak{F}^{\mathbf{A}}$ soit $\alpha^{X}$-cohérent.

D'où, en prenant $\mathfrak{F}=\boldsymbol{a}$ :

Corollatre. Pour que a soit un faisceau cohérent d'anneaux, il faut et il suffit que $Q^{x}$ soit un faisceau cohérent d'anneaux.

\section{§3. Cohomologie d'un espace à valeurs dans un faisceau}

Dans ce paragraphe, $X$ désigne un espace topologique, séparé ou non. Par un recouvrement de $X$, nous entendrons toujours un recouvrement ouvert.

18. Cochaines d'un recouvrement. Soit $\mathfrak{U}=\left\{U_{i}\right\}_{\text {it }}$ un recouvrement de $X$. Si $s=\left(i_{0}, \cdots, i_{p}\right)$ est une suite finie d'éléments de $I$, nous poserons:

$$
U_{s}=U_{i_{0} \cdots i_{p}}=U_{i_{0}} \cap \cdots \cap U_{i_{p}}
$$

Soit $\mathfrak{F}$ un faisceau de groupes abéliens sur l'espace $X .$ Si $p$ est un entier $\geqq 0$, on appelle $p$-cochaine de ll à valeurs dans $\mathfrak{F}$ une fonction $f$ qui fait correspondre à toute suite $s=\left(i_{0}, \cdots, i_{p}\right)$ de $p+1$ éléments de $I$, une section $f_{s}=f_{i_{0} \cdots i_{p}}$ de $\mathscr{F}$ au-dessus de $U_{i_{0}} \ldots_{i_{n}} .$ Les $p$-cochaînes forment un groupe abélien, noté $C^{p}(\mathfrak{l}, \mathfrak{F}) ;$ c'est le groupe produit $\prod \Gamma\left(U_{s}, \mathfrak{}\right)$, le produit étant étendu à toutes les suites $s$ de $p+1$ éléments de $I$. La famille des $C^{p}(\mathfrak{l}, \mathfrak{F}), p=0,1, \cdots$, est notée $C(\mathfrak{l}, \mathfrak{F})$. Une $p$-cochaîne est aussi appelée une cochaîne de degré $p$.

Une $p$-cochaîne $f$ est dite alternée si:

(a) $f_{i_{0} \cdots i_{p}}=0$ chaque fois que deux des indices $i_{0}, \cdots, i_{p}$ sont égaux,

(b) $f_{i_{0} \cdots i_{\sigma p}}=\varepsilon_{\sigma} f_{i_{0} \cdots i_{p}}$, si $\sigma$ est une permutation de l'ensemble $\{0, \cdots, p\}$ $\left(\varepsilon_{\sigma}\right.$ désignant la signature de $\left.\sigma\right)$.

Les $p$-cochaînes alternées forment un sous-groupe $C^{\prime p}(\mathfrak{U}, \mathfrak{F})$ du groupe $C^{p}(\mathfrak{l}, \mathfrak{F})$; la famille des $C^{\prime p}(\mathfrak{l l}, \mathfrak{F})$ est notée $C^{\prime}(\mathfrak{l}, \mathfrak{F})$.

19. Opérations simpliciales. Soit $S(I)$ le simplexe ayant pour ensemble de sommets l'ensemble $I$; un simplexe (ordonné) de $S(I)$ est une suite $s=\left(i_{0}, \cdots, i_{p}\right)$ d'éléments de $I ; p$ est appelé la dimension de $s .$ Soit $K(I)=\sum_{p=0}^{\infty} K_{p}(I)$ le complexe défini par $S(I):$ par définition, $K_{p}(I)$ est le groupe libre ayant pour base l'ensemble des simplexes de dimension $p$ de $S(I)$.

Si $s$ est un simplexe de $S(I)$, nous noterons $|s|$ l'ensemble des sommets de $s .$ Une application $h: K_{p}(I) \rightarrow K_{q}(I)$ est appelée un endomorphisme simplicial si (i) $h$ est un homomorphisme,

(ii) Pour tout simplexe $s$ de dimension $p$ de $S(I)$, on a

$$
h(s)=\sum_{s^{\prime}} c_{s}^{s^{\prime}} \cdot s^{\prime}, \quad \text { avec } c_{s}^{s^{\prime}} \in \mathbb{Z},
$$

la somme étant étendue aux simplexes $s^{\prime}$ de dimension $q$ tels que $\left|s^{\prime}\right| \subset|s|$.

Soit $h$ un endomorphisme simplicial, et soit $f \in C^{4}(\mathfrak{l}, \mathfrak{F})$ une cochaîne de degré $q$. Pour tout simplexe $s$ de dimension $p$, posons:

$$
\left({ }^{t} h f\right)_{s}=\sum_{s^{\prime}} c_{s}^{s^{\prime}} \cdot \rho_{s}^{s^{\prime}}\left(f_{s^{\prime}}\right)
$$

$\rho^{s \prime}$ désignant l'homomorphisme de restriction: $\Gamma\left(U_{s^{\prime}}, \mathfrak{}\right) \rightarrow \Gamma\left(U_{s}, \mathfrak{F}\right)$, qui a un sens, puisque $\left|s^{\prime}\right| \subset|s| .$ L'application $s \rightarrow\left({ }^{t} h f\right)_{s}$ est une $p$-cochaîne, notée ${ }^{t} h f$. L'application $f \rightarrow{ }^{t} h f$ est un homomorphisme

$$
{ }^{t} h: C^{q}(\mathfrak{U}, \mathfrak{F}) \rightarrow C^{p}(\mathfrak{l}, \mathcal{F})
$$

et l'on vérifie immédiatement les formules:

$$
{ }^{t}\left(h_{1}+h_{2}\right)={ }^{t} h_{1}+{ }^{t} h_{2}, \quad{ }^{t}\left(h_{1} \circ h_{2}\right)={ }^{t} h_{2} \circ{ }^{t} h_{1}, \quad{ }^{t} 1=1
$$

Note. Dans la pratique, on néglige fréquemment d'écrire l'homomorphisme de restriction $\rho_{s}^{s^{\prime}} .$

20. Les complexes de cochaînes. Appliquons ce qui précède à l'endomorphisme simplicial

$$
\partial: K_{p+1}(I) \rightarrow K_{p}(I)
$$

défini par la formule usuelle:

$$
\partial\left(i_{0}, \cdots, i_{p+1}\right)=\sum_{j=0}^{j=p+1}(-1)^{j}\left(i_{0}, \cdots, \hat{\imath}_{j}, \cdots, i_{p+1}\right)
$$

le signe $^{\wedge}$ signifiant, comme d'ordinaire, que le symbole au-dessus duquel il se trouve doit être omis.

Nous obtenons ainsi un homomorphisme ${ }^{t} \partial: C^{p}(\mathfrak{u}, \mathfrak{F}) \rightarrow C^{p+1}(\mathfrak{u}, \mathfrak{F})$, que nous noterons $d$; par définition, on a donc:

$$
(d f)_{i_{0} \cdots i_{p+1}}=\sum_{j=0}^{j=p+1}(-1)^{\jmath} \rho_{j}\left(f_{i_{0}} \ldots_{i} \ldots i_{p+1}\right)
$$

$\rho_{j}$ désignant l'homomorphisme de restriction

$$
\rho_{j}: \Gamma\left(U_{i_{0} \cdots i_{i}} \ldots i_{i_{p+1}}, \mathfrak{F}\right) \rightarrow \Gamma\left(U_{i_{0} \cdots i_{p+1}}, \mathfrak{F}\right)
$$

Puisque $\partial \circ \partial=0$, on a $d \circ d=0$. Ainsi $C(\mathfrak{U}, \mathfrak{F})$ se trouve muni d'un opérateur cobord qui en fait un complexe. Le $q$-ème groupe de cohomologie du complexe $C(\mathfrak{l}, \mathfrak{F})$ sera noté $H^{\varphi}(\mathfrak{l l}, \mathfrak{F}) .$ On a :

Proposition 1. $H^{0}(\mathfrak{l l}, \mathfrak{F})=\Gamma(X, \mathcal{F}) .$

Une 0 -cochaîne est un système $\left(f_{i}\right)_{1 \epsilon I}$, chaque $f_{i}$ étant une section de $\mathfrak{F}$ audessus de $U_{i}$; pour que cette cochaine soit un cocycle, il faut et il suffit que $f_{i}-f_{j}=0$ au-dessus de $U_{i} \cap U_{j}$, autrement dit qu'il existe une section $f$ de $\mathfrak{F}$ sur $X$ tout entier qui coincide avec $f_{i}$ sur $U_{i}$ pour tout $i \in I$. D'où la Proposition.

(Ainsi $H^{0}(\mathfrak{l}, \mathfrak{F})$ est indépendant de $\mathfrak{u}$; bien entendu, il n'en est pas de même de $H^{q}(\mathfrak{l}, \mathfrak{F})$, en général).

On constate immédiatement que $d f$ est alternée si $f$ est alternée; autrement dit, $d$ laisse stable $C^{\prime}(\mathfrak{U}, \mathfrak{F})$ qui forme un sous-complexe de $C(\mathbb{l}, \mathfrak{F})$. Les groupes de cohomologie de $C^{\prime}(\mathfrak{U}, \mathfrak{F})$ seront notés $H^{\prime q}(\mathfrak{U}, \mathfrak{F})$. Proposition 2. L'injection de $C^{\prime}(\mathfrak{U}, \mathfrak{F})$ dans $C(\mathfrak{U}, \mathfrak{F})$ définit un isomorphisme de $H^{\prime q}(\mathfrak{l}, \mathcal{F})$ sur $H^{q}(\mathfrak{U}, \mathfrak{F})$ pour tout $q \geqq 0$

Munissons l'ensemble $I$ d'une structure d'ordre total, et soit $h$ l'endomorphisme simplicial de $K(I)$ défini de la façon suivante:

$h\left(\left(i_{0}, \cdots, i_{q}\right)\right)=0$ si deux des indices $i_{0}, \cdots, i_{q}$ sont égaux,

$h\left(\left(i_{0}, \cdots, i_{a}\right)\right)=\varepsilon_{\sigma}\left(i_{\sigma 0}, \cdots, i_{\sigma \sigma}\right)$ si tous les indices $i_{0}, \cdots, i_{g}$ sont distincts, $\sigma$ désignant la permutation de $\{0, \cdots, q\}$ telle que $i_{o 0}<i_{\sigma 1}<\cdots<i_{\sigma q}$

On vérifie tout de suite que $h$ commute avec $\partial$, et que $h(s)=s$ si $\operatorname{dim}(s)=0$; il en résulte (cf. [7], Chap. VI, $\S 5$ ) qu'il existe un endomorphisme simplicial $k$, élevant la dimension d'une unité, et tel que $1-h=\partial \circ k+k \circ \partial .$ D'où, en passant à $C(\mathfrak{l}, \mathfrak{F})$

$$
1-{ }^{t} h={ }^{t} k \circ d+d \circ{ }^{t} k
$$

Mais on vérifie aussitôt que ${ }^{t} h$ est un projecteur de $C(\mathfrak{l}, \mathfrak{F})$ sur $C^{\prime}(\mathfrak{l}, \mathfrak{F}) ;$ comme la formule précédente montre que c'est un opérateur d'homotopie, la Proposition est démontrée. (Comparer avec [7], Chap. VI, th. 6.10).

CorolLasen. $H^{q}(\mathfrak{U}, \mathfrak{F})=0$ pour $q>\operatorname{dim}(\mathfrak{l})$.

Par définition de $\operatorname{dim}(\mathfrak{l l})$, on a $U_{i_{0} \cdots i_{n}}=\emptyset$ pour $q>\operatorname{dim}(\mathfrak{u})$, si les indices $i_{0}, \cdots, i_{q}$ sont distincts; d'où $C^{\prime q}(\mathfrak{l}, \mathfrak{F})=0$, ce qui entraîne

$$
H^{q}(\mathfrak{U}, \mathfrak{F})=H^{\prime q}(\mathfrak{U}, \mathfrak{F})=0
$$

21. Passage d'un recouvrement à un recouvrement plus fin. Un recouvrement $\mathcal{U}=\left\{U_{i}\right\}_{i \in I}$ est dit plus fin qu'un recouvrement $\mathfrak{B}=\left\{V_{j}\right\}_{j \in J}$ s'il existe une application $\tau: I \rightarrow J$ telle que $U_{i} \subset V_{r i}$ pour tout $i \in I .$ Si $f \in C^{N}(\mathfrak{V}, \mathfrak{F})$, posons:

$$
(\tau f)_{i_{0} \cdots i_{q}}=\rho_{U}^{V}\left(f_{\tau i_{0} \cdots \tau i_{q}}\right)
$$

$\rho_{U}^{v}$ désignant l'homomorphisme de restriction défini par l'inclusion de $U_{i_{0}} \ldots i_{q}$ dans $V_{r i_{0} \cdots \tau i_{0}} .$ L'application $f \rightarrow \tau f$ est un homomorphisme de $C^{q}(\mathfrak{B}, \mathfrak{F})$ dans $C^{q}(\mathfrak{l}, \mathfrak{F})$, défini pour tout $q \geqq 0$, et commutant avec $d$, donc qui définit des homomorphismes

$$
\tau^{*}: H^{q}(\mathfrak{B}, \mathfrak{F}) \rightarrow H^{q}(\mathfrak{l}, \mathfrak{F})
$$

Proposition 3. Les homomorphismes $\tau^{*}: H^{q}(\mathfrak{B}, \mathfrak{F}) \rightarrow H^{q}(\mathfrak{l l}, \mathfrak{F})$ ne dépendent que de $\mathfrak{l l}$ et $\mathfrak{B}$, et pas de l'application $\tau$ choisie.

Soient $\tau$ et $\tau^{\prime}$ deux applications de $I$ dans $J$ telles que $U_{i} \subset V_{\tau i}$ et $U_{i} \subset V_{\tau^{\prime} i}$ il nous faut montrer que $\tau^{*}=\tau^{\prime *}$.

Soit $f \epsilon C^{q}(\mathfrak{B}, \mathfrak{F}) ;$ posons

$$
(k f)_{i_{0} \cdots \tau_{q-1}}=\sum_{h=0}^{h=q-1}(-1)^{h} \rho_{h}\left(f_{\tau i_{0} \cdots \tau i_{h} \tau^{\prime} i_{h} \cdots \tau^{\prime} i_{q-1}}\right)
$$

où $\rho_{h}$ désigne l'homomorphisme de restriction défini par l'inclusion de $U_{i_{0} \cdots i_{q-1}}$ dans $V_{\tau i_{0}} \ldots \tau_{i_{h} \tau^{\prime} i_{h}} \cdots_{\tau^{\prime} i_{q-1}}$.

On vérifie par un calcul direct (cf. [7], Chap. VI, §3) que l'on a:

$$
d k f+k d f=\tau^{\prime} f-\tau f
$$

ce qui démontre la Proposition. Ainsi, si 1 est plus fin que $\mathfrak{B}$, il existe pour tout entier $q \geqq 0$ un homomorphisme canonique de $H^{q}(\mathfrak{B}, \mathfrak{F})$ dans $H^{q}(\mathfrak{l}, \mathfrak{F}) .$ Dans toute la suite, cet homomorphisme sera noté $\sigma(\mathfrak{U}, \mathfrak{V})$.

22. Groupes de cohomologie de $X$ à valeurs dans le faisceau $\mathfrak{F}$. La relation "U est plus fin que $\mathfrak{B}$ " (que nous noterons désormais $\mathfrak{l} \prec \mathfrak{B}$ ) est une relation de préordre entre recouvrements de $X$; de plus, cette relation est filtrante, car si $\mathcal{U}=\left\{U_{i}\right\}_{i \in I}$ et $\mathfrak{B}=\left\{V_{j}\right\}_{j \in J}$ sont deux recouvrements, $\mathfrak{W}=\left\{U_{i} \cap V_{j}\right\}_{(i, j) \epsilon I \times J}$ est un recouvrement qui est à la fois plus fin que $\mathcal{U}$ et plus fin que $\mathfrak{B}$.

Nous dirons que deux recouvrements $\mathfrak{l}$ et $\mathfrak{B}$ sont équivalents si l'on a $\mathfrak{l}<\mathfrak{B}$ et $\mathfrak{B} \prec \mathfrak{u} .$ Tout recouvrement $\mathfrak{U}$ est équivalent à un recouvrement $\mathfrak{u}^{\prime}$ dont l'ensemble d'indices est une partie de $\mathfrak{P}(X)$; en effet, on peut prendre pour $\mathfrak{u}^{\prime}$ l'ensemble des ouverts de $X$ appartenant à la famille 1 . On peut donc parler de l'ensemble des classes de recouvrements, pour la relation d'équivalence précédente; c'est un ensemble ordonné filtrant. $^{2}$

Si $\mathfrak{U}<\mathfrak{B}$, nous avons défini à la fin du $\mathrm{n}^{\circ}$ précédent un homomorphisme bien déterminé $\sigma(\mathfrak{U}, \mathfrak{B}): H^{q}(\mathfrak{B}, \mathfrak{F}) \rightarrow H^{q}(\mathfrak{U}, \mathfrak{F})$, défini pour tout entier $q \geqq 0$ et tout faisceau $\mathcal{F}$ sur $X$. Il est clair que $\sigma(\mathfrak{U}, \mathfrak{U})$ est l'identité, et que $\sigma(\mathfrak{U}, \mathfrak{B}) \circ \sigma(\mathfrak{B}, \mathfrak{B})=$ $\sigma(\mathfrak{U}, \mathfrak{W})$ si $\mathfrak{U} \prec \mathfrak{B} \prec \mathfrak{W}$. Il s'ensuit que, si $\mathfrak{U}$ est équivalent à $\mathfrak{Y}, \sigma(\mathfrak{U}, \mathfrak{B})$ et $\sigma(\mathfrak{V}, \mathfrak{l})$ sont des isomorphismes réciproques; autrement dit, $H^{q}(\mathfrak{M}, \mathfrak{F})$ ne dépend que de la classe du recouvrement $\mathfrak{l}$.

Défivition. On appelle g-ème groupe de cohomologie de $X$ a valeurs dans le faisceau $\mathfrak{F}$, et on note $H^{q}(X, \mathfrak{})$, la limite inductive des groupes $H^{q}(\mathfrak{U}, \mathfrak{F})$, définie suivant l'ordonné filtrant des classes de recouvrements de $X$ au moyen des homomorphismes $\sigma(\mathfrak{U}, \mathfrak{B})$.

En d'autres termes, un élément de $H^{q}(X, \mathfrak{F})$ n'est pas autre chose qu'un couple $(\mathfrak{U}, x)$, avec $x \in H^{q}(\mathfrak{U}, \mathfrak{F})$, en convenant d'identifier deux couples $(\mathfrak{l}, x)$ et $(\mathfrak{B}, y)$ s'il existe $\mathfrak{W}$, avec $\mathfrak{W} \prec \mathfrak{U}, \mathfrak{W} \prec \mathfrak{B}$, et $\sigma(\mathfrak{B}, \mathfrak{l})(x)=\sigma(\mathfrak{W}, \mathfrak{B})(y)$ dans $H^{q}(\mathfrak{F}, \mathfrak{F})$. A tout recouvrement $\mathfrak{l}$ de $X$ est donc associé un homomorphisme canonique $\sigma(\mathfrak{U}): H^{q}(\mathfrak{U}, \mathfrak{F}) \rightarrow H^{q}(X, \mathfrak{F})$.

On observera que $H^{q}(X, \mathfrak{})$ peut également être défini comme la limite inductive des $H^{q}(\mathfrak{l}, \mathfrak{F})$ suivant une famille cofinale de recouvrements $\mathfrak{l}$. Ainsi, si $X$ est quasi-compact (resp. quasi-paracompact), on peut se borner à considérer les recouvrements finis (resp. localement finis).

Lorsque $q=0$, on a, en appliquant la Proposition 1 :

Proposition 4. $H^{0}(X, \mathfrak{F})=\Gamma(X, \mathfrak{F})$

23. Homomorphismes de faisceaux. Soit $\varphi$ un homomorphisme d'un faisceau $\mathfrak{F}$ dans un faisceau $\mathrm{S}$. Si $\mathcal{l}$ est un recouvrement de $X$, faisons correspondre à tout $f \in C^{q}(\mathfrak{u}, \mathfrak{F})$ l'élément $\varphi f \in C^{q}(\mathfrak{U}, \mathcal{G})$ défini par la formule $(\varphi f)_{s}=\varphi\left(f_{s}\right) .$ L'application $f \rightarrow \varphi f$ est un homomorphisme de $C(\mathcal{l}, \mathfrak{F})$ dans $C(\mathfrak{l}, \mathcal{G})$ qui commute avec le cobord, donc qui définit des homomorphismes $\varphi^{*}: H^{q}(\mathfrak{H}, \mathfrak{F}) \rightarrow H^{q}(\mathfrak{U}, \mathcal{G})$. On a $\varphi^{*} \circ \sigma(\mathfrak{l}, \mathfrak{B})=\sigma(\mathfrak{U}, \mathfrak{B}) \circ \varphi^{*}$, d'où, par passage à la limite, des homomorphisme $_{\mathrm{S}}$

$$
\varphi^{*}: H^{q}(X, \Im) \rightarrow H^{q}(X, \mathcal{S})
$$

${ }^{2}$ Par contre, on ne pourrait parler de "l'ensemble" de tous les recouvrements, puisqu'un recouvrement est une famille dont l'ensemble d'indices est arbitraıre Lorsque $q=0, \varphi^{*}$ coincide avec l'homomorphisme de $\Gamma(X, \mathcal{F})$ dans $\Gamma(X, \mathcal{S})$ défini de façon naturelle par $\varphi$.

Dans le cas général, les homomorphismes $\varphi^{*}$ jouissent des propriétés formelles usuelles:

$$
(\varphi+\psi)^{*}=\varphi^{*}+\psi^{*}, \quad(\varphi \circ \psi)^{*}=\varphi^{*} \circ \psi^{*}, \quad 1^{*}=1
$$

En d'autres termes, pour tout $q \geq 0, H^{q}(X, \mathscr{})$ est un foncteur covariant additif de $\mathfrak{F} .$ Il en résulte notamment que, si $\mathfrak{F}$ est somme directe de deux faisceaux $\mathcal{S}_{1}$ et $\mathcal{S}_{2}, H^{q}(X, \mathfrak{F})$ est somme directe de $H^{q}\left(X, \mathrm{~S}_{1}\right)$ et de $H^{Q}\left(X, \mathrm{~S}_{2}\right)$.

Supposons que $\mathfrak{F}$ soit un faisceau de $Q$-modules. Toute section du faisceau $Q$ sur $X$ tout entier définit un endomorphisme de $\mathfrak{F}$, donc de chacun des $H^{q}(X, \mathfrak{F})$. Il s'ensuit que les $H^{q}(X, \mathcal{F})$ sont des modules sur l'anneau $\Gamma(X, \mathbb{a})$

24. Suite exacte de faisceaux: cas général. Soit $0 \rightarrow Q \stackrel{\alpha}{\rightarrow} B \stackrel{\beta}{\rightarrow}$ e $\rightarrow 0$ une suite exacte de faisceaux. Si $\mathfrak{l}$ est un recouvrement de $X$, la suite

$$
0 \rightarrow C(\mathfrak{l}, \alpha) \stackrel{\alpha}{\rightarrow} C(\mathfrak{l}, \mathbb{B}) \stackrel{\beta}{\rightarrow} C(\mathfrak{u}, \mathcal{C})
$$

est évidemment exacte, mais l'homomorphisme $\beta$ n'est pas en général surjectif. Désignons par $C_{0}(\mathfrak{l}, \mathcal{C})$ Pimage de cet homomorphisme; c'est un sous-complexe de $C(\mathfrak{U}, \mathfrak{e})$ dont les groupes de cohomologie seront notés $H_{0}^{q}(\mathfrak{l}, \mathcal{C})$. La suite exacte de complexes:

$$
0 \rightarrow C(\mathfrak{U}, Q) \rightarrow C(\mathfrak{U}, \mathbb{B}) \rightarrow C_{0}(\mathfrak{U}, \mathcal{C}) \rightarrow 0
$$

donne naissance a une suite exacte de cohomologie:

$$
\cdots \rightarrow H^{q}(\mathfrak{U}, \beta) \rightarrow H_{0}^{q}(\mathfrak{u}, \mathcal{e}) \stackrel{d}{\rightarrow} H^{q+1}(\mathfrak{l}, \alpha) \rightarrow H^{q+1}(\mathfrak{l}, \mathbb{B}) \rightarrow \cdots,
$$

où l'opérateur cobord $d$ est défini à la façon habituelle.

Soient maintenant $\mathfrak{U}=\left\{U_{i}\right\}_{i \in I}$ et $\mathfrak{B}=\left\{V_{j}\right\}_{j \epsilon J}$ deux recouvrements, et soit $\tau: I \rightarrow J$ une application telle que $U_{i} \subset V_{r i} ;$ on a donc $\mathfrak{U}<\mathfrak{B} .$ Le diagramme commutatif:

$$
\begin{array}{cc}
    0 \rightarrow C(\mathfrak{B}, \alpha) \rightarrow C(\mathfrak{B}, \mathbb{B}) \rightarrow C(\mathfrak{B}, \mathcal{C}) \\
    \tau \downarrow & \tau \downarrow & \tau \downarrow \\
    0 \rightarrow C(\mathfrak{u}, \alpha) \rightarrow C(\mathfrak{l}, \infty) \rightarrow C(\mathfrak{l}, \mathcal{C})
\end{array}
$$

montre que $\tau$ applique $C_{0}(\mathfrak{B}, \mathcal{C})$ dans $C_{0}(\mathfrak{U}, \mathcal{C})$, donc définit des homomorphismes $\tau^{*}: H_{0}^{q}(\mathfrak{M}, \mathcal{C}) \rightarrow H_{0}^{\alpha}(\mathcal{U}, \mathcal{C}) .$ De plus, les homomorphismes $\tau^{*}$ sont indépendants du choix de l'application $\tau^{*}:$ cela provient $d u$ fait que, si $f \in C_{0}^{q}(\mathscr{B}, \mathcal{C})$, on a $k f \in C_{0}^{q-1}(\mathfrak{U}, \mathcal{e})$, avec les notations de la démonstration de la Proposition 3. On a ainsi obtenu des homomorphismes canoniques $\sigma(\mathfrak{U}, \mathfrak{B}): H_{0}^{g}(\mathfrak{B}, \mathrm{C}) \rightarrow H_{0}^{q}(\mathfrak{M}, \mathcal{C})$; on neut alors définir $H_{0}^{q}(X . \mathcal{C})$ comme la limite inductive suivant $\mathfrak{l l}$ des groupes $H_{0}^{q}(\mathfrak{U}, \mathcal{C})$

% ----------------------------------------------------------------------------------------------------------------------------------------------------------------
% ----------------------------------------------------------------------------------------------------------------------------------------------------------------
% ----------------------------------------------------------------------------------------------------------------------------------------------------------------
% ----------------------------------------------------------------------------------------------------------------------------------------------------------------

Comme une limite inductive de suites exactes est une suite exacte (cf. [7], Chap. VIII, th. 5.4), on a:

Proposition $5 .$ La suite

$$
\cdots \rightarrow H^{q}(X, \mathbb{B}) \stackrel{\beta^{*}}{\rightarrow} H_{0}^{q}(X, \mathcal{C}) \stackrel{d}{\rightarrow} H^{q+1}(X, \Theta) \stackrel{\alpha^{*}}{\rightarrow} H^{q+1}(X, \mathbb{B}) \rightarrow \cdots
$$

est exacte.

( $d$ désignant l'homomorphisme obtenu par passage à la limite a partir de $\left.d: H_{0}^{q}(\mathfrak{l l}, \mathcal{C}) \rightarrow H^{q+1}(\mathfrak{l}, \alpha)\right)$

Pour pouvoir appliquer la Proposition précédente, il est commode de comparer les groupes $H_{0}^{q}(X, \mathcal{C})$ et $H^{q}(X, \mathcal{C}) .$ L'injection de $C_{0}(\mathfrak{u}, \mathrm{C})$ dans $C(\mathfrak{u}, \mathfrak{e})$ définit des homomorphismes $H_{0}^{q}(\mathfrak{U}, \mathrm{e}) \rightarrow H^{q}(\mathfrak{U}, \mathcal{C})$, d'où, par passage à la limite sur $\mathfrak{U}$, des homomorphismes:

$$
H_{0}^{q}(X, \mathcal{C}) \rightarrow H^{q}(X, \mathcal{C})
$$

Proposition 6. L'homomorphisme canonique $H_{0}^{q}(X, \mathcal{C}) \rightarrow H^{q}(X, \mathcal{C})$ est bijectif pour $q=0$ et injectif pour $q=1$

Démontrons d'abord un lemme:

LEMME 1. Soit $\mathfrak{B}=\left\{V_{j}\right\}_{j \in J}$ un recouvrement, et soit $f=\left(f_{i}\right)$ un élément de $C^{0}(\mathfrak{B}, \mathcal{e}) .$ Il existe un recouvrement $\mathfrak{l}=\left\{U_{i}\right\}_{\text {ieI }}$ et une application $\tau: I \rightarrow J$ tels que $U_{i} \subset V_{r i}$ et que $\tau f \in C_{0}^{0}(\mathfrak{l l}, \mathcal{C})$

Pour tout $x \in X$, choisissons $\tau x \epsilon J$ tel que $x \in V_{\tau x} .$ Puisque $f_{\tau x}$ est une section de $\mathcal{C}$ au-dessus de $V_{\tau x}$, il existe un voisinage ouvert $U_{x}$ de $x$, contenu dans $V_{r x}$, et une section $b_{x}$ de $\mathbb{B}$ au-dessus de $U_{x}$ tels que $\beta\left(b_{x}\right)=f_{\tau x}$ sur $U_{x} .$ Les $\left\{U_{x}\right\}_{x \in x}$ forment un recouvrement $\mathfrak{U}$ de $X$, et les $b_{x}$ forment une 0-cochaîne $b$ de $\mathfrak{l}$ à valeurs dans $\mathbb{B}$; comme $\tau f=\beta(b)$, on a bien $\tau f \in C_{0}^{0}(\mathfrak{l}, \mathcal{C}) .$

Montrons maintenant que $H_{0}^{1}(X, \mathcal{C}) \rightarrow H^{1}(X, \mathcal{C})$ est injectif. Un élément du noyau de cette application peut être représenté par un 1-cocycle $z=\left(z_{j_{0} j_{1}}\right) \epsilon$ $C_{0}^{\prime}(\mathfrak{B}, \mathcal{C})$ tel qu'il existe $f=\left(f_{j}\right) \in C^{0}(\mathfrak{B}, \mathcal{C})$ avec $d f=z$; appliquant le Lemme 1 à $f$, on trouve un recouvrement $\mathfrak{u}$ tel que $\tau f \in C_{0}^{0}(\mathfrak{l}, \mathcal{C})$, ce qui entraîne que $\tau z$ est cohomologue à dans $C_{0}(\mathfrak{l}, \mathcal{C})$, donc a pour image 0 dans $H_{0}^{1}(X, \mathrm{C}) .$ On démontre de même que $H_{0}^{0}(X, \mathcal{C}) \rightarrow H^{0}(X, \mathcal{C})$ est bijectif.

CorolLAIRE $1 .$ On a une suite exacte:

$$
0 \rightarrow H^{0}(X, \alpha) \rightarrow H^{0}(X, \mathbb{B}) \rightarrow H^{0}(X, \mathcal{C}) \rightarrow H^{1}(X, \alpha) \rightarrow H^{1}(X, \mathbb{B}) \rightarrow H^{1}(X, \mathcal{C})
$$

C'est une conséquence immédiate des Propositions 5 et 6 .

ConolLalRe 2. Si $H^{1}(X, Q)=0$, alors $\Gamma(X, B) \rightarrow \Gamma(X, \mathcal{C})$ est surjectif.

25. Suite exacte de faisceaux : cas où $X$ est paracompact. Rappelons qu'un espace $X$ est dit paracompact s'il est séparé et si tout recouvrement de $X$ admet un recouvrement localement fini plus fin. Pour un tel espace, on peut étendre la Proposition 6 à toute valeur de $q$ (j'ignore si une telle extension est possible pour des espaces non séparés): Proposition 7. Si $X$ est paracompact, l'homomorphisme canonique

$$
H_{0}^{q}(X, \mathcal{C}) \rightarrow H^{q}(X, \mathcal{C})
$$

est bijectif pour tout $q \geqq 0$

La Proposition est une conséquence immédiate du lemme suivant, analogue au Lemme 1 :

LEMME 2 . Soit $\mathfrak{B}=\left\{V_{j}\right\}_{j \epsilon J}$ un recouvrement, et soit $f=\left(f_{j_{0} \ldots j_{q}}\right)$ un élément de $C^{q}(\mathfrak{B}, \mathcal{C}) .$ Il existe un recouvrement $\mathfrak{U}=\left\{U_{i}\right\}_{i \epsilon I}$ et une application $\tau: I \rightarrow J$ tels que $U_{i} \subset V_{\text {тi }}$ et que $\tau f \in C_{0}^{q}(\mathfrak{U}, \mathcal{C})$.

Puisque $X$ est paracompact, on peut supposer $\mathfrak{B}$ localement fini. Il existe alors un recouvrement $\left\{W_{j}\right\}_{j e J}$ tel que $\bar{W}_{j} \subset V_{j} .$ Pour tout $x \in X$, choisissons un voisinage ouvert $U_{x}$ de $x$ tel que:

(a) Si $x \in V_{j}\left(\right.$ resp. $\left.x \in W_{j}\right)$, on a $U_{x} \subset V_{j}\left(\right.$ resp. $\left.U_{x} \subset W_{j}\right)$

(b) Si $U_{x} \cap W_{j} \neq \emptyset$, on a $U_{x} \subset V_{j}$,

(c) Si $x \in V_{j_{0} \ldots j_{q}}$, il existe une section $b$ de $B$ au-dessus de $U_{x}$ telle que $\beta(b)=f_{j_{0} \ldots j_{q}}$ au-dessus de $U_{x}$.

La condition (c) est réalisable, vu la définition d'un faisceau quotient, et le fait que $x$ n'appartient qu'à un nombre fini d'ensembles $V_{j_{0} \ldots j_{q}} .$ Une fois (c) vérifiée, il suffit de restreindre convenablement $U_{x}$ pour satisfaire à (a) et (b).

La famille des $\left\{U_{x}\right\}_{x e X}$ forme un recouvrement $\mathfrak{H} ;$ pour tout $x \in X$, choisissons $\tau x \in J$ tel que $x \in W_{\tau x}$. Vérifions maintenant que $\tau f$ appartient à $C_{0}^{q}(U, \mathcal{C})$, autrement dit que $f_{\tau x_{0}, \tau x_{n}}$ est image par $\beta$ d'une section de $B$ au-dessus de $U_{x_{0}} \cap \cdots \operatorname{n} U_{x_{q}} .$ Si $U_{x_{0}} \cap \cdots$ n $U_{x_{q}}$ est vide, c'est évident; sinon, on a $U_{x_{0}}$ ก $U_{x_{k}} \neq \emptyset$ pour $0 \leqq k \leqq q$, et comme $U_{x_{k}} \subset W_{\tau x_{k}}$, on a $U_{x_{0}}$ ก $W_{\tau x_{k}} \neq \emptyset$ ce qui entraîne d'après (b) que $U_{x_{0}} \subset V_{\tau x_{k}}$, d'où $x_{0} \in V_{r x_{0} \cdots \tau x_{g}} ;$ appliquant alors (c), on voit qu'il existe une section $b$ de $\mathbb{B}$ au-dessus de $U_{x_{0}}$ telle que $\beta(b)_{x}=$ $f_{r x_{0} \cdots r x_{q}}$ au-dessus de $U_{x_{0}}$, donc a fortiori au-dessus de $U_{x_{0}} \cap \cdots \cap U_{x_{q}}$, ce qui achève la démonstration.

CorolLaIre. Si $X$ est paracompact, on a une suite exacte:

$$
\cdots \rightarrow H^{q}(X, \beta) \stackrel{\beta^{*}}{\rightarrow} H^{q}(X, \mathcal{C}) \stackrel{d}{\rightarrow} H^{q+1}(X, \propto) \stackrel{\alpha^{*}}{\rightarrow} H^{q+1}(X, \mathbb{B}) \rightarrow \cdots
$$

(l'opérateur $d$ étant défini comme le composé de l'isomorphisme réciproque de $H_{0}^{q}(X, \mathcal{C}) \rightarrow H^{q}(X, \mathcal{C})$ et de $\left.d: H_{0}^{q}(X, \mathcal{C}) \rightarrow H^{q+1}(X, \alpha)\right) .$

La suite exacte précédente est appelée la suite exacte de cohomologie définie par la suite exacte de faisceaux $0 \rightarrow a \rightarrow B \rightarrow C \rightarrow 0$ donnée. Elle vaut, plus généralement, chaque fois que l'on peut démontrer que $H_{0}^{q}(X, \mathcal{C}) \rightarrow H^{q}(X, \mathcal{C})$ est bijectif (nous verrons au $\mathrm{n}^{\circ} 47$ que c'est le cas lorsque $X$ est une variété algébrique et que $Q$ est un faisceau algébrique cohérent).

26. Cohomologie d'un sous-espace fermé. Soit $\mathfrak{F}$ un faisceau sur l'espace $X$, et soit $Y$ un sous-espace de $X$. Soit $\mathfrak{F}(Y)$ le faisceau induit par $\mathfrak{F}$ sur $Y$, au sens du $\mathrm{n}^{\circ} 4 .$ Si $\mathfrak{U}=\left\{U_{i}\right\}_{i \epsilon I}$ est un recouvrement de $X$, les $U_{i}^{\prime}=Y$ \cap $U_{i}$ forment un recouvrement $\mathfrak{u}^{\prime}$ de $Y ;$ si $f_{i_{0} \cdots i_{q}}$ est une section de $\mathfrak{F}$ au-dessus de $U_{i_{0} \cdots i_{q}}$, la restriction de $f_{i_{0} \cdots i_{q}}$ à $U_{i_{0} \cdots i_{q}}^{\prime}=Y$ ก $U_{i_{0} \cdots i_{q}}$ est une section de $\mathfrak{F}(Y)$. L'opération de restriction est un homomorphisme $\rho: C(\mathfrak{l}, \mathfrak{F}) \rightarrow C\left(\mathfrak{l}^{\prime}, \mathfrak{F}(Y)\right)$, commutant avec $d$, donc définissant $\rho^{*}: H^{q}(\mathfrak{l}, \mathfrak{F}) \rightarrow H^{q}\left(\mathfrak{l}^{\prime}, \mathfrak{F}(Y)\right)$. Si $\mathfrak{U} \prec \mathfrak{B}$, on a $\mathfrak{u}^{\prime} \prec \mathfrak{B}^{\prime}$, et $\rho^{*} \circ \sigma(\mathfrak{M}, \mathfrak{B})=\sigma\left(\mathfrak{H}^{\prime}, \mathfrak{B}^{\prime}\right) \circ \rho^{*}$; donc les homomorphismes $\rho^{*}$ définissent, par passage à la limite sur $\mathfrak{U}$, des homomorphismes $\rho^{*}: H^{q}(X, \mathcal{F}) \rightarrow H^{q}(Y, \mathfrak{F}(Y))$.

Proposition 8. Supposons que $Y$ soit fermé dans $X$, et que $\mathfrak{F}$ soit nul en dehors de $Y .$ Alors $\rho^{*}: H^{q}(X, \mathfrak{}) \rightarrow H^{q}(Y, \mathcal{F}(Y))$ est bijectif pour tout $q \geqq 0$

La Proposition résulte des deux faits suivants:

(a) Tout recouvrement $\mathfrak{B}=\left\{W_{i}\right\}_{i \in I}$ de $Y$ est de la forme $\mathfrak{u}^{\prime}$, où $\mathfrak{l}$ est un recouvrement de $X$.

En effet, il suffit de poser $U_{i}=W_{i}$ u $(X-Y)$, puisque $Y$ est fermé dans $X$.

(b) Pour tout recouvrement $\mathfrak{l}$ de $X, \rho: C(\mathfrak{l}, \mathfrak{F}) \rightarrow C\left(\mathfrak{l}^{\prime}, \mathfrak{F}(Y)\right)$ est bijectif.

En effet, cela résulte de la Proposition 5 du $\mathrm{n}^{\circ} 5$, appliquée a $U_{i_{0}} \ldots i_{q}$ et au faisceau $\mathcal{F} .$

On peut aussi exprimer la Proposition 8 de la manière suivante: Si $S$ est un faisceau sur $Y$, et si $\mathcal{}^{X}$ est le faisceau obtenu en prolongeant $\mathcal{g a r} 0$ en dehors de $Y$, on a $H^{q}(Y, \mathcal{G})=H^{q}\left(X, \mathcal{}^{X}\right)$ pour tout $q \geqq 0 ;$ autrement dit, l'identification de $\mathcal{S}$ avec $\mathcal{}^{X}$ est compatible avec le passage aux groupes de cohomologie.

\section{§4. Comparaison des groupes de cohomologie de recouvrements différents}

Dans ce paragraphe, $X$ désigne un espace topologique, et $\mathcal{F}$ un faisceau sur $X .$ On se propose de donner des conditions, portant sur un recouvrement $\mathfrak{l}$ de $X$, pour que $H^{n}(\mathfrak{U}, \mathfrak{F})=H^{n}(X, \mathfrak{F})$ pour tout $n \geqq 0$.

27. Complexes doubles. Un complexe double (cf. [6], Chap. IV, §4) est un groupe abélien bigradué

$$
K=\sum_{p, q} K^{p, q}, \quad p \geqq 0, q \geqq 0
$$

muni de deux endomorphismes $d^{\prime}$ et $d^{\prime \prime}$ vérifiant les propriétés suivantes:

$$
\left\{\begin{array}{l}
    d^{\prime} \text { applique } K^{p, q} \text { dans } K^{p+1, q} \text { et } d^{\prime \prime} \text { applique } K^{p, q} \text { dans } K^{p, q+1}, \\
    d^{\prime} \circ d^{\prime}=0, d^{\prime} \circ d^{\prime \prime}+d^{\prime \prime} \circ d^{\prime}=0, d^{\prime \prime} \circ d^{\prime \prime}=0
\end{array}\right.
$$

Un élément de $K^{p, q}$ est dit bihomogène, de bidegré $(p, q)$, et de degré total $p+q .$ L'endomorphisme $d=d^{\prime}+d^{\prime \prime}$ vérifie $d \circ d=0$, et les groupes de cohomologie de $K$, muni de cet opérateur cobord, seront notés $H^{n}(K), n$ désignant le degré total

On peut également munir $K$ de l'opérateur de cobord $d^{\prime} ;$ comme $d^{\prime}$ est compatible avec la bigraduation de $K$, on obtient ainsi des groupes de cohomologie, notés $H_{I}^{p, q}(K) ;$ avec $d^{\prime \prime}$, on a les groupes $H_{I I}^{p, q}(K)$.

Nous désignerons par $K_{I I}^{q}$ le sous-groupe de $K^{0, q}$ formé des éléments $x$ tels que $d^{\prime}(x)=0$, et par $K_{I I}$ la somme directe des $K_{I I}^{q}(q=0,1, \cdots) .$ Définition analogue pour $K_{I}=\sum_{p=0}^{\infty} K_{I}^{p} .$ On notera que

$$
K_{I I}^{q}=H_{I}^{0, q}(K) \quad \text { et } \quad K_{I}^{p}=H_{I I}^{p, 0}(K)
$$

$K_{I I}$ est un sous-complexe de $K$, et l'opérateur $d$ coincide sur $K_{I I}$ avec l'opérateur $d^{\prime \prime}$. Proposition 1. Si $H_{I}^{p, q}(K)=0$ pour $p>0$ et $q \geqq 0$, l'injection $K_{I I} \rightarrow K$ définit une bijection de $H^{n}\left(K_{I I}\right)$ sur $H^{n}(K)$ pour tout $n \geqq 0$.

(Cf. [4], exposé XVII-6, dont nous reproduisons ci-dessous la démonstration).

Remplaçant $K$ par $K / K_{I I}$, on est ramené à démontrer que, si $H_{I}^{p, q}(K)=0$ pour $p \geqq 0$ et $q \geqq 0$, alors $H^{n}(K)=0$ pour tout $n \geqq 0 .$ Posons

$$
K_{h}=\sum_{q \geq h} K^{p, q}
$$

Les $K_{h}(h=0,1, \cdots)$ sont des sous-complexes emboîtés de $K$, et $K_{h} / K_{h+1}$ est isomorphe à $\sum_{p=0}^{\infty} K^{p, h}$, muni de l'opérateur cobord $d^{\prime}$. On a donc $H^{n}\left(K_{h} / K_{h+1}\right)=$ $H_{I}^{h, n-h}(K)=0$ quels que soient $n$ et $h$, d'où $H^{n}\left(K_{h}\right)=H^{n}\left(K_{h+1}\right)$. Comme $H^{n}\left(K_{h}\right)=0$ si $h>n$, on en déduit, par récurrence descendante sur $h$, que $H^{n}\left(K_{h}\right)=0$ quels que soient $n$ et $h$, et comme $K_{0}$ est égal à $K$, la Proposition est démontrée.

28. Complexe double défini par deux recouvrements. Soient $\mathfrak{l}=\left\{U_{i}\right\}_{\text {ieI }}$ et $\mathfrak{B}=\left\{V_{j}\right\}_{j e J}$ deux recouvrements de $X .$ Si $s$ est un $p$-simplexe de $S(I)$, et $s^{\prime}$ un $q$-simplexe de $S(J)$, nous désignerons par $U_{s}$ l'intersection des $U_{i}, i \in s$ (cf. $\mathrm{n}^{\circ}$ 18), par $V_{\mathfrak{s}^{\prime}}$ l'intersection des $V_{i}, j \in s^{\prime}$, par $\mathfrak{B}_{s}$ le recouvrement de $U_{3}$ formé par les $\left\{U_{s} \cap V_{j}\right\}_{j \in J}$, et par $\mathfrak{U}_{s^{\prime}}$ le recouvrement de $V_{s^{\prime}}$ formé par les $\left\{V_{s^{\prime}} \cap U_{i}\right\}_{i \in I}$

Définissons un complexe double $C(\mathfrak{U}, \mathfrak{B} ; \mathfrak{F})=\sum_{p \cdot q} C^{p, q}(\mathfrak{U}, \mathfrak{B} ; \mathfrak{F})$ de la façon suivante:

$C^{p, q}(\mathfrak{U}, \mathfrak{B} ; \mathfrak{F})=\prod \Gamma\left(U_{\mathrm{s}} \operatorname{n} V_{s^{\prime}}, \mathfrak{F}\right)$, le produit étant étendu a tous les couples $\left(s, s^{\prime}\right)$ où $s$ est un simplexe de dimension $p$ de $S(I)$ et $s^{\prime}$ un simplexe de dimension $q$ de $S(J)$

Un élément $f \in C^{p, q}(\mathfrak{l}, \mathfrak{B} ; \mathfrak{F})$ est donc un système $\left(f_{s, s^{\prime}}\right)$ de sections de $\mathfrak{F}$ sur les $U_{s}$ n $V_{s^{\prime}}$, ou encore, avec les notations du $n^{\circ} 18$, c'est un système

$$
f_{i_{0} \ldots i_{p}, j_{0} \cdots j q} \in \Gamma\left(U_{i_{0} \cdots i_{p}} \cap V_{j_{0} \cdots j_{q}}, \mathfrak{F}\right)
$$

On peut aussi identifier $C^{p, q}(\mathfrak{U}, \mathfrak{B} ; \mathcal{F})$ avec $\prod_{\varepsilon^{\prime}} C^{p}\left(\mathfrak{l}_{s^{\prime}}, \mathscr{}\right) ;$ comme, pour chaque $s^{\prime}$, on a une opération de cobord $d: C^{p}\left(\mathfrak{U}_{\varepsilon^{\prime}}, \mathcal{F}\right) \rightarrow C^{p+1}\left(\mathfrak{l}_{s^{\prime}}, \mathfrak{F}\right)$ on en déduit un homomorphisme

$$
d_{\mathfrak{u}}: C^{p, q}(\mathfrak{U}, \mathfrak{V} ; \mathfrak{F}) \rightarrow C^{p+1, q}(\mathfrak{U}, \mathfrak{B} ; \mathfrak{F})
$$

En explicitant la définition de $d_{\mathrm{u}}$, on obtient:

$$
\left(d_{\mathrm{u}} f\right)_{i_{0} \cdots i_{p+1}, j_{0} \cdots j_{q}}=\sum_{k=0}^{k=p+1}(-1)^{k}{ }_{k_{k}}\left(f_{i_{0} \ldots i_{k}} \ldots_{i_{p+1}, j_{0} \cdots j_{q}}\right)
$$

$\rho_{k}$ désignant l'homomorphisme de restriction défini par l'inclusion de

$$
U_{i_{0} \cdots i_{p+1}} \text { ก } V_{j_{0} \cdots j_{q}} \quad \text { dans } U_{i_{0} \cdots i_{k} \ldots i_{p+1}} \text { ก } V_{j_{0} \cdots j_{q}}
$$

On définit de même $d_{\mathfrak{\Re}}: C^{p, q}(\mathfrak{U}, \mathfrak{Y} ; \mathfrak{F}) \rightarrow C^{p, q+1}(\mathfrak{U}, \mathfrak{V} ; \mathfrak{F})$ et l'on a :

$$
\left(d_{\mathfrak{B}} f\right)_{i_{0} \cdots i_{p}, j_{0} \cdots j_{q+1}}=\sum_{h=q+1}^{h=q+1}(-1)^{h}{ }_{\rho_{h}}\left(f_{i_{0} \cdots i_{p}, j_{0} \cdots j_{h} \cdots j_{q+1}}\right)
$$

Il est clair que $d_{\mathfrak{u}} \circ d_{\mathrm{u}}=0, d_{\mathfrak{u}} \circ d_{\mathfrak{B}}=d_{\mathfrak{B}} \circ d_{\mathfrak{u}}, d_{\mathfrak{B}} \circ d_{\mathcal{B}}=0$. En posant alors $d^{\prime}=d_{\mathfrak{u}}, d^{\prime \prime}=(-1)^{p} d_{\mathfrak{}}$, on munit $C(\mathfrak{u}, \mathfrak{B} ; \mathfrak{F})$ d'une structure de double complexe. On peut donc appliquer à $K=C(\mathfrak{U}, \mathfrak{Y} ; \mathfrak{F})$ les définitions du $\mathrm{n}^{\circ}$ précédent; les groupes ou complexes désignés dans le cas général par $H^{n}(K)$, $H_{I}^{p, q}(K), \quad H_{I I}^{p, q}(K), K_{I}, K_{I I}$, seront notés ici $H^{n}(\mathfrak{U}, \mathfrak{B} ; \mathfrak{F}), H_{I}^{p, q}(\mathfrak{U}, \mathfrak{B} ; \mathfrak{F})$, $H_{I}^{p, q}(\mathcal{U}, \mathfrak{B} ; \mathfrak{F}), C_{I}(\mathfrak{U}, \mathfrak{B} ; \mathfrak{F})$ et $C_{I I}(\mathfrak{U}, \mathfrak{B} ; \mathfrak{F})$ respectivement.

Vu les définitions de $d^{\prime}$ et $d^{\prime \prime}$, on a immédiatement:

Proposition 2. $H_{I}^{p, q}(\mathfrak{l}, \mathfrak{B} ; \mathfrak{F})$ est isomorphe $\dot{a} \prod_{s^{\prime}} H^{p}\left(\mathfrak{l}_{s^{\prime}}, \mathfrak{F}\right)$, le produit étant étendu à tous les simplexes de dimension q de S(J). En particulier,

$$
C_{I I}^{q}(\mathfrak{U}, \mathfrak{B} ; \mathfrak{F})=H_{I}^{0, q}(\mathfrak{l}, \mathfrak{B} ; \mathfrak{F})
$$

est isomorphe $\dot{a} \prod_{s^{\prime}} H^{0}\left(\mathfrak{u}_{s^{\prime}}, \mathcal{F}\right)=C^{q}(\mathfrak{B}, \mathfrak{F}) .$

Nous noterons $\iota^{\prime \prime}$ l'isomorphisme canonique: $C(\mathfrak{B}, \mathfrak{F}) \rightarrow C_{I I}(\mathfrak{l}, \mathfrak{B} ; \mathfrak{F}) . \mathrm{Si}$ $\left(f_{j_{0} \cdots j_{q}}\right)$ est un élément de $C^{q}(\mathfrak{B}, \mathfrak{F})$, on a donc:

$$
\left(\iota^{\prime \prime} f\right)_{i_{0} \cdot j_{0} \cdots j_{q}}=\rho_{i_{0}}\left(f_{j_{0} \cdots j_{q}}\right)
$$

$\rho_{i_{0}}$ désignant l'homomorphisme de restriction défini par l'inclusion de

$$
U_{i_{0}} \cap V_{j_{0} \cdots j_{q}} \quad \text { dans } V_{j_{0} \cdots j_{q}}
$$

Bien entendu, un résultat analogue à la Proposition 2 vaut pour $H_{H i}^{p, q}(\mathfrak{l}, \mathfrak{B} ; \mathfrak{F})$, et l'on a un isomorphisme $\iota^{\prime}: C(\mathfrak{U}, \mathfrak{F}) \rightarrow C_{I}(\mathfrak{l}, \mathfrak{B} ; \mathfrak{F})$.

29. Applications. Les notations étant celles du $\mathrm{n}^{\circ}$ précédent, on a:

Proposition 3. Supposons que $H^{p}\left(\mathfrak{l}_{s^{\prime}}, \mathfrak{F}\right)=0$ pour tout $s^{\prime} c t$ tout $p>0$. Alors l'homomorphisme $H^{n}(\mathfrak{B}, \mathfrak{F}) \rightarrow H^{n}(\mathfrak{l}, \mathfrak{B} ; \mathfrak{F})$, défini par \iota", est bijectif pour tout $n \geqq 0$.

C'est une conséquence immédiate des Propositions 1 et 2 .

Avant d'énoncer la Proposition 4, démontrons un lemme:

LEmME 1. Soit $\mathfrak{W}=\left\{W_{i}\right\}_{i \in I}$ un recouvrement d'un espace $Y$, et soit $\mathfrak{F}$ un faisceau sur $Y$. S'il existe $i \in I$ tel que $W_{i}=Y$, alors $H^{p}(\mathfrak{B}, \mathfrak{F})=0$ pour tout $p>0$.

Soit $\mathfrak{M}^{\prime}$ le recouvrement de $Y$ formé de l'unique ouvert $Y$; on a évidemment $\mathfrak{W} \prec \mathfrak{B}^{\prime}$, et l'hypothèse faite sur $\mathfrak{W}$ signifie que $\mathfrak{B}^{\prime} \prec \mathfrak{W}$. Il en résulte $\left(\mathrm{n}^{\circ} 22\right)$ que $H^{p}(\mathfrak{W}, \mathfrak{F})=H^{p}\left(\mathfrak{W}^{\prime}, \mathfrak{F}\right)=0$ si $p>0$.

Proposition 4. Supposons que le recouvrement $\mathfrak{B}$ soit plus fin que le recouvrement ll. Alors $\iota^{\prime \prime}: H^{n}(\mathfrak{P}, \mathfrak{F}) \rightarrow H^{n}($ U, $\mathfrak{B} ; \mathfrak{F})$ est bijectif pour tout $n \geqq 0 .$ De plus, l'homomorphisme $\iota^{\prime} \circ \iota^{\prime \prime-1}: H^{n}(\mathfrak{l}, \mathfrak{F}) \rightarrow H^{n}(\mathfrak{B}, \mathfrak{F})$ coincide avec l'homomorphisme $\sigma(\mathfrak{B}, \mathfrak{U}) d u \mathrm{n}^{\circ} 21$

En appliquant le Lemme 1 à $\mathfrak{B}=\mathfrak{U}_{s^{\prime}}$ et $Y=V_{s^{\prime}}$, on voit que $H^{p}\left(\mathfrak{U}_{s^{\prime}}, \mathfrak{F}\right)=0$ pour tout $p>0$, et la Proposition 3 montre alors que

$$
\iota^{\prime \prime}: H^{n}(\mathfrak{B}, \mathfrak{F}) \rightarrow H^{n}(\mathfrak{l}, \mathfrak{B} ; \mathfrak{F})
$$

est bijectif pour tout $n \geqq 0$.

Soit $\tau: J \rightarrow I$ une application telle que $V_{i} \subset U_{\tau j} ;$ pour démontrer la seconde partie de la Proposition, il nous faut faire voir que, si $f$ est un $n$-cocycle de $C(\mathbb{l}, \mathfrak{F})$, les cocycles $\iota^{\prime}(f)$ et $\iota^{\prime \prime}(\tau f)$ sont cohomologues dans $C(\mathfrak{l}, \mathfrak{B} ; \mathfrak{F})$. Pour tout entier $p, 0 \leqq p \leqq n-1$, définissons $g^{p} \epsilon C^{p, n-p-1}(\mathfrak{l}, \mathfrak{V} ; \mathfrak{})$ par la formule suivante:

$$
g_{i_{0} \cdots i_{p}, j_{0} \cdots j_{n-p-1}}^{p}=\rho_{p}\left(f_{i_{0} \cdots i_{p} r j_{0} \cdots \tau_{n-p}}\right)
$$

$\rho_{p}$ désignant l'homomorphisme de restriction défini par l'inclusion de

$$
U_{i_{0} \cdots i_{p}} \cap V_{j_{0} \cdots j_{n-p-1}} \quad \text { dans } U_{i_{0} \cdots i_{p} r j_{0} \cdots+j_{n-p-1}}
$$

On vérifie par un calcul direct (en tenant compte de ce que $f$ est un cocycle) que l'on a:

$$
d^{\prime \prime}\left(g^{0}\right)=\iota^{\prime \prime}(\tau f), \cdots, d^{\prime \prime}\left(g^{p}\right)=d^{\prime}\left(g^{p-1}\right), \cdots, d^{\prime}\left(g^{n-1}\right)=(-1)^{n} \iota^{\prime}(f)
$$

d'où $d\left(g^{0}-g^{1}+\cdots+(-1)^{n-1} g^{n-1}\right)=\iota^{\prime \prime}(\tau f)-\iota^{\prime}(f)$, ce qui montre bien que $\iota^{\prime \prime}(\tau f)$ et $\iota^{\prime}(f)$ sont cohomologues.

Proposition 5. Supposons que $\mathfrak{B}$ soit plus fin que ll, et que $H^{q}\left(\mathfrak{B}_{s}, \mathfrak{F}\right)=0$ pour tout $s$ et tout $q>0$. Alors l'homomorphisme $\sigma(\mathfrak{B}, \mathfrak{l}): H^{n}(\mathfrak{u}, \mathfrak{F}) \rightarrow H^{n}(\mathfrak{B}, \mathfrak{F})$ est bijectif pour tout $n \geqq 0$

Si l'on applique la Proposition 3 en permutant les rôles de $\mathfrak{l}$ et de $\mathfrak{B}$, on voit que $\iota^{\prime}: H^{n}(\mathfrak{B}, \mathfrak{F}) \rightarrow H^{n}(\mathfrak{l}, \mathfrak{B} ; \mathfrak{})$ est bijectif. La Proposition résulte alors directement de la Proposition $4 .$

ThÉoRème 1. Soient $X$ un espace topologique, $\mathfrak{l}=\left\{U_{i}\right\}_{\text {ieI }}$ un recouvrement de $X, \mathfrak{F}$ un faisceau sur $X$. Supposons qu'il existe une famille $\mathfrak{B}^{\prime}, \alpha \in A$, de recouvrements de $X$ vérifiant les deux conditions suivantes:

(a) Pour tout recouvrement $\mathfrak{W}$ de $X$, il existe $\alpha \in A$ tel que $\mathfrak{B}^{\alpha} \prec \mathfrak{W}$.

(b) $H^{q}\left(\mathfrak{B}_{s}^{\alpha}, \mathfrak{F}\right)=0$ pour tout $\alpha \in A$, tout simplexe s de $S(I)$, et tout $q>0$

Alors $\sigma(\mathfrak{U}): H^{n}(\mathfrak{l}, \mathfrak{F}) \rightarrow H^{n}(X, \mathfrak{F})$ est bijectif pour tout $n \geqq 0$.

Puisque les $\mathfrak{B}^{\alpha}$ sont arbitrairement fins, nous pouvons supposer qu'ils sont plus fins que $\mathfrak{l}$. En ce cas l'homomorphisme

$$
\sigma\left(\mathfrak{B}^{\alpha}, \mathfrak{l}\right): H^{n}(\mathfrak{l l}, \mathfrak{F}) \rightarrow H^{n}\left(\mathfrak{B}^{\alpha}, \mathfrak{F}\right)
$$

est bijectif pour tout $n \geqq 0$, d'après la Proposition $5 .$ Comme les $\mathfrak{V}^{\text {“ }}$ sont arbitrairement fins, $H^{n}(X, \mathfrak{F})$ est limite inductive des $H^{n}\left(\mathfrak{P}^{\alpha}, \mathfrak{F}\right)$, et le théorème résulte immédiatement de là.

RemarQues. (1) Il est probable que le Théorème 1 reste valable lorsqu'on remplace la condition (b) par la condition plus faible suivante:

$\left(\mathrm{b}^{\prime}\right) \lim _{\alpha} H^{q}\left(\mathfrak{B}_{8}^{\alpha}, \mathfrak{F}\right)=0$ pour tout simplexe $s$ de $S(I)$ et tout $q>0$

(2) Le Théorème 1 est analogue à un théorème de Leray sur les recouvrements acycliques. Cf. [10], ainsi que [4], exposé XV\Pi-7.

\section{CHAPITRE II. VARIÉTÉs ALGÉBRIQUES-FAISCEAUX ALGÉBRIQUES COHÉRENTS SUR LES VARIÉTÉS AFFINES}

Dans toute la suite de cet article $K$ désigne un corps commutatif algébriquement clos, de caractéristique quelconque. 

\section{§1. Variếtés algébriques}

30. Espaces vérifiant la condition (A). Soit $X$ un espace topologique. La condition (A) est la suivante:

(A)-Toute suite décroissante de parties fermées de $X$ est stationnaire.

Autrement dit, si l'on a $F_{1} \supset F_{2} \supset F_{3} \supset \cdots$, les $F_{i}$ étant fermés dans $X$, il existe un entier $n$ tel que $F_{m}=F_{n}$ pour $m \geqq n$. Ou encore:

(A^') - L'ensemble des parties fermées de $X$, ordonné par inclusion, vérifie la condition minimale.

ExEmPLes. Munissons un ensemble $X$ de la topologie où les sous-ensembles fermés sont les parties finies de $X$ et $X$ lui-même; la condition (A) est alors vérifiée. Plus généralement, toute variété algébrique, muni de la topologie de Zariski, vérifie (A) (cf. $\left.\mathrm{n}^{\circ} 34\right)$.

Proposirion 1. (a) Si $X$ vérifie la condition (A), $X$ est quasi-compact.

(b) Si $X$ vérifie la condition (A), tout sous-espace de $X$ la vérifie aussi.

(c) Si $X$ est réunion d'une famille finie $Y_{i}$ de sous-espaces vérifiant la condition (A), alors $X$ vérifie aussi la condition (A).

Si $F_{i}$ est un ensemble filtrant décroissant de parties fermées de $X$, et si $X$ vérifie $\left(\mathrm{A}^{\prime}\right)$, il existe un $F_{i}$ contenu dans tous les autres; $\operatorname{si} \cap F_{i}=\emptyset$ il y a donc un $i$ tel que $F_{i}=\emptyset$, ce qui démontre (a)

Soit $G_{1} \supset G_{2} \supset G_{3} \supset \cdots$ une suite décroissante de parties fermées d'un sous-espace $Y$ de $X$; si $X$ vérifie (A), il existe un $n$ tel que $\vec{G}_{m}=\bar{G}_{n}$ pour $m \geqq n$, d'où $G_{m}=Y \cap \bar{G}_{m}=Y \cap \bar{G}_{n}=G_{n}$, ce qui démontre (b).

Soit $F_{1} \supset F_{2} \supset F_{3} \supset \cdots$ une suite décroissante de parties fermées d'un espace $X$ vérifiant (c); puisque les $Y_{i}$ vérifient $(A)$ il existe pour chaque $i$ un $n_{i}$ tel que $F_{m}$ ก $Y_{i}=F_{n_{i}}$ n $Y_{i}$ pour $m \geqq n_{i} ;$ si $n=\operatorname{Sup}\left(n_{i}\right)$, on a alors $F_{m}=F_{n}$ si $m \geqq n$, ce qui démontre (c).

Un espace $X$ est dit irréductible s'il n'est pas réunion de deux sous-ensembles fermés, distincts de lui-même; il revient au même de dire que deux ouverts non vides de $X$ ont une intersection non vide. Toute famille finie d'ouverts non vides de $X$ a alors une intersection non vide, et tout ouvert de $X$ est également irréductible.

Proposition 2 . Tout espace $X$ vérifiant la condition (A) est réunion d'un nombre fini de sous-espaces fermés irréductibles $Y_{i} .$ Si l'on suppose que $Y_{i}$ n'est contenu dans $Y_{i}$ pour aucun couple $(i, j), i \neq j$, l'ensemble des $Y_{i}$ est déterminé de façon unique par $X$; les $Y_{i}$ sont alors appelés les composantes irréductibles de $X$.

L'existence d'une décomposition $X=U Y_{i}$ résulte évidemment de (A). Si $Z_{k}$ est une autre décomposition de $X$, on a $Y_{i}=\cup Y_{i}$ n $Z_{k}$, et, comme $Y_{i}$ est irréductible, cela implique l'existence d'un indice $k$ tel que $Z_{k} \supset Y_{i} ;$ échangeant les rôles de $Y_{i}$ et de $Z_{k}$, on conclut de même à l'existence d'un indice $i^{\prime}$ tel que $Y_{i^{\prime}} \supset Z_{k}$; d'où $Y_{i} \subset Z_{k} \subset Y_{i^{\prime}}$, ce qui, vu l'hypothèse faite sur les $Y_{i}$ entraîne $i=i^{\prime}$ et $Y_{i}=Z_{k}$, d'où aussitôt l'unicité de la décomposition.

Proposition $3 .$ Soit $X$ un espace topologique, réunion d'une famille finie de sous-ensembles ouverts non vides $V_{i} .$ Pour que $X$ soit irréductible, il faut et il sufit que les $V_{i}$ soient irréductibles et que $V_{i}$ ก $V_{j} \neq \emptyset$ pour tout couple $(i, j)$. La nécessité de ces conditions a été signalée plus haut; montrons qu'elles sont suffisantes. Si $X=Y \cup Z$, où $Y$ et $Z$ sont fermés, on a $V_{i}=\left(V_{i} \cap Y\right) \cup\left(V_{i} \cap Z\right)$, ce qui montre que chaque $V_{i}$ est contenu soit dans $Y$, soit dans $Z$. Supposons $Y$ et $Z$ distincts de $X ;$ on peut alors trouver deux indices $i, j$ tels que $V_{i}$ ne soit pas contenu dans $Y$ et $V_{j}$ ne soit pas contenu dans $Z$; d'après ce qui précède, on a donc $V_{i} \subset Z$ et $V_{j} \subset Y$. Soit $T=V_{j}-V_{i} \cap V_{j} ; T$ est fermé dans $V_{j}$, et l'on a $V_{j}=T \cup\left(Z \cap V_{j}\right) ;$ comme $V_{j}$ est irréductible, ceci entraîne soit $T=V_{j}$, c'est-à-dire $V_{i} \cap V_{j}=\emptyset$, soit $Z \cap V_{j}=V_{j}$, c'est-à-dire $V_{j} \subset Z$ et dans les deux cas on aboutit à une contradiction, cqfd.

31. Sous-ensembles localement fermés de l'espace affine. Soit $r$ un entier $\geqq 0$, et soit $X=K^{r}$ l'espace affine de dimension $r$ sur le corps $K$. Nous munirons $X$ de la topologie de Zariski; rappelons qu'un sous-ensemble de $X$ est fermé pour cette topologie s'il est l'ensemble des zéros communs à une famille de polynômes $P^{\alpha} \epsilon K\left[X_{1}, \cdots, X_{r}\right] .$ Puisque l'anneau des polynômes est noethérien, $X$ vérifie la condition (A) du $\mathrm{n}^{\circ}$ précédent; de plus, on montre facilement que $X$ est un espace irréductible.

Si $x=\left(x_{1}, \cdots, x_{r}\right)$ est un point de $X$, nous noterons $\theta_{x}$ l'anneau local de $x$; rappelons que c'est le sous-anneau du corps $K\left(X_{1}, \cdots, X_{r}\right)$ formé des fractions rationnelles $R$ qui peuvent être mises sous la forme

$R=P / Q$, où $P$ et $Q$ sont des polynômes, et $Q(x) \neq 0$

Une telle fraction rationnelle est dite régulière en $x$; en tout point $x \in X$ où $Q(x) \neq 0$, la fonction $x \rightarrow P(x) / Q(x)$ est une fonction continue à valeurs dans $K(K$ étant muni de la topologie de Zariski) que l'on peut identifier avec $R$, le corps $K$ étant infini. Les $\mathcal{O}_{x}, x \in X$, forment donc un sous-faisceau $\mathcal{O}$ du faisceau $\mathcal{F}(X)$ des germes de fonctions sur $X$ à valeurs dans $K\left(\right.$ cf. $\left.\mathrm{n}^{\circ} 3\right)$; le faisceau $\theta$ est un faisceau d'anneaux.

Nous allons étendre ce qui précède aux sous-espaces localement fermés de $X$ (nous dirons qu'un sous-ensemble d'un espace $X$ est localement fermé dans $X$ s'il est l'intersection d'un sous-ensemble ouvert et d'un sous-ensemble fermé de $X$ ). Soit $Y$ un tel sous-espace, et soit $\mathcal{F}(Y)$ le faisceau des germes de fonctions sur $Y$ à valeurs dans $K$; si $x$ est un point de $Y$, l'opération de restriction d'une fonction définit un homomorphisme canonique

$$
\varepsilon_{x}: \mathfrak{F}(X)_{x} \rightarrow \mathfrak{F}(Y)_{x}
$$

L'image de $\mathcal{O}_{x} \operatorname{par} \varepsilon_{x}$ est un sous-anneau de $\mathfrak{F}(Y)_{x}$, que nous désignerons par $\mathcal{O}_{x, y} ;$ les $\mathcal{O}_{x, y}$ forment un sous-faisceau $\mathcal{O}_{Y}$ de $\mathfrak{F}(Y)$, que nous appellerons le faisceau des anneaux locaux de $Y$. Une section de $\mathcal{O}_{Y}$ sur un ouvert $V$ de $Y$ est donc, par définition, une application $f: V \rightarrow K$ qui est égale, au voisinage de chaque point $x \in V$, à la restriction à $V$ d'une fonction rationnelle régulière en $x$; une telle fonction $f$ sera dite régulière sur $V$; c'est une fonction continue lorsque l'on munit $V$ de la topologie induite par celle de $X$, et $K$ de la topologie de Zariski. L'ensemble des fonctions régulières en tout point de $V$ est un' anneau, l'anneau $\Gamma\left(V, \mathcal{O}_{Y}\right) ;$ observons également que, si $f \in \Gamma\left(V, \theta_{Y}\right)$ et si $f(x) \neq 0$ pour tout $x \in V$, alors $1 / f$ appartient aussi à $\Gamma\left(V, \mathcal{O}_{Y}\right)$. On peut caractériser autrement le faisceau $\mathcal{O}_{Y}$ :

Proposition 4. Soit $U$ (resp. F) un sous-espace ouvert (resp. fermé) de X, et soit $Y=U \cap F .$ Soit $I(F)$ lidéal de $K\left[X_{1}, \cdots, X_{r}\right]$ formé des polynomes nuls sur $F .$ Si $x$ est un point de $Y$, le noyau de la surjection $\varepsilon_{x}: \theta_{x} \rightarrow \mathcal{O}_{x, Y}$ est égal a l'idéal $I(F) . \mathcal{O}_{x}$ de $\mathcal{O}_{x}$.

Il est clair que tout élément de $I(F), \theta_{x}$ appartient au noyau de $\varepsilon_{x} .$ Inversement, soit $R=P / Q$ un élément de ce noyau, $P$ et $Q$ étant deux polynômes, et $Q(x) \neq 0 .$ Par hypothèse, il existe un voisinage ouvert $W$ de $x$ tel que $P(y)=0$ pour tout $y \epsilon W \cap F ;$ soit $F^{\prime}$ le complémentaire de $W$, qui est fermé dans $X$; puisque $x \notin F^{\prime}$, il existe, par définition même de la topologie de Zariski, un polynôme $P_{1}$, nul sur $F^{\prime}$ et non nul en $x$; le polynôme $P \cdot P_{1}$ appartient alors à $I(F)$, et l'on peut écrire $R=P \cdot P_{1} / Q \cdot P_{1}$, ce qui montre bien que $R \in I(F) \cdot \mathcal{O}_{x}$.

CorolLalRe. L'anneau $\mathcal{O}_{x, y}$ est isomorphe à l'anneau des fractions de $K\left[X_{1}, \cdots, X_{r}\right] / I(F)$ relatif à l'idéal maximal défini par le point $x$.

Cela résulte immédiatement de la construction de l'anneau des fractions d'un anneau quotient (cf. par exemple [8], Chap. $\mathrm{XV}, \S 5$, th.XI).

32. Applications régulières. Soit $U$ (resp. $V$ ) un sous-espace localement fermé de $K^{r}$ (resp. $\left.K^{s}\right)$. Une application $\varphi: U \rightarrow V$ est dite régulière sur $U$ (ou simplement régulière) si:

(a) $\varphi$ est continue,

(b) Si $x \in U$, et si $f \in \mathcal{O}_{\varphi(x), V}$, alors $f \circ \varphi \in \mathcal{O}_{x, U}$

Désignons les coordonnées du point $\varphi(x)$ par $\varphi_{i}(x), 1 \leqq i \leqq s .$ On a alors:

Proposition $5 .$ Pour que $\varphi: U \rightarrow V$ soit régulière sur $U$, il faut et il suffit que les $\varphi_{i}: U \rightarrow K$ soient régulières sur U pour tout $i, 1 \leqq i \leqq s .$

Comme les fonctions coordonnées sont régulières sur $V$, la condition est nécessaire. Inversement, supposons que l'on ait $\varphi_{i} \varepsilon \Gamma\left(U, \theta_{U}\right)$ pous touti; si $P\left(X_{1}, \cdots, X_{s}\right)$ est un polynôme, la fonction $P\left(\varphi_{1}, \cdots, \varphi_{s}\right)$ appartient à $\Gamma\left(U, \mathcal{O}_{U}\right)$ puisque $\Gamma\left(U, \theta_{U}\right)$ est un anneau; il s'ensuit que c'est une fonction continue sur $U$, donc que le lieu de ses zéros est fermé, ce qui prouve la continuité de $\varphi$. Si l'on a $x \in U$ et $f \in \mathcal{O}_{\varphi(x), v}$, on peut écrire localement $f$ sous la forme $f=P / Q$, où $P$ et $Q$ sont des polynômes et $Q(\varphi(x)) \neq 0 .$ La fonction $f \circ \varphi$ est alors égale à $P \circ \varphi / Q \circ \varphi$ au voisinage de $x$; d'après ce que nous venons de voir, $P \circ \varphi$ et $Q \circ \varphi$ sont régulières au voisinage de $x$; comme $Q \circ \varphi(x) \neq 0$, il en résulte que $f \circ \varphi$ est régulière au voisinage de $x$, cqfd.

La composée de deux applications régulières est régulière. Une bijection $\varphi: U \rightarrow V$ est appelée un isomorphisme birégulier (ou simplement un isomorphisme) si $\varphi$ et $\varphi^{-1}$ sont des applications régulières; il revient au même de dire que $\varphi$ est un homéomorphisme de $U$ sur $V$ qui transforme le faisceau $\mathcal{O}_{U}$ en le faisceau $\mathcal{O}_{\mathrm{V}}$.

33. Produits. Si $r$ et $r^{\prime}$ sont deux entiers $\geqq 0$, nous identifierons l'espace affine $K^{r+r^{\prime}}$ au produit $K^{r} \times K^{r^{\prime}}$. La topologie de Zariski de $K^{r+r^{\prime}}$ est plus fine que la topologie produit des topologies de Zariski de $K^{*}$ et de $K^{r^{\prime}}$; elle est même strictement plus fine si $r$ et $r^{\prime}$ sont $>0 .$ Il en résulte que, si $U$ et $U^{\prime}$ sont des sousespaces localement fermés de $K^{r}$ et de $K^{r^{\prime}}, U \times U^{\prime}$ est un sous-espace localement fermé de $K^{r+r^{\prime}}$ et le faisceau $\mathcal{O}_{U \times U^{\prime}}$ est bien défini.

Soit d'autre part $W$ un sous-espace localement fermé de $K^{t}, t \geqq 0$, et soient $\varphi: W \rightarrow U$ et $\varphi^{\prime}: W \rightarrow U^{\prime}$ deux applications. Il résulte immédiatement de la Proposition 5 que l'on a:

Proposition $6 .$ Pour que l'application $x \rightarrow\left(\varphi(x), \varphi^{\prime}(x)\right)$ soit une application régulière de W dans $U \times U^{\prime}$, il faut et il suffit que $\varphi$ et $\varphi^{\prime}$ soient régulières.

Comme toute application constante est régulière, la Proposition précédente montre que toute section $x \rightarrow\left(x, x_{0}^{\prime}\right), x_{0}^{\prime} \in U^{\prime}$, est une application régulière de $U$ dans $U \times U^{\prime}$; d'autre part, les projections $U \times U^{\prime} \rightarrow U$ et $U \times U^{\prime} \rightarrow U^{\prime}$ sont évidemment régulières.

Soient $V$ et $V^{\prime}$ des sous-espaces localement fermés de $K^{s}$ et $K^{b^{\prime}}$, et soient $\psi: U \rightarrow V$ et $\psi^{\prime}: U^{\prime} \rightarrow V^{\prime}$ deux applications. Les remarques qui précèdent, jointes à la Proposition 6, montrent que l'on a alors (cf. [1], Chap. IV):

Proposition 7. Pour que $\psi \times \psi^{\prime}: U \times U^{\prime} \rightarrow V \times V^{\prime}$ soit régulière, il faut et il suffit que $\psi$ et $\psi^{\prime}$ soient régulières.

D'où:

ConolLarre. Pour que $\psi \times \psi^{\prime}$ soit un isomorphisme birégulier, il faut et $i l$ suffit que $\psi$ et $\psi^{\prime}$ soient des isomorphismes biréguliers.

\section{Définition de la structure de variété algébrique.}

DÉFintrion. On appelle variété algébrique sur $K$ (ou simplement variété algébrique) un ensemble $X$ muni.

$1^{\circ}$ d'une topologie,

$2^{\circ}$ d'un sous-faisceau $\mathcal{O}_{\mathrm{x}}$ du faisceau $F(X)$ des germes de fonctions sur $X$ a valeurs dans $K$,

ces données étant assujetties a vérifier les axiomes $\left(\mathrm{VA}_{I}\right)$ et $\left(\mathrm{VA}_{I I}\right)$ énoncés cidessous.

Remarquons d'abord que, si $X$ et $Y$ sont munis de deux structures du type précédent, on a la notion d'isomorphisme de $X$ sur $Y$ : c'est un homéomorphisme de $X$ sur $Y$ qui transforme $\theta_{x}$ en $\theta_{Y} .$ D'autre part, si $X^{\prime}$ est un ouvert de $X$ on peut munir $X^{\prime}$ de la topologie induite et du faisceau induit: on a une notion de structure induite sur un ouvert. Ceci précisé, nous pouvons énoncer l'axiome $\left(\mathrm{VA}_{I}\right):$

$\left(\mathrm{VA}_{I}\right)$ - Il existe un recouvrement ouvert fini $\mathfrak{B}=\left\{V_{i}\right\}_{\text {ieI }}$ de l'espace $X$ tel que chaque $V_{i}$, muni de la structure induite par celle de $X$, soit isomorphe à un sousespace localement fermé $U_{i}$ d'un espace affine, muni du faisceau $\mathcal{O}_{U_{i}}$ défini au $\mathrm{n}^{\circ} 31$.

Pour simplifier le langage, nous appellerons variété préalgébrique tout espace topologique $X$ muni d'un faisceau $\theta_{x}$ vérifiant l'axiome $\left(V A_{I}\right)$. Un isomorphisme $\varphi_{i}: V_{i} \rightarrow U_{i}$ sera appelé une carte de l'ouvert $V_{i} ;$ la condition $\left(\mathrm{VA}_{I}\right)$ signifie donc qu'il est possible de recouvrir $X$ au moyen d'un nombre fini d'ouverts possédant des cartes. La Proposition $1 \mathrm{du} \mathrm{n}^{\circ} 30$ montre que $X$ vérifie alors la condition (A), donc est quasi-compact, ainsi que tous ses sous-espaces. La topologie de $X$ sera appelée "topologie de Zariski" de $X$, et le faisceau $O_{x}$ sera appelé le faisceau des anneaux locaux de $X$.

Proposition 8. Soit $X$ un ensemble, réunion d'une famille finie de sous-ensembles $X_{j}, j \epsilon J .$ Supposons que chaque $X_{j}$ soit muni d'une structure de variété préalgébrique, et que les conditions suivantes soient vérifiées:

(a) $X_{i}$ n $X_{j}$ est ouvert dans $X_{i}$ quels que soient $i, j \in J$,

(b) les structures induites par $X_{i}$ et par $X_{j}$ sur $X_{i}$ n $X_{j}$ coincident quels que soient $i, j \in J$

Il existe alors une structure de variété préalgebrique et une seule sur $X$ telle que les $X_{j}$ soient ouverts dans $X$ et que la structure induite sur chaque $X_{j}$ soit la structure donnée.

L'existence et l'unicité de la topologie de $X$ et du faisceau $\mathcal{O}_{x}$ sont immédiates; il reste à vérifier que cette topologie et ce faisceau satisfont a $\left(\mathrm{VA}_{I}\right)$, ce qui résulte du fait que les $X_{j}$ sont en nombre fini et vérifient chacun $\left(\mathrm{VA}_{I}\right)$.

Corollalre. Soient $X$ et $X^{\prime}$ deux variétés préalgébriques. Il existe sur $X \times X^{\prime}$ une structure de variété préalgébrique et une seule vérifiant la condition suivante: Si $\varphi: V \rightarrow U$ et $\varphi^{\prime}: V^{\prime} \rightarrow U^{\prime}$ sont des cartes ( $V$ étant ouvert dans $X$ et $V^{\prime}$ ouvert dans $\left.X^{\prime}\right)$, alors $V \times V^{\prime}$ est ouvert dans $X \times X^{\prime}$ et $\varphi \times \varphi^{\prime}: V \times V^{\prime} \rightarrow U \times U^{\prime}$ est une carte.

Recouvrons $X$ par un nombre fini d'ouverts $V_{i}$ ayant des cartes $\varphi_{i}: V_{i} \rightarrow U_{i}$ et soit $\left(V_{j}^{\prime}, U_{j}^{\prime}, \varphi_{j}^{\prime}\right)$, un système analogue pour $X^{\prime} .$ L'ensemble $X \times X^{\prime}$ est réunion des $V_{i} \times V_{j}^{\prime} ;$ munissons chaque $V_{i} \times V_{j}^{\prime}$ de la structure de variété préalgébrique image de celle de $U_{i} \times U_{j}^{\prime}$ par $\varphi_{i}^{-1} \times \varphi_{j}^{-1}$; les hypothèses (a) et (b) de la Proposition 8 sont applicables à ce recouvrement de $X \times X^{\prime}$, d'après le corollaire à la Proposition $7 .$ On obtient ainsi une structure de variété préalgébrique sur $X \times X^{\prime}$ qui vérifie les conditions voulues.

On peut appliquer le corollaire précédent au cas particulier $X^{\prime}=X$; ainsi $X \times X$ se trouve muni d'une structure de variété préalgébrique, et en particulier d'une topologie. Nous pouvons maintenant énoncer l'axiome $\left(V A_{I I}\right)$ :

$\left(\mathrm{VA}_{I I}\right)$ - La diagonale $\Delta$ de $X \times X$ est fermée dans $X \times X$

Supposons que $X$ soit une variété préalgébrique, obtenue par le procédé de "recollement" de la Proposition 8 ; pour que la condition $\left(\mathrm{VA}_{I I}\right)$ soit satisfaite, il faut et il suffit que $X_{i j}=\Delta \cap X_{i} \times X_{j}$ soit fermé dans $X_{i} \times X_{j} .$ Or $X_{i j}$ est l'ensemble des $(x, x)$ avec $x \in X_{i} \mathrm{n} X_{j} .$ Supposons alors qu'il existe des cartes $\varphi_{i}: X_{i} \rightarrow U_{i}$, et soit $T_{i j}=\varphi_{i} \times \varphi_{j}\left(X_{i j}\right) ; T_{i j}$ est l'ensemble des $\left(\varphi_{i}(x), \varphi_{j}(x)\right)$ pour $x$ parcourant $X_{i} \cap X_{j} .$ L'axiome $\left(\mathrm{VA}_{I I}\right)$ prend donc la forme suivante:

$\left(\mathrm{VA}_{I I}^{\prime}\right)$ - Pour tout couple $(i, j), T_{i j}$ est fermé dans $U_{i} \times U_{j}$

Sous cette forme, on reconnait l'axiome (A) de Weil (cf. [16], p. 167), à cela près que Weil ne considère que des variétés irréductibles.

ExEMPLES de variétés algébriques: Tout sous-espace localement fermé $U$ d'un espace affine, muni de la topologie induite et du faisceau $\mathcal{O}_{U}$ défini au $\mathrm{n}^{\circ} 31$, est une variété algébrique. Toute variété projective est une variété algébrique (cf. no 51). Tout espace fibré algébrique (cf. [17]) dont la base et la fibre sont des variétés algébriques est une variété algébrique. RemarQues. (1) On notera l'analogie de la condition $\left(\mathrm{VA}_{I I}\right)$ et de la condition de séparation imposée aux variétés topologiques, différentiables, analytiques.

(2) Des exemples simples montrent que la condition $\left(V \mathrm{~A}_{I I}\right)$ n'est pas une conséquence de la condition $\left(\mathrm{VA}_{I}\right)$.

35. Applications régulières, structures induites, produits. Soient $X$ et $Y$ deux variétés algébriques, $\varphi$ une application de $X$ dans $Y$. On dit que $\varphi$ est régulière si :

(a) $\varphi$ est continue,

(b) Si $x \in X$, et si $f \in \mathcal{O}_{\varphi(x), Y}$, alors $f \circ \varphi \in \mathcal{O}_{x, X}$.

De même qu'au no 32, la composée de deux applications régulières est régulière, et, pour qu'une bijection $\varphi: X \rightarrow Y$ soit un isomorphisme, il faut et il suffit que $\varphi$ et $\varphi^{-1}$ soient des applications régulières. Les applications régulières forment donc une famille de morphismes pour la structure de variété algébrique, au sens de [1], Chap. IV.

Soit $X$ une variété algébrique, et $X^{\prime}$ un sous-ensemble localement fermé de $X$. Munissons $X^{\prime}$ de la topologie induite par celle de $X$ et du faisceau $\mathcal{O}_{x}$ induit $\operatorname{par} \mathcal{O}_{X}$ (de façon plus précise, pour tout $x \in X^{\prime}$, on définit $\mathcal{O}_{x, x}$, comme l'image de $\mathcal{O}_{x, x}$ par l'homomorphisme canonique: $\left.\mathfrak{F}(X)_{x} \rightarrow \mathfrak{F}\left(X^{\prime}\right)_{x}\right) .$ L'axiome $\left(\mathrm{VA}_{I}\right)$ est vérifié: si $\varphi_{i}: V_{i} \rightarrow U_{i}$ est un système de cartes tel que $X=U V_{i}$, on pose $V_{i}^{\prime}=X^{\prime} \cap V_{i}, U_{i}^{\prime}=\varphi_{2}\left(V_{i}^{\prime}\right)$, et $\varphi_{i}: V_{i}^{\prime} \rightarrow U_{i}^{\prime}$ est un système de cartes tel que $X^{\prime}=\cup V_{i}^{\prime} .$ L'axiome $\left(\mathrm{VA}_{I I}\right)$ est vérifié du fait que la topologie de $X^{\prime} \times X^{\prime}$ est induite par celle de $X \times X$ (on pourrait aussi bien utiliser $\left.\left(\mathrm{VA}_{I I}^{\prime}\right)\right)$. On définit ainsi une structure de variété algébrique sur $X^{\prime}$, qui estdite induite par celle de $X$; on dit aussi que $X^{\prime}$ est une sous-variété de $X$ (chez Weil [16], le terme de "sous-variété" est réservé à ce que nous appelons ici une sous-variété irréductible fermée). Si $\iota$ désigne l'injection de $X^{\prime}$ dans $X, \iota$ est une application régulière; de plus, si $\varphi$ est une application d'une variété algébrique $Y$ dans $X^{\prime}$, pour que $\varphi: Y \rightarrow X^{\prime}$ soit régulière, il faut et il suffit que $\iota \circ \varphi: Y \rightarrow X$ soit régulière (ce qui justifie le terme de "structure induite", cf. [1], loc. cit.).

Si $X$ et $X^{\prime}$ sont deux variétés algébriques, $X \times X^{\prime}$ est une variété algébrique, appelée variété produit; il suffit en effet de montrer que l'axiome $\left(\mathrm{VA}_{I I}^{\prime}\right)$ est vérifié autrement dit que, si $\varphi_{i}: V_{i} \rightarrow U_{i}$ et $\varphi_{i}^{\prime}: V_{i}^{\prime} \rightarrow U_{i}^{\prime}$ sont des systèmes de cartes tels que $X=U V_{i}$ et $X^{\prime}=\cup V_{i^{\prime}}^{\prime}$, l'ensemble $T_{i j} \times T_{i^{\prime}, j^{\prime}}^{\prime}$ est alors fermé dans $U_{i} \times U_{j} \times U_{i}^{\prime}, \times U_{j^{\prime}}^{\prime}$ (les notations étant celles du no 34$)$; or cela résulte immédiatement du fait que $T_{i j}$ et $T_{i^{\prime} j^{\prime}}^{\prime}$ sont fermés dans $U_{i} \times U_{j}$ et $U_{i^{\prime}}^{\prime} \times U_{j^{\prime}}^{\prime}$ respectivement.

Les Propositions 6 et 7 sont valables sans changement pour des variétés algébriques quelconques.

Si $\varphi: X \rightarrow Y$ est une application régulière, le graphe $\Phi$ de $\varphi$ est fermé dans $X \times Y$, car c'est l'image réciproque de la diagonale de $Y \times Y$ par l'application $\varphi \times 1: X \times Y \rightarrow Y \times Y ;$ de plus, l'application $\psi: X \rightarrow \Phi$ définie par $\psi(x)=(x, \varphi(x))$ est un isomorphisme: en effet, $\psi$ est une application régulière, ainsi que $\psi^{-1}$ (puisque c'est la restriction de la projection $\left.X \times Y \rightarrow X\right)$.

36. Corps des fonctions rationnelles sur une variêté irréductible. Nous démontrerons d'abord deux lemmes de nature purement topologique: LEmme 1. Soient $X$ un espace connexe, $G$ un groupe abelien, et $\mathcal{S}$ le faisceau constant sur $X$, isomorphe a $G .$ L'application canonique $G \rightarrow \Gamma(X, \mathcal{S})$ est bijective.

Un élément de $\Gamma(X, \mathcal{G})$ n'est pas autre chose qu'une application continue de $X$ dans $G$ muni de la topologie discrète. Puisque $X$ est connexe, une telle application est constante, ce qui démontre le lemme.

Nous dirons qu'un faisceau $\mathfrak{F}$ sur un espace $X$ est localement constant si tout point de $x$ possède un voisinage $U$ tel que $\mathfrak{F}(U)$ soit constant sur $U$.

LEmme 2. Tout faisceau localement constant sur un espace irréductible est constant.

Soient $\mathfrak{F}$ le faisceau, $X$ l'espace, et posons $F=\Gamma(X, \mathfrak{F}) ;$ il nous suffira de montrer que l'homomorphisme canonique $\rho_{x}: F \rightarrow \mathfrak{F}_{x}$ est bijectif pour tout $x \in X$, car nous obtiendrons ainsi un isomorphisme du faisceau constant isomorphe à $F$ sur le faisceau $\mathfrak{F}$ donné.

Si $f \in F$, le lieu des points $x \in X$ tels que $f(x)=0$ est ouvert (d'après les propriétés générales des faisceaux), et fermé (parce que $\mathfrak{F}$ est localement constant); vu qu'un espace irréductible est connexe, ce lieu est donc, soit $\emptyset$, soit $X$, ce qui démontre déjà que $\rho_{x}$ est injectif.

Soit maintenant $m \in \mathfrak{F}_{x}$, et soit $s$ une section de $\mathfrak{F}$ au-dessus d'un voisinage $U$ de $x$ telle que $s(x)=m$; recouvrons $X$ par des ouverts non vides $U_{i}$ tels que $\mathfrak{F}\left(U_{i}\right)$ soit constant sur $U_{i} ;$ puisque $X$ est irréductible, on a $U$ n $U_{i} \neq \emptyset ;$ choisissons un point $x_{i} \epsilon U \cap U_{i} ;$ il existe évidemment une section $s_{i}$ de $\mathfrak{F}$ sur $U_{i}$ telle que $s_{i}\left(x_{i}\right)=s\left(x_{i}\right)$, et comme les sections $s$ et $s_{i}$ coincident en $x_{i}$, elles coincident dans tout $U \cap U_{i}$, puisque $U \cap U_{i}$ est irréductible, donc connexe; de même $s_{i}$ et $s_{j}$ coincident sur $U_{i} \cap U_{j}$, puisqu'elles coincident sur $U \cap U_{i} \cap U_{j} \neq \emptyset$; donc les sections $s_{i}$ définissent une section unique $s$ de $\mathcal{F}$ au-dessus de $X$, et l'on a $\rho_{x}(s)=m$, ce qui achève la démonstration.

Soit maintenant $X$ une variété algébrique irréductible. Si $U$ est un ouvert non vide de $X$, posons $Q_{U}=\Gamma\left(U, \mathcal{O}_{\mathbf{x}}\right) ; a_{U}$ est un anneau d'intégrité: en effet, supposons que l'on ait $f \cdot g=0, f$ et $g$ étant des applications régulières de $U$ dans $K$; si $F$ (resp. $G$ ) est le lieu des points $x \in U$ tels que $f(x)=0$ (resp. $g(x)=0)$, on a $U=F \cup G$, et $F$ et $G$ sont fermés dans $U$, puisque $f$ et $g$ sont continues; comme $U$ est irréductible, cela entraine $F=U$ ou $G=U$, ce qui signifie bien que $f$ ou $g$ est nul sur $U$. On peut donc parler du corps des quotients de $Q_{U}$ que nous noterons $\mathfrak{K}_{U} ;$ si $U \subset V$, l'homomorphisme $\rho_{U}: \mathbb{a}_{V} \rightarrow \alpha_{U}$ est injectif puisque $U$ est dense dans $V$, et l'on a un isomorphisme bien déterminé $\varphi_{U}^{V}$ de $\mathscr{K}_{V}$ dans $\Re_{U} ;$ le système des $\left\{\mathscr{K}_{U}, \varphi_{U}^{V}\right\}$ définit un faisceau de corps $\mathfrak{K}$; d'ailleurs $\mathcal{K}_{x}$ est canoniquement isomorphe au corps des quotients de $\mathcal{O}_{x, x}$.

Proposition 9. Pour toute variété algébrique irréductible $X$, le faisceau $\mathfrak{K}$ défini ci-dessus est un faisceau constant.

Vu le Lemme 2 , il suffit de démontrer la Proposition lorsque $X$ est une sousvariété localement fermée de l'espace affine $K^{*} ;$ soit $F$ l'adhérence de $X$ dans $K^{r}$, et soit $I(F)$ l'idéal de $K\left[X_{1}, \cdots, X_{r}\right]$ formé des polynômes nuls sur $F$ (ou sur $X$, cela revient au même). Si l'on pose $A=K\left[X_{1}, \cdots, X_{r}\right] / I(F)$, l'anneau $A$ est un anneau d'intégrité puisque $X$ est irréductible; soit $K(A)$ le corps des quotients de $A$. D'après le corollaire à la Proposition 4, on peut identifier $\mathcal{O}_{x, x}$ à l'anneau des fractions de $A$ relativement a l'idéal maximal défini par $x$; on obtient ainsi un isomorphisme du corps $K(A)$ sur le corps des fractions de $\theta_{x, x}$, et il est facile de vérifier que l'on définit ainsi un isomorphisme du faisceau constant égal à $K(A)$ sur le faisceau $\mathfrak{K}$, ce qui démontre la proposition.

D'après le Lemme 1, les sections du faisceau $\mathcal{K}$ forment un corps, isomorphe à $\widetilde_{x}$ pour tout $x \in X$, et que nous noterons $K(X) .$ On l'appelle le corps des fonctions rationnelles sur $X$; c'est une extension de type fini du corps $K$, dont le degré de transcendance sur $K$ est la dimension de $X$ (on étend cette définition aux variétés algébriques réductibles en posant $\operatorname{dim} X=$ Sup dim $Y_{i}$, si $X$ est réunion des sous-variétés fermées irréductibles $Y_{i}$ ). On identifiera en général le corps $K(X)$ avec les corps $\mathcal{K}_{x} ;$ comme l'on a $\mathcal{O}_{x, x} \subset \mathfrak{K}_{x}$, on voit que l'on identifie ainsi $\mathcal{O}_{x . x}$ à un sous-anneau de $K(X)$ (c'est l'anneau de spécialisation du point $x$ dans $K(X)$, au sens de Weil, [16], p. 77). Si $U$ est ouvert dans $X$, $\Gamma\left(U, \mathcal{O}_{x}\right)$ est donc l'intersection dans $K(X)$ des anneaux $\theta_{x, x}$ pour $x$ parcourant $U$.

Si $Y$ est une sous-variété de $X$, on a $\operatorname{dim} Y \leqq \operatorname{dim} X ;$ si en outre $Y$ est fermée, et ne contient aucune composante irréductible de $X$, on a $\operatorname{dim} Y<\operatorname{dim} X$, comme on le voit en se ramenant au cas de sous-variétés de $K^{r}$ (cf. par exemple [8], Chap. X, $\S$, th. II).

\section{§2. Faisceaux algébriques cohérents}

37. Le faisceau des anneaux locaux d'une variété algébrique. Revenons aux notations du $\mathrm{n}^{\circ} 31:$ soit $X=K^{\prime}$, et soit $\mathcal{O}$ le faisceau des anneaux locaux de $X$. On a:

Lemme 1. Le faisceau $\mathcal{O}$ est un faisceau cohérent d'anneaux, au sens du $\mathrm{n}^{\circ} 15$. Soient $x \in X, U$ un voisinage de $x$, et $f_{1}, \cdots, f_{p}$ des sections de 0 sur $U$ c'est-à-dire des fonctions rationnelles régulières en tout point de $U$; il nous faut montrer que le faisceau des relations entre $f_{1}, \cdots, f_{p}$ est un faisceau de type fini sur $\mathcal{O}$. Quitte à remplacer $U$ par un voisinage plus petit, on peut supposer que les $f_{i}$ s'écrivent $f_{i}=P_{i} / Q$, où les $P_{i}$ et $Q$ sont des polynômes, $Q$ ne s'annulant pas sur $U$. Soient maintenant $y \in U$, et $g_{i} \in \mathcal{O}_{y}$ tels que $\sum_{i=1}^{i=p} g_{i} f_{i}$ soit nulle au voisinage de $y$; on peut encore écrire les $g_{i}$ sous la forme $g_{i}=R_{i} / S$, où les $R_{i}$ et $S$ sont des polynômes, $S$ ne s'annulant pas en $y$. La relation $\sum_{i=1}^{i-p} g_{i} f_{i}=0$ au voisinage de $y "$ équivaut à la relation $" \sum_{i=1}^{i=p} R_{i} P_{i}=0$ au voisinage de $y$ ", elle-même équivalente a $\sum_{i=1}^{i=p} R_{i} P_{i}=0$. Comme le module des relations entre les polynômes $P_{i}$ est un module de type fini (puisque l'anneau des polynômes est noethérien), il s'ensuit bien que le faisceau des relations entre les $f_{i}$ est de type fini.

Soit maintenant $V$ une sous-variété fermée de $X=K^{r} ;$ pour tout $x \in X$, soit $g_{x}(V)$ l'idéal de $\mathcal{O}_{x}$ formé des éléments $f \in \mathcal{O}_{x}$ dont la restriction à $V$ est nulle au voisinage de $x$ (on a donc $g_{x}(V)=\mathcal{O}_{x}$ si $\left.x \notin V\right)$. Les $g_{x}(V)$ forment un sousfaisceau $g(V)$ du faisceau $\mathcal{O}$.

Lemme 2. Le faisceau g(V) est un faisceau cohérent de $\mathcal{O}$-modules.

Soit $I(V)$ l'idéal de $K\left[X_{1}, \cdots, X_{r}\right]$ formé des polynômes $P$ s'annulant sur $V$. D'après la Proposition $4 \mathrm{du} \mathrm{n}^{\circ} 31, g_{x}(\mathrm{~V})$ est égal à $I(V) . \mathcal{O}_{x}$ pour tout $x \in V$, et cette formule subsiste pour $x \notin V$ comme on le voit aussitôt. L'idéal $I(V)$ étant engendré par un nombre fini d'éléments, il en résulte que le faisceau $g(V)$ est de type fini, donc cohérent d'après le Lemme 1 et la Proposition $8 \mathrm{du} \mathrm{n}^{\circ} 15$.

Nous allons maintenant étendre le Lemme 1 à une variété algébrique arbitraire:

Proposition 1. Si $V$ est une variété algébrique, le faisceau $\mathcal{O}_{\mathrm{V}}$ est un faisceau cohérent d'anneaux sur $V$.

La question étant locale, nous pouvons supposer que $V$ est une sous-variété fermée de l'espace affine $K^{r}$. D'après le Lemme 2 , le faisceau $g(V)$ est un faisceau cohérent d'idéaux, donc le faisceau $\mathcal{O} / \mathrm{g}(V)$ est un faisceau cohérent d'anneaux sur $X$, d'après le Théorème 3 du $n^{\circ}$ 16. Ce faisceau d'anneaux est nul en dehors de $V$, et sa restriction à $V$ n'est autre que $\mathcal{O}_{V}\left(\mathrm{n}^{\circ} 31\right) ;$ donc le faisceau ${O}_{\mathrm{v}}$ est un faisceau cohérent d'anneaux sur $V$ (n 17, corollaire à la Proposition 11).

REMARQUE. Il est clair que la Proposition 1 vaut, plus généralement, pour toute variété préalgébrique.

38. Faisceaux algébriques cohérents. Si $V$ est une variété algébrique dont le faisceau des anneaux locaux est $\mathcal{O}_{\mathrm{V}}$, nous appellerons faisceau algébrique sur $V$ tout faisceau de $\mathcal{O}_{v}$-modules, au sens du $\mathrm{n}^{\circ} 6 ;$ si $\mathcal{F}$ et $\mathcal{S}$ sont deux faisceaux algébriques, nous dirons que $\varphi: \mathfrak{F} \rightarrow \mathcal{S}$ est un homomorphisme algébrique (ou simplement un homomorphisme) si c'est un $\mathcal{O}_{v}$-homomorphisme; rappelons que cela équivaut à dire que chacun des $\varphi_{x}: \mathfrak{F}_{x} \rightarrow \mathrm{S}_{x}$ est $\mathcal{\theta}_{x, v}$-linéaire et que $\varphi$ transforme toute section locale de $\mathcal{F}$ en une section locale de $\mathrm{S}$.

Si $\mathscr{\text { est un faisceau algébrique sur }} V$, les groupes de cohomologie $H^{q}(V, \mathfrak{F})$ sont des modules sur $\Gamma\left(V, O_{v}\right)$, cf. $n^{\circ} 23 ;$ en particulier, ce sont des espaces vectoriels sur $K$.

Un faisceau algébrique $\mathfrak{F}$ sur $V$ sera dit cohérent si c'est un faisceau cohérent de $\mathcal{O}_{v}$-modules, au sens du $\mathrm{n}^{\circ} 12$; vu la Proposition $7 \mathrm{du} \mathrm{n}^{\circ} 15$ et la Proposition 1 ci-dessus, un tel faisceau est caractérisé par le fait qu'il est localement isomorphe au conoyau d'un homomorphisme algébrique $\varphi: 0_{V}^{q} \rightarrow \mathcal{O}_{V}^{p}$.

Nous allons donner quelques exemples de faisceaux algébriques cohérents (on en verra d'autres plus tard, cf. $\mathrm{n}^{\mathrm{os}} 48,57$ notamment).

39. Faisceau d'idéaux défini par une sous-variété fermée. Soit $W$ une sousvariété fermée d'une variété algébrique $V$. Pour tout $x \in V$, soit $g_{x}(W)$ l'idéal de $\mathcal{O}_{x, v}$ formé des éléments $f$ dont la restriction à $W$ est nulle au voisinage de $x$; soit $g(W)$ le sous-faisceau de $\mathcal{O}_{V}$ formé par les $g_{x}(W) .$ On a la Proposition suivante, qui généralise le Lemme 2:

Proposition 2. Le faisceau $\mathfrak{g}(W)$ est un faisceau algébrique cohérent.

La question étant locale, nous pouvons supposer que $V$ (donc aussi $W$ ) est une sous-variété fermée de l'espace affine $K^{r} .$ Il résulte alors du Lemme 2 , appliqué à $W$, que le faisceau d'idéaux défini par $W$ dans $K^{r}$ est de type fini; il s'ensuit que $g(W)$, qui en est l'image par l'homomorphisme canonique $\mathcal{O} \rightarrow \mathcal{O}_{\boldsymbol{v}}$, est également de type fini, donc est cohérent d'après la Proposition $8 \mathrm{du} \mathrm{n}^{\circ} 15$ et la Proposition $1 \mathrm{du} \mathrm{n}^{\circ} 37$. Soit $\mathcal{O}_{W}$ le faisceau des anneaux locaux de $W$, et soit $\mathcal{O}_{W}^{V}$ le faisceau sur $V$ obtenu en prolongeant $\mathcal{O}_{w}$ par 0 en dehors de $W\left(\right.$ cf. $\left.\mathrm{n}^{\circ} 5\right)$; ce faisceau est canoniquement isomorphe à $\mathcal{O}_{V} / g(W)$, autrement dit, on a une suite exacte:

$$
0 \rightarrow g(W) \rightarrow \mathcal{O}_{V} \rightarrow \mathcal{O}_{W}^{V} \rightarrow 0
$$

Soit alors $\mathcal{F}$ un faisceau algébrique sur $W$, et soit $\mathcal{F}^{v}$ le faisceau obtenu en prolongeant $\mathscr{\text { par }} 0$ en dehors de $W$; on peut considérer $\mathscr{F}^{V}$ comme un faisceau de $\mathcal{O}_{w}^{v}$-modules, donc aussi comme un faisceau de $\mathcal{O}_{v}$-modules dont l'annulateur contient $g(W)$. On a:

Proposition $3 .$ Si $\mathfrak{F}$ est un faisceau algébrique cohérent sur $W, \mathfrak{F}^{v}$ est un faisceau algébrique cohérent sur $V$. Inversement, si $\mathcal{S}$ est un faisceau algébrique cohérent sur $V$ dont l'annulateur contient $g(W)$, la restriction de S d $W$ est un faisceau algébrique cohérent sur $W$.

Si $\mathfrak{F}$ est un faisceau algébrique cohérent sur $W, \mathfrak{F}^{V}$ est un faisceau cohérent de $\theta_{w}^{v}$-modules (n $^{\circ} 17$, Proposition 11), donc un faisceau cohérent de $\mathcal{O}_{v}$-modules (no 16, Théorème 3). Inversement, si $\mathcal{\text { est un faisceau algébrique cohérent sur }}$ $V$, dont l'annulateur contient $g(W)$, S peut être considéré comme un faisceau de $\mathcal{O}_{V} / g(W)$-modules, et c'est un faisceau cohérent (n $^{\circ} 16$, Théorème 3 ); la restriction de $\mathcal{S}$ à $W$ est alors un faisceau cohérent de $\mathcal{O}_{w}$-modules $\left(\mathrm{n}^{\circ} 17\right.$, Proposition 11).

Ainsi, tout faisceau algébrique cohérent sur $W$ peut être identifié à un faisceau algébrique cohérent sur $V$ (et cette identification ne change pas les groupes de cohomologie, d'après la Proposition 8 du $n^{\circ} 26$ ). En particulier, tout faisceau algébrique cohérent sur une variété affine (resp. projective) peut être considéré comme un faisceau algébrique cohérent sur l'espace affine (resp. projectif); nous ferons fréquemment usage de cette possibilité par la suite.

RemarQue. Soit $\mathcal{S}$ un faisceau algébrique cohérent sur $V$, qui soit nul en dehors de $W$; l'annulateur de $\mathcal{S}$ ne contient pas nécessairement $\mathfrak{g}(W)$ (autrement dit, $\mathcal{S}$ ne peut pas toujours être considéré comme un faisceau algébrique cohérent sur $W)$; tout ce que l'on peut affirmer, c'est qu'il contient une puissance de $g(W)$

40. Faisceaux d'idéaux fractionnaires. Soit $V$ une variété algébrique irréductible, et soit $K(V)$ le faisceau constant des fonctions rationnelles sur $V$ (cf. $\left.\mathrm{n}^{\circ} 36\right) ; K(V)$ est un faisceau algébrique, qui n'est pas cohérent si $\operatorname{dim} V>0$. Un sous-faisceau algébrique $\mathfrak{F}$ de $K(V)$ peut être appelé un "faisceau d'idéaux fractionnaires", puisque chaque $\mathcal{F}_{x}$ est un idéal fractionnaire de $\mathcal{O}_{x, v}$

Proposition 4 . Pour qu'un sous-faisceau algébrique $\mathfrak{F}$ de $K(V)$ soit cohérent, il faut et il suffit qu'il soit de type fini.

La nécessité est triviale. Pour démontrer la suffisance, il suffit de prouver que $K(V)$ vérifie la condition (b) de la définition 2 du $\mathrm{n}^{\circ} 12$, autrement dit que, si $f_{1}, \cdots, f_{p}$ sont des fractions rationnelles, le faisceau $R\left(f_{1}, \cdots, f_{p}\right)$ est de type fini. Si $x$ est un point de $V$, on peut trouver des fonctions $g_{i}$ et $h$, telles que $f_{i}=g_{i} / h, g_{i}$ et $h$ étant régulières sur un voisinage $U$ de $x$, et $h$ étant non nulle sur $U$; le faisceau $R\left(f_{1}, \cdots, f_{p}\right)$ est alors égal au faisceau $R\left(g_{1}, \cdots, g_{p}\right)$ qui est de type fini, puisque $\mathcal{O}_{V}$ est un faisceau cohérent d'anneaux.

41. Faisceau associê à un espace fibré à fibre vectorielle. Soit $E$ un espace fibré algébrique, à fibre vectorielle de dimension $r$, et de base une variété algébrique $V$; par définition, la fibre type de $E$ est l'espace vectoriel $K^{r}$, et le groupe structural est le groupe linéaire $\mathbf{G L}(r, K)$ opérant sur $K^{r}$ à la facon usuelle (pour la définition d'un espace fibré algébrique, cf. [17]; voir aussi $[15], \mathrm{n}^{\circ} 4$ pour les espaces fibrés analytiques à fibres vectorielles).

Si $U$ est un ouvert de $V$, soit $s(E)_{V}$ l'ensemble des sections de $E$ régulières sur $U$; si $V \supset U$, on a un homomorphisme de restriction $\varphi_{U}^{V}: s(E)_{v} \rightarrow s(E)_{U}$ d'où un faisceau $\delta(E)$, appelé le faisceau des germes de sections de $E .$ Du fait que $E$ est un espace fibré à fibre vectorielle, chaque $s(E)_{U}$ est un $\Gamma\left(U, \mathcal{O}_{V}\right)$-module, et il s'ensuit que $s(E)$ est un faisceau algébrique sur $V$. Si l'on identifie localement $E$ à $V \times K^{r}$, on voit que

Proposition $5 .$ Le faisceau $\$(E)$ est localement isomorphe à $\mathcal{O}_{V}^{i} ;$ en particulier, c'est un faisceau algébrique cohérent.

Inversement, il est facile de voir que tout faisceau algébrique $\mathfrak{F}$ sur $V$, localement isomorphe à $\mathcal{O}_{V}^{r}$, est isomorphe à un faisceau $\delta(E)$, où $E$ est déterminé a un isomorphisme près (cf. [15], pour le cas analytique).

Si $V$ est une variété sans singularités, on peut prendre pour $E$ l'espace fibré des $p$-covecteurs tangents à $V(p$ étant un entier $\geqq 0) ;$ soit $\Omega^{p}$ le faisceau $s(E)$ correspondant; un elément de $\Omega_{x}^{p}, x \in V$, n'est pas autre chose qu'une forme différentielle de degré $p$ sur $V$, régulière en $x$. Si l'on pose $h^{p, q}=\operatorname{dim}_{K} H^{\varphi}\left(V, \Omega^{p}\right)$ on sait que, dans le cas classique (et si $V$ est projective), $h^{p, q}$ est égal à la dimension de l'espace des formes harmoniques de type $(p, q)$ (théorème de Dolbeault $^{3}$ ), et, si $B_{n}$ désigne le $n$-ème nombre de Betti de $V$, on a $B_{n}=\sum_{p+q=n} h^{p, q}$. Dans le cas général, on pourrait prendre la formule précédente pour définition des nombres de Betti d'une variété projective sans singularités (nous verrons en effet au no 66 que les $h^{p, 7}$ sont finis). Il conviendrait d'étudier leurs propriétés et notamment de voir s'ils coïncident avec ceux qui interviennent dans les conjectures de Weil relatives aux variétés sur les corps finis." Signalons seulement ici qu'ils vérifient la "dualité de Poincare" $B_{n}=B_{2 m-n}$ lorsque $V$ est irréductible et de dimension $m$.

Les groupes de cohomologie $H^{q}(V, \delta(E))$ interviennent aussi dans d'autres questions, notamment dans le théorème de Riemann-Roch, ainsi que dans la classification des espaces fibrés algébriques de base $V$, et de groupe structural le groune affine $x \rightarrow a x+b$ (cf. $[17], \S 4$, où est traité le cas où $\operatorname{dim} V=1$ ).

\section{§3. Faisceaux algébriques cohêrents sur les variétés affines}

42. Variétés affines. Une variété algébrique $V$ est dite affine si elle est isomorphe à une sous-variété fermée d'un espace affine. Le produit de deux variétés

3 P. Dolbeault. Sur la cohomologie des variétés analytiques complexes. C. R. Paris, 236, 1953, p. $175-177$

${ }^{4}$ Bulletin Amer. Math. Soc., 55,1949 , p. 507 . affines est une variété affine; toute sous-variété fermée d'une variété affine est une variété affine.

Un sous-ensemble ouvert $U$ d'une variété algébrique $X$ est dit affine si, muni de la structure de variété algébrique induite par celle de $X$, c'est une variété affine.

Proposition 1. Soient $U$ et $V$ deux sous-ensembles ouverts d'une variété algébrique $X .$ Si $U$ et $V$ sont affines, $U$ n $V$ est affine.

Soit $\Delta$ la diagonale de $X \times X$; d'après le $n^{\circ} 35$, l'application $x \rightarrow(x, x)$ est un isomorphisme birégulier de $X$ sur $\Delta$; donc la restriction de cette application à $U \cap V$ est un isomorphisme birégulier de $U \cap V$ sur $\Delta \cap U \times V$. Comme $U$ et $V$ sont des variétés affines, $U \times V$ est aussi une variété affine; d'autre part, $\Delta$ est fermée dans $X \times X$ d'après l'axiome $\left(\mathrm{VA}_{I I}\right)$, donc $\Delta \cap U \times V$ est fermée dans $U \times V$, et c'est bien une variété affine, cqfd.

(Il est facile de voir que cette Proposition est en défaut pour les variétés préalgébriques: l'axiome $\left(\mathrm{VA}_{I I}\right) \mathrm{y}$ joue un rôle essentiel).

Introduisons maintenant une notation qui sera utilisée dans toute la suite de ce paragraphe: si $V$ est une variété algébrique, et $f$ une fonction régulière sur $V$, nous noterons $V_{f}$ le sous-ensemble ouvert de $V$ formé des points $x \in V$ tels que $f(x) \neq 0$

Proposition $2 .$ Si $V$ est une variété algébrique affine, et $f$ une fonction régulière sur $V$, l'ouvert $V_{f}$ est un ouvert affine.

Soit $W$ le sous-ensemble de $V \times K$ formé des couples $(x, \lambda)$ tels que $\lambda . f(x)=1$; il est clair que $W$ est fermé dans $V \times K$, donc est une variété affine. Pour tout $(x, \lambda) \in W$, posons $\pi(x, \lambda)=x$; l'application $\pi$ est une application régulière de $W$ dans $V_{f} .$ Inversement, pour tout $x \in V_{f}$, posons $\omega(x)=(x, 1 / f(x))$; l'application $\omega: V_{f} \rightarrow W$ est régulière, et l'on a $\pi \circ \omega=1, \omega \circ \pi=1$, donc $V_{f}$ et $W$ sont isomorphes, cqfd.

Proposition 3. Soient $V$ une sous-variété fermée de $K^{*}, F$ un sous-ensemble fermé de $V$, et $U=V-F$. Les ouverts $V_{P}$ forment une base pour la topologie de U lorsque P parcourt l'ensemble des polynômes nuls sur $F$.

Soit $U^{\prime}=V-F^{\prime}$ un ouvert de $U$, et soit $x \in U^{\prime} ;$ il nous faut montrer qu'il existe $P$ tel que $V_{P} \subset U^{\prime}$ et $x \in V_{P} ;$ autrement dit, $P$ doit être nul sur $F^{\prime}$ et non nul en $x$; l'existence d'un tel polynôme résulte simplement de la définition de la topologie de $K^{r}$.

THÉonEme 1. Les ouverts affines d'une variété algébrique $X$ forment une base d'ouverts pour la topologie de $X$.

La question étant locale, on peut supposer que $X$ est une sous-variété localement fermée d'un espace affine $K^{r}$; dans ce cas, le théorème résulte immédiatement des Propositions 2 et $3 .$

CorolLalre. Les recouvrements de $X$ formés d'ouverts affines sont arbitrairement fins.

On notera que, si $\mathfrak{U}=\left\{U_{i}\right\}_{i \in I}$ est un tel recouvrement, les $U_{i_{0} \cdots i_{p}}$ sont tous des ouverts affines, d'après la proposition $1 .$

43. Quelques propriétés préliminaires des variétés irréductibles. Soit $V$ une sous-variété fermée de $K^{r}$, et soit $I(V)$ l'idéal de $K\left[X_{1}, \cdots, X_{r}\right]$ formé des polynômes nuls sur $V$; soit $A$ l'anneau quotient $K\left[X_{1}, \cdots, X_{r}\right] / I(V) ;$ on a un homomorphisme canonique

$$
\iota: A \rightarrow \Gamma\left(V, \mathcal{O}_{V}\right)
$$

qui est injectif par définition même de $I(V)$.

Propositron 4. Si $V$ est irréductible, $\iota: A \rightarrow \Gamma\left(V, \mathcal{O}_{V}\right)$ est bijectif.

(En fait, ceci vaut pour toute sous-variété fermée de $K^{*}$, comme nous le montrerons au $\mathrm{n}^{\circ}$ suivant).

Soit $K(V)$ le corps des fractions de $A$; d'après le $n^{\circ} 36$, on peut identifier $\mathcal{O}_{x, v}$ à l'anneau des fractions de $A$ relativement à l'idéal maximal $\mathrm{m}_{x}$ formé par les polynômes nuls en $x$, et l'on a $\Gamma\left(V, \mathcal{O}_{v}\right)=\mathrm{A}=\bigcap_{x \in V} \mathcal{O}_{x, v}$ (tous les $\theta_{x, v}$ étant considérés comme des sous-anneaux de $K(V)) .$ Mais tout idéal maximal de $A$ est égal à l'un des $\mathfrak{m}_{x}$, puisque $K$ est algébriquement clos (théorème des zéros de Hilbert); il en résulte immédiatement (cf. [8], Chap. XV, §5, th. X) que $A=\bigcap_{x \in V} \mathcal{O}_{x, v}=\Gamma\left(V, \mathcal{O}_{v}\right)$, cqfd.

Proposition 5. Soient $X$ une variété algébrique irréductible, $Q$ une fonction régulière sur $X$, et $P$ une fonction régulière sur $X_{\mathbf{Q}} .$ Alors, pour tout $n$ assez grand, la fonction rationnelle $Q^{n} P$ est régulière sur $X$ tout entier.

Vu la quasi-compacité de $X$, la question est locale; d'après le Théorème 1, on peut donc supposer que $X$ est une sous-variété fermée de $K^{r}$. La Proposition précédente montre alors que $Q$ est un élément de $\left.A=K \mid X_{1}, \cdots, X_{r}\right] / I(X)$. L'hypothèse faite sur $P$ signifie que, pour tout point $x \in X_{Q}$, on peut écrire $P=P_{x} / Q_{x}$, avec $P_{x}$ et $Q_{x}$ dans $A$, et $Q_{x}(x) \neq 0 ;$ si a désigne l'idéal de $A$ engendré par les $Q_{x}$, la variété des zéros de $\mathfrak{a}$ est contenue dans la variété des zéros de $Q$; en vertu du théorème des zéros de Hilbert, ceci entraine $Q^{n} \epsilon \mathfrak{a}$ pour $n$ assez grand, d'où $Q^{n}=\sum R_{x} \cdot Q_{x}$ et $Q^{n} P=\sum R_{x} \cdot P_{x}$ avec $R_{x} \in A$, ce qui montre bien que $Q^{n} P$ est régulière sur $X$.

(On aurait pu également utiliser le fait que $X_{Q}$ est affine si $X$ l'est, et appliquer la Proposition 4 à $X_{\mathbf{Q}}$ ).

Proposition 6. Soient $X$ une variété algébrique irréductible, $Q$ une fonction régulière sur $X, \mathfrak{F}$ un faisceau algébrique cohérent sur $X$, et s une section de $\mathfrak{F}$ audessus de $X$ dont la restriction à $X_{Q}$ soit nulle. Alors, pour tout $n$ assez grand, la section $Q^{n}$ s est nulle sur $X$ tout entier.

La question étant encore locale, nous pouvons supposer:

(a) que $X$ est une sous-variété fermée de $K^{\prime}$,

(b) que $\mathcal{F}$ est isomorphe au conoyau d'un homomorphisme $\varphi: 0_{x}^{p} \rightarrow \mathcal{O}_{x}^{g}$,

(c) que $s$ est image d'une section $\sigma$ de $\mathcal{O}_{X}^{q}$.

(En effet, toutes ces conditions sont vérifiées localement).

Posons $A=\Gamma\left(X, \mathcal{O}_{\boldsymbol{x}}\right)=K\left[X_{1}, \cdots, X_{r}\right] / I(X) .$ La section $\sigma$ peut être identifiée à un système de $q$ éléments de $A .$ Soient, d'autre part,

$$
t_{1}=\varphi(1,0, \cdots, 0), \cdots, t_{p}=\varphi(0, \cdots, 0,1)
$$

les $t_{i}, 1 \leqq i \leqq p$, sont des sections de $\mathcal{O}_{X}^{q}$ au-dessus de $X$, donc peuvent aussı être identifiés à des systèmes de $q$ éléments de $A$. L'hypothèse faite sur $s$ signifie que, pour tout $x \in X_{Q}$, on a $\sigma(x) \in \varphi\left(\Theta_{x, X}^{p}\right)$, c'est-à-dire que $\sigma$ peut s'écrire sous la forme $\sigma=\sum_{i=11}^{i=p} f_{i} \cdot t_{i}$, avec $f_{i} \in \mathcal{O}_{x, X} ;$ ou, en chassant les dénominateurs, qu'il existe $Q_{x} \bar{\epsilon} A, Q_{x}(x) \neq 0$, tel que $Q_{x} \cdot \sigma=\sum_{i=1}^{i=p} R_{i} \cdot t_{i}$, avec $R_{i} \in A$. Le raisonnement fait plus haut montre alors que, pour tout $n$ assez grand, $Q^{n}$ appartient à l'idéal engendré par les $Q_{x}$, d'où $Q^{n} \sigma(x) \in \varphi\left(\theta_{x, X}^{p}\right)$ pour tout $x \in X$, ce qui signifie bien que $Q^{n} s$ est nulle sur $X$ tout entier.

\section{Nullité de certains groupes de cohomologie.}

Proposition 7. Soient $X$ une variété affine irréductible, $Q_{i}$ une famille finie de fonctions régulières sur $X$, ne s'annulant pas simultanément, et $1 \mathrm{l}$ le recouvrement ouvert de $X$ formé par les $X_{\boldsymbol{e}_{i}}=U_{i} .$ Si $\mathfrak{F}$ est un sous-faisceau algébrique cohérent de $\mathcal{O}_{X}^{p}$, on a $H^{q}(\mathfrak{U}, \mathfrak{F})=0$ pour tout $q>0$.

Quitte à remplacer Ul par un recouvrement équivalent, on peut supposer qu'aucune des fonctions $Q_{i}$ n'est identiquement nulle, autrement dit que l'on a $U_{i} \neq \emptyset$ pour tout $i$.

Soit $f=\left(f_{i_{0} \cdots i_{q}}\right)$ un $q$-cocycle de $\mathbb{l}$ à valeurs dans $\mathcal{F} .$ Chaque $f_{i_{0} \cdots i_{q}}$ est une section de $\mathfrak{F}$ sur $U_{i_{0}} \ldots_{i_{g}}$, donc peut être identifié à un système de $p$ fonctions régulières sur $U_{i_{i}} \ldots_{i} ;$ appliquant la Proposition 5 à $Q=Q_{i_{0}} \cdots Q_{i_{n}}$, on voit que, pour tout $n$ assez grand, $g_{i_{0} \cdots i_{q}}=\left(Q_{i_{0}} \cdots Q_{i_{q}}\right)^{n} f_{i_{0} \cdots i_{q}}$ est un système de $p$ fonctions régulières sur $X$ tout entier, autrement dit est une section de $\theta^{p}$ au-dessus de $X$. Choisissons un entier $n$ tel que ceci soit valable pour tous les svstèmes $i_{0}, \cdots, i_{o}$, ce qui est possible puisque ces systèmes sont en nombre fini. Considérons l'image de $g_{i_{0} \cdots i_{q}}$ dans le faisceau cohérent $\mathcal{O}_{X}^{p} / \mathfrak{F}$; c'est une section nulle sur $U_{i_{0}} \cdots i_{q} ;$ appliquant alors la Proposition 6, on voit que, pour tout $m$ assez grand, le produit de cette section par $\left(Q_{i_{0}} \cdots Q_{i_{q}}\right)^{m}$ est nul sur $X$ tout entier, ce qui signifie que $\left(Q_{i_{0}} \cdots Q_{i_{q}}\right)^{m} g_{i_{0} \cdots i_{q}}$ est une section de $\mathcal{F}$ sur $X$ tout entier. En posant $N=m+n$, on voit donc que Pon a construit des sections $h_{i_{0} \cdots i_{q}}$ de $\mathcal{F}$ au-dessus de $X$, qui coincident avec $\left(Q_{i_{0}} \cdots Q_{i_{q}}\right)^{N} f_{i_{0} \cdots i_{q}}$ sur $U_{i_{0} \cdots i_{q}}$.

Comme les $Q_{i}^{N}$ ne s'annulent pas simultanément, il existe des fonctions

$$
R_{i} \in \Gamma\left(X, \mathcal{O}_{x}\right)
$$

telles que $\sum R_{i} \cdot Q_{i}^{N}=1 .$ Posons alors, pour tout système $i_{0}, \cdots, i_{q-1}$ :

$$
k_{i_{0} \cdots i_{q-1}}=\sum_{i} R_{i} \cdot h_{i i_{0} \cdots i_{q-1}} /\left(Q_{i_{0}} \cdots Q_{i_{q-1}}\right)^{N}
$$

ce qui a un sens, puisque $Q_{i_{0}} \cdots Q_{i_{q-1}}$ est différent de 0 sur $U_{i_{0} \cdots i_{q-1}}$.

On définit ainsi une cochaine $k \in C^{q-1}(\mathfrak{l}, \mathfrak{F}) .$ Je dis que $f=d k$, ce qui démontrera la Proposition.

Il faut vérifier que $(d k)_{i_{0} \cdots i_{q}}=f_{i_{0} \cdots i_{q}} ;$ il suffira de montrer que ces deux sections coïncident sur $U=\cap U_{i}$, car elles coincideront alors partout, puisque ce sont des systèmes de $p$ fonctions rationnelles sur $X$ et que $U \neq \emptyset$. Or, au-dessus de $U$, on peut écrire

$$
k_{i_{0} \cdots i_{q-1}}=\sum_{i} R_{i} \cdot Q_{i}^{N} \cdot f_{i i_{0} \cdots i_{q-1}}
$$

d'où

$$
(d k)_{i_{0} \cdots i_{q}}=\sum_{j=0}^{j=q}(-1)^{q} \sum_{i} R_{i} \cdot Q_{i}^{N} \cdot f_{i i_{0} \cdots \hat{i}_{i} \cdots i_{q}}
$$

% ----------------------------------------------------------------------------------------------------------------------------------------------------------------
% ----------------------------------------------------------------------------------------------------------------------------------------------------------------
% ----------------------------------------------------------------------------------------------------------------------------------------------------------------
% ----------------------------------------------------------------------------------------------------------------------------------------------------------------

et, en tenant compte de ce que $f$ est un cocycle,

$$
(d k)_{i_{0} \cdots i_{q}}=\sum_{i} R_{i} \cdot Q_{i}^{N} \cdot f_{i_{0} \cdots i_{q}}=f_{i_{0} \cdots i_{q}}
$$

cqfd.

Corollatiee 1. $H^{q}(X, \mathfrak{F})=0$ pour $q>0$

En effet la Proposition 3 montre que les recouvrements du type utilisé dans la Proposition 7 sont arbitrairement fins.

CorolLarre 2. L'homomorphisme $\Gamma\left(X, \mathcal{O}_{X}^{p}\right) \rightarrow \Gamma\left(X, \mathcal{O}_{X}^{p} / \mathfrak{F}\right)$ est surjectif.

Cela résulte du Corollaire 1 ci-dessus et du Corollaire 2 à la Proposition $6 \mathrm{du}$ no $24 .$

CorolLAIRE $3 .$ Soit $V$ une sous-variété fermée de $K^{r}$, et soit

$$
A=K\left[X_{1}, \cdots, X_{r}\right] / I(V)
$$

L'homomorphisme $\iota: A \rightarrow \Gamma\left(V, \mathcal{O}_{v}\right)$ est bijectif.

On applique le Corollaire 2 ci-dessus avec $X=K^{r}, p=1, \mathscr{F}=g(V)$, faisceau d'idéaux défini par $V$; on obtient que tout élément de $\Gamma\left(V, O_{V}\right)$ est restriction d'une section de $\mathcal{O}$ sur $X$, c'est-à-dire d'un polynôme, d'après la Proposition 4 appliquée à $X$.

45. Sections d'un faisceau algébrique cohérent sur une variêté affine.

THÉonème 2. Soit $\mathcal{F}$ un faisceau algébrique cohérent sur une variété affine $X$. Pour tout $x \in X$, le $\mathcal{O}_{x, x}$-module $\mathfrak{F}_{x}$ est engendré par les éléments de $\Gamma(X, \mathcal{F})$

Puisque $X$ est affine, on peut la plonger comme sous-variété fermée dans un espace affine $K^{\prime} ;$ en prolongeant le faisceau $\mathcal{F}$ par 0 en dehors de $X$, on obtient un faisceau algébrique cohérent sur $K^{r}$ (cf. no 39$)$, et on est ramené à prouver le théorème pour ce nouveau faisceau. Autrement dit, nous pouvons supposer que $X=K^{r}$.

Vu la définition des faisceaux cohérents, il existe un recouvrement de $X$ formé d'ouverts au-dessus desquels $\mathfrak{F}$ est isomorphe à un quotient d'un faisceau $0^{p}$. Utilisant la Proposition 3, on voit qu'il existe un nombre fini de polynômes $Q_{i}$, ne s'annulant pas simultanément, et tels qu'il existe au-dessus de chaque $U_{i}=X_{Q_{i}}$ un homomorphisme surjectif $\varphi_{i}: 0^{p_{i}} \rightarrow \mathfrak{F} ;$ on peut en outre supposer qu'aucun de ces polynômes n'est identiquement nul.

Le point $x$ appartient à l'un des $U_{i}$, disons $U_{0} ;$ il est clair que $\mathscr{F}_{x}$ est engendré par les sections de $\mathfrak{F}$ sur $U_{0} ;$ comme $Q_{0}$ est inversible dans $\mathcal{O}_{x}$, il nous suffira donc de démontrer le lemme suivant:

LEMME 1. Si $s_{0}$ est une section de $\mathfrak{F}$ au-dessus de $U_{0}$, il existe un entier $N$ et une section s de $\mathfrak{F}$ au-dessus de $X$ tels que $s=Q_{0}^{N} \cdot s_{0}$ au-dessus de $U_{0}$.

D'après la Proposition $2, U_{i}$ ก $U_{0}$ est une variété affine, évidemment irréductible; en appliquant le Corollaire 2 de la Proposition 7 à cette variété et à $\varphi_{i}: 0^{p_{i}} \rightarrow \mathscr{F}$, on voit qu'il existe une section $\sigma_{0 i}$ de $\theta^{p_{i}}$ sur $U_{i} \cap U_{0}$ telle que $\varphi_{i}\left(\sigma_{0 i}\right)=s_{0}$ sur $U_{i} \cap U_{0} ;$ comme $U_{i} \cap U_{0}$ est le lieu des points de $U_{i}$ où $Q_{0}$ ne s'annule pas, on peut appliquer la Proposition 5 à $X=U_{i}, Q=Q_{0}$, et l'on voit ainsi qu'il existe, pour $n$ assez grand, une section $\sigma_{i}$ de $\mathcal{O}^{p_{i}}$ au-dessus de $U_{i}$ qui coincide avec $Q_{0}^{n} \cdot \sigma_{0 i}$ au-dessus de $U_{i} \cap U_{0} ;$ en posant $s_{i}^{\prime}=\varphi_{i}\left(\sigma_{i}\right)$, on obtient une section de $\mathfrak{F}$ au-dessus de $U_{i}$ qui coïncide avec $Q_{0}^{n} \cdot s_{0}$ au-dessus de $U_{i} \cap U_{0} .$ Les sections $s_{i}^{\prime}$ et $s_{j}^{\prime}$ coïncident sur $U_{i}$ n $U_{j} \cap U_{0} ;$ en appliquant la Proposition 6 à $s_{i}^{\prime}-s_{j}^{\prime}$ on voit que, pour $m$ assez grand, on a $Q_{0}^{m} \cdot\left(s_{i}^{\prime}-s_{j}^{\prime}\right)=0$ sur $U_{i}$ n $U_{j}$ tout entier. Les $Q_{0}^{m} \cdot s_{i}^{\prime}$ définissent alors une section unique $s$ de F sur $X$, et l'on a $s=Q_{0}^{n+m} s_{0}$ sur $U_{0}$, ce qui démontre le lemme, et achève la démonstration du Théorème 2 .

CorolLaIRe 1. Le faisceau $\mathcal{F}$ est isomorphe a un faisceau quotient d'un faisceau $\boldsymbol{O}_{\mathbf{X}}^{p} .$

Puisque $\mathfrak{F}_{x}$ est un $\mathcal{O}_{x, x}$-module de type fini, il résulte du théorème ci-dessus qu'l existe un nombre fini de sections de $\mathcal{F}$ engendrant $\mathcal{F}_{x} ;$ d'après la Proposition 1 du n 12, ces sections engendrent aussi $F_{y}$ pour $y$ assez voisin de $x .$ L'espace $X$ étant quasi-compact, on en conclut qu'il existe un nombre fini de sections $s_{1}, \cdots, s_{p}$ de $\mathcal{F}$ engendrant $\mathfrak{F}_{x}$ pour tout $x \in X$, ce qui signifie bien que $\mathfrak{F}$ est isomorphe à un faisceau quotient du faisceau $\mathcal{O}_{X}^{p}$.

CorolLaIRE 2. Soit $Q \stackrel{\alpha}{\rightarrow} B \stackrel{\beta}{\rightarrow}$ e une suite exacte de faisceaux algébriques cohérents sur une variété affine $X .$ La suite $\Gamma(X, \mathbb{Q}) \stackrel{\alpha}{\rightarrow} \Gamma(X, \beta) \stackrel{\beta}{\rightarrow} \Gamma(X, \mathcal{C})$ est alors exacte.

On peut supposer, comme dans la démonstration du Théorème 2, que $X$ est l'espace affine $K^{r}$, donc est irréductible. Posons $\mathscr{I}=\operatorname{Im}(\alpha)=\operatorname{Ker}(\beta)$; tout revient à voir que $\alpha: \Gamma(X, Q) \rightarrow \Gamma(X, g)$ est surjectif. Or, d'après le Corollaire 1, on peut trouver un homomorphisme surjectif $\varphi: \theta_{x}^{p} \rightarrow Q$, et, d'après le Corollaire 2 à la Proposition $7, \alpha \circ \varphi: \Gamma\left(X, \mathcal{O}_{X}^{p}\right) \rightarrow \Gamma(X, \mathscr{})$ est surjectif; il en est a fortiori de même de $\alpha: \Gamma(X, \alpha) \rightarrow \Gamma(X, g)$, cqfd.

46. Groupes de cohomologie d'une variété affine à valeurs dans un faisceau algébrique cohérent. Nous allons généraliser la Proposition 7:

THÉonème 3. Soient $X$ une variété affine, $Q_{i}$ une famille finie de fonctions régulières sur $X$, ne s'annulant pas simultanément, et $\mathfrak{l l}$ le recouvrement ouvert de $X$ formé par les $X_{{ }_{i}}=U_{i} .$ Si $\mathfrak{F}$ est un faisceau algébrique cohérent sur $X$, on a $H^{q}(\mathcal{l}, \mathfrak{F})=0$ pour tout $q>0$

Supposons d'abord $X$ irréductible. D'après le Corollaire 1 au Théorème 2 , on peut trouver une suite exacte

$$
0 \rightarrow \mathbb{R} \rightarrow \mathcal{O}_{X}^{p} \rightarrow \mathfrak{F} \rightarrow 0
$$

La suite de complexes: $0 \rightarrow C(\mathfrak{l}, \Omega) \rightarrow C\left(\mathfrak{U}, \mathcal{O}_{X}^{p}\right) \rightarrow C(\mathfrak{U}, \mathfrak{F}) \rightarrow 0$ est exacte en effet, cela revient à dire que toute section de $\mathfrak{s u r}$ un $U_{i_{0} \cdots i_{q}}$ est image d'une section de $\mathcal{O}_{X}^{p}$ sur $U_{i_{0} \cdots i_{q}}$, ce qui résulte du Corollaire 2 à la Proposition 7 , appliqué à la variété irréductible $U_{i_{0}} \ldots_{i_{q}} .$ Cette suite exacte de complexes donne naissance à une suite exacte de cohomologie:

$$
\cdots \rightarrow H^{q}\left(\mathfrak{u}, \mathcal{O}_{x}^{p}\right) \rightarrow H^{q}(\mathfrak{U}, \mathcal{F}) \rightarrow H^{q+1}(\mathfrak{u}, Q) \rightarrow \cdots
$$

et comme $H^{q}\left(\mathfrak{l}, \mathcal{O}_{X}^{p}\right)=H^{q+1}(\mathfrak{u}, \Omega)=0$ pour $q>0$ d'après la Proposition 7 on en conclut que $H^{q}(\mathfrak{l}, \mathcal{F})=0$. Passons maintenant au cas général. On peut plonger $X$ comme sous-variété fermée dans un espace affine $K^{r}$; d'après le Corollaire 3 à la Proposition 7, les fonctions $Q_{i}$ sont induites par des polynômes $P_{i} ;$ soit d'autre part $R_{j}$ un système fini de générateurs de l'idéal $I(X)$. Les fonctions $P_{i}, R_{j}$ ne s'annulent pas simultanément sur $K^{r}$, donc définissent un recouvrement ouvert $\mathfrak{l}^{\prime}$ de $K^{r}$; soit $\mathfrak{F}^{\prime}$ le faisceau obtenu en prolongeant $\mathfrak{F}$ par 0 en dehors de $X$; en appliquant ce que nous venons de démontrer à l'espace $K^{\prime}$, aux fonctions $P_{i}, R_{j}$, et au faisceau $F^{\prime}$, on voit que $H^{q}\left(\mathfrak{U}^{\prime}, \mathfrak{F}^{\prime}\right)=0$ pour $q>0 .$ Comme on vérifie immédiatement que le complexe $C\left(\mathfrak{u}^{\prime}, \mathfrak{F}^{\prime}\right)$ est isomorphe au complexe $C(11, \mathfrak{F})$, il s'ensuit bien que $H^{q}(\mathfrak{l}, \mathfrak{F})=0$, cqfd.

CorolLatre 1. Si $X$ est une variété affine, et $\mathfrak{F}$ un faisceau algébrique cohérent sur $X$, on a $H^{q}(X, F)=0$ pour tout $q>0$.

En effet les recouvrements du type utilisé dans le théorème précédent sont arbitrairement fins.

CorolLaIRe $2 .$ Soit $0 \rightarrow Q \rightarrow \mathbb{B} \rightarrow \mathcal{C} \rightarrow 0$ une suite exacte de faisceaux sur une variété affine $X .$ Si le faisceau a est algébrique cohérent, l'homomorphisme $\Gamma(X, \mathbb{B}) \rightarrow \Gamma(X, \mathcal{C})$ est surjectif

Cela résulte du Corollaire 1, où l'on fait $q=1$.

47. Recouvrements des variétés algébriques par des ouverts affines.

Proposition 8. Soit $X$ une variété affine, et soit $\mathfrak{l}=\left\{U_{i}\right\}_{i \in I}$ un recouvrement fini de $X$ par des ouverts affines. Si $\mathfrak{F}$ est un faisceau algébrique cohérent sur $X$, on a $H^{q}(\mathfrak{l}, \mathfrak{F})=0$ pour tout $q>0$.

D'après la Proposition 3, il existe des fonctions régulières $P_{j}$ sur $X$ telles que le recouvrement $\mathfrak{B}=\left\{X_{P_{j}}\right\}$ soit plus fin que $\mathfrak{u}$. Pour tout $\left(i_{0}, \cdots, i_{p}\right)$ le recouvrement $\mathfrak{B}_{i_{0} \cdots i_{p}}$ induit par $\mathfrak{B}$ sur $U_{i_{0} \cdots i_{p}}$ est défini par les restrictions des $P_{j}$ à $U_{i_{0} \cdots i_{p}}$; comme $U_{i_{0} \cdots i_{p}}$ est une variété affine d'après la Proposition 1 , on peut lui appliquer le Théorème 3 , et on en conclut que $H^{q}\left(\mathfrak{B}_{i_{0} \cdots i_{p}}, \mathfrak{F}\right)=0$ pour tout $q>0 .$ Appliquant alors la Proposition $5 \mathrm{du} \mathrm{n}^{\circ} 29$, on voit que

$$
H^{q}(\mathfrak{U}, \mathfrak{F})=H^{q}(\mathfrak{Y}, \mathfrak{F})
$$

et, comme $H^{q}(\mathfrak{V}, \mathfrak{F})=0$ pour $q>0$ d'après le Théorème 3, la Proposition est démontrée.

THÉokème 4. Soient $X$ une variété algébrique, $\mathfrak{F}$ un faisceau algébrique cohérent sur $X$, et $\mathfrak{l l}=\left\{U_{i}\right\}_{\text {itI }}$ un recouvrement fini de $X$ par des ouverts affines. L'homomorphisme $\sigma(\mathfrak{l}): H^{n}(\mathfrak{U}, \mathfrak{F}) \rightarrow H^{n}(X, \mathfrak{})$ est bijectif pour tout $n \geqq 0$

Considérons la famille $\mathfrak{B}^{\alpha}$ des recouvrements finis de $X$ par des ouverts affines. D'après le corollaire au Théorème 1, ces recouvrements sont arbitrairement fins. D'autre part, pour tout système $\left(i_{0}, \cdots, i_{p}\right)$, le recouvrement $\mathfrak{B}_{i_{0} \cdots i_{\mathfrak{n}}}^{\alpha}$ induit par $\mathfrak{B}^{\alpha}$ sur $U_{i_{0}} \ldots i_{p}$ est un recouvrement par des ouverts affines, d'après la Proposition 1; d'après la Proposition 8, on a donc $H^{q}\left(\mathfrak{B}_{i_{0} \cdots i_{p}}^{\alpha}, \mathfrak{F}\right)=0$ pour $q>0$. Les conditions (a) et (b) du Théorème 1, no 29 , étant vérifiées, le théorème en résulte.

THÉonÈme $5 .$ Soient $X$ une variété algébrique, et $\mathcal{U}=\left\{U_{i}\right\}_{\text {ieI }}$ un recouvrement fini de $X$ par des ouverts affines. Soit $0 \rightarrow Q \rightarrow B \rightarrow \mathcal{C} \rightarrow 0$ une suite exacte de faisceaux sur $X$, le faisceau Q étant algébrique cohérent. L'homomorphisme canonique $H_{0}^{q}(\mathfrak{l l}, \mathfrak{e}) \rightarrow H^{q}(\mathfrak{u}, \mathcal{C})$ (cf. no 24$)$ est bijectif pour tout $q \geqq 0$.

Il suffit évidemment de montrer que $C_{0}(\mathfrak{l l}, \mathcal{C})=C(\mathfrak{l}, \mathcal{C})$, c'est-à-dire que toute section de e au-dessus de $U_{i_{0} \ldots i_{0}}$ est image d'une section de $\mathbb{B}$ au-dessus de $U_{i_{0} \ldots i_{q}}$, ce qui résulte du Corollaire 2 au Théorème 3 .

CorolLatre 1. Soit $X$ une variété algébrique, et soit $0 \rightarrow Q \rightarrow B \rightarrow \mathcal{C} \rightarrow 0$ une suite exacte de faisceaux sur $X$, le faisceau Q étant algébrique cohérent. L'homomorphisme canonique $H_{0}^{q}(X, \mathcal{C}) \rightarrow H^{q}(X, \mathcal{C})$ est bijectif pour tout $q \geqq 0$.

C'est une conséquence immédiate des Théorèmes 1 et 5 .

Corollatre 2 . On a une suite exacte:

$$
\cdots \rightarrow H^{q}(X, B) \rightarrow H^{q}(X, \mathcal{C}) \rightarrow H^{q+1}(X, \alpha) \rightarrow H^{q+1}(X, \mathbb{B}) \rightarrow \cdots
$$

\section{$\S$ 4. Correspondance entre modules de type fini et faisceaux algébriques cohérents}

48. Faisceau associé à un module. Soient $V$ une variété affine, $\mathcal{O}$ le faisceau des anneaux locaux de $V ;$ l'anneau $A=\Gamma(V, \mathcal{O})$ sera appelé l'anneau de coordonnées de $V$, c'est une algèbre sur $K$ qui n'a pas d'autre élément nilpotent que 0. Si $V$ est plongée comme sous-variété fermée dans un espace affine $K^{\prime}$, on sait (cf. no 44) que $A$ s'identifie a l'algèbre quotient de $K\left[X_{1}, \cdots, X_{r}\right]$ par l'idéal des polynômes nuls sur $V$; il s'ensuit que l'algèbre $A$ est engendrée par un nombre fini d'éléments.

Inversement, on vérifie aisément que, si $A$ est une $K$-algèbre commutative sans élément nilpotent (autre que 0 ) et engendrée par un nombre fini d'éléments, il existe une variété affine $V$ telle que $A$ soit isomorphe à $\Gamma(V, \theta) ;$ de plus $V$ est déterminée à un isomorphisme près par cette propriété (on peut identifier $V$ à l'ensemble des caractères de $A$, muni de la topologie usuelle).

Soit $M$ un $A$-module; $M$ définit sur $V$ un faisceau constant, que nous noterons encore $M$; de même $A$ définit un faisceau constant, et le faisceau $M$ peut être considéré comme un faisceau de $A$-modules. Posons $Q(M)=0 \otimes_{A} M$, le faisceau $\mathcal{O}$ étant aussi considéré comme un faisceau de $A$-modules; il est clair que $\alpha(M)$ est un faisceau algébrique sur $V$. De plus, si $\varphi: M \rightarrow M^{\prime}$ est un $A$-homomorphisme, on a un homomorphisme $Q(\varphi)=1 \otimes \varphi: \mathbb{a}(M) \rightarrow Q\left(M^{\prime}\right) ;$ en d'autres termes, $Q(M)$ est un foncteur covariant du module $M$

Proposition 1. Le foncteur $\mathbb{Q}(M)$ est exact.

Soit $M \rightarrow M^{\prime} \rightarrow M^{\prime \prime}$ une suite exacte de $A$-modules. Il nous faut voir que la suite $Q(M) \rightarrow Q\left(M^{\prime}\right) \rightarrow Q\left(M^{\prime \prime}\right)$ est exacte, autrement dit que, pour tout $x \in V$, la suite:

$$
\mathcal{O}_{x} \otimes_{A} M \rightarrow \mathcal{O}_{x} \otimes_{A} M^{\prime} \rightarrow \mathcal{O}_{x} \otimes_{A} M^{\prime \prime}
$$

est exacte.

Or $\mathcal{O}_{x}$ n'est pas autre chose que l'anneau de fractions $A_{s}$ de $A, S$ étant l'ensemble des $f \in A$ tels que $f(x) \neq 0$ (pour la définition d'un anneau de fractions, cf. [8], [12] ou [13]). La Proposition 1 est donc un cas particulier du résultat suivant: LEmme 1. Soient $A$ un anneau, $S$ une partie multiplicativement stable de $A$ ne contenant pas $0, A_{s} l^{\prime}$ 'anneau de fractions de A relativement d $S .$ Si $M \rightarrow M^{\prime} \rightarrow M^{\prime \prime}$ est une suite exacte de $A$-modules, la suite $A_{s} \otimes_{A} M \rightarrow A_{s} \otimes_{A} M^{\prime} \rightarrow A_{s} \otimes_{A} M^{\prime \prime}$ est exacte.

Désignons par $M_{s}$ l'ensemble des fractions $m / s$, avec $m \in M, s \in S$, deux fractions $m / s$ et $m^{\prime} / s^{\prime}$ étant identifiées s'il existe $s^{\prime \prime} \in S$ tel que $s^{\prime \prime}\left(s^{\prime} \cdot m-s \cdot m^{\prime}\right)=0$ on voit facilement que $M_{s}$ est un $A_{s}$-module, et que P'application

$$
a / s \otimes m \rightarrow a \cdot m / s
$$

est un isomorphisme de $A_{s} \otimes_{A} M$ sur $M_{s}$; on est donc ramené à montrer que la suite

$$
M_{s} \rightarrow M_{s}^{\prime} \rightarrow M_{s}^{\prime \prime}
$$

est exacte, ce qui est immédiat.

Proposition 2. $\alpha(M)=0$ entraîne $M=0$.

Soit $m$ un élément de $M ;$ si $Q(M)=0$, on a $1 \otimes m=0$ dans $\theta_{x} \otimes_{A} M$ pour tout $x \in V .$ D'après ce qui précède, $1 \otimes m=0$ équivaut a l'existence d'un élément $s \epsilon A, s(x) \neq 0$, tel que $s \cdot m=0 ;$ l'annulateur de $m$ dans $M$ n'est donc contenu dans aucun idéal maximal de $A$, ce qui entraîne qu'il est égal à $A$, d'où $m=0$.

Proposition 3. Si M est un A-module de type fini, $a(M)$ est un faisceau algébrique cohérent sur $V$.

Puisque $M$ est de type fini et que $A$ est noethérien, $M$ est isomorphe au conoyau d'un homomorphisme $\varphi: A^{q} \rightarrow A^{p}$, et $a(M)$ est isomorphe au conoyau de $\alpha(\varphi): \alpha\left(A^{q}\right) \rightarrow Q\left(A^{p}\right) .$ Comme $Q\left(A^{p}\right)=0^{p}$ et $Q\left(A^{q}\right)=\theta^{q}$, il en résulte bien que $\alpha(M)$ est cohérent.

49. Module associé à un faisceau algébrique. Soit $\mathfrak{F}$ un faisceau algébrique sur $V$, et soit $\Gamma(\mathfrak{F})=\Gamma(V, \mathfrak{}) ;$ puisque $\mathfrak{F}$ est un faisceau de $\boldsymbol{O}$-modules, $\Gamma(\mathfrak{F})$ est muni d'une structure naturelle de $A$-module. Tout homomorphisme algébrique $\varphi: \mathfrak{F} \rightarrow \mathrm{G}$ définit un $A$-homomorphisme $\Gamma(\varphi): \Gamma(\mathfrak{F}) \rightarrow \Gamma(\mathrm{G}) .$ Si l'on a une suite exacte de faisceaux algébriques cohérents $\mathfrak{F} \rightarrow \mathcal{S} \rightarrow \mathfrak{C}$, la suite

$$
\Gamma(\mathfrak{F}) \rightarrow \Gamma(\mathrm{S}) \rightarrow \Gamma(\mathfrak{C})
$$

est exacte $\left(\mathrm{n}^{\circ} 45\right) ;$ en appliquant ceci à une suite exacte $\theta^{p} \rightarrow \mathfrak{F} \rightarrow 0$, on voit que $\Gamma(\mathfrak{F})$ est un $A$-module de type fini si $F$ est cohérent.

Les foncteurs $Q(M)$ et $\Gamma(\mathfrak{F})$ sont "réciproques" l'un de l'autre:

ThÉoRÈME 1. (a) Si $M$ est un A-module de type fini, $\Gamma(\alpha(M))$ est canoniquement isomorphe à $M$.

(b) Si $\mathcal{F}$ est un faisceau algébrique cohérent sur V, $\mathbb{A}(\Gamma(\mathfrak{F}))$ est canoniquement isomorphe a $\mathcal{F} .$

Démontrons d'abord (a). Tout élément $m \in M$ définit une section $\alpha(m)$ de $\alpha(M)$ par la formule: $\alpha(m)(x)=1 \otimes m \in \mathcal{O}_{x} \otimes_{A} M$; d'où un homomorphisme $\alpha: M \rightarrow \Gamma(Q(M)) .$ Lorsque $M$ est un module libre de type fini, $\alpha$ est bijectif (il suffit de le voir lorsque $M=A$, auquel cas c'est évident); si $M$ est un module de type fini quelconque, il existe une suite exacte $L^{1} \rightarrow L^{0} \rightarrow M \rightarrow 0$, où $L^{0}$ et $L^{1}$ sont libres de type fini; la suite $Q\left(L^{1}\right) \rightarrow a\left(L^{0}\right) \rightarrow \alpha(M) \rightarrow 0$ est exacte, donc aussi la suite $\Gamma\left(\alpha\left(L^{1}\right)\right) \rightarrow \Gamma\left(\alpha\left(L^{0}\right)\right) \rightarrow \Gamma(\alpha(M)) \rightarrow 0$. Le diagramme commutatif:

![](https://cdn.mathpix.com/cropped/171be0f982e6ae195d9eb998d7a9d582-06.jpg?height=110&width=386&top_left_y=226&top_left_x=248)

montre alors que $\alpha: M \rightarrow \Gamma(\alpha(M))$ est bijectif, ce qui démontre (a).

Soit maintenant $\mathfrak{F}$ un faisceau algébrique cohérent sur $V$. Si l'on associe à tout $s \in \Gamma(\mathfrak{F})$ l'élément $s(x) \in \mathfrak{F}_{x}$, on obtient un $A$-homomorphisme: $\Gamma(\mathfrak{F}) \rightarrow \mathfrak{F}_{x}$ qui se prolonge en un $\Theta_{x}$-homomorphisme $\beta_{x}: \mathcal{O}_{x} \otimes_{A} \Gamma(\mathfrak{F}) \rightarrow \mathfrak{F}_{x}$; on vérifie facilement que les $\beta_{x}$ forment un homomorphisme de faisceaux $\beta: \alpha(\Gamma(\mathfrak{})) \rightarrow \mathfrak{F} .$ Lorsque $\mathscr{F}=\theta^{p}$, l'homomorphisme $\beta$ est bijectif; il en résulte, par le même raisonnement que ci-dessus, que $\beta$ est bijectif pour tout faisceau algébrique cohérent $\mathcal{F}$, ce qui démontre (b).

RemarQues. (1) On peut également déduire (b) de (a); cf. n 65 , démonstration de la Proposition 6 .

(2) Nous verrons au Chapitre III comment il faut modifier la correspondance précédente lorsqu'on étudie les faisceaux cohérents sur l'espace projectif.

50. Modules projectifs et espaces fibrés à fibres vectorielles. Rappelons ([6], Chap. I, th. 2.2) qu'un $A$-module $M$ est dit projectif s'il est facteur direct d'un $A$-module libre.

Proposition $4 .$ Soit $M$ un A-module de type fini. Pour que $M$ soit projectif, il faut et il suffit que le $\mathcal{O}_{x}$-module $\mathcal{O}_{x} \otimes_{A} M$ soit libre pour tout $x \in V$.

Si $M$ est projectif, $\mathcal{O}_{x} \otimes_{A} M$ est $\mathcal{O}_{x}$-projectif, donc $\mathcal{O}_{x}$-libre puisque $\mathcal{O}_{x}$ est un anneau local (cf. [6], Chap. VIII, th. 6.1').

Réciproquement, si tous les $\mathcal{O}_{x} \otimes_{A} M$ sont libres, on a

$$
\operatorname{dim}(M)=\operatorname{Sup} \operatorname{dim}_{x \in V}\left(O_{x} \otimes_{A} M\right)=0 \quad \text { (cf. [6], Chap. VII, Exer. 11), }
$$

ce qui entraine que $M$ est projectif ([6], Chap. VI, §2).

Remarquons que, si $\mathfrak{F}$ est un faisceau algébrique cohérent sur $V$, et si $\mathfrak{F}_{x}$ est isomorphe à $\mathcal{O}_{x}^{p}, \mathfrak{F}$ est isomorphe à $\theta^{p}$ au-dessus d'un voisinage de $x$; si cette propriété est vérifiée en tout point $x \in V$, le faisceau $\mathfrak{F}$ est donc localement isomorphe à un faisceau $\theta^{p}$, l'entier $p$ restant constant sur toute composante connexe de $V$. En appliquant ceci au faisceau $Q(M)$, on obtient:

CorolLaIRe. Soit $\mathfrak{F}$ un faisceau algébrique cohérent sur une variété affine connexe $V$. Les trois propriétés suivantes sont équivalentes:

(i) $\Gamma(\mathfrak{F})$ est un A-module projectif.

(ii) $\mathfrak{F}$ est localement isomorphe à un faisceau $\mathcal{O}^{p}$.

(iii) $\mathcal{F}$ est isomorphe au faisceau des germes de sections d'un espace fibré algébrique a fibre vectorielle de base $V$. En outre, l'application $E \rightarrow \Gamma(\delta(E))(E$ désignant un espace fibré à fibre vectorielle) met en correspondance biunivoque les classes d'espaces fibrés et les classes de $A$-modules projectifs de type fini; dans cette correspondance, un espace fibré trivial correspond à un module libre, et réciproquement.

Signalons que, lorsque $V=K^{r}$ (auquel cas $\left.A=K\left[X_{1}, \cdots, X_{r}\right]\right)$, on ignore s'il existe des $A$-modules projectifs de type fini qui ne soient pas libres, ou, ce qui revient au même, s'il existe des espaces fibrés algébriques à fibres vectorielles, de base $K^{\top}$, et non triviaux.

\section{CHAPITRE III. FAISCEAUX ALGÉBRIQUES COHÉRENTS SUR LES VARIÉTÉS PROJECTIVES}

\section{§1. Variétés projectives}

51. Notations. (Les notations introduites ci-dessous seront utilisées sans référence dans toute la suite du chapitre).

Soit $r$ un entier $\geqq 0$, et soit $Y=K^{r+1}-\{0\} ;$ le groupe multiplicatif $K^{*}$ des éléments $\neq 0$ de $K$ opère sur $Y$ par la formule:

$$
\lambda\left(\mu_{0}, \cdots, \mu_{r}\right)=\left(\lambda \mu_{0}, \cdots, \lambda \mu_{r}\right)
$$

Deux points $y$ et $y^{\prime}$ seront dits équivalents s'il existe $\lambda \in K^{*}$ tel que $y^{\prime}=\lambda y$; l'espace quotient de $Y$ par cette relation d'équivalence sera noté $\mathbf{P}_{r}(K)$, ou simplement $X$; c'est l'espacc projectif de dimension $r$ sur $K$; la projection canonique de $Y$ sur $X$ sera notée $\pi$.

Soit $I=\{0,1, \cdots, r\} ;$ pour tout $i \in I$, nous désignerons par $t_{i}$ la $i$-ème fonction coordonnée sur $K^{r+1}$, définie par la formule:

$$
t_{i}\left(\mu_{0}, \cdots, \mu_{r}\right)=\mu_{i}
$$

Nous désignerons par $V_{i}$ le sous-ensemble ouvert de $K^{r+1}$ formé des points où $t_{i}$ est $\neq 0$, et par $U_{i}$ l'image de $V_{i}$ par $\pi$; les $\left\{U_{i}\right\}_{i \in I}$ forment un recouvrement $\mathfrak{u}$ de $X$. Si $i \in I$ et $j \in I$, la fonction $t_{j} / t_{i}$ est régulière sur $V_{i}$, et invariante par $K^{*}$, donc définit une fonction sur $U_{i}$ que nous noterons encore $t_{j} / t_{i}$; pour $i$ fixé, les fonctions $t_{j} / t_{i}, j \neq i$, définissent une bijection $\psi_{i}: U_{i} \rightarrow K^{r}$.

Nous munirons $K^{r+1}$ de sa structure de variété algébrique, et $Y$ de la structure induite. Nous munirons également $X$ de la topologie quotient de celle de $Y$ : un sous-ensemble fermé de $X$ est donc l'image par $\pi$ d'un cône fermé de $K^{r+1}$. $\mathrm{Si}_{U} \boldsymbol{e s t}_{\text {ouvert dans }} X$ nous noserons $A_{y}=\Gamma\left(\pi^{-1}(U), \mathcal{O}_{Y}\right)$ : c'est l'anneau des fonctions régulières sur $\pi^{-1}(U) .$ Soit $A_{U}^{0}$ le sous-anneau de $A_{U}$ formé des éléments invariants par $K^{*}$ (c'est-à-dire des fonctions homogènes de degré 0 ). Lorsque $V \supset U_{,}$on a un homomorphisme de restriction $\varphi_{U}^{V}: A_{V}^{0} \rightarrow A_{U}^{0}$, et le système des $\left(A_{U}^{0}, \varphi_{U}^{V}\right)$ définit un faisceau $\mathcal{O}_{x}$ que l'on peut considérer comme un sous-faisceau du faisceau $\mathcal{F}(X)$ des germes de fonctions sur $X$. Pour qu'une fonction $f$, définie au voisinage de $x$, appartienne à $\mathcal{O}_{x, x}$, il faut et il suffit qu'elle coincide localement avec une fonction de la forme $P / Q$, où $P$ et $Q$ sont deux polynômes homo- gènes de même degré en $t_{0}, \cdots, t_{r}$, avec $Q(y) \neq 0$ pour $y \in \pi^{-1}(x)$ (ce que nous écrirons plus brièvement $Q(x) \neq 0$ )

Proposition 1. L'espace projectif $X=\mathbf{P}_{r}(K)$, muni de la topologie et du faisceau précédents, est une variété algébrique.

Les $U_{i}, i \in I$, sont des ouverts de $X$, et on vérifie tout de suite que les bijections $\psi_{i}: U_{i} \rightarrow K^{r}$ définies ci-dessus sont des isomorphismes biréguliers, ce qui montre que l'axiome $\left(\mathrm{VA}_{I}\right)$ est satisfait. Pour démontrer que $\left(\mathrm{VA}_{I I}\right)$ l'est aussi, il faut voir que la partie de $K^{r} \times K^{r}$ formée des couples $\left(\psi_{i}(x), \psi_{j}(x)\right)$, où $x \in U_{i} \cap U_{j}$, est fermée, ce qui ne présente pas de difficultés.

Dans la suite, $X$ sera toujours muni de la structure de variété algébrique qui vient d'être définie; le faisceau $\mathcal{O}_{x}$ sera simplement noté $\mathcal{O}$. Une variété algébrique $V$ sera dite projective si elle est isomorphe à une sous-variété fermée d'un espace projectif. L'étude des faisceaux algébriques cohérents sur les variétés projectives peut se ramener à l'étude des faisceaux algébriques cohérents sur les $\mathbf{P}_{r}(K)$, cf. $\mathrm{n}^{\circ} 39$

52. Cohomologie des sous-variétés de l'espace projectif. Appliquons le Théorème 4 du no 47 au recouvrement $\mathcal{U}=\left\{U_{i}\right\}_{i \in I}$, défini au $\mathrm{n}^{\circ}$ précédent: c'est possible puisque chacun des $U_{i}$ est isomorphe à $K^{r}$. On obtient ainsi:

Proposition 2. Si $\mathcal{F}$ est un faisceau algébrique cohérent sur $X=\mathbf{P}_{r}(K), l^{\prime} h o m o-$ morphisme $\sigma(\mathfrak{l}): H^{n}(\mathfrak{l}, \mathcal{F}) \rightarrow H^{n}(X, \mathcal{F})$ est bijectif pour tout $n \geqq 0$

Puisque $\mathfrak{l l}$ est formé de $r+1$ ouverts, on a (cf. no 20, corollaire à la Proposition 2):

Corollatre. $H^{n}(X, \mathfrak{F})=0$ pour $n>r$

Ce dernier résultat peut être généralisé de la façon suivante:

Proposition 3. Soit $V$ une variété algébrique, isomorphe à une sous-variété localement fermée d'un espace projectif $X .$ Soit $\mathfrak{F}$ un faisceau algébrique cohérent sur $V$, et soit $W$ une sous-variété de $V$ telle que $\mathfrak{F}$ soit nul en dehors de $W$. On a alors $H^{n}(V, \mathscr{})=0 \quad$ pour $n>\operatorname{dim} W$.

En particulier, prenant $W=V$, on voit que l'on a:

CorolLalRe. $H^{n}(V, \mathscr{})=0 \quad$ pour $n>\operatorname{dim} V$.

Identifions $V$ à une sous-variété localement fermée de $X=\mathrm{P}_{r}(K) ;$ il existe un ouvert $U$ de $X$ tel que $V$ soit fermée dans $U$. Nous supposerons que $W$ est fermée dans $V$, ce qui est évidemment licite; alors $W$ est fermée dans $U$. Posons $F=X-U$. Avant de démontrer la Proposition 3 , établissons deux lemmes:

LEMME 1. Soit $k=\operatorname{dim} W ;$ il existe $k+1$ polynômes homogènes $P_{i}\left(t_{0}, \cdots, t_{r}\right)$, de dearés $>0$. nuls sur $F$, et ne s'annulant pas simultanément sur $W$.

(Par abus de langage, on dit qu'un polynôme homogène $P$ s'annule en un point $x$ de $\mathbf{P}_{r}(K)$ s'il s'annule sur $\pi^{-1}(x)$.

Raisonnons par récurrence sur $k$, le cas où $k=-1$ étant trivial. Choisissons un point sur chaque composante irréductible de $W$, et soit $P_{1}$ un polynôme homogène nul sur $F$, de degré $>0$, et ne s'annulant en aucun de ces points (l'existence de $P_{1}$ résulte de ce que $F$ est fermé, compte tenu de la définition de la topologie de $\left.\mathbf{P}_{r}(K)\right) .$ Soit $W^{\prime}$ la sous-variété de $W$ formée des points $x \in W$ tels que $P_{1}(x)=0 ;$ vu la construction de $P_{1}$, aucune composante irréductible de $W$ n'est contenue dans $W^{\prime}$, et il s'ensuit (cf. $\mathrm{n}^{\circ} 36$ ) que $\operatorname{dim} W^{\prime}<k$. En appliquant l'hypothèse de récurrence à $W^{\prime}$, on voit qu'il existe $k$ polynômes homogènes $P_{2}, \cdots, P_{k+1}$, nuls sur $F$, et ne s'annulant pas simultanément sur $W^{\prime}$; il est clair que les polynômes $P_{1}, \cdots, P_{k+1}$ vérifient les conditions voulues.

LEmme 2. Soit $P\left(t_{0}, \cdots, t_{r}\right)$ un polynôme homogène de degré $n>0 .$ L'ensemble $X_{P}$ des points $x \in X$ tels que $P(x) \neq 0$ est un ouvert affine de $X$.

Si l'on fait correspondre à tout point $y=\left(\mu_{0}, \cdots, \mu_{r}\right) \in Y$ le point d'un espace $K^{N}$ convenable qui a pour coordonnées tous les monômes $\mu_{0}^{m_{0}} \cdots \mu_{r}^{m_{r}}, m_{0}+\cdots+$ $m_{r}=n$, on obtient, par passage au quotient, une application $\varphi_{n}: X \rightarrow \mathbf{P}_{N-1}(K)$. Il est classique, et d'ailleurs facile à vérifier, que $\varphi_{n}$ est un isomorphisme birégulier de $X$ sur une sous-variété fermée de $\mathbf{P}_{N-1}(K)$ ("variété de Veronese"); or $\varphi_{n}$ transforme l'ouvert $X_{P}$ en le lieu des points de $\varphi_{n}(X)$ non situés sur un certain hyperplan de $\mathbf{P}_{N-1}(K)$; comme le complémentaire d'un hyperplan est isomorphe à un espace affine, on en conclut que $X_{P}$ est bien isomorphe à une sous-variété fermée d'un espace affine.

Démontrons maintenant la Proposition 3. Prolongeons le faisceau $\mathfrak{F}$ par 0 sur $U-V ;$ nous obtenons un faisceau algébrique cohérent sur $U$, que nous noterons encore $\mathcal{F}$, et l'on sait (cf. $\mathrm{n}^{\circ} 26$ ) que $H^{n}(U, \mathfrak{F})=H^{n}(V, \mathfrak{}) .$ Soient d'autre part $P_{1}, \cdots, P_{k+1}$ des polynômes homogènes vérifiant les conditions du Lemme 1 ; soient $P_{k+2}, \cdots, P_{h}$ des polynômes homogènes de degrés $>0$, nuls sur $W$ u $F$, et ne s'annulant simultanément en aucun point de $U-W$ (pour obtenir de tels polynômes, il suffit de prendre un système de générateurs homogènes de l'idéal défini $\operatorname{par} W$ u $F$ dans $\left.K\left[t_{0}, \cdots, t_{r}\right]\right) .$ Pour tout $i, 1 \leqq i \leqq h$, soit $V_{i}$ l'ensemble des points $x \in X$ tels que $P_{i}(x) \neq 0 ;$ on a $V_{i} \subset U$, et les hypothèses faites ci-dessus montrent que $\mathfrak{B}=\left\{V_{i}\right\}$ est un recouvrement ouvert de $U$ : de plus. le Lemme 2 montre que les $V_{i}$ sont des ouverts affines, d'où $H^{n}(\mathfrak{B}, \mathfrak{F})=H^{n}(U, \mathfrak{F})=H^{n}(V, \mathfrak{F})$ pour tout $n \geqq 0 .$ D'autre part, si $n>k$ et si les indices $i_{0}, \cdots, i_{n}$ sont distincts, l'un de ces indices est $>k+1$, et $V_{i_{0} \ldots i_{n}}$ ne rencontre pas $W$; on en conclut que le groupe des cochaînes alternées $C^{\prime n}(\mathfrak{B}, \mathfrak{})$ est nul si $n>k$, ce qui entraîne bien $H^{n}(\mathfrak{F}, \mathfrak{F})=0$, d'après la Proposition $2 \mathrm{du} \mathrm{n}^{\circ} 20$.

53. Cohomologie des courbes algébriques irréductibles. Si $V$ est une variété alǵbrique irréductible de dimension 1, les sous-ensembles fermés de $V$, distincts de $V$ sont les sous-ensembles finis. Si $F$ est une partie finie de $V$, et $x$ un point de $F$, nous poserons $V_{x}^{F}=(V-F) \cup\{x\} ;$ les $V_{x}^{p}, x \in F$, forment un recouvrement ouvert fini $\mathfrak{B}^{F}$ de $V$.

LEMME 3 . Les recouvrements $\mathfrak{B}^{F}$ du type précédent sont arbitrairement fins.

Soit $\mathcal{U}=\left\{U_{i}\right\}_{i s}$ un recouvrement ouvert de $V$, que l'on peut supposer fini, puisque $V$ est quasi-compact. On peut également supposer $U_{i} \neq \emptyset$ pour tout $i \in I$. Si l'on pose $F_{i}=V-U_{i}, F_{i}$ est donc fini, et il en est de même de $F=U_{i \epsilon} F_{i} .$ Montrons que $\mathfrak{B}^{F}<\mathfrak{u}$, ce qui démontrera le lemme. Soit $x \in F$; il existe $i \in I$ tel que $x \in F_{i}$, puisque les $U_{i}$ recouvrent $V ;$ on a alors $F-\{x\} \supset F_{i}$ puisque $F \supset F_{i}$, ce qui signifie que $V_{x} \subset \cup_{i}$, et démontre bien que $\mathfrak{B}^{\prime}<\mathfrak{U} .$ LEmme 4. Soient $\mathfrak{F}$ un faisceau sur $V$, et $F$ une partie finie de $V$. On a

$$
H^{n}\left(\mathfrak{B}^{p}, \mathfrak{F}\right)=0
$$

pour $n \geqq 2$.

Posons $W=V-F ;$ il est clair que $V_{x_{0}}^{F} \cap \cdots$ n $V_{x_{n}}^{P}=W$ si les $x_{0}, \cdots, x_{n}$ sont distincts, et si $n \geqq 1$. Si l'on pose $G=\Gamma(W, \mathfrak{})$, il en résulte que le complexe alterné $C^{\prime}\left(\mathfrak{P}^{P}, \mathcal{F}\right)$ est isomorphe, en dimensions $\geqq 1$, à $C^{\prime}(S(F), G), S(F)$ désignant le simplexe ayant $F$ pour ensemble de sommets. Il s'ensuit que

$$
H^{n}\left(\mathfrak{P}^{P}, \mathfrak{F}\right)=H^{n}(S(F), G)=0 \text { pour } n \geqq 2
$$

la cohomologie d'un simplexe étant triviale.

Les Lemmes 3 et 4 entraînent évidemment:

Proposition 4. Si $V$ est une courbe algébrique irréductible, et $\mathfrak{F}$ un faisceau quelconque sur $V$, on a $H^{n}(V, \mathfrak{F})=0$ pour $n \geqq 2$.

ReMARQUE. J'ignore si un résultat analogue au précédent est valable pour les variétés de dimension quelconque.

§2. Modules gradués et faisceaux algébriques cohérents sur l'espace projectif

54. L'opération $\mathfrak{F}(n) .$ Soit $\mathfrak{F}$ un faisceau algébrique sur $X=\mathbf{P}_{r}(K) .$ Soit $\mathfrak{F}_{i}=\mathfrak{F}\left(U_{i}\right)$ la restriction de $\mathfrak{F}$ à $U_{i}\left(\right.$ cf. $\left.\mathrm{n}^{\circ} 51\right) ; n$ désignant un entier quelconque, soit $\theta_{i i}(n)$ l'isomorphisme de $\mathcal{F}_{i}\left(U_{i} \cap U_{j}\right)$ sur $\mathcal{F}_{i}\left(U_{i} \cap U_{j}\right)$ défini par la multiplication par la fonction $t_{i}^{n} / t_{i}^{n}$ : cela a un sens. puisque $t_{i} / t_{i}$ est une fonction régulière sur $U_{i}$ ก $U_{j}$ et à valeurs dans $K^{*}$. On a $\theta_{i j}(n) \circ \theta_{j k}(n)=\theta_{i k}(n)$ en tout point de $U_{i}$ ก $U_{j} \cap U_{k} ;$ on peut donc appliquer la Proposition $4 \mathrm{du} \mathrm{n}^{\circ} 4$, et l'on obtient ainsi un faisceau algébrique, noté $\mathfrak{F}(n)$, défini par recollement des faisceaux $\mathfrak{F}_{i}=\mathfrak{F}\left(U_{i}\right)$ au moyen des isomorphismes $\theta_{i j}(n)$.

On a des isomorphismes canoniques: $\mathfrak{F}(0) \approx \mathfrak{F}, \mathcal{F}(n)(m) \approx \mathfrak{F}(n+m)$. De plus, $\mathfrak{F}(n)$ est localement isomorphe à $\mathfrak{F}$, donc cohérent si $\mathfrak{F}$ l'est; il en résulte également que toute suite exacte $\mathfrak{F} \rightarrow \mathcal{F}^{\prime} \rightarrow \mathfrak{F}^{\prime \prime}$ de faisceaux algébriques donne naissance à une suite exacte $\mathcal{F}(n) \rightarrow \mathfrak{F}^{\prime}(n) \rightarrow \mathfrak{F}^{\prime \prime}(n)$ pour tout $n \in \mathbf{Z}$.

On peut appliquer ce qui précède au faisceau $\mathfrak{F}=\mathcal{O}$, et l'on obtient ainsi les faisceaux $\mathcal{O}(n), n \in \mathrm{Z}$. Nous allons donner une autre description de ces faisceaux: si $U$ est ouvert dans $X$, soit $A_{U}^{n}$ la partie de $A_{U}=\Gamma\left(\pi^{-1}(U), \theta_{Y}\right)$ formée des fonctions homogènes de degré $n$ (c'est-a-dire vérifiant l'identité $f(\lambda y)=\lambda^{n} f(y)$ pour $\lambda \epsilon K^{*}$, et $\left.y \in \pi^{-1}(U)\right)$; les $A_{U}^{n}$ sont des $A_{U}^{0}$-modules, donc donnent naissance à un faisceau algébrique, que nous désignerons par $\mathcal{O}^{\prime}(n)$. Un élement de $\mathcal{O}^{\prime}(n)_{x}$, $x \in X$, peut donc être identifié à une fraction rationnelle $P / Q, P$ et $Q$ étant des polynômes homogènes tels que $Q(x) \neq 0$ et que deg $P-\operatorname{deg} Q=n$

Proposition 1. Les faisceaux $\mathcal{O}(n)$ et $\mathcal{O}^{\prime}(n)$ sont canoniquement isomorphes.

Par définition, une section de $\mathcal{O}(n)$ sur un ouvert $U \subset X$ est un système $\left(f_{i}\right)$ de sections de $\mathcal{O}$ sur les $U$ ก $U_{i}$, avec $f_{i}=\left(t_{j}^{n} / t_{i}^{n}\right) \cdot f_{j}$ sur $U \cap U_{i}$ ก $U_{j}$; les $f_{j}$ neuvent être identifiées à des fonctions régulières et homogènes de degré 0 sur les $\pi^{-1}(U) \cap \pi^{-1}\left(U_{i}\right)$; posons $g_{i}=t_{i}^{n} \cdot f_{i} ;$ on a alors $g_{i}=g_{j}$ en tout point de $\pi^{-1}(U) \cap \pi^{-1}\left(U_{i}\right) \cap \pi^{-1}\left(U_{j}\right)$, donc les $g_{i}$ sont restrictions d'une fonction unique $g$, régulière sur $\pi^{-1}(U)$, et homogène de degré $n$. Inversement, une telle fonction $g$ définit un système $\left(f_{i}\right)$ en posant $f_{i}=g / t_{i}^{n} .$ L'application $\left(f_{i}\right) \rightarrow g$ est donc un isomorphisme de $\mathcal{O}(n) \operatorname{sur} \mathcal{O}^{\prime}(n)$.

Dans la suite, nous identifierons le plus souvent $\mathcal{O}(n)$ et $\mathcal{O}^{\prime}(n)$ au moyen de l'isomorphisme précédent. On observera qu'une section de $\mathcal{O}^{\prime}(n)$ au-dessus de $X$ n'est pas autre chose qu'une fonction régulière sur $Y$ et homogène de degré $n .$ Si l'on suppose $r \geqq 1$, une telle fonction est identiquement nulle pour $n<0$, et c'est un polynôme homogène de degré $n$ pour $n \geqq 0$.

Proposition 2. Pour tout faisceau algébrique $\mathcal{F}$, les faisceaux $\mathfrak{F}(n)$ et $\mathfrak{F} \otimes_{0} \mathcal{O}(n)$ sont canoniquement isomorphes.

Puisque $\mathcal{O}(n)$ est obtenu à partir des $\mathcal{O}_{i}$ par recollement au moyen des $\theta_{i j}(n)$, $F \otimes O(n)$ est obtenu à partir des $\mathfrak{F}_{i} \otimes \mathcal{O}_{i}$ par recollement au moyen des isomorphismes $1 \otimes \theta_{i j}(n) ;$ en identifiant $\mathcal{F}_{i} \otimes \mathcal{O}_{i}$ à $\mathfrak{F}_{i}$, on retrouve bien la définition de $\mathfrak{F}(n)$

Dans la suite, nous ferons également l'identification de $\mathcal{F}(n)$ et de $\mathfrak{F} \otimes \mathcal{O}(n)$.

55. Sections de $\mathfrak{( n )} .$ Démontrons d'abord un lemme sur les variétés affines, qui est tout à fait analogue au Lemme $1 \mathrm{du} \mathrm{n}^{\circ} 45:$

Lemme 1. Soient $V$ une variété affine, $Q$ une fonction régulière sur $V$, et $V_{Q}$ l'ensemble des points $x \in V$ tels que $Q(x) \neq 0 .$ Soit $\mathfrak{F}$ un faisceau algébrique cohérent sur $V$, et soit s une section de $\mathfrak{F}$ au-dessus de $V_{\mathbf{Q}} .$ Alors, pour tout $n$ assez grand, il existe une section $s^{\prime}$ de $\mathfrak{F}$ au-dessus de $V$ tout entier, telle que $s^{\prime}=Q^{n} s$ au-dessus de $V_{Q}$

En plongeant $V$ dans un espace affine, et prolongeant $\mathfrak{F}$ par 0 en dehors de $V$, on se ramène au cas où $V$ est un espace affine, donc est irréductible. D'après le Corollaire 1 au Théorème 2 du $\mathrm{n}^{\circ} 45$, il existe un homomorphisme surjectif $\varphi: \Theta_{V}^{p} \rightarrow \mathfrak{F} ;$ d'après la Proposition $2 \mathrm{du} \mathrm{n}^{\circ} 42, V_{Q}$ est un ouvert affine, et il existe donc $\left(\mathrm{n}^{\circ} 44\right.$, Corollaire 2 à la Proposition 7$)$ une section $\sigma$ de $\mathcal{O}_{V}^{p}$ au-dessus de $V_{\odot}$ telle que $\varphi(\sigma)=s .$ On peut identifier $\sigma$ à un système de $p$ fonctions régulières sur $V_{Q} ;$ appliquant à chacune de ces fonctions la Proposition 5 du $\mathrm{n}^{\circ} 43$, on voit qu'il existe une section $\sigma^{\prime}$ de $\mathcal{O}_{V}^{p}$ sur $V$ telle que $\sigma^{\prime}=Q^{n} \sigma$ sur $V_{Q}$, pourvu que $n$ soit assez grand. En posant $s^{\prime}=\varphi\left(\sigma^{\prime}\right)$, on obtient bien une section de $\mathfrak{F}$ sur $V$ telle que $s^{\prime}=Q^{n} s$ sur $V_{Q}$

TuÉonème 1. Soit $\mathfrak{F}$ un faisceau algébrique cohérent sur $X=\mathbf{P}_{r}(K) .$ Il existe un entier $n(\mathfrak{F})$ tel que, pour tout $n \geqq n(\mathfrak{F})$, et tout $x \in X$, le $\mathcal{O}_{x}$-module $\mathfrak{F}(n)_{x}$ soit engendré par les éléments de $\Gamma(X, \mathscr{( n )})$.

Par définition de $\mathfrak{F}(n)$, une section $s$ de $\mathfrak{F}(n)$ sur $X$ est un système $\left(s_{i}\right)$ de sections de $\mathcal{F}$ sur les $U_{i}$, vérifiant les conditions de cohérence:

$$
s_{i}=\left(t_{j}^{n} / t_{i}^{n}\right) \cdot s_{j} \text { sur } U_{i} \cap U_{j}
$$

nous dirons que $s_{i}$ est la $i$-ème composante de s.

D'autre part, puisque $U_{i}$ est isomorphe à $K^{r}$, il existe un nombre fini de sections $s_{i}^{\alpha}$ de $\mathfrak{F}$ sur $U_{i}$ qui engendrent $\mathcal{F}_{x}$ pour tout $x \in U_{i}\left(\mathrm{n}^{\circ} 45\right.$, Corollaire 1 au Théorème 2); si, pour un certain entier $n$, on peut trouver des sections $s^{\alpha}$ de $\mathfrak{F}(n)$ dont les $i$-èmes composantes soient les $s_{i}^{\alpha}$, il est évident que $\Gamma(X, \mathfrak{F}(n))$ engendre $\mathcal{F}(n)_{x}$ pour tout $x \in U_{i} .$ Le Théorème 1 sera donc démontré si nous prouvons le Lemme suivant:

LEmme 2. Soit $s_{i}$ une section de $\mathfrak{F}$ au-dessus de $U_{i}$. Pour tout $n$ assez grand, $i l$ existe une section s de $\mathfrak{F}(n)$ dont la $i$-ème composante est égale à $s_{i}$.

Appliquons le Lemme 1 à la variété affine $V=U_{j}$, à la fonction $Q=t_{i} / t_{j}$ et à la section $s_{i}$ restreinte à $U_{i}$ n $U_{j} ;$ c'est licite, puisque $t_{i} / t_{j}$ est une fonction régulière sur $U_{j}$ dont le lieu des zéros est $U_{j}-U_{i}$ n $U_{j} .$ On en conclut qu'il existe un entier $p$ et une section $s_{i}^{\prime}$ de $\mathfrak{F}$ sur $U_{i}$ tels que $s_{i}^{\prime}=\left(t_{i}^{p} / t_{i}^{p}\right) \cdot s_{i}$ sur $U_{i}$ ก $U_{j}$; pour $j=i$, ceci entraine $s_{i}^{\prime}=s_{i}$, ce qui permet d'écrire la formule précédente $s_{j}^{\prime}=\left(t_{i}^{p} / t_{j}^{p}\right) \cdot s_{i}^{\prime}$

Les $s_{j}^{\prime}$ étant définies pour tout indice $j$ (avec le même exposant $p$ ), considérons $s_{i}^{\prime}-\left(t_{k}^{p} / t_{i}^{p}\right) \cdot s_{k}^{\prime} ;$ c'est une section de $\mathfrak{F}$ sur $U_{i}$ ก $U_{k}$ dont la restriction à $U_{i}$ ก $U_{j} \cap U_{k}$ est nulle; en lui appliquant la Proposition 6 du n $^{\circ} 43$, on voit que, pour tout entier $q$ assez grand, on a $\left(t_{i}^{q} / t_{j}^{q}\right)\left(s_{j}^{\prime}-\left(t_{k}^{p} / t_{j}^{p}\right) \cdot s_{k}^{\prime}\right)=0$ sur $U_{j}$ ก $U_{k}$; si on pose alors $s_{j}=\left(t_{i}^{q} / t_{j}^{q}\right) \cdot s_{j}^{\prime}$, et $n=p+q$, la formule précédente s'écrit $s_{i}=\left(t_{k}^{n} / t_{i}^{n}\right), s_{k}$, et le svstème $s=\left(s_{i}\right)$ est bien une section de $\mathfrak{F}(n)$ dont la $i$-ème composante est égale à $s_{i}$, cqfd.

Coroluatre. Tout faisceau algébrique cohérent $\mathfrak{F}$ sur $X=\mathbf{P}_{r}(K)$ est isomorphe d un faisceau quotient d'un faisceau $\mathcal{O}(n)^{p}, n$ et $p$ étant des entiers convenables.

D'anrès le théorème qui précède. il existe un entier $n$ tel que $\mathfrak{F}(-n)_{x}$ soit engendré par $\Gamma(X, \mathfrak{F}(-n))$ pour tout $x \in X ;$ vu la quasi-compacité de $X$, cela équivaut à dire que $\mathcal{F}(-n)$ est isomorphe à un faisceau quotient du faisceau $\theta^{p}, p$ étant un entier $\geqq 0$ convenable. Il en résulte alors que $\mathcal{F} \approx \mathfrak{F}(-n)(n)$ est isomorphe à un faisceau quotient de $\mathcal{O}(n)^{p} \approx \mathcal{O}^{p}(n)$

56. Modules gradués. Soit $S=K\left[t_{0}, \cdots, t_{r}\right]$ l'algèbre des polynômes en $t_{0}, \cdots, t_{r} ;$ pour tout entier $n \geqq 0$, soit $S_{n}$ le sous-espace vectoriel de $S$ formé par les polynômes homogènes de degré $n ;$ pour $n<0$, on posera $S_{n}=0$. L'algèbre $S$ est somme directe des $S_{n}, n \in \mathrm{Z}$, et l'on a $S_{p} \cdot S_{q} \subset S_{p+q} ;$ autrement dit, $S$ est une algèbre graduée

Rappelons qu'un $S$-module $M$ est dit gradué lorsqu'on s'est donné une décomposition de $M$ comme somme directe: $M=\sum_{n e z} M_{n}$, les $M_{n}$ étant des sous-groupes de $M$ tels que $S_{p} \cdot M_{q} \subset M_{p+q}$, pour tout couple d'entiers $(p, q)$. Un élément de $M_{n}$ est dit homogène de degré $n ;$ un sous-module $N$ de $M$ est dit homogène s'il est somme directe des $N$ n $M_{n}$, auquel cas c'est un $S$-module gradué. Si $M$ et $M^{\prime}$ sont deux $S$-modules gradués, un $S$-homomorphisme

$$
\varphi: M \rightarrow M^{\prime}
$$

est dit homogène de degré $s$ si $\varphi\left(M_{n}\right) \subset M_{n+s}^{\prime}$ pour tout $n \in \mathrm{Z}$. Un S-homomorphisme homogène de degré 0 sera appelé simplement un homomorphisme.

Si $M$ est un $S$-module gradué, et $n$ un entier, nous noterons $M(n)$ le $S$-module gradué:

$$
M(n)=\sum_{p \in z} M(n)_{p}, \text { avec } M(n)_{p}=M_{n+p}
$$

On a donc $M(n)=M$ en tant que $S$-module, mais un élément homogène de degré $p$ dans $M(n)$ est homogène de degré $n+p$ dans $M$; autrement dit, $M(n)$ se déduit de $M$ en abaissant les degrés de $n$ unités.

Nous désignerons par e la classe des $S$-modules gradués $M$ tels que $M_{n}=0$ pour $n$ assez grand. Si $A \rightarrow B \rightarrow C$ est une suite exacte d'homomorphismes de $S$-modules gradués, les relations $A \in \mathcal{C}$ et $C \in \mathcal{C}$ entrainent évidemment $B \in \mathrm{C} ;$ autrement dit, e est bien une classe, au sens de [14], Chap. I. De façon générale, nous utiliserons la terminologie introduite dans l'article précité; en particulier, un homomorphisme $\varphi: A \rightarrow B$ sera dit e-injectif (resp. e-surjectif) si $\operatorname{Ker}(\varphi) \in \mathcal{C}$ (resp. si Coker $(\varphi) \in \mathcal{C})$, et e-bijectif s'il est à la fois e-injectif et e-surjectif.

Un S-module gradué $M$ est dit de type fini s'il est engendré par un nombre fini d'éléments; nous dirons que $M$ vérifie la condition (TF) s'il existe un entier $p$ tel que le sous-module $\sum_{n \geq p} M_{n}$ de $M$ soit de type fini; il revient au même de dire que $M$ est $\mathcal{C}$-isomorphe a un module de type fini. Les modules vérifiant (TF) forment une classe contenant $\mathcal{C}$.

Un $S$-module gradué $L$ est dit libre (resp. libre de type fini) s'il admet une base (resp. une base finie) formée d'éléments homogènes, autrement dit s'il est isomorphe à une somme directe (resp. une somme directe finie) de modules $S\left(n_{i}\right)$.

57. Faisceau algébrique associé à un $S$-module gradué. Si $U$ est une partie non vide de $X$, nous noterons $S(U)$ le sous-ensemble de $S=K\left[t_{0}, \cdots, t_{r}\right]$ formé des polynômes homogènes $Q$ tels que $Q(x) \neq 0$ pour tout $x \in U ; S(U)$ est un sous-ensemble multiplicativement stable de $S$, ne contenant pas 0 . Pour $U=X$, on écrira $S(x)$ au lieu de $S(\{x\})$

Soit $M$ un $S$-module gradué. Nous désignerons par $M_{U}$ l'ensemble des fractions $m / Q$, avec $m \in M, Q \in S(U), m$ et $Q$ étant homogènes de même degré; on identifie deux fractions $m / Q$ et $m^{\prime} / Q^{\prime}$ s'il existe $Q^{\prime \prime} \in S(U)$ tel que

$$
Q^{\prime \prime}\left(Q^{\prime} . m-Q . m^{\prime}\right)=0
$$

il est clair que l'on définit bien ainsi une relation d'équivalence entre couples $(m, Q)$. Pour $U=x$, on écrira $M_{x}$ au lieu de $M_{|x|}$.

Appliquant ceci à $M=S$, on trouve pour $S_{U}$ l'anneau des fractions rationnelles $P / Q$, où $P$ et $Q$ sont des polynômes homogènes de même degré et $Q \in S(U)$; si $M$ est un $S$-module gradué quelconque, on peut munir $M_{U}$ d'une structure de $S_{U}$-module en posant:

$$
\begin{aligned}
    m / Q+m^{\prime} / Q^{\prime} &=\left(Q^{\prime} m+Q m^{\prime}\right) / Q Q^{\prime} \\
    (P / Q) \cdot\left(m / Q^{\prime}\right) &=P m / Q Q^{\prime} .
\end{aligned}
$$

Si $U \subset V$, on a $S(V) \subset S(U)$, d'où des homomorphismes canoniques

$$
\varphi_{U}^{V}: M_{v} \rightarrow M_{U}
$$

le système $\left(M_{U}, \varphi_{U}^{v}\right)$, où $U$ et $V$ parcourent les ouverts non vides de $X$, définit donc un faisceau que nous noterons $Q(M) ;$ on vérifie tout de suite que

$$
\lim _{x \in U} M_{U}=M_{x}
$$

c'est-à-dire que $a(M)_{x}=M_{x} .$ On a en particulier $Q(S)=\mathcal{O}$, et comme les $M_{U}$ sont des $S_{v}$-modules, il s'ensuit que $Q(M)$ est un faisceau de $Q(S)$-modules, c'est-à-dire un faisceau algébrique sur $X$. Tout homomorphisme $\varphi: M \rightarrow M^{\prime}$ définit de façon naturelle des homomorphismes $S_{U}$-linéaires $\varphi_{U}: M_{U} \rightarrow M_{U}^{\prime}$, d'où un homomorphisme de faisceaux $\alpha(\varphi): \mathbb{Q}(M) \rightarrow Q\left(M^{\prime}\right)$, que nous noterons souvent $\varphi$. On a évidemment

$$
a(\varphi+\psi)=\alpha(\varphi)+a(\psi), a(1)=1, \alpha(\varphi \circ \psi)=\alpha(\varphi) \circ a(\psi)
$$

L'opération $\Omega(M)$ est donc un foncteur additif covariant, défini sur la catégorie des $S$-modules gradués, et à valeurs dans la catégorie des faisceaux algébriques $\operatorname{sur} X$

(Les définitions ci-dessus sont tout à fait analogues à celles du $\S 4$ du Chap. II; on observera toutefois que $S_{U}$ n'est pas l'anneau de fractions de $S$ relativement à $S(U)$, mais seulement sa composante homogène de degré 0 .)

58. Premières propriétés du foncteur $Q(M)$.

Proposition 3. Le foncteur $\mathbb{Q}(M)$ est un foncteur exact.

Soit $M \stackrel{\alpha}{\rightarrow} M^{\prime} \stackrel{\beta}{\rightarrow} M^{\prime \prime}$ une suite exacte de $S$-modules gradués, et montrons que la suite $M_{x} \stackrel{\alpha}{\rightarrow} M_{x}^{\prime} \stackrel{\beta}{\rightarrow} M_{x}^{\prime \prime}$ est aussi exacte. Soit $m^{\prime} / Q \in M_{x}^{\prime}$ un élément du noyau de $\beta$; vu la définition de $M_{x}^{\prime \prime}$, il existe $R \in S(x)$ tel que $R \beta\left(m^{\prime}\right)=0 ;$ mais alors il existe $m \in M$ tel que $\alpha(m)=R m^{\prime}$, et l'on a $\alpha(m / R Q)=m^{\prime} / Q$, cqfd. (Comparer avec le no 48, Lemme 1.)

Proposition 4. Si M est un S-module gradué, et si n est un entier, $Q(M(n))$ est canoniquement isomorphe a $Q(M)(n)$.

Soient $i \in I, x \in U_{i}$, et $m / Q \in M(n)_{x}$, avec $m \in M(n)_{p}, Q \in S(x), \operatorname{deg} Q=p$. Posons:

$$
\eta_{i, x}(m / Q)=m / t_{i}^{n} Q \in M_{x}
$$

ce qui est licite puisque $m \in M_{n+p}$ et $t_{i}^{n} Q \in S(x) .$ On voit immédiatement que $\eta_{i, x}: M(n)_{x} \rightarrow M_{x}$ est bijectif pour tout $x \in U_{i}$, et définit un isomorphisme $\eta_{i}$ de $a(M(n))$ sur $Q(M)$ au-dessus de $U_{i} .$ En outre, on a $\eta_{i} \circ \eta_{j}^{-1}=\theta_{i j}(n)$ audessus de $U_{i}$ n $U_{j} .$ Vu la définition de l'opération $\mathcal{F}(n)$, et la Proposition 4 du $\mathrm{n}^{\circ} 4$, cela montre bien que $\boldsymbol{Q}(M(n))$ est isomorphe à $\mathrm{C}(M)(n)$.

CorolLatre. $a(S(n))$ est canoniquement isomorphe a $\mathcal{O}(n)$.

En effet, on a déjà dit que $Q(S)$ était isomorphe à $\mathcal{O}$.

(Il est d'ailleurs évident directement que $Q(S(n))$ est isomorphe à $\mathcal{O}^{\prime}(n)$, puisque ${O}^{\prime}(n)_{x}$ est justement formé des fractions rationnelles $P / Q$, telles que deg $P-\operatorname{deg} Q=n$, et $Q \in S(x) .)$

Proposition $5 .$ Soit $M$ un S-module gradué vérifiant la condition (TF). Le faisceau algébrique $Q(M)$ est alors un faisceau cohérent, et, pour que $Q(M)=0$, il faut et il suffit que $M \in \mathcal{C}$.

Si $M \in$ e, pour tout $m \in M$ et tout $x \in X$, il existe $Q \in S(x)$ tel que $Q m=0$ : il suffit de prendre $Q$ de degré assez grand; on a donc $M_{x}=0$, d'où $Q(M)=0$. Soit maintenant $M$ un $S$-module gradué vérifiant la condition (TF); il existe un sous-module homogène $N$ de $M$, de type fini, tel que $M / N \in \mathcal{C} ;$ en appliquant ce qui précède, ainsi que la Proposition 3, on voit que $Q(N) \rightarrow Q(M)$ est bijectif, et il suffit donc de prouver que $Q(N)$ est cohérent. Puisque $N$ est de type fini, il existe une suite exacte $L^{1} \rightarrow L^{0} \rightarrow N \rightarrow 0$, où $L^{0}$ et $L^{1}$ sont des modules libres de type fini. D'après la Proposition 3, la suite $Q\left(L^{1}\right) \rightarrow Q\left(L^{0}\right) \rightarrow Q(N) \rightarrow 0$ est exacte. Mais, d'après le corollaire à la Proposition $4, a\left(L^{0}\right)$ et $\alpha\left(L^{1}\right)$ sont isomorphes à des sommes directes finies de faisceaux $\mathcal{O}\left(n_{i}\right)$, donc sont cohérents. Il s'ensuit bien que $\alpha(N)$ est cohérent.

Soit enfin $M$ un $S$-module gradué vérifiant (TF), et tel que $\alpha(M)=0 ; \mathrm{vu}$ ce qui précède, on peut supposer $M$ de type fini. Si $m$ est un élément homogène de $M$, soit $\mathfrak{a}_{m}$ l'annulateur de $m$, c'est-à-dire l'ensemble des polynômes $Q \in S$ tels que $Q . m=0$; il est clair que $\mathfrak{a}_{m}$ est un idéal homogène. De plus, l'hypothèse $M_{x}=0$ pour tout $x \in X$ entraine que la variété des zéros de $\mathfrak{a}_{m}$ dans $K^{r+1}$ est vide où réduite à $\{0\}$; le théorème des zéros de Hilbert montre alors que tout polynôme homogène de degré assez grand appartient à $\mathfrak{a}_{m}$. Appliquant ceci à un système fini de générateurs de $M$, on en conclut aussitôt que $M_{p}=0$ pour $p$ assez grand, ce qui achève la démonstration.

En combinant les Propositions 3 et 5 on obtient:

Proposition 6 . Soient $M$ et $M^{\prime}$ deux S-modules gradués vérifiant la condition (TF), et soit $\varphi: M \rightarrow M^{\prime}$ un homomorphisme de M dans $M^{\prime} .$ Pour que

$$
\alpha(\varphi): Q(M) \rightarrow Q\left(M^{\prime}\right)
$$

soit injectif (resp. surjectif, bijectif), il faut et il suffit que \varphi soit e-injectif (resp. e-surjectif, e-bijectif).

59. $S$-module gradué associé à un faisceau algébrique. Soit $\mathfrak{F}$ un faisceau algébrique sur $X$, et posons:

$$
\Gamma(F)=\sum_{n e z} \Gamma(\mathfrak{F})_{n}, \quad \text { avec } \quad \Gamma(F)_{n}=\Gamma(X, \mathcal{F}(n))
$$

Le groupe $\Gamma(\mathfrak{F})$ est un groupe gradué; nous allons le munir d'une structure de $S$-module. Soit $s \in \Gamma(X, \mathcal{F}(q))$ et soit $P \in S_{p} ;$ on peut identifier $P$ à une section de $\mathcal{O}(p)$ (cf. $\left.\mathrm{n}^{\circ} 54\right)$, donc $P \otimes s$ est une section de $\mathcal{O}(p) \otimes \mathfrak{F}(q)=\mathfrak{F}(q)(p)=$ $\mathfrak{F}(p+q)$, en utilisant les isomorphismes du $\mathrm{n}^{\circ} 54$; nous avons ainsi défini une section de $\mathcal{F}(p+q)$ que nous noterons $P . s$ au lieu de $P \otimes$ s. L'application $(P, s) \rightarrow P . s$ munit $\Gamma(\mathfrak{F})$ d'une structure de $S$-module compatible avec sa graduation.

On peut aussi définir $P . s$ au moyen de ses composantes sur les $U_{i}$ : si les composantes de $s$ sont $s_{i} \in \Gamma\left(U_{i}, F\right)$, avec $s_{i}=\left(t_{j}^{q} / t_{i}^{q}\right) \cdot s_{j}$ sur $U_{i}$ ก $U_{j}$, on a $(P . s)_{i}=\left(P / t_{i}^{p}\right) \cdot s_{i}$, ce qui a bien un sens, puisque $P / t_{i}^{p}$ est une fonction régulière $\operatorname{sur} U_{i}$.

Pour pouvoir comparer les foncteurs $Q(M)$ et $\Gamma(\mathfrak{F})$ nous allons définir deux homomorphismes canoniques:

$$
\alpha: M \rightarrow \Gamma(Q(M)) \quad \text { et } \quad \beta: a(\Gamma(\mathfrak{})) \rightarrow \mathfrak{F}
$$

DÉFinition de $\alpha .$ Soit $M$ un $S$-module gradué, et soit $m \in M_{0}$ un élément homogène de degré 0 de $M$. L'élément $m / 1$ est un élément bien défini de $M_{x}$, et varie continûment avec $x \in X$; donc $m$ définit une section $\alpha(m)$ de $Q(M)$. Si maintenant $m$ est homogène de degré $n, m$ est homogène de degré 0 dans $M(n)$, donc définit une section $\alpha(m)$ de $Q(M(n))=\alpha(M)(n)$ (cf. Proposition 4). D'où la définition de $\alpha: M \rightarrow \Gamma(\alpha(M))$, et il est immédiat que c'est un homomorphisme.

DéFinition de $\beta$. Soit $\mathfrak{F}$ un faisceau algébrique sur $X$, et soit $s / Q$ un élément de $\Gamma(\mathfrak{F})_{x}$, avec $s \in \Gamma(X, \mathfrak{F}(n)), Q \in S_{n}$, et $Q(x) \neq 0 .$ La fonction $1 / Q$ est homogène de degré $-n$, et régulière en $x$, c'est donc une section de $\theta(-n)$ au voisinage de $x$; il s'ensuit que $1 / Q \otimes s$ est une section de $\mathcal{O}(-n) \otimes \mathfrak{F}(n)=\mathfrak{F}$ au voisinage de $x$, donc définit un élément de $\mathfrak{F}_{x}$, que nous noterons $\beta_{x}(s / Q)$, car il ne dépend que de $s / Q$. On peut également définir $\beta_{x}$ en utilisant les composantes $s_{i}$ de $s:$ si $x \in U_{i}, \beta_{x}(s / Q)=\left(t_{i}^{n} / Q\right)_{. s_{i}}(x)$. La collection des homomorphismes $\beta_{x}$ définit l'homomorphisme $\beta: \mathbb{Q}(\Gamma(\mathcal{F})) \rightarrow \mathfrak{F} .$

Les homomorphismes $\alpha$ et $\beta$ sont reliés par les Propositions suivantes, qui se démontrent par un calcul direct:

Proposition 7. Soit $M$ un S-module gradué. Le composé des homomorphismes $\alpha(M) \rightarrow Q(\Gamma(\alpha(M))) \rightarrow Q(M)$ est l'identité.

(Le premier homomorphisme est défini par $\alpha: M \rightarrow \Gamma(\alpha(M))$, et le second est $\beta$, appliqué à $\mathscr{F}=Q(M) .)$

Proposition 8. Soit $\mathfrak{F}$ un faisceau algébrique sur $X$. Le composé des homomorphismes $\Gamma(\mathfrak{F}) \rightarrow \Gamma(\alpha(\Gamma(\mathfrak{F}))) \rightarrow \Gamma(\mathfrak{F})$ est l'identité.

(Le premier homomorphisme est $\alpha$, appliqué à $M=\Gamma(\mathfrak{F})$, tandis que le second est défini $\operatorname{par} \beta: \mathbb{Q}(\Gamma(\mathfrak{F})) \rightarrow \mathfrak{F}$.)

Nous montrerons au $\mathrm{n}^{\circ} 65$ que $\beta: \mathbb{\alpha}(\Gamma(\mathfrak{F})) \rightarrow \mathfrak{F}$ est bijectif si $\mathfrak{F}$ est cohérent, et que $\alpha: M \rightarrow \Gamma(\alpha(M))$ est $\mathcal{C}$-bijectif si $M$ vérifie la condition (TF).

60. Cas des faisceaux algébriques cohérents. Démontrons d'abord un résultat préliminaire:

Propositron 9. Soit $\mathscr{L}$ un faisceau algébrique sur $X$, somme directe d'un nombre fini de faisceaux $\mathcal{O}\left(n_{i}\right) .$ Alors $\Gamma(\mathfrak{L})$ vérifie $(\mathrm{TF})$, et $\beta: \mathbb{Q}(\Gamma(\mathscr{})) \rightarrow \mathcal{L}$ est bijectif.

On se ramène tout de suite à $\mathcal{L}=\mathcal{O}(n)$, puis à $\mathscr{L}=\mathcal{O}$. Dans ce cas, on sait que $\Gamma(\mathcal{O}(p))=S_{p}$ pour $p \geqq 0$, donc on a $S \subset \Gamma(\mathcal{O})$, le quotient appartenant à $c .$ Il s'ensuit d'abord que $\Gamma(\theta)$ vérifie (TF), puis que $\alpha(\Gamma(\theta))=Q(S)=\mathcal{O}$, cqfd.

(On observera que l'on a $\Gamma(\theta)=S$ si $r \geqq 1 ;$ par contre, si $r=0, \Gamma(0)$ n'est même pas un $S$-module de type fini.)

THÉoRÈme $2 .$ Pour tout faisceau algébrique cohérent $\mathfrak{F}$ sur $X$, il existe un S-module gradué $M$, vérifiant (TF), tel que $Q(M)$ soit isomorphe à $\mathfrak{F}$.

D'après le corollaire au Théorème 1, il existe une suite exacte de faisceaux algébriques:

$$
\mathcal{L}^{1} \stackrel{\varphi}{\rightarrow} \mathcal{L}^{0} \rightarrow \mathcal{F} \rightarrow 0
$$

où $\mathcal{L}^{1}$ et $\mathscr{\Omega}^{0}$ vérifient les hypothèses de la Proposition précédente. Soit $M$ le conoyau de l'homomorphisme $\Gamma(\varphi): \Gamma\left(\mathscr{L}^{1}\right) \rightarrow \Gamma\left(£^{0}\right) ;$ d'après la Proposition 9 , $M$ vérifie la condition (TF). En appliquant le foncteur $Q$ à la suite exacte:

$$
\Gamma\left(\mathcal{L}^{1}\right) \rightarrow \Gamma\left(\mathfrak{L}^{0}\right) \rightarrow M \rightarrow 0
$$

on obtient la suite exacte:

$$
Q\left(\Gamma\left(\mathscr{L}^{1}\right)\right) \rightarrow Q\left(\Gamma\left(\mathscr{2}^{0}\right)\right) \rightarrow \mathbb{Q}(M) \rightarrow 0
$$

Considérons le diagramme commutatif suivant:

![](https://cdn.mathpix.com/cropped/171be0f982e6ae195d9eb998d7a9d582-17.jpg?height=101&width=363&top_left_y=346&top_left_x=270)

D'après la Proposition 9, les deux homomorphismes verticaux sont bijectifs. Il en résulte que $Q(M)$ est isomorphe à $\mathfrak{F}$, cqfd.

\section{$\S$ 3. Cohomologie de l'espace projectif à valeurs dans un faisceau algébrique cohérent}

61. Les complexes $C_{k}(M)$ et $C(M) .$ Nous conservons les notations des $\mathrm{n}^{\mathrm{os}}$ 51 et 56 . En particulier, $I$ désignera l'intervalle $\{0,1, \cdots, r\}$, et $S$ désignera l'algèbre graduée $K\left[t_{0}, \cdots, t_{r}\right]$

Soient $M$ un $S$-module gradué, $k$ et $q$ deux entiers $\geqq 0$; nous allons définir un groupe $C_{k}^{q}(M):$ un élément de $C_{k}^{q}(M)$ est une application

$$
\left(i_{0}, \cdots, i_{q}\right) \rightarrow m\left\langle i_{0} \cdots i_{q}\right\rangle
$$

qui fait correspondre à toute suite $\left(i_{0}, \cdots, i_{q}\right)$ de $q+1$ éléments de $I$ un élément homogène de degré $k(q+1)$ de $M$, dépendant de façon alternée de $i_{0}, \cdots, i_{q} .$ En particulier, on a $m\left\langle i_{0} \cdots i_{q}\right\rangle=0$ si deux des indices $i_{0}, \cdots, i_{q}$ sont égaux. On définit de façon évidente l'addition dans $C_{k}^{q}(M)$, ainsi que la multiplication par un élément $\lambda \in K$, et $C_{k}^{q}(M)$ est un espace vectoriel sur $K$.

Si $m$ est un élément de $C_{k}^{q}(M)$, définissons $d m \in C_{k}^{q+1}(M)$ par la formule:

$$
(d m)\left\langle i_{0} \cdots i_{q+1}\right\rangle=\sum_{j=0}^{j=q+1}(-1)^{j} t_{i_{i}}^{k} \cdot m\left\langle i_{0} \cdots \hat{\imath}_{j} \cdots i_{q+1}\right\rangle
$$

On vérifie par un calcul direct que $d \circ d=0$; donc, la somme directe $C_{k}(M)=\sum_{q=0}^{q=r} C_{k}^{q}(M)$, munie de l'opérateur cobord $d$, est un complexe, dont le $q$-ème groupe de cohomologie sera noté $H_{k}^{q}(M)$.

(Signalons, d'après [11], une autre interprétation des éléments de $C_{k}^{q}(M):$ introduisons $r+1$ symboles différentiels $d x_{0}, \cdots, d x_{r}$, et faisons correspondre à tout $m \in C_{k}^{q}(M)$ la "forme différentielle" de degré $q+1:$

$$
\omega_{m}=\sum_{i_{0}<\cdots<i_{q}} m\left\langle i_{0} \cdots i_{i}\right\rangle d x_{i_{0}} \wedge \cdots \wedge d x_{i_{q}}
$$

Si l'on pose $\alpha_{k}=\sum_{i=0}^{i=-0} t_{i}^{k} d x_{i}$, on voit que l'on a :

$$
\omega_{d m}=\alpha_{k} \wedge \omega_{m}
$$

autrement dit, l'opération de cobord se transforme en la multiplication extérieure par la forme $\alpha_{k}$ ).

Si $h$ est un entier $\geqq k$, soit $\rho_{k}^{h}: C_{k}^{q}(M) \rightarrow C_{h}^{q}(M)$ l'homomorphisme défini par la formule:

$$
\rho_{k}^{h}(m)\left\langle i_{0} \cdots i_{q}\right\rangle=\left(t_{i_{0}} \cdots t_{i_{q}}\right)^{h-k} m\left\langle i_{0} \cdots i_{q}\right\rangle
$$

On a $\rho_{k}^{h} \circ d=d \circ \rho_{k}^{h}$, et $\rho_{h}^{l} \circ \rho_{k}^{h}=\rho_{k}^{l}$ si $k \leqq h \leqq l .$ On peut donc définir le complexe $C(M)$, limite inductive du système $\left(C_{k}(M), \rho_{k}^{h}\right)$ pour $k \rightarrow+\infty .$ Les groupes de cohomologie de ce complexe seront notés $H^{q}(M) .$ Puisque la cohomologie commute avec les limites inductives (cf. [6], Chap. V, Prop. 9.3*), on a:

$$
H^{q}(M)=\lim _{k \rightarrow \infty} H_{k}^{q}(M)
$$

Tout homomorphisme $\varphi: M \rightarrow M^{\prime}$ définit un homomorphisme

$$
\varphi: C_{k}(M) \rightarrow C_{k}\left(M^{\prime}\right)
$$

par la formule: $\varphi(m)\left\langle i_{0} \cdots i_{q}\right\rangle=\varphi\left(m\left\langle i_{0} \cdots i_{q}\right\rangle\right)$, d'où, par passage à la limite, $\varphi: C(M) \rightarrow C\left(M^{\prime}\right) ;$ de plus ces homomorphismes commutent avec le cobord, et définissent donc des homomorphismes

$$
\varphi: H_{k}^{q}(M) \rightarrow H_{k}^{q}\left(M^{\prime}\right) \text { et } \varphi: H^{q}(M) \rightarrow H^{q}\left(M^{\prime}\right)
$$

Si l'on a une suite exacte $0 \rightarrow M \rightarrow M^{\prime} \rightarrow M^{\prime \prime} \rightarrow 0$, on a une suite exacte de complexes $0 \rightarrow C_{k}(M) \rightarrow C_{k}\left(M^{\prime}\right) \rightarrow C_{k}\left(M^{\prime \prime}\right) \rightarrow 0$, d'où une suite exacte de cohomologie:

$$
\cdots \rightarrow H_{k}^{q}\left(M^{\prime}\right) \rightarrow H_{k}^{q}\left(M^{\prime \prime}\right) \rightarrow H_{k}^{q+1}(M) \rightarrow H_{k}^{q+1}\left(M^{\prime}\right) \rightarrow \cdots
$$

Mêmes résultats pour $C(M)$ et les $H^{q}(M)$.

REMARQUE. Nous verrons plus loin (cf. $\mathrm{n}^{\circ} 69$ ) que l'on peut exprimer les $H_{k}^{q}(M)$ au moyen des $\mathrm{Ext}_{3}^{q}$.

62. Calcul des $H_{k}^{q}(M)$ pour certains modules $M .$ Soient $M$ un $S$-module gradué, et $m \in M$ un élément homogène de degré $0 .$ Le système des $\left(t_{i}^{k} . m\right)$ est un 0 -cocycle de $C_{k}(M)$, que nous noterons $\alpha^{k}(m)$, et que nous identifierons à sa classe de cohomologie. On obtient ainsi un homomorphisme $K$-linéaire $\alpha^{k}: M_{0} \rightarrow H_{k}^{0}(M)$; comme $\alpha^{h}=\rho_{k}^{h} \circ \alpha^{k}$ si $h \geqq k$, les $\alpha^{k}$ définissent par passage à la limite un homomorphisme $\alpha: M_{0} \rightarrow H^{\circ}(M) .$

Introduisons encore deux notations:

Si $\left(P_{0}, \cdots, P_{h}\right)$ sont des éléments de $S$, nous noterons $\left(P_{0}, \cdots, P_{h}\right) M$ le sous-module de $M$ formé des éléments $\sum_{i=0}^{i=h} P_{i}: m_{i}$, avec $m_{i} \in M ;$ si les $P_{i}$ sont homogènes, ce sous-module est homogène.

Si $P$ est un élément de $S$, et $N$ un sous-module de $M$, nous noterons $N: P$ le sous-module de $M$ formé des éléments $m \in M$ tels que $P . m \in N$; on a évidemment $N: P \supset N ;$ si $N$ et $P$ sont homogènes, $N: P$ est homogène.

Ces notations étant précisées, on a: Proposition 1. Soient $M$ un S-module gradué et $k$ un entier $\geqq 0 .$ Supposons que, pour tout $i \in I$, on ait:

$$
\left(t_{0}^{k}, \cdots, t_{i-1}^{k}\right) M: t_{i}^{k}=\left(t_{0}^{k}, \cdots, t_{i-1}^{k}\right) M
$$

Alors:

(a) $\alpha^{k}: M_{0} \rightarrow H_{k}^{0}(M)$ est bijectif $($ si $r \geqq 1)$.

(b) $H_{k}^{q}(M)=0$ pour $0<q<r$

(Pour $i=0$, l'hypothèse signifie que $t_{0}^{x} . m=0$ entraîne $m=0$.)

Cette Proposition est un cas particulier d'un résultat dû à de Rham [11] (le résultat de de Rham étant d'ailleurs valable même si l'on ne suppose pas les $m\left\langle i_{0} \cdots i_{q}\right\rangle$ homogènes). Voir aussi [6], Chap. VIII, $\S 4$, où est traité un cas particulier, suffisant pour les applications que nous allons faire.

Nous allons appliquer la Proposition 1 au $S$-module gradué $S(n)$ :

Proposition 2. Soient $k$ un entier $\geqq 0, n$ un entier quelconque. Alors:

(a) $\alpha^{k}: S_{n} \rightarrow H_{k}^{0}(S(n))$ est bijectif $($ si $r \geqq 1)$.

(b) $H_{k}^{q}(S(n))=0 \quad$ pour $0<q<r$.

(c) $H_{k}^{r}(S(n))$ admet pour base (sur K) les classes de cohomologie des monômes $t_{0}^{\alpha_{0}} \cdots t_{r}^{\alpha_{r}}$, avec $0 \leqq \alpha_{i}<k$ et $\sum_{i=0}^{i=r} \alpha_{i}=k(r+1)+n$

Il est clair que le $S$-module $S(n)$ vérifie les hypothèses de la Proposition 1, ce qui démontre (a) et (b). D'autre part, pour tout $S$-module gradué $M$, on a $H_{k}^{r}(M)=M_{k(r+1)} /\left(t_{0}^{k}, \cdots, t_{r}^{k}\right) M_{k r} ;$ or les monômes

$$
t_{0}^{\alpha_{0}} \cdots t_{r}^{\alpha_{r}}, \alpha_{i} \geqq 0, \sum_{i=0}^{i=r} \alpha_{i}=k(r+1)+n
$$

forment une base de $S(n)_{k(r+1)}$, et ceux de ces monômes pour lesquels l'un au moins des $\alpha_{i}$ est $\geqq k$ forment une base de $\left(t_{0}^{k}, \cdots, t_{r}^{k}\right) S(n)_{k r}$; d'où (c).

Il est commode d'écrire les exposants $\alpha_{i}$ sous la forme $\alpha_{i}=k-\beta_{i} .$ Les conditions énoncées dans (c) s'écrivent alors:

$$
0<\beta_{i} \leqq k \text { et } \quad \sum_{i=0}^{i=r} \beta_{i}=-n
$$

La deuxième condition, jointe à $\beta_{i}>0$, entraîne $\beta_{i} \leqq-n-r ;$ si donc $k \geq-n-r$. la condition $\beta_{i} \leqq k$ est conséquence des deux précédentes. D'où:

CorolLaine 1. Pour $k \geqq-n-r, H_{k}^{r}(S(n))$ admet pour base les classes de cohomologie des monômes $\left(t_{0} \cdots t_{r}\right)^{k} / t_{0}^{00} \cdots t_{r}^{0}$, avec $\beta_{i}>0$ et $\sum_{i=0}^{i=f} \beta_{i}=-n$

On a également:

CorolLalRe 2. Si $h \geqq k \geqq-n-r$, l'homomorphisme

$$
\rho_{k}^{h}: H_{k}^{q}(S(n)) \rightarrow H_{h}^{q}(S(n))
$$

est bijectif pour tout $q \geqq 0$.

Pour $q \neq r$, cela résulte des assertions (a) et (b) de la Proposition 2. Pour $q=r$, cela résulte du Corollaire 1, compte tenu de ce que $\rho_{k}^{n}$ transforme

$$
\left(t_{0} \cdots t_{r}\right)^{k} / t_{0}^{\beta_{0}} \cdots t_{r}^{\beta r} \text { en }\left(t_{0} \cdots t_{r}\right)^{h} / t_{0}^{\beta_{0}} \cdots t_{r}^{\beta_{r}}
$$

CorolLatre 3. L'homomorphisme $\alpha: S_{n} \rightarrow H^{0}(S(n))$ est bijectif si $r \geqq 1$, ou si $n \geqq 0 .$ On a $H^{q}(S(n))=0$ pour $0<q<r$, et $H^{r}(S(n))$ est un espace vectoriel de dimension $\left(\begin{array}{c}-n-1 \\ r\end{array}\right)$ sur $K$.

L'assertion relative à $\alpha$ résulte de la Proposition $2,(a)$, dans le cas ou $r \geqq 1$; elle est immédiate si $r=0$ et $n \geqq 0$. Le reste du Corollaire est une conséquence évidente des Corollaires 1 et 2 (en convenant que le coefficient binômial $\left(\begin{array}{l}a \\ r\end{array}\right)$ est nul si $a<r$ ).

63. Propriétés générales des $H^{q}(M)$.

Proposition $3 .$ Soit $M$ un S-module gradué vérifiant la condition (TF). Alors:

(a) Il existe un entier $k(M)$ tel que $\rho_{k}^{n}: H_{k}^{q}(M) \rightarrow H_{h}^{4}(M)$ soit bijectif pour $h \geqq k \geqq k(M)$ et $q$ quelconque.

(b) $H^{q}(M)$ est un espace vectoriel de dimension finie sur $K$ pour tout $q \geqq 0$.

(c) $I l$ existe un entier $n(M)$ tel que, pour $n \geqq n(M), \alpha: M_{n} \rightarrow H^{0}(M(n))$ soit bijectif, et que $H^{q}(M(n))$ soit nul pour tout $q>0$.

On se ramène tout de suite au cas où $M$ est de type fini. Nous dirons alors que $M$ est de dimension $\leq s(s$ étant un entier $\geqq 0)$ s'il existe une suite exacte:

$$
0 \rightarrow L^{s} \rightarrow L^{s-1} \rightarrow \cdots \rightarrow L^{0} \rightarrow M \rightarrow 0
$$

où les $L^{i}$ soient des $S$-modules gradués libres de type fini. D'après le théorème des syzygies de Hilbert (cf. [6], Chap. VIII, th. 6.5), cette dimension est toujours $\leqq r+1 .$

Nous démontrerons la Proposition par récurrence sur la dimension de $M$. Si elle est $0 . M$ est libre de type fini, i.e. somme directe de modules $S\left(n_{i}\right)$, et la Proposition résulte des Corollaires 2 et 3 à la Proposition $2 .$ Supposons que $M$ soit de dimension $\leqq s$, et soit $N$ le noyau de $L^{0} \rightarrow M$. Le $S$-module gradué $N$ est de dimension $\leqq s-1$, et l'on a une suite exacte:

$$
0 \rightarrow N \rightarrow L^{0} \rightarrow M \rightarrow 0
$$

Vu l'hypothèse de récurrence, la Proposition est vraie pour $N$ et $L^{0} .$ En appliquant le lemme des cina ([7]. Chap. I, Lemme 4.3) au diagramme commutatif:

$$
\begin{array}{cccc}
    H_{k}^{q}(N) & \rightarrow H_{k}^{q}\left(L^{0}\right) & \rightarrow H_{k}^{q}(M) \rightarrow H_{k}^{q+1}(N) & \rightarrow H_{k}^{q+1}\left(L^{0}\right) \\
    \downarrow & \downarrow & \downarrow & \downarrow & \downarrow \\
    H_{h}^{q}(N) & \rightarrow H_{h}^{q}\left(L^{0}\right) & \rightarrow H_{h}^{q}(M) \rightarrow H_{h}^{q+1}(N) & \rightarrow H_{h}^{q+1}\left(L^{0}\right)
\end{array}
$$

où $h \geq k \geq \operatorname{Sup}\left(k(N), k\left(L^{0}\right)\right)$, on démontre (a), d'où évidemment (b), puisque les $H_{k}^{q}(M)$ sont de dimension finie sur $K$. D'autre part, la suite exacte

$$
H^{q}\left(L^{0}(n)\right) \rightarrow H^{q}(M(n)) \rightarrow H^{q+1}(N(n))
$$

montre que $H^{q}(M(n))=0$ pour $n \geqq \operatorname{Sup}\left(n\left(L^{0}\right), n(N)\right)$. Considérons enfin le diagramme commutatif:

![](https://cdn.mathpix.com/cropped/171be0f982e6ae195d9eb998d7a9d582-20.jpg?height=110&width=539&top_left_y=1131&top_left_x=182)

% ----------------------------------------------------------------------------------------------------------------------------------------------------------------
% ----------------------------------------------------------------------------------------------------------------------------------------------------------------
% ----------------------------------------------------------------------------------------------------------------------------------------------------------------
% ----------------------------------------------------------------------------------------------------------------------------------------------------------------

pour $n \geqq n(N)$, on a $H^{1}(N(n))=0 ;$ on en déduit que $\alpha: M_{n} \rightarrow H^{0}(M(n))$ est bijectif pour $n \geqq \operatorname{Sup}\left(n\left(L^{0}\right), n(N)\right)$, ce qui achève la démonstration de la Proposition.

64. Comparaison des groupes $H^{q}(M)$ et $H^{q}(X, \alpha(M)) .$ Soit $M$ un $S$-module gradué, et soit $\alpha(M)$ le faisceau algébrique sur $X=\mathbf{P}_{r}(K)$ défini à partir de $M$ par le procédé du $\mathrm{n}^{\circ} 57$. Nous allous comparer $C(M)$ avec $C^{\prime}(\mathfrak{U}, \alpha(M))$, complexe des cochaînes alternées du recouvrement $\mathfrak{U}=\left\{U_{i}\right\}_{i \in I}$ à valeurs dans le faisceau $\alpha(M)$

Soit $m \epsilon C_{k}^{q}(M)$, et soit $\left(i_{0}, \cdots, i_{q}\right)$ une suite de $q+1$ éléments de $I$. Le polynôme $\left(t_{i_{0}} \cdots t_{i_{0}}\right)^{k}$ appartient visiblement à $S\left(U_{i_{0} \cdots i_{0}}\right)$, avec les notations du $\mathrm{n}^{\circ}$ 57. Il en résulte que $m\left\langle i_{0} \cdots i_{q}\right\rangle /\left(t_{i_{0}} \cdots t_{i_{q}}\right)^{k}$ appartient à $M_{U}$, où $U=U_{i_{0}} \cdots i_{q}$, donc définit une section de $\varrho(M)$ au-dessus de $U_{i_{0}} \ldots_{q_{q}} .$ Lorsque $\left(i_{0}, \cdots, i_{q}\right)$ varie, le système formé par ces sections est une $q$-cochaîne alternée de $\mathfrak{u}$, à valeurs dans $\boldsymbol{a}(M)$, que nous noterons $\iota_{k}(m) .$ On voit tout de suite que $\iota_{k}$ commute avec $d$, et que $\iota_{k}=\iota_{h} \circ \rho_{k}^{h}$ si $h \geqq k$. Par passage à la limite inductive, les $\iota_{k}$ définissent donc un homomorphisme $\iota: C(M) \rightarrow C^{\prime}(\mathfrak{l}, \alpha(M))$, commutant avec $d$.

Proposition 4. Si $M$ vérifie la condition $(\mathrm{TF}), \iota: C(M) \rightarrow C^{\prime}(\mathfrak{U}, Q(M))$ est bijectif.

Si $M \in \mathcal{C}$, on a $M_{n}=0$ pour $n \geqq n_{0}$, d'où $C_{k}(M)=0$ pour $k \geqq n_{0}$, et $C(M)=0$. Comme tout $S$-module vérifiant (TF) est e-isomorphe à un module de type fini, ceci montre que l'on peut se borner au cas où $M$ est de type fini. On peut alors trouver une suite exacte $L^{1} \rightarrow L^{0} \rightarrow M \rightarrow 0$, où $L^{1}$ et $L^{0}$ sont libres de type fini. D'après les Propositions 3 et 5 du $n^{\circ} 58$, la suite

$$
Q\left(L^{1}\right) \rightarrow Q\left(L^{0}\right) \rightarrow Q(M) \rightarrow 0
$$

est une suite exacte de faisceaux algébriques cohérents; comme les $U_{i_{0} \ldots i_{q}}$ sont des ouverts affines, la suite

$$
C^{\prime}\left(\mathfrak{U}, \alpha\left(L^{1}\right)\right) \rightarrow C^{\prime}\left(\mathfrak{U}, \mathbb{Q}\left(L^{0}\right)\right) \rightarrow C^{\prime}(\mathfrak{U}, Q(M)) \rightarrow 0
$$

est une suite exacte (cf. $\mathrm{n}^{\circ} 45$, Corollaire 2 au Théorème 2). Le diagramme commutatif

![](https://cdn.mathpix.com/cropped/9337510346c782fee82ef71daf8d2a35-01.jpg?height=121&width=485&top_left_y=964&top_left_x=203)

montre alors que, si la Proposition est vraie pour les modules $L^{1}$ et $L^{0}$, elle l'est pour $M$. Nous sommes donc ramenés au cas particulier d'un module libre de type fini, puis, par décomposition en somme directe, au cas où $M=S(n)$.

Dans ce cas, on a $Q(S(n))=O(n) ;$ une section $f_{i_{0} \cdots i_{q}}$ de $\mathcal{O}(n)$ sur $U_{i_{0}} \ldots i_{q}$ est, par définition même de ce faisceau, une fonction régulière sur $V_{i_{0}} n \cdots$ n $V_{i_{q}}$ et homogène de degré $n$. Comme $V_{i_{0}} n \cdots \cap V_{i_{q}}$ est l'ensemble des points de $K^{r+1}$ où la fonction $t_{i_{0}} \cdots t_{i_{q}}$ est $\neq 0$, il existe un entier $k$ tel que

$$
f_{i_{0} \cdots i_{q}}=P\left\langle i_{0} \cdots i_{q}\right\rangle /\left(t_{i_{0}} \cdots t_{i_{q}}\right)^{k}
$$

$P\left\langle i_{0} \cdots i_{q}\right\rangle$ étant un polynôme homogène de degré $n+k(q+1)$, c'est-à-dire de degré $k(q+1)$ dans $S(n) .$ Ainsi, toute cochaîne alternée $f \in C^{\prime}(\mathfrak{U}, \mathcal{O}(n))$ définit un système $P\left\langle i_{0} \cdots i_{q}\right\rangle$ qui est un élément de $C_{k}(S(n))$; d'où un homomorphisme

$$
\nu: C^{\prime}(\mathfrak{H}, \mathcal{O}(n)) \rightarrow C(S(n))
$$

Comme on vérifie tout de suite que $\iota \circ \nu=1$ et $\nu \circ \iota=1$, il en résulte que $\iota$ est bijectif, ce qui achève la démonstration.

ConolLatre. \iota deffinit un isomorphisme de $H^{q}(M)$ sur $H^{q}(X, \alpha(M))$ pour tout $q \geqq 0$

En effet, on sait que $H^{\prime q}(\mathfrak{U}, \alpha(M))=H^{q}(\mathfrak{l}, \alpha(M))\left(\mathrm{n}^{\circ} 20\right.$, Proposition 2), et que $H^{q}(\mathfrak{u}, \alpha(M))=H^{q}(X, \alpha(M))\left(\mathrm{n}^{\circ} 52\right.$, Proposition 2 , qui est applicable puisque $Q(M)$ est cohérent).

RemarQue. Il est facile de voir que $\iota: C(M) \rightarrow C^{\prime}(\mathfrak{U}, \alpha(M))$ est injectif, même si $M$ ne vérifie pas la condition (TF).

\section{Applications.}

Proposition 5. Si $M$ est un S-module gradué vérifiant la condition (TF), l'homomorphisme $\alpha: M \rightarrow \Gamma(\alpha(M))$, défini au $\mathrm{n}^{\circ} 59$, est e-bijectif.

Il faut voir que $\alpha: M_{n} \rightarrow \Gamma(X, \alpha(M(n)))$ est bijectif pour $n$ assez grand. Or, d'après la Proposition $4, \Gamma(X, a(M(n)))$ s'identifie à $H^{0}(M(n))$; la Proposition résulte donc de la Proposition $3,(c)$, compte tenu du fait que l'homomorphisme $\alpha$ est transformé par l'identification précédente en l'homomorphisme défini au début du $\mathrm{n}^{\circ} 62$, et également noté $\alpha$.

Proposition 6. Soit $\mathfrak{F}$ un faisceau algébrique cohérent sur $X .$ Le S-module gradué $\Gamma(\mathfrak{F})$ vérifie la condition (TF), et l'homomorphisme $\beta: \mathbb{Q}(\Gamma(\mathfrak{F})) \rightarrow \mathfrak{F}$, défini $a u \mathrm{n}^{\circ} 59$, est bijectif.

D'après le Théorème $2 \mathrm{du} \mathrm{n}^{\circ} 60$, on peut supposer que $\mathfrak{F}=\alpha(M)$, où $M$ est un module vérifiant (TF). D'après la Proposition précédente, $\alpha: M \rightarrow \Gamma(Q(M))$ est e-bijectif; comme $M$ vérifie (TF), il s'ensuit que $\Gamma(\alpha(M))$ la vérifie aussi. Appliquant la Proposition $6 \mathrm{du} \mathrm{n}^{\circ} 58$, on voit que $\alpha: \mathrm{a}(M) \rightarrow \mathbb{}(\Gamma(\alpha(M)))$ est bijectif. Puisque le composé: $\alpha(M) \stackrel{\alpha}{\rightarrow} Q(\Gamma(\alpha(M))) \stackrel{\beta}{\rightarrow} \alpha(M)$ est l'identité (n ${ }^{\circ} 59$, Proposition 7$)$, il s'ensuit que $\beta$ est bijectif, cqfd.

Proposition 7. Soit $\mathcal{F}$ un faisceau algébrique cohérent sur $X .$ Les groupes $H^{q}(X, \mathcal{F})$ sont des espaces vectoriels de dimension finie sur $K$ pour tout $q \geqq 0$, et l'on a $H^{q}(X, \mathfrak{F}(n))=0$ pour $q>0$ et $n$ assez grand.

On peut supposer, comme ci-dessus, que $\mathcal{F}=\alpha(M)$, où $M$ est un module vérifiant (TF). La Proposition résulte alors de la Proposition 3 et du corollaire à la Proposition $4 .$

Proposition 8. On a $H^{q}(X, O(n))=0$ pour $0<q<r$, et $H^{*}(X, \mathcal{O}(n))$ est un espace vectoriel de dimension $\left(\begin{array}{c}-n-1 \\ r\end{array}\right)$ sur $K$, admettant pour base les classes de cohomologie des cocycles alternés de lu

$$
f_{01 \ldots r}=1 / t_{0}^{\beta_{0}} \cdots t_{r}^{\beta_{r}}, \quad \text { avec } \quad \beta_{i}>0 \text { et } \sum_{i=0}^{i=r} \beta_{i}=-n
$$

On a $\theta(n)=\alpha(S(n))$, d'où $H^{q}(X, \mathcal{O}(n))=H^{q}(S(n))$, d'après le corollaire a la Proposition 4; la Proposition résulte immédiatement de là et des corollaires à la Proposition 2 .

On notera en particulier que $H^{r}(X, \theta(-r-1))$ est un espace vectoriel de dimension 1 sur $K$, admettant pour base la classe de cohomologie du cocycle $f_{01 \ldots r}=1 / t_{0} \cdots t_{r}$

66. Faisceaux algébriques cohérents sur les variétés projectives. Soit $V$ une sous-variété fermée de l'espace projectif $X=\mathbf{P}_{r}(K)$, et soit $\mathfrak{F}$ un faisceau algébrique cohérent sur $V$. En prolongeant $\mathcal{F}$ par 0 en dehors de $V$, on obtient un faisceau algébrique cohérent sur $X$ (cf. $\left.\mathrm{n}^{\circ} 39\right)$, noté $\mathfrak{F}^{x}$; on sait que $H^{q}\left(X, \mathcal{F}^{X}\right)=$ $H^{q}(V, \mathfrak{F})$. Les résultats du $\mathrm{n}^{\circ}$ précédent s'appliquent donc aux groupes $H^{q}(V, \mathfrak{F}) .$ On obtient tout d'abord (compte tenu du no 52 ):

THÉORÈME $1 .$ Les groupes $H^{q}(V, \mathcal{F})$ sont des espaces vectoriels de dimension finie sur $K$, nuls pour $q>\operatorname{dim} V$.

En particulier, pour $q=0$, on a:

Conollatre. $\Gamma(V, \mathfrak{F})$ est un espace vectoriel de dimension finie sur $K$.

(Il est naturel de conjecturer que le théorème ci-dessus est valable pour toute variété complète, au sens de Weil [16].)

Soit $U_{i}^{\prime}=U_{i} \cap V ;$ les $U_{i}^{\prime}$ forment un recouvrement ouvert $\mathfrak{U}^{\prime}$ de $V .$ Si $\mathfrak{F}$ est un faisceau algébrique sur $V$, soit $\mathfrak{F}_{i}=\mathfrak{F}\left(U_{i}^{\prime}\right)$, et soit $\theta_{i j}(n)$ l'isomorphisme de $\mathfrak{F}_{j}\left(U_{i}^{\prime} \cap U_{j}^{\prime}\right)$ sur $\mathcal{F}_{i}\left(U_{i}^{\prime} \cap U_{j}^{\prime}\right)$ défini par la multiplication par $\left(t_{j} / t_{i}\right)^{n} .$ On notera $\mathcal{F}(n)$ le faisceau obtenu a partir des $\mathfrak{F}_{i}$ par recollement au moyen des $\theta_{i j}(n) .$ L'opération $\mathfrak{F}(n)$ jouit des mêmes propriétés que celle définie au $\mathrm{n}^{\circ} 54$ et qu'elle généralise; en particulier $\mathcal{F}(n)$ est canoniquement isomorphe à $\mathcal{F} \otimes \mathcal{O}_{v}(n)$

On a $F^{x}(n)=\mathfrak{F}(n)^{X}$. Appliquant alors le Théorème $1 \mathrm{du} \mathrm{n}^{\circ} 55$, ainsi que la Proposition 7 du $\mathrm{n}^{\circ} 65$, on obtient:

THÉorème 2. Soit $\mathcal{F}$ un faisceau algébrique cohérent sur $V$. Il existe un entier $m(\mathfrak{F})$ tel que l'on ait, pour tout $n \geqq m(\mathfrak{F})$ :

(a) Pour tout $x \in V$, le $\boldsymbol{O}_{x, v}$-module $\mathfrak{F}(n)_{x}$ est engendré par les éléments de $\Gamma(V, \mathcal{F}(n))$

(b) $H^{q}(V, \mathcal{F}(n))=0$ pour tout $q>0$.

Remarque. Il est essentiel d'observer que le faisceau $\mathfrak{F}(n)$ ne dépend pas seulement de $\mathfrak{F}$ et de $n$, mais aussi du plongement de $V$ dans l'espace projectif $X$. Plus précisément, soit $P$ l'espace fibré principal $\pi^{-1}(V)$, de groupe structural le groupe $K^{*} ; n$ étant un entier, faisons opérer $K^{*}$ sur $K$ par la formule:

$$
(\lambda, \mu) \rightarrow \lambda^{-n} \mu \quad \text { si } \lambda \in K^{*} \text { et } \mu \in K
$$

Soit $E^{n}=P \times_{k} . K$ l'espace fibré associé à $P$ et de fibre type $K$, muni des opérateurs précédents; soit $s\left(E^{n}\right)$ le faisceau des germes de sections de $E^{n}$ (cf. $\left.\mathrm{n}^{\circ} 41\right)$. En tenant compte du fait que les $t_{i} / t_{j}$ forment un système de changement de cartes pour $P$, on vérifie tout de suite que $\$\left(E^{n}\right)$ est canoniquement isomorphe à $\mathcal{O}_{V}(n) .$ La formule $\mathfrak{F}(n)=\mathfrak{F} \otimes \mathcal{O}_{v}(n)=\mathfrak{F} \otimes \delta\left(E^{n}\right)$ montre alors que l'opération $\mathcal{F} \rightarrow \mathfrak{F}(n)$ ne dépend que de la classe de l'espace fibré P défini par le plongement $V \rightarrow X .$ En particulier, si $V$ est normale, $\mathscr{F}(n)$ ne dépend que de la classe d'équivalence linéaire des sections hyperplanes de $V$ dans le plongement considéré (cf. [17]).

67. Un complément. Si $M$ est un $S$-module gradué vérifiant (TF), nous noterons $M^{4}$ le $S$-module gradué $\Gamma(\alpha(M))$. Nous avons vu au $\mathrm{n}^{\circ} 65$ que $\alpha: M \rightarrow M^{4}$ est e-bijectif. Nous allons donner des conditions permettant d'affirmer que $\alpha$ est bijectif.

Proposition 9 . Pour que $\alpha: M \rightarrow M^{4}$ soit bijectif, il faut et il suffit que les conditions suivantes soient vérifiées:

(i) Si $m \in M$ est tel que $t_{i} \cdot m=0$ pour tout $i \in I$, alors $m=0$.

(ii) Si des éléments $m_{i} \in M$, homogènes et de même degré, vérifient la relation $t_{j} \cdot m_{i}-t_{i} \cdot m_{j}=0$ pour tout couple $(i, j)$, il existe $m \in M$ tel que $m_{i}=t_{i} \cdot m$

Montrons que les conditions (i) et (ii) sont vérifiées par $M^{4}$, ce qui prouvera leur nécessité. Pour (i), on peut supposer que $m$ est homogène, c'est-à-dire est une section de $Q(M(n)) ;$ dans ce cas, la condition $t_{i} \cdot m=0$ entraîne que $m$ est nulle sur $U_{i}$, et, ceci ayant lieu pour tout $i \in I$, on a bien $m=0$. Pour (ii), soit $n$ le degré des $m_{i}$; on a donc $m_{i} \in \Gamma(\alpha(M(n))) ;$ comme $1 / t_{i}$ est une section de $\mathcal{O}(-1)$ sur $U_{i}, m_{i} / t_{i}$ est une section de $Q(M(n-1))$ sur $U_{i}$, et la condition $t_{j} \cdot m_{i}-t_{i} \cdot m_{j}=0$ montre que ces diverses sections sont les restrictions d'une section unique $m$ de $\mathbb{Q}(M(n-1))$ sur $X$; il reste à comparer les sections $t_{i} \cdot m$ et $m_{i} ;$ pour montrer qu'elles coïncident sur $U_{j}$, il suffit de voir que $t_{j}\left(t_{i} \cdot m-m_{i}\right)=0$ sur $U_{i}$, ce qui résulte de la formule $t_{i} \cdot m_{i}=t_{i} \cdot m_{j}$ et de la définition de $m$.

Montrons maintenant que (i) entraîne que $\alpha$ soit injectif. Pour $n$ assez grand, on sait que $\alpha: M_{n} \rightarrow M_{n}^{4}$ est bijectif, et l'on peut donc raisonner par récurrence descendante sur $n ;$ si $\alpha(m)=0$, avec $m \in M_{n}$, on aura $t_{i} \alpha(m)=\alpha\left(t_{i} \cdot m\right)=0$ et l'hypothèse de récurrence, applicable puisque $t_{i} \cdot m \in M_{n+1}$, montre que $m=0$. Montrons enfin que (i) et (ii) entraînent que $\alpha$ soit surjectif. On peut, comme précédemment, raisonner par récurrence descendante sur $n$. Si $m^{\prime} \in M_{n}^{q}$, l'hypothèse de récurrence montre qu'il existe $m_{i} \in M_{n+1}$ tels que $\alpha\left(m_{i}\right)=t_{i} \cdot m^{\prime} ;$ on a $\alpha\left(t_{j} \cdot m_{i}-t_{i} \cdot m_{j}\right)=0$, d'où $t_{j} \cdot m_{i}-t_{i} \cdot m_{j}=0$, puisque $\alpha$ est injectif. La condition (ii) entraîne alors l'existence de $m \in M_{n}$ tel que $t_{i} \cdot m=m_{i} ;$ on a $t_{i}\left(m^{\prime}-\alpha(m)\right)=0$, ce qui montre que $m^{\prime}=\alpha(m)$, et achève la démonstration.

Remarques. (1) La démonstration montre que la condition (i) est nécessaire et suffisante pour que $\alpha$ soit injectif.

(2) On peut exprimer (i) et (ii) ainsi: l'homomorphisme $\alpha^{1}: M_{n} \rightarrow H_{1}^{0}(M(n))$ est bijectif pour tout $n \in \mathrm{Z} .$ D'ailleurs, la Proposition 4 montre que l'on peut identifier $M^{4}$ au $S$-module $\sum_{n \in z} H^{0}(M(n))$, et il serait facile de tirer de là une démonstration purement algébrique de la Proposition 9 (sans utiliser le faisceau $\boldsymbol{Q}(M))$

\section{\$4. Relations avec les foncteurs Exts}

68. Les foncteurs $\operatorname{Ext}_{3}^{q}$. Nous conservons les notations du $\mathrm{n}^{\circ} 56 .$ Si $M$ et $N$ sont deux $S$-modules gradués, nous désignerons par $\operatorname{Hom}_{s}(M, N)_{n}$ le groupe des $S$-homomorphismes homogènes de degré $n$ de $M$ dans $N$, et par $\operatorname{Hom}_{s}(M, N)$ le groupe gradué $\sum_{n \in z} \operatorname{Hom}_{s}(M, N)_{n} ;$ c'est un $S$-module gradué; lorsque $M$ est de type fini il coïncide avec le $S$-module de tous les $S$-homomorphismes de $M$ dans $N$.

Les foncteurs dérivés (cf. [6], Chap. V) du foncteur $\operatorname{Hom}_{s}(M, N)$ sont les foncteurs $\operatorname{Ext}_{s}^{q}(M, N), q=0,1, \cdots$. Rappelons brièvement leur définition: $^{5}$

On choisit une "résolution" de $M$, c'est-à-dire une suite exacte:

$$
\cdots \rightarrow L^{q+1} \rightarrow L^{q} \rightarrow \cdots \rightarrow L^{0} \rightarrow M \rightarrow 0
$$

où les $L^{q}$ sont des $S$-modules gradués libres, et les applications des homomorphismes (c'est-à-dire, comme d'ordinaire, des S-homomorphismes homogènes de degré 0 ). Si l'on pose $C^{q}=\operatorname{Hom}_{s}\left(L^{q}, N\right)$, l'homomorphisme $L^{q+1} \rightarrow L^{q}$ définit par transposition un homomorphisme $d: C^{q} \rightarrow C^{q+1}$, vérifiant $d \circ d=0 ;$ ainsi $C=\sum_{q \geq 0} C^{q}$ se trouve muni d'une structure de complexe, et le $q$-ème groupe de cohomologie de $C$ n'est autre, par définition, que $\operatorname{Ext}_{s}^{q}(M, N) ;$ on montre qu'il ne dépend pas de la résolution choisie. Comme les $C^{q}$ sont des $S$-modules gradués, et que $d: C^{q} \rightarrow C^{q+1}$ est homogène de degré 0, les $\operatorname{Ext}_{s}^{q}(M, N)$ sont des $S$-modules gradués par des sous-espaces $\operatorname{Ext}_{s}^{q}(M, N)_{n} ;$ les $\operatorname{Ext}_{s}^{q}(M, N)_{n}$ sont les groupes de cohomologie du complexe formé par les $\operatorname{Hom}_{s}\left(L^{q}, N\right)_{n}$, c'est-àdire sont les foncteurs dérivés du foncteur $\operatorname{Hom}_{s}(M, N)_{n}$.

Rappelons les principales propriétés des Ext $_{s}^{q}$ :

$\operatorname{Ext}_{s}^{0}(M, N)=\operatorname{Hom}_{s}(M, N) ; \operatorname{Ext}_{s}^{q}(M, N)=0$ pour $q>r+1$ si $M$ est $\mathrm{de}$ type fini (à cause du théorème des syzygies de Hilbert, cf. [6], Chap. VIII, th. 6.5); $\operatorname{Ext}_{s}^{q}(M, N)$ est un $S$-module de type fini si $M$ et $N$ sont de type fini (car on peut choisir une résolution où les $L^{q}$ soient de type fini); pour tout $n \in \mathrm{Z}$, on a des isomorphismes canoniques:

$$
\operatorname{Ext}_{s}^{q}(M(n), N) \approx \operatorname{Ext}_{s}^{q}(M, N(-n)) \approx \operatorname{Ext}_{s}^{q}(M, N)(-n)
$$

Les suites exactes:

$$
0 \rightarrow N \rightarrow N^{\prime} \rightarrow N^{\prime \prime} \rightarrow 0 \text { et } \quad 0 \rightarrow M \rightarrow M^{\prime} \rightarrow M^{\prime \prime} \rightarrow 0
$$

${ }^{6}$ Lorsque $M$ n'est pas un module de type fini, les $\operatorname{Ext}_{S}^{t}(M, N)$ définis ci-dessus peuvent différer des $\operatorname{Ext}_{S}^{\mathcal{S}}(M, N)$ définis dans [6]: cela tient à ce que Hom, $(M, N)$ n'a pas le même sens dans les deux cas. Cependant, toutes les démonstrations de [6] sont valables sans changement dans le cas envisagé ici: cela se voit, soit directement, soit en appliquant l'Appendice de [6]. donnent naissance à des suites exactes:

$$
\begin{aligned}
    &\cdots \rightarrow \operatorname{Ext}_{s}^{q}(M, N) \rightarrow \operatorname{Ext}_{s}^{q}\left(M, N^{\prime}\right) \rightarrow \operatorname{Ext}_{s}^{q}\left(M, N^{\prime \prime}\right) \rightarrow \operatorname{Ext}_{s}^{q+1}(M, N) \rightarrow \cdots \\
    &\cdots \rightarrow \operatorname{Ext}_{S}^{q}\left(M^{\prime \prime}, N\right) \rightarrow \operatorname{Ext}_{S}^{q}\left(M^{\prime}, N\right) \rightarrow \operatorname{Ext}_{s}^{q}(M, N) \rightarrow \operatorname{Ext}_{s}^{q+1}\left(M^{\prime \prime}, N\right) \rightarrow \cdots
\end{aligned}
$$

69. Interprétation des $H_{k}^{q}(M)$ au moyen des $\operatorname{Ext}_{s}^{q} .$ Soit $M$ un $S$-module gradué, et soit $k$ un entier $\geqq 0 .$ Posons:

$$
B_{k}^{q}(M)=\sum_{n \in Z} H_{k}^{q}(M(n))
$$

avec les notations du n $^{\circ} 61$.

On obtient ainsi un groupe gradué, isomorphe au $q$-ème groupe de cohomologie du complexe $\sum_{n \in z} C_{k}(M(n))$; ce complexe peut être muni d'une structure de $S$-module compatible avec sa graduation en posant

$$
(P \cdot m)\left\langle i_{0} \cdots i_{q}\right\rangle=P \cdot m\left\langle i_{0} \cdots i_{q}\right\rangle, \text { si } P \in S_{p}, \text { et } m\left\langle i_{0} \cdots i_{q}\right\rangle \in C_{k}^{q}(M(n))
$$

comme l'opérateur cobord est un $S$-homomorphisme homogène de degré 0 , il s'ensuit que les $B_{k}^{q}(M)$ sont eux-mêmes des $S$-modules gradués.

Nous poserons

$$
B^{q}(M)=\lim _{k \rightarrow \infty} B_{k}^{q}(M)=\sum_{n \in Z} H^{q}(M(n))
$$

Les $B^{q}(M)$ sont des $S$-modules gradués. Pour $q=0$, on a

$$
B^{0}(M)=\sum_{n \in Z} H^{0}(M(n))
$$

et l'on retrouve le module noté $M^{4}$ au no 67 (lorsque $M$ vérifie la condition (TF)). Pour chaque $n \in \mathrm{Z}$, on a défini au no 62 une application linéaire $\alpha: M_{n} \rightarrow H^{0}(M(n))$; on vérifie immédiatement que la somme de ces applications définit un homomorphisme, que nous noterons encore $\alpha$, de $M$ dans $B^{0}(M)$.

Proposition $1 .$ Soit $k$ un entier $\geqq 0$, et soit $J_{k}$ l'idéal $\left(t_{0}^{k}, \cdots, t_{r}^{k}\right)$ de S. Pour tout S-module gradué $M$, les S-modules gradués $B_{k}^{q}(M)$ et $\operatorname{Ext}_{S}^{q}\left(J_{k}, M\right)$ sont isomorphes.

Soit $L_{k}^{q}, q=0, \cdots, r$, le $S$-module gradué libre admettant pour base des éléments $e\left\langle i_{0} \cdots i_{q}\right\rangle, 0 \leqq i_{0}<i_{1}<\cdots<i_{q} \leqq r$, de degré $k(q+1)$; on définit un opérateur $d: L_{k}^{q+1} \rightarrow L_{k}^{q}$ et un opérateur $\varepsilon: L_{k}^{0} \rightarrow J_{k}$ par les formules:

$$
\begin{aligned}
    d\left(e\left(i_{0} \cdots i_{q+1}\right\rangle\right) &=\sum_{j=0}^{j=q+1}(-1)^{j} t_{i}^{k} \cdot e\left\langle i_{0} \cdots \hat{\imath}_{j} \cdots i_{q+1}\right\rangle \\
    \varepsilon(e\langle i\rangle) &=t_{i}^{k}
\end{aligned}
$$

LEmme 1. La suite d'homomorphismes:

$$
0 \rightarrow L_{k}^{r} \stackrel{d}{\longrightarrow} L_{k}^{r-1} \rightarrow \cdots \rightarrow \quad L_{k}^{0} \stackrel{\varepsilon}{\longrightarrow} \quad J_{k} \rightarrow 0
$$

est une suite exacte.

Pour $k=1$, ce résultat est bien connu (cf. [6], Chap. VIII, $\S 4)$; le cas général se démontre de la même manière (ou s'y ramène); on peut également utiliser le théorème démontré dans [11]. La Proposition 1 se déduit immédiatement du Lemme, si l'on remarque que le complexe formé par les $\operatorname{Hom}_{s}\left(L_{k}^{q}, M\right)$ et le transposé de $d$ n'est autre que le complexe $\sum_{n e Z} C_{k}(M(n))$

CorolLaIRe 1. $H_{k}^{q}(M)$ est isomorphe $a \operatorname{Ext}_{s}^{q}\left(J_{k}, M\right)_{0}$.

En effet ces deux groupes sont les composantes de degré 0 des groupes gradués $B_{k}^{q}(M)$ et $\operatorname{Ext}_{s}^{q}\left(J_{k}, M\right)$

CorolLatRe 2. $H^{q}(M)$ est isomorphe à $\lim _{k \rightarrow \infty} \operatorname{Ext}_{S}^{q}\left(J_{k}, M\right)_{0}$.

On voit facilement que l'homomorphisme $\rho_{k}^{h}: H_{k}^{q}(M) \rightarrow H_{h}^{q}(M)$ du $\mathrm{n}^{\circ} 61$ est transformé par l'isomorphisme du Corollaire 1 en l'homomorphisme de

$$
\operatorname{Ext}_{s}^{q}\left(J_{k}, M\right)_{0} \text { dans } \operatorname{Ext}_{s}^{q}\left(J_{h}, M\right)_{0}
$$

défini par l'inclusion $J_{h} \rightarrow J_{k}$; d'où le Corollaire 2 .

RemarQUe. Soit $M$ un $S$-module gradué de type fini; $M$ définit (cf. $\mathrm{n}^{\circ} 48$ ) un faisceau algébrique cohérent $\mathfrak{F}^{\prime}$ sur $K^{r+1}$, donc sur $Y=K^{r+1}-\{0\}$, et l'on peut vérifier que $H^{q}\left(Y, \mathfrak{F}^{\prime}\right)$ est isomorphe à $B^{q}(M)$.

70. Définition des foncteurs $T^{q}(M) .$ Définissons d'abord la notion de module dual d'un $S$-module gradué. Soit $M$ un $S$-module gradué; pour tout $n \in \mathrm{Z}, M_{n}$ est un espace vectoriel sur $K$, dont nous désignerons l'espace vectoriel dual par $\left(M_{n}\right)^{\prime} .$ Posons:

$$
M^{*}=\sum_{n \in z} M_{n}^{*}, \quad \text { avec } \quad M_{n}^{*}=\left(M_{-n}\right)^{\prime}
$$

Nous allons munir $M^{*}$ d'une structure de $S$-module compatible avec sa graduation; pour tout $P \in S_{p}$, l'application $m \rightarrow P . m$ est une application $K$-linéaire de $M_{-n-p}$ dans $M_{-n}$, donc définit par transposition une application $K$-linéaire $\operatorname{de}\left(M_{-n}\right)^{\prime}=M_{n}^{*}$ dans $\left(M_{-n-p}\right)^{\prime}=M_{n+p}^{*} ;$ ceci définit la structure de $S$-module de $M^{*}$. On aurait également pu définir $M^{*}$ comme $\operatorname{Hom}_{s}(M, K)$, en désignant par $K$ le $S$-module gradué $S /\left(t_{0}, \cdots, t_{r}\right)$

Le $S$-module gradué $M^{*}$ est appelé le dual de $M ;$ on a $M^{* *}=M$ si chacun des $M_{n}$ est de dimension finie sur $K$, ce qui est le cas si $M=\Gamma(\mathfrak{F}), \mathfrak{F}$ étant un faisceau algébrique cohérent sur $X$, ou bien si $M$ est de type fini. Tout homomorphisme $\varphi: M \rightarrow N$ définit par transposition un homomorphisme de $N^{*}$ dans $M^{*}$. Si la suite $M \rightarrow N \rightarrow P$ est exacte, la suite $P^{*} \rightarrow N^{*} \rightarrow M^{*}$ l'est aussi; autrement dit, $M^{*}$ est un foncteur contravariant, et exact, du module $M$. Lorsque $I$ est un idéal homogène de $S$, le dual de $S / I$ n'est autre que l' "inverse system" de $I$, au sens de Macaulay (cf. $\left.[9], \mathrm{n}^{\circ} 25\right)$.

Soient maintenant $M$ un $S$-module gradué, et $q$ un entier $\geqq 0 .$ Nous avons défini au $\mathrm{n}^{\circ}$ précédent le $S$-module gradué $B^{q}(M) ;$ le module dual de $B^{q}(M)$ sera noté $T^{q}(M)$. On a donc, par définition:

$$
T^{q}(M)=\sum_{n \in Z} T^{q}(M)_{n}, \quad \text { avec } \quad T^{q}(M)_{n}=\left(H^{q}(M(-n))\right)^{\prime}
$$

Tout homomorphisme $\varphi: M \rightarrow N$ définit un homomorphisme de $B^{q}(M)$ dans $B^{q}(N)$, d'où un homomorphisme de $T^{q}(N)$ dans $T^{q}(M)$; ainsi les $T^{q}(M)$ sont des foncteurs contravariants de $M$ (nous verrons d'ailleurs au no 72 qu'ils peuvent s'exprimer très simplement en fonction des $\mathrm{Ext}_{s}$ ). Toute suite exacte:

$$
0 \rightarrow M \rightarrow N \rightarrow P \rightarrow 0
$$

donne naissance à une suite exacte:

$$
\cdots \rightarrow B^{q}(M) \rightarrow B^{q}(N) \rightarrow B^{q}(P) \rightarrow B^{q+1}(M) \rightarrow \cdots
$$

d'où, par transposition, une suite exacte:

$$
\cdots \rightarrow T^{q+1}(M) \rightarrow T^{q}(P) \rightarrow T^{q}(N) \rightarrow T^{q}(M) \rightarrow \cdots
$$

L'homomorphisme $\alpha: M \rightarrow B^{0}(M)$ définit par transposition un homomorphisme $\alpha^{*}: T^{0}(M) \rightarrow M^{*}$.

Puisque $B^{q}(M)=0$ pour $q>r$, on a $T^{q}(M)=0$ pour $q>r$

71. Détermination de $T^{r}(M) .$ (Dans ce $n^{\circ}$, ainsi que dans le suivant, nous supposerons que l'on a $r \geqq 1 ;$ le cas $r=0$ conduit à des énoncés quelque peu différents, et d'ailleurs triviaux).

Nous désignerons par $\Omega$ le $S$-module gradué $S(-r-1)$; c'est un module libre, admettant pour base un élément de degré $r+1$. On a vu au $\mathrm{n}^{\circ} 62$ que $H^{\prime}(\Omega)=H_{k}^{r}(\Omega)$ pour $k$ assez grand, et que $H_{k}^{r}(\Omega)$ admet une base sur $K$ formé de l'unique élément $\left(t_{0} \cdots t_{r}\right)^{k} / t_{0} \cdots t_{r} ;$ l'image dans $H^{r}(\Omega)$ de cet élément sera notée $\xi ; \xi$ constitue donc une base de $H^{r}(\Omega)$.

Nous allons maintenant définir un produit scalaire $\langle h, \varphi\rangle$ entre éléments $h \in B^{r}(M)_{-n}$ et $\varphi \in \operatorname{Hom}_{s}(M, \Omega)_{n}, M$ étant un $S$-module gradué quelconque. L'élément $\varphi$ peut être identifié à un élément de $\operatorname{Hom}_{s}(M(-n), \Omega)_{0}$, c'est-à-dire à un homomorphisme de $M(-n)$ dans $\Omega$; il définit donc, par passage aux groupes de cohomologie, un homomorphisme de $H^{r}(M(-n))=B^{r}(M)_{-n}$ dans $H^{r}(\Omega)$, que nous noterons encore $\varphi$. L'image de $h$ par cet homomorphisme est donc un multiple scalaire de $\xi$, et nous définirons $\langle h, \varphi\rangle$ par la formule:

$$
\varphi(h)=\langle h, \varphi\rangle \xi
$$

Pour tout $\varphi \in \operatorname{Hom}_{s}(M, \Omega)_{n}$, la fonction $h \rightarrow\langle h, \varphi\rangle$ est une forme linéaire sur $B^{r}(M)_{-n}$, donc peut être identifiée à un élément $\nu(\varphi)$ du dual de $B^{r}(M)_{-n}$, qui n'est autre que $T^{r}(M)_{n} .$ Nous avons ainsi défini une application homogène de degré 0

$$
\nu: \operatorname{Hom}_{s}(M, \Omega) \rightarrow T^{r}(M)
$$

et la formule $\langle P . h, \varphi\rangle=\langle h, P . \varphi\rangle$ montre que $\nu$ est un $S$-homomorphisme.

Proposition 2. L'homomorphisme $\nu: \operatorname{Hom}_{s}(M, \Omega) \rightarrow T^{r}(M)$ est bijectif.

Nous démontrerons d'abord la Proposition lorsque $M$ est un module libre. Si $M$ est somme directe de sous-modules homogènes $M^{\alpha}$, on a:

$$
\operatorname{Hom}_{s}(M, \Omega)_{n}=\prod_{\alpha} \operatorname{Hom}_{s}\left(M^{\alpha}, \Omega\right)_{n} \quad \text { et } \quad T^{r}(M)_{n}=\prod_{\alpha} T^{r}\left(M^{\alpha}\right)_{n}
$$

Ainsi, si la proposition est vraie pour les $M^{\alpha}$, elle l'est pour $M$, et cela ramène le cas des modules libres au cas particulier d'un module libre à un seul générateur, c'est-à-dire au cas où $M=S(m) .$ On peut alors identifier $\operatorname{Hom}_{s}(M, \Omega)_{n}$ à Hom $_{s}(S, S(n-m-r-1))_{0}$, c'est-à-dire à l'espace vectoriel des polynômes homogènes de degré $n-m-r-1$. Donc $\operatorname{Hom}_{s}(M, \Omega)_{n}$ admet pour base la famille des monômes $t_{0}^{\gamma_{0}} \cdots t_{r}^{\gamma_{r}}$, avec $\gamma_{i} \geqq 0$ et $\sum_{i=0}^{1=+0} \gamma_{i}=n-m-r-1$. D'autre part, nous avons vu au n ${ }^{\circ} 62$ que $H_{k}^{r}(S(m-n))$ admet pour base (si $k$ est assez grand) la famille des monômes $\left(t_{0} \cdots t_{r}\right)^{k} / t_{0}^{8} \cdots t_{r}^{\beta_{r}}$, avec $\beta_{i}>0$ et $\sum_{i=0}^{i=r} \beta_{i}=n-m .$ En posant $\beta_{i}=\gamma_{i}^{\prime}+1$, on peut écrire ces monômes sous la forme $\left(t_{0} \cdots t_{r}\right)^{k-1} / t_{0}^{\gamma_{0}} \cdots t_{r}^{\gamma_{r}}$, avec $\gamma_{i}^{\prime} \geqq 0$ et $\sum_{i=0}^{i=r} \gamma_{i}^{\prime}=n-m-r-1$. En remontant à la définition de $\langle h, \varphi\rangle$, on constate alors que le produit scalaire:

$$
\left\langle\left(t_{0} \cdots t_{r}\right)^{k-1} / l_{0}^{\gamma_{0}^{\prime}} \cdots t_{r}^{\gamma_{r}^{\prime}}, t_{0}^{\gamma_{0}} \cdots t_{r}^{\gamma_{r}}\right\rangle
$$

est toujours nul, sauf si $\gamma_{i}=\gamma_{i}^{\prime}$ pour tout $i$, auquel cas il est égal à $1 .$ Cela signifie que $\nu$ transforme la base des $t_{0}^{\gamma_{0}} \cdots t_{r}^{\gamma_{r}}$ en la base duale de la base des $\left(t_{0} \cdots t_{r}\right)^{k-1} / t_{0}^{\gamma_{0}} \cdots t_{r}^{\gamma_{r}}$, donc est bijectif, ce qui achève de prouver la Proposition dans le cas où $M$ est libre.

Passons maintenant au cas général. Choisissons une suite exacte

$$
L^{1} \rightarrow L^{0} \rightarrow M \rightarrow 0
$$

où $L^{0}$ et $L^{1}$ soient libres. Considérons le diagramme commutatif suivant:

![](https://cdn.mathpix.com/cropped/9337510346c782fee82ef71daf8d2a35-09.jpg?height=98&width=542&top_left_y=612&top_left_x=170)

La première ligne de ce diagramme est une suite exacte, d'après les propriétés générales du foncteur $\mathrm{Hom}_{s}$; la seconde est aussi une suite exacte, car c'est la suite duale de la suite

$$
B^{r}\left(L^{1}\right) \rightarrow B^{r}\left(L^{0}\right) \rightarrow B^{r}(M) \rightarrow 0
$$

suite qui est elle-même exacte, à cause de la suite exacte de cohomologie des $B^{q}$, et $d u$ fait que $B^{r+1}(M)=0$ quel que soit le $S$-module gradué $M .$ D'autre part, les deux homomorphismes verticaux

$$
\nu: \operatorname{Hom}_{s}\left(L^{0}, \Omega\right) \rightarrow T^{r}\left(L^{0}\right) \text { et } \nu: \operatorname{Hom}_{s}\left(L^{1}, \Omega\right) \rightarrow T^{r}\left(L^{1}\right)
$$

sont bijectifs, on vient de le voir. Il s'ensuit que

$$
\nu: \operatorname{Hom}_{s}(M, \Omega) \rightarrow T^{r}(M)
$$

est également bijectif, ce qui achève la démonstration.

72. Détermination des $T^{q}(M)$. Nous allons démontrer le théorème suivant, qui généralise la Proposition 2:

THÉonème 1. Soit $M$ un S-module gradué. Pour $q \neq r$, les S-modules gradués $T^{r-q}(M)$ et $\operatorname{Ext}_{S}^{q}(M, \Omega)$ sont isomorphes. De plus, on a une suite exacte:

$0 \rightarrow \operatorname{Ext}_{s}^{r}(M, \Omega) \rightarrow T^{0}(M) \stackrel{\alpha^{*}}{\longrightarrow} M^{*} \rightarrow \operatorname{Ext}_{s}^{r+1}(M, \Omega) \rightarrow 0$ Nous allons utiliser la caractérisation axiomatique des foncteurs dérivés donnée dans [6], Chap. III, $\S 5 .$ Pour cela, définissons d'abord de nouveaux foncteurs $E^{q}(M)$ de la façon suivante:

Pour $q \neq r, r+1$,

$$
E^{q}(M)=T^{r-q}(M)
$$

Pour $q=r$,

$$
E^{r}(M)=\operatorname{Ker}\left(\alpha^{*}\right)
$$

Pour $q=r+1$,

$$
E^{r+1}(M)=\operatorname{Coker}\left(\alpha^{*}\right)
$$

Les $E^{q}(M)$ sont des foncteurs additifs contravariants de $M$, jouissant des propriétés suivantes:

(i) $E^{0}(M)$ est isomorphe à $\operatorname{Hom}_{s}(M, \Omega)$.

C'est ce qu'affirme la Proposition 2 .

(ii) Si $L$ est libre, $E^{q}(L)=0$ pour tout $q>0$.

Il suffit de le vérifier pour $L=S(n)$, auquel cas cela résulte du no $62 .$

(iii) $A$ toute suite exacte $0 \rightarrow M \rightarrow N \rightarrow P \rightarrow 0$ est associée une suite d'opérateurs cobords $d^{q}: E^{q}(M) \rightarrow E^{q+1}(P)$, et la suite:

$$
\cdots \quad \rightarrow \quad E^{q}(P) \quad \rightarrow \quad E^{q}(N) \rightarrow E^{q}(M) \stackrel{d^{q}}{\rightarrow} E^{q+1}(P) \rightarrow \cdots
$$

est exacte.

La définition de $d^{q}$ est évidente si $q \neq r-1, r:$ c'est l'homomorphisme de $T^{r-q}(M)$ dans $T^{r-q-1}(P)$ défini au no $70 .$ Pour $q=r-1$ ou $r$, on utilise le diagramme commutatif suivant:

![](https://cdn.mathpix.com/cropped/9337510346c782fee82ef71daf8d2a35-10.jpg?height=99&width=506&top_left_y=699&top_left_x=193)

Ce diagramme montre tout d'abord que l'image de $T^{1}(M)$ est contenue dans le noyau de $\alpha^{*}: T^{0}(P) \rightarrow P^{*}$, qui n'est autre que $E^{r}(P) .$ D'où la définition de $d^{r-1}: E^{r-1}(M) \rightarrow E^{r}(P)$

Pour définir $d^{r}: \operatorname{Ker}\left(T^{0}(M) \rightarrow M^{*}\right) \rightarrow \operatorname{Coker}\left(T^{0}(P) \rightarrow P^{*}\right)$, on utilise le procédé de [6], Chap. III, Lemme 3.3: si $x \in \operatorname{Ker}\left(T^{0}(M) \rightarrow M^{*}\right)$, il existe $y \in P^{*}$ et $z \in T^{0}(N)$ tels que $x$ soit image de $z$ et que $y$ et $z$ aient même image dans $N^{*}$; on pose alors $d^{r}(x)=y$.

L'exactitude de la suite

$$
\cdots \rightarrow E^{q}(P) \quad \rightarrow \quad E^{q}(N) \rightarrow E^{q}(M) \stackrel{d^{q}}{\longrightarrow} E^{q+1}(P) \rightarrow \cdots
$$

résulte de l'exactitude de la suite

$$
\cdots \rightarrow T^{r-q}(P) \rightarrow T^{r-q}(N) \rightarrow T^{r-q}(M) \rightarrow T^{r-q-1}(P) \rightarrow \cdots
$$

et de [6], loc. cit.

(iv) L'isomorphisme de (i) et les opérateurs $d^{q}$ de (iii) sont "naturels".

Cela résulte immédiatement de leurs définitions. Comme les propriétés (i) à (iv) caractérisent les foncteurs dérivés du foncteur $\operatorname{Hom}_{s}(M, \Omega)$, on a $E^{q}(M) \approx \operatorname{Ext}_{s}^{q}(M, \Omega)$, ce qui démontre le Théorème.

Corollatre 1. Si $M$ vérifie (TF), $H^{q}(M)$ est isomorphe a l'espace vectoriel dual de $\operatorname{Ext}_{s}^{r-q}(M, \Omega)_{0}$ pour tout $q \geqq 1$

En effet, nous savons que $H^{q}(M)$ est un espace vectoriel de dimension finie dont le dual est isomorphe à $\operatorname{Ext}_{s}^{r-q}(M, \Omega)_{0}$

CorolLaIRe 2. Si M vérifie (TF), les $T^{q}(M)$ sont des S-modules gradués de type fini pour $q \geqq 1$, et $T^{0}(M)$ vérifie (TF).

On peut remplacer $M$ par un module de type fini sans changer les $B^{q}(M)$, donc les $T^{q}(M)$. Les $\operatorname{Ext}_{s}^{r-q}(M, \Omega)$ sont alors des $S$-modules de type fini, et l'on a $M^{*} \epsilon$ e, d'où le Corollaire.

\section{$\S$ 5. Applications aux faisceaux algébriques cohérents}

73. Relations entre les foncteurs $\mathrm{Ext}_{s}^{q}$ et $\mathrm{Ext}_{\theta_{x}}^{q} .$ Soient $M$ et $N$ deux $S$-modules gradués. Si $x$ est un point de $X=\mathbf{P}_{r}(K)$, nous avons défini au $\mathrm{n}^{\circ} 57$ les $\mathcal{O}_{x}^{-}$ modules $M_{x}$ et $N_{x} ;$ nous allons mettre en relation les $\operatorname{Ext}_{\theta_{x}}^{q}\left(M_{x}, N_{x}\right)$ avec le $S$-module gradué $\operatorname{Ext}_{s}^{q}(M, N):$

Proposition 1. Supposons que M soit de type fini. Alors:

(a) Le faisceau $Q\left(\mathrm{Hom}_{s}(M, N)\right)$ est isomorphe au faisceau $\operatorname{Hom}_{\mathcal{O}}(\alpha(M), \mathbb{Q}(N))$.

(b) Pour tout $x \in X$, le $\mathcal{O}_{x}$-module $\operatorname{Ext}_{s}^{q}(M, N)_{x}$ est isomorphe au $\mathcal{O}_{x}$-module $\operatorname{Ext}_{\theta_{-}}^{q}\left(M_{x}, N_{x}\right)$

Définissons d'abord un homomorphisme $\iota_{x}: \operatorname{Hom}_{s}(M, N)_{x} \rightarrow \operatorname{Hom}_{\theta_{x}}\left(M_{x}, N_{x}\right)$. Un élément du premier module est une fraction $\varphi / P$, avec $\varphi \epsilon \operatorname{Hom}_{s}(M, N)_{n}$, $P \in S(x), P$ homogène de degré $n$; si $m / P^{\prime}$ est un élément de $M_{x}, \varphi(m) / P P^{\prime}$ est un élément de $N_{x}$ qui ne dépend que de $\varphi / P$ et de $m / P^{\prime}$, et l'application $m / P^{\prime} \rightarrow$ $\varphi(m) / P P^{\prime}$ est un homomorphisme $\iota_{x}(\varphi / P): M_{x} \rightarrow N_{x} ;$ ceci définit $\iota_{x} .$ D'après la Proposition $5 \mathrm{du} \mathrm{n}^{\circ} 14, \mathrm{Hom}_{\theta_{x}}\left(M_{x}, \mathcal{N}_{x}\right)$ peut être identifié à

$$
\operatorname{Hom}_{\mathcal{\theta}}(\mathbb{Q}(M), \mathbb{Q}(N))_{x}
$$

cette identification transforme $\iota_{x}$ en

$$
\iota_{x}: \alpha\left(\operatorname{Hom}_{s}(M, N)\right)_{x} \rightarrow \operatorname{Hom}_{\theta}(\alpha(M), \alpha(N))_{x},
$$

et l'on vérifie facilement que la collection des $\iota_{x}$ est un homomorphisme

$$
\iota: a\left(\operatorname{Hom}_{s}(M, N)\right) \rightarrow \operatorname{Hom}_{\theta}(Q(M), Q(N))
$$

Lorsque $M$ est un module libre de type fini, $\iota_{x}$ est bijectif : en effet, il suffit de le voir lorsque $M=S(n)$, auquel cas c'est immédiat.

Si maintenant $M$ est un $S$-module gradué de type fini quelconque, choisissons une résolution de $M$ :

$$
\cdots \rightarrow L^{q+1} \rightarrow L^{q} \rightarrow \cdots \rightarrow L^{0} \rightarrow M \rightarrow 0
$$

oì les $L^{q}$ soient libres de type fini. et considérons le complexe $C$ formé par les $\operatorname{Hom}_{s}\left(L^{q}, N\right)$. Les groupes de cohomologie de $C$ sont les $\operatorname{Ext}_{s}^{2}(M, N)$; autrement dit, si l'on désigne par $B^{q}$ et $Z^{q}$ les sous-modules de $C^{q}$ formés respectivement des cobords et des cocycles, on a des suites exactes:

et

$$
\begin{gathered}
    0 \rightarrow Z^{q} \rightarrow C^{q} \rightarrow B^{q+1} \rightarrow 0 \\
    0 \rightarrow B^{q} \rightarrow Z^{q} \rightarrow \operatorname{Ext}_{s}^{q}(M, N) \rightarrow 0
\end{gathered}
$$

Comme le foncteur $Q(M)$ est exact, les suites

$$
0 \rightarrow Z_{x}^{q} \rightarrow C_{x}^{q} \rightarrow B_{x}^{q+1} \rightarrow 0
$$

et

$$
0 \rightarrow B_{x}^{q} \rightarrow Z_{x}^{q} \rightarrow \operatorname{Ext}_{s}^{q}(M, N)_{x} \rightarrow 0
$$

sont aussi exactes.

Mais, d'après ce qui précède, $C_{x}^{q}$ est isomorphe à $\operatorname{Hom}_{\theta_{x}}\left(L_{x}^{q}, N_{x}\right) ;$ les $\operatorname{Ext}_{s}^{t}(M, N)_{x}$ sont isomorphes aux groupes de cohomologie du complexe formé par les $\operatorname{Hom}_{\theta_{z}}\left(L_{x}^{q}, N_{x}\right)$, et, comme les $L_{x}^{q}$ sont évidemment $\mathcal{\theta}_{x}$-libres, on retrouve bien la définition des $\operatorname{Ext}_{\theta_{x}}^{q}\left(M_{x}, N_{x}\right)$, ce qui démontre (b); pour $q=0$, ce qui précède montre que $\iota_{x}$ est bijectif, donc $\iota$ est un isomorphisme, d'où (a).

74. Nullité des groupes de cohomologie $H^{q}(X, \mathfrak{F}(-n))$ pour $n \rightarrow+\infty$.

THÉoREme 1. Soit $\mathcal{F}$ un faisceau algébrique cohérent sur $X$, et soit $q$ un entier $\geq 0$. Les deux conditions suivantes sont équivalentes:

(a) $H^{q}(X, \mathcal{F}(-n))=0$ pour $n$ assez grand.

(b) $\operatorname{Ext}_{\theta_{x}}^{r-q}\left(\mathfrak{F}_{x}, \mathcal{O}_{x}\right)=0$ pour tout $x \in X$.

D'après le Théorème 2 du $\mathrm{n}^{\circ} 60$, on peut supposer que $\mathcal{F}=\alpha(M)$, où $M$ est un $S$-module gradué de type fini, et, d'après le $\mathrm{n}^{\circ} 64, H^{q}(X, \mathfrak{F}(-n))$ est isomorphe à $H^{q}(M(-n))=B^{q}(M)_{-n} ;$ donc la condition (a) équivaut à

$$
T^{q}(M)_{n}=0
$$

pour $n$ assez grand, c'est-à-dire à $T^{q}(M) \in \mathcal{C} .$ D'après le Théorème $1 \mathrm{du} \mathrm{n}^{\circ} 72$ et le fait que $M^{*} \epsilon \mathcal{C}$ puisque $M$ est de type fini, cette dernière condition équivaut a $\operatorname{Ext}_{s}^{r-q}(M, \Omega) \in \mathcal{C} ;$ puisque $\operatorname{Ext}_{s}^{r-q}(M, \Omega)$ est un $S$-module de type fini,

$$
\operatorname{Ext}_{s}^{r-q}(M, \Omega) \in \mathcal{C}
$$

équivaut à $\operatorname{Ext}_{s}^{r-q}(M, \Omega)_{x}=0$ pour tout $x \in X$, d'après la Proposition $5 \mathrm{du}$ no 58 ; enfin, la Proposition 1 montre que $\operatorname{Ext}_{s}^{r-q}(M, \Omega)_{x}=\operatorname{Ext}_{\theta_{x}}^{r-q}\left(M_{x}, \Omega_{x}\right)$, et comme $M_{x}$ est isomorphe à $\mathfrak{F}_{x}$, et $\Omega_{x}$ isomorphe à $\mathcal{O}(-r-1)_{x}$, donc à $\mathcal{O}_{x}$, ceci achève la démonstration.

Pour énoncer le Théorème 2 , nous aurons besoin de la notion de dimension d'un $\mathcal{\theta}_{x}$-module. Rappelons $([6]$, Chap. VI, $\S 2)$ qu'un $\mathcal{O}_{x}$-module de type fini $P$ est dit de dimension $\leqq p$ s'il existe une suite exacte de $\mathcal{O}_{x}$-modules:

$$
0 \rightarrow L_{p} \rightarrow L_{p-1} \rightarrow \cdots \rightarrow L_{0} \rightarrow P \rightarrow 0
$$

où chaque $L_{p}$ soit libre (cette définition équivaut à celle de [6], loc. cit., du fait que tout $\mathcal{O}_{x}$-module projectif de type fini est libre-cf. [6], Chap. VI\PiI, Th. $\left.6.1^{\prime}\right)$. Tout $\mathcal{O}_{x}$-module de type fini est de dimension $\leqq r$, d'après le théorème des syzygies (cf. [6], Chap. VIII, Th. $6.2^{\prime}$ ).

Lemme 1. Soient $P$ un $\mathcal{O}_{x}$-module de type fini, et soit $p$ un entier $\geqq 0$. Les deux conditions suivantes sont équivalentes:

(i) $P$ est de dimension $\leqq p$.

(ii) $\operatorname{Ext}_{\theta_{x}}^{m}\left(P, \mathcal{O}_{x}\right)=0$ pour tout $m>p$

Il est clair que (i) entraîne (ii). Démontrons que (ii) entraîne (i) par récurrence descendante sur $p$; pour $p \geqq r$, le Lemme est trivial, puisque (i) est toujours vérifié; passons maintenant de $p+1$ à $p ;$ soit $N$ un $\theta_{x}$-module de type fini quelconque. On peut trouver une suite exacte $0 \rightarrow R \rightarrow L \rightarrow N \rightarrow 0$, où $L$ est libre de type fini (parce que $\mathcal{O}_{x}$ est noethérien). La suite exacte

$$
\operatorname{Ext}_{\theta_{x}}^{p+1}(P, L) \rightarrow \operatorname{Ext}_{\theta_{x}}^{p+1}(P, N) \rightarrow \operatorname{Ext}_{\theta_{x}}^{p+2}(P, R)
$$

montre que $\operatorname{Ext}_{\theta_{x}}^{p+1}(P, N)=0:$ en effet, on a $\operatorname{Ext}_{\theta_{x}}^{p+1}(P, L)=0$ d'après la condition (ii), et $\operatorname{Ext}_{\theta_{x}}^{p+2}(P, R)=0$ puisque $\operatorname{dim} P \leqq p+1$ d'après l'hypothèse de récurrence. Comme cette propriété caractérise les modules de dimension $\leqq p$, le Lemme est démontré.

En combinant le Lemme et le Théorème 1, on obtient:

THÉonème $2 .$ Soit $\mathfrak{F}$ un faisceau algébrique cohérent sur $X$, et soit $p$ un entier $\geqq 0$. Les deux conditions suivantes sont équivalentes:

(a) $H^{q}(X, \mathfrak{F}(-n))=0$ pour $n$ assez grand et $0 \leqq q<p$.

(b) Pour tout $x \in X$, le $\mathcal{O}_{x}$-module $\mathcal{F}_{x}$ est de dimension $\leqq r-p$.

75. Variétés sans singularités. Le résultat suivant joue un rôle essentiel dans l'extension au cas abstrait du "théorème de dualité" de $[15]$ :

THÉonème 3. Soit $V$ une sous-variété sans singularités de l'espace projectif $P_{r}(K) ;$ supposons que toutes les composantes irréductibles de $V$ aient la même dimension $p .$ Soit $F$ un faisceau algébrique cohérent sur $V$ tel que, pour tout $x \in V$, $\mathfrak{F}_{x}$ soit un module libre sur $\boldsymbol{\theta}_{x, \boldsymbol{v}}$. On a alors $H^{q}(V, \mathfrak{F}(-n))=0$ pour $n$ assez grand et $0 \leqq q<p$

D'après le Théorème 2, tout revient à montrer que $\mathcal{O}_{x, v}$, considéré comme $\mathcal{O}_{x}$-module, est de dimension $\leqq r-p$. Désignons par $\mathfrak{g}_{x}(V)$ le noyau de Phomomorphisme canonique $\varepsilon_{x}: \mathcal{O}_{x} \rightarrow \mathcal{O}_{x, V} ;$ puisque le point $x$ est simple sur $V$, on sait (cf. [18], th. 1) que cet idéal est engendré par $r-p$ éléments $f_{1}, \cdots, f_{r-p}$, et le théorème de Cohen-Macaulay (cf. [13], p. 53, prop. 2 ) montre que l'on a

$$
\left(f_{1}, \cdots, f_{i-1}\right): f_{i}=\left(f_{1}, \cdots, f_{i-1}\right) \quad \text { pour } 1 \leqq i \leqq r-p
$$

Désignons alors par $L_{a}$ le $\mathcal{O}_{x}$-module libre admettant pour base des éléments $e\left\langle i_{1} \cdots i_{q}\right\rangle$ correspondant aux suites $\left(i_{1}, \cdots, i_{q}\right)$ telles que

$$
1 \leqq i_{1}<i_{2}<\cdots<i_{q} \leqq r-p
$$

pour $q=0$, prenons $L_{0}=\mathcal{O}_{x}$, et posons:

$$
\begin{aligned}
    d\left(e\left(i_{1} \cdots i_{q}\right\rangle\right) &=\sum_{j=1}^{j=q}(-1)^{j} f_{i_{i}} \cdot e\left(i_{1} \cdots \hat{\imath}_{j} \cdots i_{q}\right\rangle \\
    d(e\langle i\rangle) &=f_{i}
\end{aligned}
$$

D'après [6], Chap. VIII, prop. 4.3, la suite

$0 \rightarrow L_{r-p} \quad \stackrel{d}{\longrightarrow} \quad L_{r-p-1} \quad \stackrel{d}{\longrightarrow} \quad \cdots \quad \stackrel{d}{\longrightarrow} \quad L_{0} \stackrel{\varepsilon_{x}}{\longrightarrow} \quad \mathcal{O}_{x, v} \rightarrow 0$

est exacte, ce qui démontre bien que $\operatorname{dim}_{\theta_{x}}\left(\mathcal{O}_{x, v}\right) \leqq r-p$, cqfd.

CorolLAIRE. On a $H^{q}\left(V, \mathcal{O}_{v}(-n)\right)=0$ pour $n$ assez grand et $0 \leqq q<p$.

REMARQUE. La démonstration ci-dessus s'applique, plus généralement, chaque fois que l'idéal $g_{x}(V)$ admet un système de $r-p$ générateurs, autrement dit lorsque la variété $V$ est localement une intersection complète, en tout point.

76. Variêtés normales. Nous aurons besoin du Lemme suivant:

LEmME 2 . Soit $M$ un $\mathcal{O}_{x}$-module de type fini, et soit $f$ un élément non inversible de $\mathcal{O}_{x}$, tel que la relation f.m $=0$ entraîne $m=0$ si $m \in M$. La dimension $d u$ $\mathcal{O}_{x}$-module M/fM est alors égale a la dimension de M augmentée d'une unité.

Par hypothèse, on a une suite exacte $0 \rightarrow M \stackrel{\alpha}{\rightarrow} M \rightarrow M / f M \rightarrow 0$, où $\alpha$ est la multiplication par $f$. Si $N$ est un $\mathcal{O}_{x}$-module de type fini, on a donc une suite exacte:

$\begin{aligned} \cdots \rightarrow \operatorname{Ext}_{\theta_{x}}^{q}(M, N) \stackrel{\alpha}{\longrightarrow} \operatorname{Ext}_{\theta_{x}}^{q}(M, N) \rightarrow \operatorname{Ext}_{\theta_{x}}^{q+1}(M / f M, N) & \rightarrow \\ \operatorname{Ext}_{\theta_{x}}^{q+1}(M, N) & \rightarrow \cdots \end{aligned}$

Notons $p$ la dimension de $M$. En faisant $q=p+1$ dans la suite exacte précédente, on voit que $\operatorname{Ext}_{\theta_{x}}^{p+2}(M / f M, N)=0$, ce qui entraîne ([6], Chap. VI, $\S 2$ ) que $\operatorname{dim}(M / f M) \leqq p+1 .$ D'autre part, puisque $\operatorname{dim} M=p$, on peut choisir un $N$ tel que $\operatorname{Ext}_{\theta_{x}}^{p}(M, N) \neq 0 ;$ en faisant alors $q=p$ dans la suite exacte ci-dessus, on voit que $\operatorname{Ext}_{\theta_{x}}^{p+1}(M / f M, N)$ s'identifie au conoyau de

$$
\operatorname{Ext}_{\mathcal{O}_{x}}^{p}(M, N) \stackrel{\alpha}{\longrightarrow} \operatorname{Ext}_{\mathcal{O}_{x}}^{p}(M, N)
$$

comme ce dernier homomorphisme n'est autre que la multiplication par $f$, et que $f$ n'est pas inversible dans l'anneau local $\theta_{x}$, il résulte de [6], Chap. VIII, prop. 5.1' que ce conoyau est $\neq 0$, ce qui montre que $\operatorname{dim} M / f M \geqq p+1$ et achève la démonstration.

Nous allons maintenant démontrer un résultat qui est en rapport étroit avec le "lemme d'Enriques-Severi", dû à Zariski [19]:

THÉoRÈe 4. Soit $V$ une sous-variété irréductible, normale, de dimension $\geqq 2$, de l'espace projectif $\mathbf{P}_{r}(K) .$ Soit $\mathfrak{F}$ un faisceau algébrique cohérent sur $V$ tel que, pour tout $x \in V, \mathfrak{F}_{x}$ soit un module libre sur $\mathcal{O}_{x, \mathrm{v}} .$ On a alors $H^{1}(V, \mathfrak{F}(-n))=0$ pour $n$ assez grand.

D'après le Théorème 2, tout revient à montrer que $\mathcal{O}_{x, v}$, considéré comme $\mathcal{O}_{x}$-module, est de dimension $\leqq r-2$. Choisissons d'abord un élément $f \in \mathcal{O}_{x}$ tel que $f(x)=0$ et que l'image de $f$ dans $\mathcal{O}_{x, v}$ ne soit pas nulle; c'est possible $\mathrm{du}$ fait que $\operatorname{dim} V>0 .$ Puisque $V$ est irréductible, $\mathcal{O}_{x, v}$ est un anneau d'intégrité, et l'on peut appliquer le Lemme 2 au couple $\left(\mathcal{O}_{x, v}, f\right) ;$ on a donc:

$$
\operatorname{dim} \mathcal{O}_{x, v}=\operatorname{dim} \mathcal{O}_{x, v} /(f)-1, \text { avec }(f)=f . \mathcal{O}_{x, v}
$$

Puisque $\mathcal{O}_{x, v}$ est un anneau intégralement clos, tous les idéaux premiers $\mathfrak{p}^{\alpha}$ de l'idéal principal $(f)$ sont minimaux (cf. [12], p. 136 , ou $\left.[9], \mathrm{n}^{\circ} 37\right)$, et aucun d'eux n'est donc égal à l'idéal maximal $\mathrm{m}$ de $\mathcal{O}_{x, v}$ (sinon, on aurait $\left.\operatorname{dim} V \leqq 1\right)$. On peut donc trouver un élément $g \in \mathrm{m}$ n'appartenant à aucun des $\mathfrak{p}^{\alpha}$; cet élément $g$ n'est pas diviseur de 0 dans l'anneau quotient $\mathcal{O}_{x, v} /(f) ;$ en appelant $\bar{g}$ un représentant de $g$ dans $\mathcal{O}_{x}$, on voit que l'on peut appliquer le Lemme 2 au couple $\left(\mathcal{O}_{x, \mathrm{v}} /(f), \bar{g}\right) ;$ on a donc:

$$
\operatorname{dim} \mathcal{O}_{x, v} /(f)=\operatorname{dim} \mathcal{O}_{x, v} /(f, g)-1
$$

Mais, d'après le théorème des syzygies déjà cité, on a $\operatorname{dim} \mathcal{O}_{x, v} /(f, g) \leqq r ;$ d'où $\operatorname{dim} \mathcal{O}_{x, v} /(f) \leqq r-1$ et $\operatorname{dim} \mathcal{O}_{x, v} \leqq r-2$, cqfd.

CoRolLAIRE. On a $H^{1}\left(V, \mathcal{O}_{V}(-n)\right)=0$ pour $n$ assez grand.

REMARQUEs. (1) Le raisonnement fait ci-dessus est classique en théorie des syzygies. Cf. par exemple W. Gröbner, Moderne Algebraische Geometrie, $152.6$ et 153.1.

(2) Même si la dimension de $V$ est $>2$, on peut avoir $\operatorname{dim} \mathcal{O}_{x, y}=r-2$. C'est notamment le cas lorsque $V$ est un cône dont la section hyperplane $W$ est une variété projectivement normale et irrégulière (i.e. $\left.H^{1}\left(W, \mathcal{O}_{w}\right) \neq 0\right)$.

77. Caractérisation homologique des variétés $k$-fois de première espèce. Soit $M$ un $S$-module gradué de type fini. On démontre, par un raisonnement identique à celui du Lemme 1 :

LEmMe 3. Pour que $\operatorname{dim} M \leqq k$, il faut et il suffit que $\operatorname{Ext}_{s}^{q}(M, S)=0$ pour $q>k$

Puisque $M$ est gradué, on a $\operatorname{Ext}_{s}^{q}(M, \Omega)=\operatorname{Ext}_{s}^{q}(M, S)(-r-1)$, donc la condition ci-dessus équivaut à $\operatorname{Ext}_{s}^{q}(M, \Omega)=0$ pour $q>k$. Compte tenu du Théorème 1 du no 72, on en conclut:

Proposition 2. (a) Pour que $\operatorname{dim} M \leqq r$, il faut et il suffit que $\alpha: M_{n} \rightarrow$ $H^{0}(M(n))$ soit injectif pour tout $n \in \mathbf{Z}$.

(b) Si $k$ est un entier $\geqq 1$, pour que $\operatorname{dim} M \leqq r-k$, il faut et il suffit que $\alpha: M_{n} \rightarrow H^{0}(M(n))$ soit bijectif pour tout $n \in \mathbf{Z}$, et que $H^{q}(M(n))=0$ pour $0<q<k$ et tout $n \in \mathbf{Z}$

Soit $V$ une sous-variété fermée de $\mathbf{P}_{r}(K)$, et soit $I(V)$ l'idéal des polynômes homogènes nuls sur $V$. Posons $S(V)=S / I(V)$, c'est un $S$-module gradué dont le faisceau associé n'est autre que $\dot{O}_{v} .$ Nous dirons $^{6}$ que $V$ est une sous-variété $" k$-fois de première espèce" de $\mathbf{P}_{r}(K)$ si la dimension du $S$-module $S(V)$ est $\leqq r-k$. Il est immédiat que $\alpha: S(V)_{n} \rightarrow H^{0}\left(V, \mathcal{O}_{V}(n)\right)$ est injectif pour tout $n \in \mathrm{Z}$, donc toute variété est 0 -fois de première espèce. En appliquant la Proposition précédente à $M=S(V)$, on obtient:

${ }^{6}$ Cf. P. Dubreil, Sur la dimension des ideaux de polynômes, J. Math. Pures App., 15 , 1936, p. $271-283$. Voir aussi W. Gröbner, Moderne Algebraische Geometrie, $\S 5$. Proposition 3. Soit $k$ un entier $\geqq 1 .$ Pour que la sous-variété $V$ soit $k$-fois de première espèce, il faut et il suffit que les conditions suivantes soient vérifiées pour tout $n \in \mathbf{Z}:$

(i) $\alpha: S(V)_{n} \rightarrow H^{0}\left(V, \mathcal{O}_{V}(n)\right)$ est bijectif.

(ii) $H^{q}\left(V, \mathcal{O}_{V}(n)\right)=0$ pour $0<q<k$.

(La condition (i) peut aussi s'exprimer en disant que la série linéaire découpée sur $V$ par les formes de degré $n$ est complète, ce qui est bien connu.)

En comparant avec le Théorème 2 (ou en raisonnant directement), on obtient:

Corollalne. Si $V$ est $k$-fois de première espèce, on a $H^{q}\left(V, \mathcal{O}_{V}\right)=0$ pour

$0<q<k$ et, pour tout $x \in V$, la dimension du $\mathcal{O}_{x}$-module $\mathcal{O}_{x, \mathrm{v}}$ est $\leqq r-k$.

Si $m$ est un entier $\geqq 1$, notons $\varphi_{m}$ le plongement de $\mathbf{P}_{r}(K)$ dans un espace projectif de dimension convenable donné par les monômes de degré $m$ (cf. [8], Chap. XVI, $\S 6$, ou bien $\mathrm{n}^{\circ} 52$, démonstration du Lemme 2). Le corollaire cidessus admet alors la réciproque suivante:

Proposition 4. Soit $k$ un entier $\geqq 1$, et soit $V$ une sous-variété connexe et fermée de $\mathbf{P}_{r}(K)$. Supposons que $H^{q}\left(V, \mathcal{O}_{v}\right)=0$ pour $0<q<k$, et que, pour tout $x \in V$, la dimension du $\mathcal{O}_{x}$-module $\mathcal{O}_{x, v}$ soit $\leqq r-k$.

Alors, pour tout $m$ assez grand, $\varphi_{m}(V)$ est une sous-variété $k$-fois de première espèce.

Du fait que $V$ est connexe, on a $H^{0}\left(V, \mathcal{O}_{V}\right)=K .$ En effet, si $V$ est irréductible, c'est évident (sinon $H^{0}\left(V, \mathcal{O}_{V}\right)$ contiendrait une algèbre de polynômes, et ne serait pas de dimension finie sur $K$ ); si $V$ est réductible, tout élément $f \in H^{0}\left(V, \mathcal{O}_{V}\right)$ induit une constante sur chacune des composantes irréductibles de $V$, et ces constantes sont les mêmes, à cause de la connexion de $V$.

Du fait que $\operatorname{dim} \mathcal{\theta}_{x, v} \leqq r-1$, la dimension algébrique de chacune des composantes irréductibles de $V$ est au moins égale à $1 .$ Il en résulte que

$$
H^{0}\left(V, \mathcal{O}_{V}(-n)\right)=0
$$

pour $n>0$ (car si $f \in H^{0}\left(V, \mathcal{O}_{v}(-n)\right)$ et $f \neq 0$, les $f^{\kappa} . g$, avec $g \in S(V)_{n k}$ formeraient un sous-espace vectoriel de $H^{0}\left(V, \mathcal{O}_{V}\right)$ de dimension $>1$ ).

Ceci étant précisé, notons $V_{m}$ la sous-variété $\varphi_{m}(V)$; on a évidemment

$$
\mathcal{O}_{v_{m}}(n)=\mathcal{O}_{V}(n m)
$$

Pour $m$ assez grand, les conditions suivantes sont satisfaites:

(a) $\alpha: S(V)_{n m} \rightarrow H^{0}\left(V, \mathcal{O}_{V}(n m)\right)$ est bijectif pour tout $n \geqq 1$

Cela résulte de la Proposition 5 du $\mathrm{n}^{\circ} 65$.

(b) $H^{q}\left(V, \mathcal{O}_{V}(n m)\right)=0$ pour $0<q<k$ et pour tout $n \geqq 1$.

Cela résulte de la Proposition 7 du $\mathrm{n}^{\circ} 65$.

(c) $H^{q}\left(V, \mathcal{O}_{V}(n m)\right)=0$ pour $0<q<k$ et pour tout $n \leqq-1$

Cela résulte du Théorème 2 du $\mathrm{n}^{\circ} 74$, et de l'hypothèse faite sur les $\mathcal{O}_{x, v}$ D'autre part, on a $H^{0}\left(V, \mathcal{O}_{v}\right)=K, H^{0}\left(V, \mathcal{O}_{v}(n m)\right)=0$ pour tout $n \leqq-1$, et $H^{q}\left(V, \mathcal{O}_{v}\right)=0$ pour $0<q<k$, en vertu de l'hypothèse. Il s'ensuit que $V_{m}$ vérifie toutes les hypothèses de la Proposition 3, cqfd.

CorolLatre. Soit $k$ un entier $\geqq 1$, et soit $V$ une variété projective sans singularités, de dimension $\geqq k$. Pour que $V$ soit birégulièrement isomorphe à une sousvariété $k$-fois de première espèce d'un espace projectif convenable, il faut et il sufit que $V$ soit connexe et que $H^{q}\left(V, \mathcal{O}_{V}\right)=0$ pour $0<q<k$.

La nécessité est évidente, d'après la Proposition 3. Pour démontrer la suffisance, il suffit de remarquer que $\mathcal{O}_{x, v}$ est alors de dimension $\leqq r-k$ (cf. $\left.\mathrm{n}^{\circ} 75\right)$ et d'appliquer la Proposition précédente.

78. Intersections complètes. Une sous-variété $V$ de dimension $p$ de l'espace projectif $\mathbf{P}_{r}(K)$ est une intersection complète si l'idéal $I(V)$ des polynômes nuls sur $V$ admet un système de $r-p$ générateurs $P_{1}, \cdots, P_{r-p} ;$ dans ce cas, toutes les composantes irréductibles de $V$ ont la dimension $p$, d'après le théorème de Macaulay (cf. $\left.[9], \mathrm{n}^{\circ} 17\right) .$ Il est bien connu qu'une telle variété est $p$-fois de première espèce, ce qui entraîne déjà que $H^{Q}\left(V, \mathcal{O}_{V}(n)\right)=0$ pour $0<q<p$, comme nous venons de le voir. Nous allons déterminer $H^{p}\left(V, \mathcal{O}_{V}(n)\right)$ en fonction des degrés $m_{1}, \cdots, m_{r-p}$ des polynômes homogènes $P_{1}, \cdots, P_{r-p}$.

Soit $S(V)=S / I(V)$ l'anneau de coordonnées projectives de $V .$ D'après le théorème $1 \mathrm{du} \mathrm{n}^{\circ} 72$, tout revient à déterminer le $S$-module $\operatorname{Ext}_{s}^{r-p}(S(V), \Omega)$. Or, on a une résolution analogue à celle du $\mathrm{n}^{\circ} 75:$ on prend pour $L^{q}$ le $S$-module gradué libre admettant pour base des éléments $e\left\langle i_{1} \cdots i_{q}\right\rangle$ correspondant aux suites $\left(i_{1}, \cdots, i_{q}\right)$ telles que $1 \leqq i_{1}<i_{2}<\cdots<i_{q} \leqq r-p$, et de degrés $\sum_{j=1}^{j=q} m_{j} ;$ pour $L^{0}$, on prend $S .$ On pose:

$$
\begin{gathered}
    d\left(e\left\langle i_{1} \cdots i_{q}\right\rangle\right)=\sum_{j=1}^{j=q}(-1)^{j} P_{i_{j}} \cdot e\left\langle i_{1} \cdots \hat{\imath}_{j} \cdots i_{q}\right\rangle \\
    d(e(i\rangle)=P_{i}
\end{gathered}
$$

La suite $0 \rightarrow L^{r-p} \stackrel{d}{\rightarrow} \cdots \stackrel{d}{\rightarrow} L^{0} \rightarrow S(V) \rightarrow 0$ est exacte ([6], Chap. VIII, Prop. 4.3). Il en résulte que les $\operatorname{Ext}_{s}^{q}(S(V), \Omega)$ sont les groupes de cohomologie du complexe formé par les $\operatorname{Hom}_{s}\left(L^{q}, \Omega\right) ;$ mais on peut identifier un élément de $\operatorname{Hom}_{s}\left(L^{q}, \Omega\right)_{n}$ à un système $f\left(i_{1} \cdots i_{a}\right\rangle$, où les $f\left(i_{1} \cdots i_{a}\right\rangle$ sont des polynômes homogènes de degrés $m_{i_{1}}+\cdots+m_{i_{q}}+n-r-1 ;$ une fois cette identification faite, l'opérateur cobord est donné par la formule usuelle:

$$
(d f)\left\langle i_{1} \cdots i_{q+1}\right\rangle=\sum_{j=1}^{j=q+1}(-1)^{j} P_{i_{i}} \cdot f\left\langle i_{1} \cdots \hat{\imath}_{j} \cdots i_{q+1}\right\rangle
$$

Le théorème de Macaulay déjà cité montre que l'on est dans les conditions de $[11]$, et l'on retrouve bien le fait que $\operatorname{Ext}_{3}^{9}(S(V), \Omega)=0$ pour $q \neq r-p$. D'autre part, $\operatorname{Ext}_{s}^{r-p}(S(V), \Omega)_{n}$ est isomorphe au sous-espace de $S(V)$ formé des éléments homogènes de degré $N+n$, avec $N=\sum_{i=1}^{i=r-p} m_{i}-r-1$. Compte tenu du Théorème 1 du n $^{\circ} 72$, on obtient:

Proposition 5. Soit $V$ une intersection complète, définie par des polynômes homogènes $P_{1}, \cdots, P_{r-p}$, de degrés $m_{1}, \cdots, m_{r-p}$ (a) L'application $\alpha: S(V)_{n} \rightarrow H^{0}\left(V, \mathcal{O}_{V}(n)\right)$ est bijective pour tout $n \in \mathbf{Z}$.

(b) $H^{q}\left(V, \mathcal{O}_{v}(n)\right)=0$ pour $0<q<p$ et tout $n \in \mathbf{Z}$.

(c) $H^{p}\left(V, \mathcal{O}_{V}(n)\right)$ est isomorphe a l'espace vectoriel dual de $H^{0}\left(V, \mathcal{O}_{V}(N-n)\right)$, avec $N=\sum_{i=1}^{\langle r-p} m_{i}-r-1$

On notera, en particulier, que $H^{p}\left(V, \mathcal{O}_{V}\right)$ n'est nul que si $N<0$.

\section{§6. Fonction caractéristique et genre arithmétique}

79. Caractêristique d'Euler-Poincaré. Soit $V$ une variété projective, et soit $\mathfrak{F}$ un faisceau algébrique cohérent sur $V$. Posons:

$$
h^{q}(V, \mathfrak{F})=\operatorname{dim}_{K} H^{q}(V, \mathfrak{F})
$$

Nous avons vu $\left(\mathrm{n}^{\circ} 66\right.$, Théorème 1 ) que les $h^{q}(V, \mathfrak{F})$ sont finis pour tout entier $q$, et nuls pour $q>\operatorname{dim} V .$ On peut donc définir un entier $\chi(V, \mathcal{F})$ en posant:

$$
\chi(V, \mathfrak{F})=\sum_{q=0}^{\infty}(-1)^{q} h^{q}(V, \mathfrak{F}) .
$$

C'est la caractéristique d'Euler-Poincaré de $V$, à valeurs dans $\mathfrak{F}$.

LEMME 1. Soit $0 \rightarrow L_{1} \rightarrow \cdots \rightarrow L_{p} \rightarrow 0$ une suite exacte, les $L_{i}$ étant des espaces vectoriels de dimension finie sur $K$, et les homomorphismes $L_{i} \rightarrow L_{i+1}$ étant $K$ linéaires. On a alors:

$$
\sum_{q=1}^{q=p}(-1)^{q} \operatorname{dim}_{K} L_{q}=0
$$

On raisonne par récurrence sur $p$, le lemme étant évident si $p \leqq 3 ;$ si $L_{p-1}^{\prime}$ désigne le noyau de $L_{p-1} \rightarrow L_{p}$, on a les deux suites exactes:

$$
\begin{aligned}
    &0 \rightarrow L_{1} \rightarrow \cdots \rightarrow L_{p-1}^{\prime} \rightarrow 0 \\
    &0 \rightarrow L_{p-1}^{\prime} \rightarrow L_{p-1} \rightarrow L_{p} \rightarrow 0
\end{aligned}
$$

En appliquant l'hypothèse de récurrence à chacune de ces suites, on voit que $\sum_{q=1}^{q=p-2}(-1)^{q} \operatorname{dim} L_{q}+(-1)^{p-1} \operatorname{dim} L_{p-1}^{\prime}=0$, et

$$
\operatorname{dim} L_{p-1}^{\prime}-\operatorname{dim} L_{p-1}+\operatorname{dim} L_{p}=0
$$

d'où aussitôt le Lemme.

Proposition 1. Soit $0 \rightarrow Q \rightarrow B \rightarrow \mathcal{C} \rightarrow 0$ une suite exacte de faisceaux algébriques cohérents sur une variété projective $V$, les homomorphismes $Q \rightarrow \mathbb{B}$ et $\mathbb{B} \rightarrow \mathbb{C}$ étant K-linéaires. On a alors:

$$
\chi(V, \mathbb{B})=\chi(V, \alpha)+\chi(V, \mathcal{e})
$$

D'après le Corollaire 2 au Théorème $5 \mathrm{du} \mathrm{n}^{\circ} 47$, on a une suite exacte de cohomologie:

$$
\cdots \rightarrow H^{q}(V, B) \rightarrow H^{q}(V, \mathcal{C}) \rightarrow H^{q+1}(V, \alpha) \rightarrow H^{q+1}(V, \beta) \rightarrow \cdots
$$

En appliquant le Lemme 1 à cette suite exacte d'espaces vectoriels, on obtient la Proposition.

Propositron 2. Soit $0 \rightarrow \mathfrak{F}_{1} \rightarrow \cdots \rightarrow \mathfrak{F}_{p} \rightarrow 0$ une suite exacte de faisceaux algé- briques cohérents sur une variété projective $V$, les homomorphismes $\mathcal{F}_{i} \rightarrow \mathcal{F}_{i+1}$ étant algébriques. On a alors:

$$
\sum_{q=1}^{q=p}(-1)^{q} \chi\left(V, \mathcal{F}_{q}\right)=0
$$

On raisonne par récurrence sur $p$, la Proposition étant un cas particulier de la Proposition 1 si $p \leqq 3$. Si l'on désigne $\operatorname{par} \mathscr{F}_{p-1}^{\prime}$ le noyau de $\mathcal{F}_{p-1} \rightarrow \mathfrak{F}_{v}$, le faisceau $\mathcal{F}_{p-1}^{\prime}$ est algébrique cohérent puisque $\mathfrak{F}_{p-1} \rightarrow \mathfrak{F}_{p}$ est un homomorphisme algébrique. On peut donc appliquer l'hypothèse de récurrence aux deux suites exactes

$$
\begin{aligned}
    &0 \rightarrow \mathfrak{F}_{1} \rightarrow \cdots \rightarrow \mathfrak{F}_{p-1}^{\prime} \rightarrow 0 \\
    &0 \rightarrow \mathfrak{F}_{p-1}^{\prime} \rightarrow \mathcal{F}_{p-1} \rightarrow \mathfrak{F}_{p} \rightarrow 0
\end{aligned}
$$

et la Proposition en résulte aussitôt.

80. Relation avec la fonction caractéristique d'un $S$-module gradué.

Soit $\mathfrak{\text { un faisceau algébrique cohérent sur l'espace }} \mathbf{P}_{r}(K) ;$ nous écrirons $\chi(\mathfrak{F})$ au lieu de $\chi\left(\mathbf{P}_{r}(K), \mathfrak{F}\right)$. On a:

Proposition 3. $\chi(\mathfrak{F}(n))$ est un polynôme en $n$ de degré $\leqq r$.

D'après le Théorème 2 du $n^{\circ} 60$, il existe un $S$-module gradué $M$, de type fini, tel que $a(M)$ soit isomorphe à $\mathfrak{F} .$ En appliquant à $M$ le théorème des syzygies de Hilbert, on obtient une suite exacte de $S$-modules gradués:

$$
0 \rightarrow L^{r+1} \rightarrow \cdots \rightarrow L^{0} \rightarrow M \rightarrow 0
$$

où les $L^{q}$ soint libres de type fini. En appliquant le foncteur $Q$ à cette suite, obtient une suite exacte de faisceaux:

$$
0 \rightarrow \mathcal{L}^{r+1} \rightarrow \cdots \rightarrow \mathcal{L}^{0} \rightarrow \mathfrak{F} \rightarrow 0
$$

où chaque $\mathcal{L}^{q}$ est isomorphe à une somme directe finie de faisceaux $\mathcal{O}\left(n_{i}\right) .$ La Proposition 2 montre que $\chi(\mathfrak{F}(n))$ est égal à la somme alternée des $\chi\left(\mathscr{L}^{q}(n)\right)$ ce qui nous ramène au cas du faisceau $O\left(n_{i}\right) .$ Or il résulte du no 62 que l'on a $\chi(\mathcal{O}(n))=\left(\begin{array}{c}n+r \\ r\end{array}\right)$, ce qui est bien un polynôme en $n$, de degré $\leqq r ;$ d'où la Proposition.

Proposition 4. Soit $M$ un S-module gradué vérifiant la condition (TF), et soit $\mathfrak{F}=\boldsymbol{Q}(M) .$ Pour tout $n$ assez grand, on a $\chi(\mathfrak{F}(n))=\operatorname{dim}_{\mathrm{K}} M_{n}$.

En effet, on sait $\left(\mathrm{n}^{\circ} 65\right)$ que, pour $n$ assez grand, l'homomorphisme $\alpha: M_{n} \rightarrow$ $H^{0}(X, \mathfrak{F}(n))$ est bijectif, et $H^{q}(X, \mathfrak{F}(n))=0$ pour tout $q>0 ;$ on a alors

$$
\chi(\mathfrak{F}(n))=h^{0}(X, \mathfrak{F}(n))=\operatorname{dim}_{K} M_{n}
$$

On retrouve ainsi le fait bien connu que $\operatorname{dim}_{K} M_{n}$ est un polynôme en $n$ pour $n$ assez grand: ce polynôme, que nous noterons $P_{M}$, est appelé la fonction caractéristique de $M ;$ pour tout $n \in \mathrm{Z}$, on a $P_{M}(n)=\chi(F(n))$, et, en particulier, pour $n=0$, on voit que le terme constant de $P_{M}$ est égal à $\chi(F)$.

Appliquons ceci à $M=S / I(V), I(V)$ étant l'idéal homogène de $S$ formé des polynômes nuls sur une sous-variété fermée $V$ de $\mathbf{P}_{r}(K)$. Le terme constant de $P_{M}$ est appelé, dans ce cas, le genre arithmétique de $V$ (cf. [19]); comme d'autre part on a $Q(M)=\mathcal{O}_{V}$, on obtient:

Proposition $5 .$ Le genre arithmétique d'une variété projective $V$ est égal a

$$
\chi\left(V, \mathcal{O}_{V}\right)=\sum_{q=0}^{\infty}(-1)^{q} \operatorname{dim}_{K} H^{q}\left(V, \mathcal{O}_{V}\right)
$$

RemarQues. (1) La Proposition précédente met en évidence le fait que le genre arithmétique est indépendant du plongement de $V$ dans un espace projectif, puisqu'il en est de même des $H^{q}\left(V, \mathcal{O}_{v}\right)$.

(2) Le genre arithmétique virtuel (défini par Zariski dans [19]) peut également être ramené à une caractéristique d'Euler-Poincaré. Nous reviendrons ultérieurement sur cette question, étroitement liée au théorème de Riemann-Roch.

(3) Pour des raisons de commodité, nous avons adopté une définition du genre arithmétique légèrement différente de la définition classique (cf. [19]). Si toutes les composantes irréductibles de $V$ ont la même dimension $p$, les deux définitions sont reliées par la formule suivante: $\chi\left(V, \mathcal{O}_{V}\right)=1+(-1)^{p} p_{a}(V)$.

81. Degré de la fonction caractéristique. Si $\mathfrak{F}$ est un faisceau algébrique cohérent sur une variété algébrique $V$, nous appellerons support de $\mathscr{F}$, et nous noterons $\operatorname{Supp}(\mathfrak{})$, l'ensemble des points $x \in V$ tels que $\mathscr{_}{x} \neq 0 .$ Du fait que $\mathscr{\text { est un }}$ faisceau de type fini, cet ensemble est fermé: en effet, si l'on a $\mathscr{F}_{x}=0$, la section nulle engendre $\mathfrak{F}_{x}$, donc aussi $\mathfrak{F}_{y}$ pour $y$ assez voisin de $x\left(\mathrm{n}^{\circ} 12\right.$, Proposition 1 ), ce qui signifie que le complémentaire de Supp(F) est ouvert.

Soit $M$ un $S$-module gradué de type fini, et soit $\mathfrak{F}=\alpha(M)$ le faisceau défini par $M$ sur $\mathbf{P}_{r}(K)=X$. On peut déterminer Supp $(\mathfrak{F})$ à partir de $M$ de la manière suivante:

Soit $0=\bigcap_{\alpha} M^{\alpha}$ une décomposition de 0 comme intersection de sous-modules primaires homogènes $M^{\alpha}$ de $M$, les $M^{\alpha}$ correspondant aux idéaux premiers homogènes $\mathfrak{p}^{\alpha}$ (cf. [12], Chap. IV); on supposera que cette décomposition est "la plus courte possible", i.e. qu'aucun des $M^{\alpha}$ n'est contenu dans l'intersection des autres. Pour tout $x \in X$, chaque $\mathfrak{p}^{\alpha}$ définit un idéal premier $\mathfrak{p}_{x}^{\alpha}$ de l'anneau local $\mathcal{O}_{x}$, et l'on a $\mathfrak{p}_{x}^{\alpha}=\mathcal{O}_{x}$ si et seulement si $x$ n'appartient pas à la variété $V^{\alpha}$ définie par l'idéal $\mathfrak{p}^{\alpha} .$ On a de même $0=\bigcap_{\alpha} M_{x}^{\alpha}$ dans $M_{x}$, et l'on vérifie sans difficulté que l'on obtient ainsi une décomposition primaire de 0 dans $M_{x}$, les $M_{x}^{\alpha}$ correspondant aux idéaux premiers $\mathfrak{p}_{x}^{\alpha} ;$ si $x \notin V^{\alpha}$, on a $M_{x}^{\alpha}=M_{x}$, et, si l'on se borne à considérer les $M_{x}^{\alpha}$ tels que $x \in V^{\alpha}$, on obtient une décomposition "la plus courte possible" (cf. [12], Chap. IV, th. 4 , où sont établis des résultats analogues). On en conclut aussitôt que $M_{x} \neq 0$ si et seulement si $x$ appartient à l'une des variétés $V^{\alpha}$, autrement dit $\operatorname{Supp}(\mathfrak{F})=\bigcup_{\alpha} V^{\alpha}$.

Proposition 6. Si $\mathfrak{F}$ est un faisceau algébrique cohérent sur $\mathbf{P}_{r}(K)$, le degré $d u$ polynôme $\chi(\mathfrak{F}(n))$ est égal à la dimension de $\operatorname{Supp}(\mathfrak{F})$.

Nous raisonnerons par récurrence sur $r$, le cas $r=0$ étant trivial. On peut supposer que $\mathcal{F}=\alpha(M)$, où $M$ est un $S$-module gradué de type fini; utilisant les notations introduites ci-dessus, nous devons montrer que $\chi(\mathscr{F}(n))$ est un polynôme de degré $q=$ Sup $\operatorname{dim} V^{\alpha}$. Soit $t$ une forme linéaire homogène n'appartenant à aucun des idéaux premiers $\mathfrak{p}^{\alpha}$, sauf éventuellement a l'idéal premier "impropre" $\mathfrak{p}^{0}=\left(t_{0}, \cdots, t_{r}\right) ;$ une telle forme existe du fait que le corps $K$ est infini. Soit $E$ l'hyperplan de $X$ d'équation $t=0$. Considérons la suite exacte:

$$
0 \rightarrow \mathcal{O}(-1) \rightarrow \mathcal{O} \rightarrow \mathcal{O}_{B} \rightarrow 0
$$

où $\theta \rightarrow \mathcal{O}_{B}$ est l'homomorphisme de restriction, tandis que $\mathcal{O}(-1) \rightarrow \mathcal{O}$ est l'homomorphisme $f \rightarrow$ t.f. Par produit tensoriel avec $\mathfrak{F}$, on obtient la suite exacte:

$$
\mathfrak{F}(-1) \rightarrow \mathfrak{F} \rightarrow \mathfrak{F}_{E} \rightarrow 0, \quad \text { avec } \quad \mathcal{F}_{E}=\mathfrak{F} \otimes_{\mathcal{O}} \mathcal{O}_{E}
$$

Au-dessus de $U_{i}$, on peut identifier $\mathfrak{F}(-1)$ à $\mathfrak{,}$ et cette identification transforme l'homomorphisme $\mathfrak{F}(-1) \rightarrow \mathfrak{F}$ défini ci-dessus en la multiplication par $t / t_{i} ;$ du fait que $t$ a été choisie en dehors des $\mathfrak{p}^{\alpha}, t / t_{i}$ n'appartient à aucun des idéaux premiers de $M_{x}=\mathscr{F}_{x}$ si $x \in U_{i}$, et l'homomorphisme précédent est injectif (cf. [12], p. 122, th. $\left.7, \mathrm{~b}^{\prime \prime \prime}\right)$. On a donc la suite exacte:

$$
0 \rightarrow \mathfrak{F}(-1) \rightarrow \mathcal{F} \rightarrow \mathfrak{F}_{E} \rightarrow 0
$$

d'où, pour tout $n \in \mathrm{Z}$, la suite exacte:

$$
0 \rightarrow \mathfrak{F}(n-1) \rightarrow \mathfrak{F}(n) \rightarrow \mathfrak{F}_{E}(n) \rightarrow 0 .
$$

En appliquant la Proposition 1, on voit que:

$$
\chi(F(n))-\chi(\mathscr{( n - 1 )})=\chi\left(\mathfrak{F}_{E}(n)\right)
$$

Mais le faisceau $\mathfrak{F}_{E}$ est un faisceau cohérent de $\mathcal{O}_{E}$-modules, autrement dit est un faisceau algébrique cohérent sur $E$, qui est un espace projectif de dimension $r-1$. De plus, $\mathscr{F}_{x, E}=0$ signifie que l'endomorphisme de $\mathcal{F}_{x}$ défini par la multiplication par $t / t_{i}$ est surjectif, ce qui entraîne $\mathcal{F}_{x}=0$ (cf. [6], Chap. VIII, prop. 5.1'). Il s'ensuit que $\operatorname{Supp}\left(\mathfrak{F}_{E}\right)=E \cap \operatorname{Supp}(\mathfrak{F})$, et, comme $E$ ne contient aucune des variétés $V^{\alpha}$, il s'ensuit par un résultat connu que la dimension de $\operatorname{Supp}\left(\mathscr{F}_{E}\right)$ est égale à $q-1$. L'hypothèse de récurrence montre alors que $\chi\left(\mathscr{F}_{E}(n)\right)$ est un polynôme de degré $q-1 ;$ comme c'est la différence première de la fonction $\chi(\mathfrak{F}(n))$, cette dernière est donc bien un polynôme de degré $q$.

Remaroues. (1) La Proposition 6 était bien connue lorsque $\mathcal{F}=\mathcal{O} / \mathrm{g}, \mathrm{g}$ étant un faisceau cohérent d'idéaux. Cf. $[9], \mathrm{n}^{\circ} 24$, par exemple.

(2) La démonstration précédente n'utilise pas la Proposition 3 , et la démontre donc à nouveau.

PARIS

\section{BI BLIOGRAPHIE}

[1] N. BoURBAKI. Théorie des Ensembles. Paris (Hermann).

[2] H. CARTAN. Séminaire E.N.S., 1950-1951.

[3] H. CARTAN. Séminaire E.N.S., 1951-1952.

[4] H. CARTAN. Séminaire E.N.S., 1953-1954.

5] H. CARTAN. Variétés analytiques complexes et cohomologie. Colloque de Bruxelles, (1953), p. 41-55. [6] H. CARTAN and S. ErLENBERG. Homological Algebra. Princeton Math. Ser., nº 19.

[7] S. EILENBERG and N. E. STEENROD. Foundations of Algebraic Topology. Princeton Math. Ser., $n^{\circ} 15 .$

[8] W. V. D. HodGe and D. PEDOE. Methods of Algebraic Geometry. Cambridge Univ. Press.

[9] W. KRULL. Idealtheorie. Ergebnisse IV-3. Berlin (Springer).

[10] J. LERAY. L'anneau spectral et l'anneau filtré d'homologie d'un espace localement compact et d'une application continue. J. Math. Pures App., $29,(1950)$, p. 1-139.

[11] G. De Ruam. Sur la division de formes et de courants par une forme linéaire. Comment. Math. Helv., 28,(1954), p. 346-352.

[12] P. SAMUEL. Commutative Algebra (Notes by D. Hertzig). Cornell Univ., $1953 .$

[13] P. SamueL. Algèbre locale. Mém. Sci. Math., CXXIII, Paris, $1953 .$

[14] J-P. SERRE. Groupes d'homotopie et classes de groupes abéliens. Ann. of Math., 58 , (1953), р. 258-294.

[15] J-P. SeRRE. Un théorème de dualité. Comment. Math. Helv., $29,(1955)$, p. 9-26.

[16] A. WEIL. Foundations of Algebraic Geometry. Colloq. XXIX.

[17] A. WerL. Fibre-spaces in Algebraic Geometry (Notes by A. Wallace). Chicago Univ., 1952

[18] O. ZARISKI. The concept of a simple point of an abstract algebraic variety. Trans. Amer. Math. Soc., $62,(1947)$, p. $1-52 .$

[19] O, ZARISKI, Complete linear sustems on normal varieties and a generalization of a lemma of Enriques-Severi. Ann. of Math., 55 , (1952), p. 552-592.