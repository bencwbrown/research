\documentclass[11pt]{amsart}

\usepackage[parfill]{parskip}    
\usepackage{graphicx}
\usepackage{amssymb, amsfonts,mathabx}
\usepackage{epstopdf}

\usepackage{amsmath, amsthm}
\usepackage[all,cmtip]{xy}

\usepackage{tikz}
\usepackage{tikz-cd}
\usetikzlibrary{matrix,arrows,patterns,calc,through,backgrounds,fadings, decorations}
\usetikzlibrary{decorations.pathreplacing}

\newtheorem{theorem}{Theorem}[section]
\newtheorem{lemma}[theorem]{Lemma}
\newtheorem*{lemma*}{Lemma}
\newtheorem{proposition}[theorem]{Proposition}
\newtheorem{corollary}[theorem]{Corollary}
\newtheorem{definition}[theorem]{Definition\rm}
\newtheorem{conjecture}[theorem]{Conjecture}
\newtheorem{remark}{\it Remark\/}
\newtheorem{example}{\it Example\/}

\newcommand{\st}{\ensuremath{:}}% such that
\newcommand{\ie}{\emph{i.e.} }
\newcommand{\eg}{\emph{e.g.} }
\newcommand{\cf}{\emph{cf.} }
\newcommand{\al}{\alpha}
\newcommand{\la}{\lambda}
\newcommand{\w}{\omega}
\newcommand{\m}{\mu}
\newcommand{\n}{\nu}
\newcommand{\e}{\epsilon}
\newcommand{\K}{K\"ahler }
\newcommand{\HK}{hyperk\"ahler }
\newcommand{\into}{\hookrightarrow}
\newcommand{\PP}{\mathbb{P}}
\newcommand{\RR}{\mathbb{R}}
\newcommand{\CC}{\mathbb{C}}
\newcommand{\QQ}{\mathbb{Q}}
\newcommand{\FF}{\mathbb{F}}
\newcommand{\ZZ}{\mathbb{Z}}
\newcommand{\NN}{\mathbb{N}}
\newcommand{\HH}{\mathbb{H}}
\newcommand{\vp}{\varphi}
\newcommand{\mc}[1]{\mathcal{#1}}
\newcommand{\mcL}{\mathcal{L}}
\newcommand{\mcO}{\mathcal{O}}
\newcommand{\mf}[1]{\mathfrak{#1}}
\newcommand{\mfg}{\mathfrak{g}}
\newcommand{\mfh}{\mathfrak{h}}
\newcommand{\mft}{\mathfrak{t}}

\DeclareMathOperator{\Lie}{\text{Lie}}
\DeclareMathOperator{\Aut}{Aut}
\DeclareMathOperator{\Tr}{Tr}
\DeclareMathOperator{\Image}{Im}
\DeclareMathOperator{\Ad}{Ad}
\DeclareMathOperator{\Diff}{Diff}
\DeclareMathOperator{\Vect}{Vect}
\DeclareMathOperator{\Sympl}{Sympl}
\DeclareMathOperator{\Span}{Span}

\usepackage{hyperref}

\title{Preprint 01}
\author{Benjamin C. W. Brown}
\address[Benjamin Brown]{School of Mathematics and Maxwell Institute, The University of Edinburgh, Peter Guthrie Tait Road, Edinburgh EH9 3FD, United Kingdom}
\email{B.Brown@ed.ac.uk}
\date{\today}  
\thanks{}                                         
\begin{document}

\begin{abstract}

\end{abstract}

\maketitle

\section{Introduction}

\subsection*{Acknowledgements}  

\section{Symplectic Toric Manifolds \& Orbifolds} \label{sec:symplectic-toric-manifolds-orbifolds}

\section{Hypertoric Manifolds \& Symplectic Cutting} \label{sec:hypertoric-manifolds-symplectic-cutting}



\subsection{Symplectic Toric Manifolds} \label{subsec:toric-manifolds}

\subsection{Symplectic Toric Orbifolds} \label{subsec:toric-orbifolds}

The symplectic toric manifolds and their associated Delzant polytopes in the previous subsection were generalised to symplectic toric orbifolds in \cite{LT-1997}, where the associated polytope \emph{non-basic}, that is we weaken the conditions on the edge vectors to each vertex so that they no longer need to be form a $\ZZ$-basis.











\section{Index Theory and Equivariant Localisation} \label{sec:index-theory-localisation}

\subsection{Equivariant Index Formula} \label{subsec:equivariant-index}

Let $T$ be an $n$-torus, $M$ a $T$-manifold equipped with pre-quantisation data $(\mcL, \langle\, ,\, \rangle, \nabla)$ and a $T$-equivariant complex structure.





\providecommand{\bysame}{\leavevmode\hbox to3em{\hrulefill}\thinspace}
\providecommand{\href}[2]{#2}

\bibliographystyle{unsrt}
\bibliography{bibliography}

\end{document}  










